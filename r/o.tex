\chapter*{О}
\addcontentsline{toc}{chapter}{О}

\textbf{Объедки} – народное название гастронома на углу дома по Большой Житомирской, 8.\\

\medskip

\textbf{Обсерваторный яр} – засыпан в начале 20 века, находился у склона с обсерваторией, у перекрестка Коцюбинского и Обсерваторной. В него до 1880-х свозили снег и лед с улиц, а позже навоз и мусор. На пригорке над бывшим яром стоит школа №155.\\

\medskip

\textbf{Омелютинка}, ручей

Начало открытого русла:\\50°24'33.7"N 30°33'33.3"E

Конец открытого русла:\\ 50°24'17.3"N 30°33'42.7"E

Протекает вдоль Тимирязевской улицы, в овраге за забором ботанического сада за хоздвором, между горами Караваевкой и Бусовой. Вытекает из трубы и следует в заросшем, темном овраге, левый берег коего (гора Караваевка) крутой и суглинный. Там много клещей. Местами Омелютинка разливается в овраге на несколько рукавов.

Ручей уходит около здания Киевзеленстроя внизу Тимирязевской в коллектор, именуемый диггерами Net Cave из-за обилия паутины. Под землей течет на север и вытекает из трубы в Выдубицкое озеро.

Название ручья происходит от бывшей улицы Омелютинской, которая стала частью Тимирязевской. Тимирязевская = Омелютинская + Военно-Кладбищенская. При этом Омелютинская с запада ограничивала Еврейское кладбище, а с востока его границей была нынешняя Кленовая аллея, прежняя улица Еврейско-кладбищенская. Улица же Омелютинская названа так от усадьбы Омелютинского, которая была там по крайней мере на стыке 19-20 годов.

Раньше, ручей начинался выше на северо-восток от хоздвора, и образовывал несколько прудов по ходу – сохранилась даже дамба\footnote{50°24'43.24"N,  30°33'38.95"E}. В ложбине одного из этих прудов сейчас Японский сад.

Судя по всему, у ручья было два истока:

Северный, ближе к бывшей улице Караваевской, ныне аллее ботсада: 

50°24'46.11"N 30°33'45.87"E

Южный, ближе ко Кленовой аллее (прежней улице Еврейскокладбищенской):

50°24'48.84"N 30°33'40.01"E

Ниже хоздвора, гора левого берега ручья слывет Караваевкой.

Подробнее о ручье читайте в «Ереси о Киеве».\\

\medskip

\textbf{Орбита} – длинный магазин, занимавший первый этаж углового дома на бульваре Леси Украинки, 19. Внутри формой повторял угол, занимал весь его, на два крыла. Просуществовал по 1990-е. Там продавали телевизоры и прочую технику. Люди, встречаясь, говорили – «возле Орбиты».\\

\medskip

\textbf{Орбита} – кинотеатр в угловом доме Крещатик, 29, на самом углу. В какой-то мере протянул по начало нулевых, там был уже интернет-клуб, а в девяностые на втором этаже продавались игровые приставки и картриджи к ним, а также видеокассеты. Закрыт в 2009 году, последний показанный фильм – «Франкенштейн», цена билетов на то время – 24 гривны для взрослых, 15 для детей.\\

\medskip

\textbf{Орешка} – ореховая роща на склоне Зверинецкой горы между бульваром Дружбы народов и улицей Мичурина. Напротив холма со Зверинецким кладбищем. Одно из мест катаний на санках в 1990-е. С нулевых постепенно застраивается сверху и снизу. В рощу можно попасть, выйдя на остановке «улица Струтинского». Некогда часть Орешки уже была застроена частными усадьбами, а местность террасирована. К Орешке примыкало Святое озеро сверху, и Наводницкий ручей снизу.\\

\medskip

\textbf{Ореховатка} – речка, названная по хутору Ореховому. Но сначала о ней самой.

Речка существует в открытом русле как водоток, соединяющий один за другим пруды в Голосеевском парке. Еще в первой половине 20 века пруд был один, он и сейчас есть – который наибольший и самый нижний, возле станции метро «Голосеевская». А остальные пруды были болотом (хотя возможно, в более отдаленном прошлом на месте болота существовали пруды).

К среднему пруду, со стороны теперешнего Голосеевского проспекта, простиралось кладбище\footnote{50°23'26.1"N 30°30'12.2"E} – эдак напротив перекрестка Голосеевского проспекта с Маричанской улицей (последняя названа в честь речки Маричанки безосновательно – Маричанка там не протекала). Кладбище вписывалось в эдакий ромб, образуемый поныне существующими тропами в том месте.

Современная водная система Ореховатки имеет два истока.

Северный, исток А, начинается (уже под землей, конечно) на территории Института рака, в овраге по четной стороне улицы Сеченова, вдоль номеров 10, 6, потом пересекает улицу и проходит под южной стороной Сеченова 3, выходит к Голосеевскому проспекту около номера 108 корпус 2, пересекает проспект и углубляется в лес Голосеевского парка, где вскоре вливается\footnote{50°23'14.4"N 30°29'49.1"E} к первому из Верхних Ореховатских прудов, а их три, и разделены они дамбами.

Южный, исток Б, течет сюда через лес примерно от ул. Героев Обороны, 10. Он вливается в начало второго из Верхних прудов.

На территории парка ручей протекает на поверхности. После Верхних прудов идет Средний, затем через некоторое расстояние – Нижний, что возле Голосеевской площади. На этом отрезке со стороны Голосеевского проспекта в Ореховатку впадает\footnote{Примерно 50°23'30.3"N 30°30'32.3"E} приток, что появляется на поверхности в лесу за остановкой «Михаила Стельмаха». 

С улицей Стельмаха связан и другой приток. 
От самого нижнего пруда Ореховатка уходит в коллектор. Со стороны Голосеевской площади подходил приток, что вливался в Ореховатку ниже по течению от пруда, т.е. на северо-восток от него, там где сейчас здание на улице Голосеевской, 53\footnote{50°23'49.5"N 30°30'44.4"E}. Этот приток начинался у пересечения\footnote{50°23'52.0"N 30°30'00.7"E} улиц Стельмаха (ее верховье) и Водогонной и протекал по большому яру наискосок к площади, на юго-восток. Верховья яра хорошо заметны у стадиона при школе на Васильковской 12-А.

После Нижнего пруда, Ореховатка течет в подземном коллекторе вдоль нечетной стороны улицы Голосеевской до проспекта Науки и промзоны, пройдя под которой пересекает улицу Гринченко и сразу выливается из прямоугольного портала в Лыбедь\footnote{50°24'19.0"N 30°31'44.0"E}, близ восточного угла маргаринового завода на Саперно-Слободской, на противоположной стороне от него. На вид, поток воды меньше чем у Совки.

Сразу после устья Ореховатки, сама Лыбедь из поверхностного русла прячется в коллектор.

Коллектор же Ореховатки диггеры называют Ореховатка, Туманная или Маргариновая (от одноименного завода).\\

\medskip


\textbf{Ореховой хутор} – хутор, известный в 19 веке в Голосеево, от него и Ореховатские пруды названы, и речка Ореховатка, через пруды бегущая.\\

\medskip

\textbf{Остров, Нахаловка}, хутор

50°22'54"N 30°34'31"E

Хаотичный поселок без адресов, расположен в устье Лыбеди, за канализационной станцией «Правобережная», откуда несет особенным запахом. Самодельные дома, деревянные заборы.

По рассказам старожила, еще в тридцатые годы 20 века сюда перебирались жить из Осокорков, так возник хутор Остров. На карте тридцатых годов там виден, среди песков и верболозов, растянувшийся в некотором отдалении от берега поселок на десятка два дворов, подписанный Осокорки – стало быть, переселенцы называли его так. Лыбедь повыше хутора поворачивала на юг, оставляя хутор между собой и Днепром, и впадала в залив Николайчик. 

В конце 1960, при строительстве пятой ТЭЦ и вообще промзоны Телички (имеет название поселок Комсомольский), хутор, по словам старожила, стали теснить и снесли много домов – из чего можно заключить, что дома стояли тогда не только на правом берегу Лыбеди, но и на Левом.

Потомков «первопоселенцев» тут на 21 век почти не осталось, селятся все, кто хочет и может.\\

\medskip


\textbf{Отрадный}, район – возник в 1914 году как хутор, находился по железной дороге юго-запад\-нее хутора Грушки, около нынешнего перекрестка улицы Радищева с переулком Радищева. Насчитывал несколько улиц с переулками, было много садов. Одна из улиц была названа в честь коллежского регистратора Константина Яниховского – Константиновской (позже Радищева). Именно он выкупил в указанном году земли у местных крестьян и основал хутор Отрадный. Ныне на месте самого хутора – промзона.\\


\medskip

\textbf{Охмадет} – «Охраны материнства и детства», целый комплекс детских медицинских учреждений между проспектом Победы и улицей Андрющенко. Возник на месте нескольких дореволюционных больниц Товарищества чернорабочих.\\


%Летом 1915 года, направляясь на фронт, в Киеве остановился полк, который прежде стоял в станице Отрадной на Кубани. Подразделения расположились на юго-западной окраине города во временных палатках. Для штаба солдаты соорудили несколько домиков и обсадили их деревьями. Среди широкого поля создался хутор, который в память о станице был назван Отрадным