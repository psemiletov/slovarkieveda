\chapter*{Д}
\addcontentsline{toc}{chapter}{Д}


\textbf{Даронова гора} – на стыке 19 и 20 веков, так назывался склон между нынешним Олимпийским стадионом и Косым Капониром. Насколько я понимаю, Даронова гора – другое название Черепановой горы.\\

\textbf{Далекий брод} – в середине 19 века, переезд на Лыбеди в районе около центрального вокзала, ниже впадения Скомороха, как урочище подписан на карте 1860 года на правой стороне Лыбеди, не доходя от устья Скомороха до Батыевой горы.\\

\textbf{Дары} – бытовавшее по девяностые годы название известного овощного магазина «Дари ланів» на Бастионной, 1, на первом этаже. В 21 веке его помещение занимает то автосалон, то банк. 

В советское время, рядом с «Дарами» слева располагался магазин «Трикотаж» с разными тканями, иголками, наперстками и готовой одеждой.\\

\textbf{Дача Хрущова} – остатки жилой усадьбы в Кмитовом яру. В 1889 году тут купил участок земли Октавиан Бельский, аптекарский помощник. Спустя четыре года он начал по проекту архитектора Николая Казанского строить тут особняк, а заимев уже собственную аптеку на Подоле, Бельский поставил рядом с особняком в Кмитовом яру доходный дом.

После революции местность, имевшую вид парка с двумя особняками, до тридцатых годов не трогали, ходили сюда все кому не лень, а потом обнесли забором и устроили жилье для номенклатуры. До 1937 года тут проживал Балицкий, нарком внутренних дел УССР. Потом его расстреляли, усадьбу отдали под пионерлагерь детей сотрудников НКВД.

В 1943 году место детей занял Хрущев, а когда он перебрался в Москву, «дача» осталась резиденцией первых секретарей ЦК. По слухам, под дачей был зал для заседаний и шел подземный ход со зданиями бывшей партийной школы. В 1978 году рядом построили Институт педиатрии, акушерства и гинекологии и усадьба вошла в его территорию.

В усадьбе дачи лежит мрачное верховье Кмитова яра с каскадом прудов, через которые перетекает ручей – один из истоков Глубочицы. Поверх переброшены мостики, или один мостик, не помню. Всё носит живописные следы разрушения. По дорожкам мамы катают в колясках детей.\\

\textbf{Дача Хрущова} – другая, см. Васильчикова дача.\\

\textbf{Дачная} – окрестности автостанции Дачная (проспект Победы, 142А) близ Академгородка.\\

\textbf{ДВС} – см. Водогон.\\

\textbf{Дворцовый сад} – в 19 веке, название нынешнего Городского сада, там где стадион Динамо.\\

\textbf{Девич-гора, Дивич-гора} – ныне, и примерно со второй половины 19 века известна как Лысая гора, та что ниже Зверинца. Упоминается в документах 16 века, как владение Михайловского монастыря – Дивич гора, земля Орининская, Орыновская.

Представляет собой большой, покрытый лиственными деревьями и разнотравьем наверху холм с остатками Лысогорского форта. На Лысой горе растут и уничтожаются древние, возрастом много веков, дубы.\\

\textbf{Дерперская роща} – на середину 19 века занимала Кристерову горку и окрестности. Занимала местность от нынешней Кристеровой горки (что к северу от улицы Осипова) и на юг к ручью Коноплянке и прудам на оном. В роще находился хутор графа Эстерази, как раз где сейчас известен старинный дом Кристера. До Эстерази, там был хутор Дерпера, от чего и пошло название.\\

\textbf{Дехтяги, Дегтяри} – район между Нивками и Галаганами, частный сектор по улице Януша Корчака от улицы Эстонской до Краснодарской. Еще в 1940-х считался хутором. Хутор был основан крестьянином села Беличи, Василием Дехтяренко. Между домами номер 22 и 28 по улице Ставропольской, а точнее в Ставропольском переулке, есть маленькое, но старое кладбище\footnote{50°27'56.7"N 30°24'46.3"E}. В 1970 году останки с кладбища перенесли на Берковецкое, в 1980-х здесь устроили спортивную площадку, но там никто из местных не играл, ибо знали, что это место кладбища. На 2021 год – заросший травой пустырь. Окрестности кладбища и есть исторический центр Дегтярей. В районе Нивок проживает много людей с Фамилией Дехтягенко.\\
%В 1876 году Григорий Галаган отдал в Дегтярях усадьбу, под ремесленное училище. 

\textbf{Дегтяры} – условно так можно назвать овраг межу горами Валовой и Детинкой, там где улица Дегтярная. Плотно застроен новыми домами, которые путают с близлежащим кварталом Воздвиженкой, названным так по Воздвиженской улице и занявшим урочища Гончары и Кожемяки. В склоне над Дегтярами начинается ручей Киянка, приток Глубочицы.\\

\textbf{Деловой двор} – существовал в 19 веке, отсюда и Деловая улица. Стоял на перекрестке Деловой и нынешней Большой Васильковской (Красноармейской).\\

\textbf{Детинка, Дытынка} 

50°27'31.7"N 30°30'42.2"E

Гора на север от БЖ, эдакий длинный отрог, что лежит между Клинцом, Замковой, Валовой и Старокиевской горами, или, если угодно, между ярами-урочищами Дегтяры и Гончары (улица Дегтярная и Гончарная). Крыши строек по обе стороны Детинки вровень с плато самой горы. Она имеет любопытный, удобный для обороны вид сверху – толстое основание у материка, затем узкий перешеек, затем расширение, как бы голова. Достаточно охранять перешеек малыми силами.

Склоны из суглинка, заросли травой и деревьями. Самая северная часть отрога – лысая. По верху всей горы идет тропа от Пейзажной аллеи (БЖ).\\ 

\textbf{Дидоровка}
 
50°22'32"N 30°30'9"E

Удольный пруд в Голосеево, к южной стороне водоема примыкает очень высокая гора с лыжной трассой, а берега оборудованы избушками для пикников, за сидение в избушках берется плата. К Дидоровке добираются по горной дороге от Сельхозакадемии, что пролегает прямо через ботанический сад, мимо Голосеевской пустыни.

В первой половине 20 века, Дидоровка относилась к лежащим от нее на сервер землям совхоза Голосиевка, что был по дороге к Покровскому Голосеевскому монастырю (пустыни).\\

\textbf{Диск} – парк в Корчеватом. Назван так по одноименному кинотеатру, развалины коего еще виднеются.\\

\textbf{Дниструиха}, озеро

См. Корчи.\\

\textbf{Довнар} – улица Довнар-Запольского и окрестности, от Ванды Василевской до Лукьяши. По адресам 1/12, 3/1, 3/2, 4 лежит микрорайон домов в стиле конструктивизма, построенный в 1920-30-х (чего по ним не скажешь), по проекту архитектора Михаила Аничкина. Часть домов «конструктивистского» микрорайона относится к улице Коперника – номера 20, 18.\\


\textbf{Дом Гинсбурга} - первый киевский небоскреб в 11 этажей, стоял на холме чуть севернее нынешней гостиницы "Украина" на Институтской улице - через улицу напротив южного крыла Института благородных девиц. Короче говоря, там где сейчас автостоянка перед гостиницей. 

Построен в 1910-12 годах по проекту архитекторов Адольфа Минкус и Фёдора Троупянского. Строителем и владельцем был купец первой гильдии Лев Борисович Гинсбург. Стоимость здания оценивалась в полтора миллиона рублей. На постройку ушло 12 миллионов кирпичей. 

Высота каждого этажа равнялась 4 метрам, здание венчала башня в 10 метров. Дом после постройки использовался как гостиница и под сдачу квартир. В 1917 году, аренда в нем квартиры на год стоила 1300-1700 рублей. Сам Гинсбург жил в двухэтажном особняке неподалеку.

После революции дом Гинсбурга национализировали и превратили в коммуналку. В 1941 году в нем квартировал Иван Кудря, хранивший там свои документы для "легенды" и прочие важнын в подпольной деятельности вещи, и когда небоскреб вместе с многими другими зданиями на Крещатика был взорван в 24 сентября 1941 года - раньше писали, что немцами, нынче пишут, что нашими.
 
В Киеве на улице Городецкого, 9 есть еще один "дом Гинсбурга", поменьше, тоже был доходный дом - роскошный, украшенный скульптурами. Он тоже пострадал во время взрыва, лишившись трех башенок.

Стоит рядом с кинотеатром "Украина".\\

\textbf{Дом Рыбальского}

50°27'48"N   30°30'53"E

Адрес: ул. Нискольско-Притисская, 5

В этом доме, примыкающем к Флоровскому монастырю, некогда проживал Георгрий Рыбальский, бывший киевским войтом в 1797-1813 годах. Само одноэтажное здание, кажется, построено еще раньше. В нем был подвальный этаж. 

Дом пережил подольский пожар 1811 года, а когда в 1813 году Рыбальский умер, дом унаследовали потомки. В 1857 году здание перекупили и сделали рядом двухэтажную пристройку - она больше первичного дома и уходит как бы вглубь Флоровского монастыря.

Если стоять лицом к дому, то справа будут ворота, а еще правее них еще один одноэтажный дом, тоже под номером 5. Так вот дом Рыбальского - только "левый", тот, что расположен дальше от входа в монастырь.


Глядя на этот по нынешним меркам небогатый, в пять окон, домик, на ум приходят несколько мыслей. Либо войт Рыбальский был скромен, либо тогда остальные жили много хуже. Или же у людей были другие потребности.\\


\textbf{Дом старой почты} - местное название трехэтажного, 1951 года постройки дома по Татарской 32Б. Раньше там, на первом этаже, была почта, а еще прежде - конюшня, клуб, кинотеатр.

В бытность там почты, оная использовалась как "здание гестапо" в фильме "Нина" о киевских подпольщиках, семье Сосниных.\\

\textbf{ДОК}

50°24'24"N 30°34'36"E

Дерево-обрабатывающий комбинат, расположен в промзоне Теличке. Рабочих его называли «доковцами». На 2021 год, в части корпусов расположились другие фирмы. К заводу подходит улица Деревообрабатывающая.\\

\textbf{Дом Самолёт} – дом справа от станции метро «Арсенальная». Построен в 1930-х по плану Каракиса.\\

\textbf{Древлянская площадь} – во второй половине 19, начале 20 веков – площадь у перекрестка нынешних улиц Дегтярёвской, Якира и Зоологической. Неподалеку площади начинал по поверхности течь ручей Скоморох. А по другую сторону от площади и Старожитомирской дороги (ныне Дегтярёвская), близ западной стороны площади, по 1880-е располагался ипподром.\\

\textbf{Дружбы} – станция метро Дружбы Народов.\\

\textbf{Душегубица} – на плане Ушакова 1695 года, Наводницкий овраг обозначен как «боярак Душегубица». «Душегубица» более нигде мне не встречалась в земельных документах. А вдруг это искаженное упоминание «дороги на Выдубичи», как обычно подписано примерно там на других, хотя и более позднейших картах?\\

\textbf{Дымерский шлях} – давняя дорога из Киева через Межигорье, Вышгород, затем в Дымер и Чернобыль.

%Дурка