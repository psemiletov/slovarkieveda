\chapter*{Д}
\addcontentsline{toc}{chapter}{Д}


\textbf{Даронова гора} – на стыке 19 и 20 веков, так назывался склон между нынешним Олимпийским стадионом и Косым Капониром. Насколько я понимаю, Даронова гора – другое название Черепановой горы.\\

\medskip


\textbf{Далекий брод} – в середине 19 века, переезд на Лыбеди в районе около центрального вокзала, ниже впадения Скомороха, как урочище подписан на карте 1860 года на правой стороне Лыбеди, не доходя от устья Скомороха до Батыевой горы.\\

\medskip

\textbf{Дары} – бытовавшее по девяностые годы название известного овощного магазина «Дари ланів» на Бастионной, 1, на первом этаже. В 21 веке его помещение занимает то автосалон, то банк. 

В советское время, рядом с «Дарами» слева располагался магазин «Трикотаж» с разными тканями, иголками, наперстками и готовой одеждой.\\

\medskip

\textbf{Дача Хрущова} – остатки жилой усадьбы в Кмитовом яру. В 1889 году тут купил участок земли Октавиан Бельский, аптекарский помощник. Спустя четыре года он начал по проекту архитектора Николая Казанского строить тут особняк, а заимев уже собственную аптеку на Подоле, Бельский поставил рядом с особняком в Кмитовом яру доходный дом.

После революции местность, имевшую вид парка с двумя особняками, до тридцатых годов не трогали, ходили сюда все кому не лень, а потом обнесли забором и устроили жилье для номенклатуры. До 1937 года тут проживал Балицкий, нарком внутренних дел УССР. Потом его расстреляли, усадьбу отдали под пионерлагерь детей сотрудников НКВД.

В 1943 году место детей занял Хрущев, а когда он перебрался в Москву, «дача» осталась резиденцией первых секретарей ЦК. По слухам, под дачей был зал для заседаний и шел подземный ход со зданиями бывшей партийной школы. В 1978 году рядом построили Институт педиатрии, акушерства и гинекологии и усадьба вошла в его территорию.

В усадьбе дачи лежит мрачное верховье Кмитова яра с каскадом прудов, через которые перетекает ручей – один из истоков Глубочицы. Поверх переброшены мостики, или один мостик, не помню. Всё носит живописные следы разрушения. По дорожкам мамы катают в колясках детей.\\

\medskip

\textbf{Дача Хрущова} – другая, см. Васильчикова дача.\\

\medskip


\textbf{Дачная} – окрестности автостанции Дачная (проспект Победы, 142-А) близ Академгородка.\\

\medskip

\textbf{ДВС} – см. Водогон.\\

\medskip

\textbf{Дворцовый сад} – в 19 веке, название нынешнего Городского сада, там где стадион Динамо.\\

\medskip

\textbf{Девич-гора, Дивич-гора} – ныне, и примерно со второй половины 19 века известна как Лысая гора, та что ниже Зверинца. Упоминается в документах 16 века, как владение Михайловского монастыря – Дивич гора, земля Орининская, Орыновская.

Представляет собой большой, покрытый лиственными деревьями и разнотравьем наверху холм с остатками Лысогорского форта. На Лысой горе растут и уничтожаются древние, возрастом много веков, дубы.\\

\medskip

\textbf{Дерперская роща} – на середину 19 века занимала Кристерову горку и окрестности. Занимала местность от нынешней Кристеровой горки (что к северу от улицы Осипова) и на юг к ручью Коноплянке и прудам на оном. В роще находился хутор графа Эстерази, как раз где сейчас известен старинный дом Кристера. До Эстерази, там был хутор Дерпера, от чего и пошло название.\\

\medskip

\textbf{Дехтяги, Дегтяри} – район между Нивками и Галаганами, частный сектор по улице Януша Корчака от улицы Эстонской до Краснодарской. Еще в 1940-х считался хутором. Хутор был основан крестьянином села Беличи, Василием Дехтяренко. Между домами номер 22 и 28 по улице Ставропольской, а точнее в Ставропольском переулке, есть маленькое, но старое кладбище\footnote{50°27'56.7"N 30°24'46.3"E}. В 1970 году останки с кладбища перенесли на Берковецкое, в 1980-х здесь устроили спортивную площадку, но там никто из местных не играл, ибо знали, что это место кладбища. На 2021 год – заросший травой пустырь. Окрестности кладбища и есть исторический центр Дегтярей. В районе Нивок проживает много людей с Фамилией Дехтягенко.\\

\medskip
%В 1876 году Григорий Галаган отдал в Дегтярях усадьбу, под ремесленное училище. 

\textbf{Дегтяры} – условно так можно назвать овраг межу горами Валовой и Детинкой, там где улица Дегтярная. Плотно застроен новыми домами, которые путают с близлежащим кварталом Воздвиженкой, названным так по Воздвиженской улице и занявшим урочища Гончары и Кожемяки. В склоне над Дегтярами начинается ручей Киянка, приток Глубочицы.\\

\medskip

\textbf{Деловой двор} – существовал в 19 веке, отсюда и Деловая улица. Стоял на перекрестке Деловой и нынешней Большой Васильковской (Красноармейской).\\

\medskip

\textbf{Детинка, Дытынка} 

50°27'31.7"N 30°30'42.2"E

Гора на север от БЖ, эдакий длинный отрог, что лежит между Клинцом, Замковой, Валовой и Старокиевской горами, или, если угодно, между ярами-урочищами Дегтяры и Гончары (улица Дегтярная и Гончарная). Крыши строек по обе стороны Детинки вровень с плато самой горы. Она имеет любопытный, удобный для обороны вид сверху – толстое основание у материка, затем узкий перешеек, затем расширение, как бы голова. Достаточно охранять перешеек малыми силами.

Склоны из суглинка, заросли травой и деревьями. Самая северная часть отрога – лысая. По верху всей горы идет тропа от Пейзажной аллеи (БЖ).\\ 

\medskip

\textbf{Дидоровка}
 
50°22'32"N 30°30'9"E

Удольный пруд в Голосеево, к южной стороне водоема примыкает очень высокая гора с лыжной трассой, а берега оборудованы избушками для пикников, за сидение в избушках берется плата. К Дидоровке добираются по горной дороге от Сельхозакадемии, что пролегает прямо через ботанический сад, мимо Голосеевской пустыни.

В первой половине 20 века, Дидоровка относилась к лежащим от нее на сервер землям совхоза Голосиевка, что был по дороге к Покровскому Голосеевскому монастырю (пустыни).\\

\medskip

\textbf{Диск} – парк в Корчеватом. Назван так по одноименному кинотеатру, развалины коего еще виднеются.\\

\medskip

\textbf{Дниструиха}, озеро

См. Корчи.\\

\medskip

\textbf{Довнар} – улица Довнар-Запольского и окре\-стности, от Ванды Василевской до Лукьяши. По адресам 1/12, 3/1, 3/2, 4 лежит микрорайон домов в стиле конструктивизма, построенный в 1920-30-х (чего по ним не скажешь), по проекту архитектора Михаила Аничкина. Часть домов «конструктивистского» микрорайона относится к улице Коперника – номера 20, 18.\\

\medskip

\textbf{Дом Гинсбурга} – первый киевский небоскреб в 11 этажей, стоял на холме чуть севернее нынешней гостиницы «Украина» на Институтской улице – через улицу напротив южного крыла Института благородных девиц. Короче говоря, там где сейчас автостоянка перед гостиницей. 

Построен в 1910-12 годах по проекту архитекторов Адольфа Минкуса и Фёдора Троупянского. Строителем и владельцем был купец первой гильдии Лев Борисович Гинсбург. Стоимость здания оценивалась в полтора миллиона рублей. На постройку ушло 12 миллионов кирпичей. 

Высота каждого этажа равнялась 4 метрам, здание венчала башня в 10 метров. Дом после постройки использовался как гостиница и под сдачу квартир. В 1917 году, аренда в нем квартиры на год стоила 1300-1700 рублей. Сам Гинсбург жил в двухэтажном особняке неподалеку.

После революции дом Гинсбурга национализировали и превратили в коммуналку. В 1941 году в нем квартировал Иван Кудря, хранивший там свои документы для «легенды» и прочие важные в подпольной деятельности вещи, и когда небоскреб вместе с многими другими зданиями на Крещатика был взорван в 24 сентября 1941 года – раньше писали, что немцами, нынче пишут, что нашими.
 
В Киеве на улице Городецкого, 9 есть еще один «дом Гинсбурга», поменьше, тоже был доходный дом – роскошный, украшенный скульптурами. Он тоже пострадал во время взрыва, лишившись трех башенок. Стоит рядом с кинотеатром «Украина».\\


\medskip


\textbf{Дом Понятовского}

%https://ru.wikipedia.org/wiki/%D0%9A%D0%B8%D0%B5%D0%B2%D1%81%D0%BA%D0%BE%D0%B5_%D0%B4%D0%B2%D0%BE%D1%80%D1%8F%D0%BD%D1%81%D0%BA%D0%BE%D0%B5_%D1%81%D0%BE%D0%B1%D1%80%D0%B0%D0%BD%D0%B8%D0%B5
%Маврикием Понятовским по проекту А. В. Беретти. Г

Не сохранился, снесен в 1976 году. Помещик, камергер Ламберт-Маврикий Понятовский\footnote{Усадьба была куплена его отцом, полковником Иосифом Понятовским, в 1828 году.} в 1851 году 
возвел, несколько не доведя строительство до конца, кирпичный трехэтажный дом примерно по месту северного перечения Майдана и Крещатика\footnote{50.450560, 30.524440}, по несколько переделанному в Петербурге проекту киевского городского архитектора Людвига Станзани.

Газета «Киевские епархиальные ведомости» писала:

\begin{quotation}
Дом этот представлял собой в середине ХІХ в. наглухо забитую безлюдную постройку. Как это вышло? Дом строил польский магнат Понятовский. Этому вельможе какой-то ксёндз навеял, что с завершением постройки он умрет. Перепуганный вельможа прекратил сооружение своего дворца, забил окна и двери и дом превратил в такое легендарное здание. В конце 1850-х годов с этого дома сняли темную легенду, оживили его, открыли окна и двери. Тут была устроена художественная выставка.\end{quotation}

Затем на первом этаже Понятовский сдавал помещения под магазины, а в 1861 году продал здание Дворянское депутатскому собранию Киевской губернии.

Прежде чем я продолжу о судьбе здания, проследим судьбу Понятовского. Он умер 8 мая 1878 года в другом своем доме, на Липках. Вдова Понятовского заказала жившему в Риме скульптору Бродскому за 30000 франков мраморную сидячую статую покойного. В 1882-м Понятовская намеревалась поставить оную статую в Александровском костеле, близ престола, как бы уравнивая усопшего мужа со святыми, но духовенство не поддержало начинание. Что до самого Понятовского, то могила его поныне есть на Байковом...

Итак, дом Понятовского стал домом Дворянского собрания, где оно собственно сходилось на собрания, устраивало маскарады, концерты и выставки, а в 1865 году там в трех комнтатах была устроена, хотя через несколько лет переехала, еще и Публичная библиотека Киева. В 1895 году Собрание сдало часть своей усадьбы в аренду Киевскому обществу взаимного кредита, и то пристроило своё здание, фасадом подобное бывшему дому Понятовского. 

После революции 1917 года в здании Собрания, на Крещатике 16, поместилась военная комендатура – то немцев, то красных – первым советским комендантом города было Щорс. Затем здание стало Домом работников просвещения, потом Домом учителя, хотя на первом этаже продолжали располагаться разные магазины.

Еще позже там расположились цеха типографского предприятия Полиграфкнига, и в них работала моя бабушка, Татьяна Федоровна Бородина.

Наконец на том же углу появился Дом Профсоюзов...\\

\medskip

%\textbf{Дом с ирисами}
%https://kyivcity.net/kiev/dom-s-irisami-v-kieve-gde-naxoditsya-pochemu-tak-nazyvaetsya-i-v-kakom-on-sostoyanii-foto/
%\medskip

\textbf{Дом Рыбальского}

50°27'48"N   30°30'53"E

Адрес: ул. Нискольско-Притисская, 5

В этом доме, примыкающем к Флоровскому монастырю, некогда проживал Георгий Рыбальский, бывший киевским войтом в 1797-1813 годах. Само одноэтажное здание, кажется, построено еще раньше. В нем был подвальный этаж.

Дом пережил подольский пожар 1811 года, а когда в 1813 году Рыбальский умер, дом унаследовали потомки. В 1857 году здание перекупили и сделали рядом двухэтажную пристройку – она больше первичного дома и уходит как бы вглубь Флоровского монастыря.

Если стоять лицом к дому, то справа будут ворота, а еще правее них еще один одноэтажный дом, тоже под номером 5. Так вот дом Рыбальского – только «левый», тот, что расположен дальше от входа в монастырь.

Глядя на этот по нынешним меркам небогатый, в пять окон, домик, на ум приходят несколько мыслей. Либо войт Рыбальский был скромен, либо тогда остальные жили много хуже. Или же у людей были другие потребности.\\

\medskip

\textbf{Дом старой почты} – местное название трехэтажного, 1951 года постройки дома по Татарской 32-Б. Раньше там, на первом этаже, была почта, а еще прежде – конюшня, клуб, кинотеатр.

В бытность там почты, оная использовалась как «здание гестапо» в фильме «Нина» о киевских подпольщиках, семье Сосниных.\\

\medskip


\textbf{ДОК}

50°24'24"N 30°34'36"E

Дерево-обрабатывающий комбинат, расположен в промзоне Теличке. Рабочих его называли «доковцами». На 2021 год, в части корпусов расположились другие фирмы. К заводу подходит улица Деревообрабатывающая.\\

\medskip

\textbf{Дом Самолёт} – дом справа от станции метро «Арсенальная». Построен в 1930-х по плану Каракиса.\\

\medskip

\textbf{Древлянская площадь} – во второй половине 19, начале 20 веков – площадь у перекрестка нынешних улиц Дегтярёвской, Якира и Зоологической. Неподалеку площади начинал по поверхности течь ручей Скоморох. А по другую сторону от площади и Старожитомирской дороги (ныне Дегтярёвская), близ западной стороны площади, по 1880-е располагался ипподром.\\

\medskip

\textbf{Дружбы} – станция метро Дружбы Народов.\\

\medskip

\textbf{Душегубица} – на плане Ушакова 1695 года, Наводницкий овраг обозначен как «боярак Душегубица». «Душегубица» более нигде мне не встречалась в земельных документах. А вдруг это искаженное упоминание «дороги на Выдубичи», как обычно подписано примерно там на других, хотя и более позднейших картах?\\

\medskip

\textbf{Дымерский шлях} – давняя дорога из Киева через Межигорье, Вышгород, затем в Дымер и Чернобыль.

%Дурка