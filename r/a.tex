\chapter*{А}
\addcontentsline{toc}{chapter}{А}


\textbf{Авиагородок} - район, лежающий на северо-запад от станции метро "Святошин", ограниченный проспектом Победы, бульваром Вернадского, улицей Краснова и железной дорогой. Изначально был застроен частными домами работников близлежащего (к востоку) авиазавода (ныне "Антонов"). Сейчас там дома всех сортов - и остаток частного сектора, и двухэтажки, и хрущовки.\\

\textbf{Автострада} - старое название бульвара Дружбы Народов.\\

\textbf{Академка, Академ} – Академгородок, жилой массив на месте села Беличи. По начало 20 века, среди высоток, на улице Прилужной оставались частные дома Беличей, например заброшенный дом среди болота, на Прилужной 2\footnote{50°27'42"N 30°20'23"E} – непосредственно на восток от Ушакова, 18. В декабре 2013 году дом на исчезнувшей улице горел, а до того стоял целый, с решетками на окнах, электрификацией и газовой трубой. Другие дома на Прилужной, уцелевшие – номера 9 и 11, с запада примыкающие к школе и стадиону.\\

\textbf{Александровская гора} - дореволюционное, не прижившееся навание Владимирской горки купно с верхним ея уступом, где Михайловский монастырь.\\

\textbf{Александровская площадь} – до революции, нынешняя Контрактовая площадь.\\ 

\textbf{Александровская слободка (Костопальня)} – местность к северу от Краснозвездного проспекта, между улицами Клинической, Народной, Ольги Кобылянской. На начало 21 века это частный сектор, расположенный по верху холма. Улицы полого спускаются оттуда в Божков Яр, и хотя имеют стройное расположение, там нехитро заблудиться.

Район граничит с Соломенкой, Монтажником, Проневщиной.

Жители слободки в 1913 году обращались в городскую думу, поясняя возникновение слободки следующим образом:

\begin{quotation}
В последние годы жизнь в Киеве настолько подорожала и квартиры по\-днялись в цене, что бедному рабочему люду стало невозможно жить в городе. Между нами – мастеровыми Юго-Западной железной дороги – возникла мысль создать селение на окраине города. Таким образом, на пахотных землях села Совки возник наш поселок.

На приобретение участков под уса\-дьбы мы потратили наши последние деньги, а на обустроение жилья были вынуждены обратиться за ссудой. 

Наш поселок каждый год увеличивается и сейчас насчитывает около 300 усадеб, причем 200 усадеб уже имеют жилье с населением до 2000 душ.
\end{quotation}

На то время в поселке было 9 улиц, привожу названия первоначальные и последующие: Преображенская (проходит насквозь слободки), Гоголевская (Космодемьянской), Садовая (Краснопартизанская), Суворовская (Белгородская), Алексеевская, Озёрная, Николаевская (Горвица), Скобелевская (ныне Златопольская), Успенская (Херсонская, не сохранилась), Дачная (Пироговского),  Васильевская.

Слободка входила в состав Никольско-Борща\-говской волости Киевского уезда и губернии. Слободку присоединили к Киеву в 1923 году.

Одно из прежних названий слободки – Костопальня, от заводов по переработке костей, что работали тут в 19 веке, выжигая (перепаливая) кости. Полученный уголь шел на производство сахара, в частности для сахарорафинадного завода на Демиевке. Вонь в округе стояла страшная. Кости заводам поставлялись с Соломенского скотомогильника и Бактериологического института в Протасовом яру.

Улица Яслинская на Костопальне называется так от построенного там в 1912 году, за деньги предпринимателя Гладенюка, приюта или яслей для сирот.\\

\textbf{Александрия}, город – в 1909 году собирались из поселков Верхней и Нижней Соломенки, Протасова яра, Кучмина яра и Батыевой горы образовать административную единицу – город Александрию.\\

\textbf{Алексеевский парк} – после 1914 года под него обустроили территорию Всероссийской выставки 1913 со всеми ее монументальными и не очень зданиями числом около двухсот. На месте этого угасшего после гражданской войны парка ныне – стадион «Олимпийский», Театр оперетты, Планетарий, Институт физкультуры и окрестности.\\

\textbf{Аллея падших} – в конце 1960-х, начале 1970-х, часть бульвара Шевченко от памятника Ленина (снесен в 2014, стоял внизу бульвара) к перекрестку с Пушкинской. Служила местом сходбищ хиппи и прочих неформалов. Некоторые хиппи носили на шее жестяные дореволюционные бляхи дворников.\\

\textbf{Анатомикум} - так называли в 19 веке здание анатомического театра, где сейчас Музей медицины на Хмельницкого, 37. Здание построено в 1853 по проекту Александра Беретти - сына Викетиня Беретти (архитектора, по чьим проектам построены Обсерватория, Институт благородных девиц, Киевский универ).\\

\textbf{Анненковская улица} – старое название Лютеранской.\\

\textbf{Аносовский сад, Аносовский парк} – на стыке 19-20 веков, небольшой сад вдоль нынешней улицы Лаврской, между Николаевским военным собором и старообрядческой церковью. Примерно окрестности нынешней Аллеи Славы к мемориалу Вечной славы.\\

\textbf{Антифеевка} – см. Старая поляна.\\

\textbf{Антифеевка} – невесть как возникший топоним около Кирилловской церкви и Шполянки.\\


%\textbf{Антифеевка} – другая Антифеевка, на Куренёвке, на север от Кирилловской церкви, можно сказать часть Копыловки либо ее параллельное название. Антифеев был квартальным околоточным, которому давали взятки самосёлы, чтобы их не трогали. Граничит со Шполянкой.\\

\textbf{Аполонник} – знаменитый в середине 20 века продуктовый магазин на углу Нижнеюрковской и Фрунзе, там, где теперь дом 2-6/32. Назывался так по фамилии владельца, купца Кузьмы Ефимовича Полонника. Здание снесено в 1986 году, теперь там новый дом.\\

\textbf{Аркадия} – дореволюционный сад стыка 19-20 веков, располагался между Бибиковским бульваром, анатомическим театром на нем, улицей Фундуклеевской и Малой Владимирской (Чкалова-Гончара). В современных пределах – район между Коцюбинского, Шевченко, Гончара, Хмельницкого.\\ 

\textbf{Арсенал}, совхоз – в 1940-х находился между Совками и Жулянами, южнее начала улицы Кайсарова, на одной линии с Нижним Ореховатским прудом, что в парке им. Рыльского.\\

\textbf{Арсенальский дом} – хрущовка по адресу Бастионная 13, с неизменным с советских времен продуктовым магазином на первом этаже. Дом был ведомственным, от завода «Арсенал», построенный для рабочих-арсенальцев.\\

\textbf{Артиллерийская лаборатория} на Зверинце – занимала местность между Госпитальным кладбищем и нынешними улицами Зверинецкой и Тимирязевской.\\

\textbf{Афанасьевский яр} – см. Святославский яр.\\

\textbf{Афанасьевский ручей} – течет в коллекторе под улицами Чапаева, Чкалова (Гончара) к цирку, впадает\footnote{50°26'26.4"N 30°29'27.8"E} в Лыбедь ниже Скомороха, у разворотного кольца трамвая. Длина коллектора – около километра.\\

\textbf{Афанасьевский яр} – еще один Афанасьевский яр, однако в Демиевке, невесть где именно. Упомянут в адресных справочниках того времени, например «Весь Киев» 1915 года, где в яру том две усадьбы: Ефимова П. И. и Курченко И. К.
