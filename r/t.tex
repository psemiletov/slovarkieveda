\chapter*{Т}
\addcontentsline{toc}{chapter}{Т}

\textbf{Табачка}

50°26'58.3"N 30°29'16.6"E

Снесена в 21 веке, Киевская табачная фабрика на улице Володарского (Златоустовская). Производила  «Столичные», «Славутич», «Приму» (последняя была выпущена впервые именно на этой фабрике). Находилась в начале улицы между домами 3 и 23, насыщая окрестности запахом табака.\\

\medskip

\textbf{Теличка} – бывшая «деревня Теличко», селение, разделялась на Верхнюю и Нижнюю. В документе «Описанию Посполитым слободи Зверинця в Уезде киевском состоящого, учиненногое 1786 года», среди описаний владений Выдубицкого монастыря сказано:

\begin{quotation}
Там же на Либеди между кирпичными артилерийскими заводами близ мельнички\footnote{Выдубицкая мельница, стояла на малой плотине на Лыбеди.} в обивательской избе Григория Телички продается только ценная горелка в оног обывателя монастирской посуды: кварта – 1; полкварта – 1; чвертка – 1, жестяныя.\end{quotation}

\medskip

\textbf{Теличка}, озеро

Озеро существовало по крайней мере по 1960-е, ниже Выдубицкого озера, отделенное от него железной дорогой. Застроено промзоной вдоль улицы Деревообрабатывающей. Озеро было большое, примерно между улицами Стройиндустрии и Баренбойма. Представляло собой один из остатков прежнего, ближайшего к правому берегу Днепра рукава Старика, от которого остались, с севера на юг, Выдубицкое озеро, озеро Теличка, а потом Лысогорский рукав (примыкавший вплотную к Лысой горе). Между озером Теличкой и Лысогорским рукавом и было устье Лыбеди.

На карте 1902 года озеро Теличка названо Стариком.\\


\medskip

\textbf{Темп} – существовавший по 1990-е гастроном на первом этаже углового дома по бульвару Дружбы Народов, 34/2. Длинный магазин на много отделов. В западной части дома был еще хлебный магазин с отдельным входом, через возвышенное, со ступеньками крыльцо. Сейчас заполнение первого этажа разными магазинами совершенно переменилось. На Левом берегу тоже был свой Темп, в Дарнице на Симферопольской, 9.\\

\medskip

\textbf{Тесные улицы} – в 16-18 веках, местность к юго-западу от нынешней Севастопольской площади. В 16 веке принадлежала Михайловскому златоверхому монастырю. Продолжались в сторону вокзала Кучминым яром.\\

\medskip

\textbf{Три могилы} – на плане 1750 года урочище примерно в районе где улица Нагорная смыкается со Смородинским спуском. Нарисованы, один над другим, три кургана (могилы).\\

\medskip


\textbf{Трипдача} – Татарка, одноэтажный старенький домик номер 4 на улице Отто Шмидта. Раньше тут был кожно-венерический диспансер.\\


\medskip

\textbf{Трипольский гостинец} – старинная дорога по берегу Днепра от Лыбедь до Триполья.\\


\medskip

\textbf{Троицкая площадь} – одна из таковых в дореволюционном Киеве, поглощена Куренёвским парком. В середине 19 века, к северу к ней примыкал земляной пороховой погреб, а еще севернее – казармы.\\

\medskip

\textbf{Троицкая площадь} – около Республиканского, ныне Олимпийского стадиона. Прежде Базарная.\\

\medskip

\textbf{Троллейбусное депо №1} – по 2015 год годы находилось у перекрестка улиц Ивана Кудри и Красноармейской (Б. Васильковская), теперь там жилой комплекс «Французский квартал». Рядом, на другой стороне улицы Ивана Кудри\footnote{50°24'49"N 30°31'45"E}, был завод электротранспорта, основанный в 1904 году, и в 2005 году перенесенный в Подольское трамвайное депо. Завод чинил троллейбусы, трамваи, вагоны метро, до закупок чешских троллейбусов, до 1970-х производил свои под маркой «Киев».\\


\medskip

\textbf{Тропа Хо Ши Мина}

50°27'02.6"N 30°31'36.8"E

Координаты условны. Тропа в зарослях и между дворами на задворках гостиницы Днепр и за Октябрьским дворцом. Условно говоря внутренняя часть угла Крещатика и Грушевского. Начиналась слева от дворца, шла к гостинице «Днепр», сворачивала к дому номер 4. Проходила в том числе во двориках зданий 19 века, мимо Стены Цоя с его портретом и граффити.

На начало лета 2021 года можно подойти только к началу Тропы. Если подняться ко Дворцу и пойти в его сторону, обращенную к Крещатику, будет спуск в виде аллеи через некие деревья и кусты. Он приведет вас к железному забору, за которым – котлован с убогими, вдохновляющими домами и мостиком над двором. Вдоль левого идет убитая лестница и упирается в калитку в заборе, возле коего нанесен мусор. Смотрите иллюстрации в разделе картинок. Дальше калитки не пройти, то есть узкий коридор внутренних дворов, в том числе со стеной Цоя, недоступен.

Раньше был также проходняк в доме что слева, со стороны конца Крещатика.\\

\medskip

\textbf{Труба} – подземный переход под Майданом, в определенные годы там собирались нефоры.\\

\medskip

%\textbf{Турец} – «смуговина Турец» у Иорданского озера, из актов 17-18 веков. Пролив, вытекавший из Иорданского озера по Оболони, к старинному валу, что длился от Юркового ставка.

\textbf{Турецкий городок} – он же Кадетский гай от одноименной улицы. Граничит с Проневщиной и Соломенкой – с первой туда добираются пехом через Верхний каскад Совских прудов, с Соломинки же – на троллейбусе и маршрутках. Изначально построен (дома в 10-16 этажей) турками, на немецкие деньги, в девяностые, для выведенного из Восточной Германии советского «военного контингента».\\

\medskip

\textbf{Турецкое кладбище} – в 19 веке, примыкало к южной части Еврейского кладбища на Зверинце. Обозначено на карте 1879 года, причем меньшим, нежели Еврейское.\\

\medskip

\textbf{Тюремный замок} – в 19, начале 20 веков – здание нынешнего КПЗ на Лукьяновке. Граничил на западе с Древлянской площадью, а на востоке с Лукьяновской. Вдоль замка проходила с одной стороны Старо-Житомирская дорога (ныне там Дехтеревская улица), а параллельно, но с другой стороны, была улица Тюремная. Вдоль восточной границы замка протекал, выруливая из яра с нынешней Белорусской улицей, ручей Скоморох, бегущий потом вдоль Бердичевской улицы.
