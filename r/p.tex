\chapter*{П}
\addcontentsline{toc}{chapter}{П}

\textbf{Палестина}

50°27'17"N 30°31'38"E

Здание насосной станции и водного фильтра «Палестина» сохранилось поныне, это небольшой зеленоватый, зефирный домик слева, как если спускаться по Владимирскому спуску. Имеет адрес Владимирский спуск, 4. Построено по проекту архитектора Шлифера в 1886-87 годах, сюда поступала вода из нижней машинной станции, что при набережной Днепра, близ колонны Магдебургкому праву.

\medskip

\textbf{Пан-гора}

50°28'30.7"N 30°28'33.6"E

Пригорок, склон между началом улицы Нагорной и Подольским спуском. Если ехать или спускаться последним, то сразу у его верха, справа будет выпуклый травяной холм с рощей наверху, туда взбирается тропа. Этом холм и есть Пан-гора. В хорошую погоду можно видеть отдыхающих прямо на склоне людей. Ниже по спуску от Пан-горы, в сторону приближающегося колена Смородинского спуска – другие урочища – Туи, Нижние Туи, Дальняя горка.\\

\medskip

\textbf{Панкратьевский ручей} – он же, у диггеров, Бродвей.

Устье в Днепр: 50°26'06.0"N 30°33'55.4"E

Название от Панкратьевского\footnote{Полковник Пётр Прокофьевич Панкратьев (1757-1810) был с 1802 года киевским гражданским губернатором. Его похоронили в 1810 году на кладбище при бывшем Кирилловском монастыре рядом с женой, Елизаветой Ивановной Литке. Могила с памятником-колонной, увенчанной шаром и крестом, была разрушена непонятно когда, при расширении территории психиатрической больницы.} яра, который был засыпан дамбой в 1853 году – упорядочивали Панкратьевский спуск (позже названный Днепровским) – и дорогу спуска переименовали в честь царя, в Николаевскую. Осадочную воду из яра стали отводить в т.н. Царский колодец – коллектор, куда присоединили и окрестные дренажки 15, 16, 27, 28, известные среди диггеров как Николка – от Никольского спуска.

Царский колодец, глубиной около 20 метров, соединяет Аскольдовку с Николкой.\\ 

\medskip

\textbf{Пантюхов хутор} – на стыке 19-20 веков, большой хутор в Бабьем яру, на северном его склоне, западнее Кирилловской больницы. Принадлежал семейству Пантюховых. Известна работа врача и антрополога Ивана Ивановича Пантюхова «Куреневка. Медико-антропологи\-ческий очерк» (Киев, 1904, печатано в «Трудах Общества Киевских Врачей» за 1903-1904 гг.). Сын его, Михаил Пантюхов, был писателем-декадентом и умер в Кирилловской больнице для умалишенных, вероятно покончил с собой. Некоторые считают его прообразом Мастера у Булгакова.\\

\medskip

\textbf{Паньковщина} – село, затем хутор к юго-востоку от речки Мокрой, на правом берегу Лыбеди, примерно на Батыевой горе и частично в Кучмином яру. Название, бытовавшее еще в первой половине 19 столетия, постепенно вытеснилось расширившейся Соломенкой.\\

\medskip

\textbf{Паньковщина} – деревня или слобода, на левом берегу Лыбеди, напротив одноименного давнего села, затем хутора Паньковщины (см. выше). Деревня Паньковщина занимала нынешнюю местность между Университетом и центральным вокзалом.

Название вероятно произошло от семейства земян (землевладельцев) Паньковичей, Василия и сына его Максима, известных по документам в 16 веке. Оба были наместниками (управляющими на местах) владениями киевских Софийских митрополитов и, судя по жалобам, отжимали церковные земли как у своих патронов, так и прочих монастырей.\\

\medskip

\textbf{Паньковский ручей} – приток Лыбеди, истоки его заточены в дренажку в ботсаду им. Фомина, впадает в Лыбедь перед мостом на улице Льва Толстого. Самое начало ручья сокрыто под красным корпусом универа, а в его подвал можно попасть через один из двух подземных коридоров-коллекторов.\\

\medskip

\textbf{Парк Ватутина} – Марьинский парк. Поколение СССР называет его именно так.\\

\medskip


\textbf{Парк Космонавтов}

50°30'02.4"N 30°27'09.8"E

Старое народное название сквера на улице Вышгородской, в советское время тут на детской площадке были две горки-ракеты, а также стенд с портретами космонавтов.\\

\medskip

\textbf{Пасынча беседа} – летописное урочище, уп\-оминается в летописи в связи с церковью святого Ильи, «яже есть над ручьем конец пасынце беседы и козаре се бо бе сборная церкви, мнози бо беша Варязе христьяни», куда князь Игорь водил для клятвы христианскую часть своей дружины, состоящей из народа Руси. Сопоставить в современных ориентирах я не берусь, в отличие от археологов.\\

\medskip

\textbf{Пентагон} – в 1970-80 так именовали здание главного управлении милиции на Софийской площади. Ибо пятиугольно это здание.\\

\medskip

\textbf{Перевесище} – летописное урочище. Согласно Нестору, «а Перевесище бе вне города». На плане Киева 1860 года Перевесище обозначено на отрезке от низа Шелковичной до начала Жандарской (Саксаганского), то есть удолье с Дворцом спорта.\\

\medskip

\textbf{Передславино} – Владимир поселил Рогнеду на Лыбеди в месте, где при Несторе было «селце Передславино». Улица Предславинская и предположение археологов о том, где находилось «селце», основаны на домыслах.\\

\medskip

\textbf{Переезд} – железнодорожный переезд в месте, где проспект Науки выруливает мимо Китаево к Корчеватому, пересекая улицу Пироговский шлях.\\

\medskip

\textbf{Пересечение}

50.458649519053665, 30.421054559785173

Путепровод  чуть восточнее метро «Берестейская», где пересекаются проспект Победы и улицы Лагерная и Дегтеревская.\\

\medskip

\textbf{Пески}  – думается, так в определенное время именовался Зверинецкий склон, где ныне улица Мичурина. Почему так думаю?

В «Путеводителе Киева» 1884 года издания есть список улиц и урочищ, не обозначенных на прилагаемом к путеводителю плане. И там же усадьбы, лежащие по этим улицам. И вот речь заходит об окрестностях Зверинца. Читаю – улица «Наводницкая (на зверинец)», усадьбы таких-то. «Набережная р. Днепра (от Наводницкой)» – там частные дома и прачечная военного госпиталя. Затем идет про улицу Зверинецкую (тогдашнюю, не нынешннюю), делится она на две части, по направлениям.

«Зверинецкая от завода Доната к пруду». Чугунолитейный и машиностроительный Завод Федора Доната располагался, с 1855 или 1862 до 1915 года там, где Суворовское училище. Что за пруд имеется в виду, я не знаю. Упомянуты усадьбы: «Инженера Доната, Исаевой, Иванова» и т.д.

Затем в путеводителе прописана «Зверинецкая (к Выдубецкому монастырю), идущая по горе, с усадьбами – св. Троицкого монастыря (Ионы), Выдубицкого монастыря, Хмелька, Смирновой» и т.д.

И далее сказано: «На Песках – Гранчиной, Черепановой, Ковалева», много-много фамилий, и потом рядом идут «Матевеенковой, Зайченко» – а это известные персонажи истории об открытии Зверинецких пещер, и на время открытия оных в 1880-х они жили на Ломаковской улице, ныне Мичурина.\\

\medskip

\textbf{Песчаный} – ручей, левый приток Лыбеди, на большой протяженности взят в коллектор, кроме территории зоопарка.

Начинается около Троллейбусное депо № 2 КП «Киевпасстранс» и дома на Дегтяревской 35/9, где было засыпанное ныне верховье яра. 

Оттуда в коллекторе ручей пересекает улицу Довженко и проходит под домом Довженко, 14/1, затем, несколько отклоняясь от природного русла на юг (на природном сейчас стоит детсад 112), следует вдоль улицы Молдавской под пожарным депо (Молдавская 3А), мимо  бойлерной, ныряет под Западный РКС ПАО «Киевэнерго» (Довженко 12Б), оттуда к бизнес-центру по адресу Дегтярёвская 25А, и проходит вдоль северной границы парка Пушкина, между нею и промзоной. Затем пересекает улицу Зоологическую немного к северо-востоку от 5-го номера\footnote{В точке примерно 50°27'30.1"N 30°27'40.0"E}, после чего переходит в зоопарк чуть южнее сенохранилища\footnote{50°27'31"N 30°27'42"E}.

В зоопарке по ходу ручья устроены пруды.

Ручей коллектором выходит из зоопарка у перекрестка Тбилисского переулка и улицы Виктора Ермолы, прямо от зеленых служебных ворот зоопарка.

Под улицей Ермолы, ближе к четной стороне, коллектор идет к перекрестку с улицей Ванды Василевской, пересекает ее и дом 6/8 левее арки, там где АТБ, и следует дальше по улице Ермолы почти до самого ее конца, а затем уходит под восточный угол дома на проспекте Победы, 22. Далее коллектор пересекает отрезок Провиантской улицы, выходящий тут к проспекту Победы, затем пересекает сам проспект и углубляется во дворы оного по нечетной стороне, откуда выходит к дому на Борщаговской, 8. Пересекает на юг Борщаговскую улицу, и за автомойкой двумя трубами впадает\footnote{50°26'50.2"N 30°28'26.1"E} в Лыбедь.\\

\medskip

\textbf{Петербурская слобода} – небольшой поселок, показанный на карте 1833 года у подножия холма Ландшафтного парка, на протяженности от долготы моста Патона и по долготу Лавры. Мимо нее шла дорога к Наводницкой переправе. На углу холма, при въезде в устье оврага Наводничей, стоял шлагбаум Московской заставы. Сверху над слободой, по холму, рос Комендантский сад. Хотя на картах слободу не балуют отображением, на плане 1885 года там же – «обывательские строения».\\

\medskip


\textbf{Петровка} – собирательное название окрестностей станции метро «Петровка», а также одноименный книжно-дисковый рынок и толкучка (по выходным дням) при оном.\\

\medskip

%расширить

\textbf{Петровка} – местное, в 1950-70, название улицы Петровской, когда она была населена.\\

\medskip

\textbf{Петропавловская площадь} – на Куренёвке, некогда площадь, занимавшая место нынешнего сквера близ Птички (Куренёвского Птичьего рынка). 

Ныне сохранился осколок частного сектора с адресами Петропавловской площади, вдоль трамвайной линии 16 маршрута, между нею и Птичкой – усадьбы от 3 до 11. Номер 11 – двухэтажный, около него трамвайное кольцо и площадь, где как и вдоль всей «улицы Петропавловская площадь» по выходным собирается барахолка. Там же – смычка с улицей Семена Скляренко и на север – с железнодорожным мостом над нею, и высокой железнодорожной же насыпью, поросшей старыми уже деревьями. За нею к северу начинается Приорка и лежащая параллельно насыпи улица Резервная – еще один клочок частного сектора.\\

\medskip

%Печерский рынок


\textbf{Печеры, Пичеры, Печары} – по крайней мере в 17 веке так называли нынешнее сердце Печерска, то бишь Лавру и около.

Боплан писал:

\begin{quotation}На полмили ниже Киева лежит селение Печеры, с большим монастырем, в коем обыкновенно живет митрополит или патриарх. Близ монастыря под горою находятся пещеры [...]
\end{quotation}

Во время Боплана рядом с Лаврой действовал еще Николаевский монастырь:

\begin{quotation}
Между Киевом и Печерами, на горе, омываемой Днепром, в живописной местности лежит Николаевский монастырь, принадлежащий русским монахам. Монахи употребляют в пищу только рыбу; впрочем имеют право выходить из обители для прогулки и посещения знакомых.
\end{quotation}

Речь идет о Пустынно-Николаевском монастыре, что находился тогда на Аскольдовой могиле, а в 1696 году перенесен на место Военного Николаевского собора.\\

\medskip

\textbf{Печерск} – как бы другой Печерск, отличный от известного. В узком смысле, так жители улицы Бастионной и примыкающих Евгении Бош (Катерины Билокур), Подвысоцкого, Киквидзе называют окрестности, от Печерского моста и до ботсада. Печерский мост назван так от близлежащего Пещерного сада, принадлежавшего лавре – он был ближе к Суворовскому училищу.\\ 

\medskip

\textbf{Печерский ипподром}

50°26'10.64"N 30°32'56.37"E

Пространство его сохраняется по 2020 год между улицами Суворова, Лейпцигской и сев\-еро-западными валами Печерской крепости (Успенский и Петровский ее бастионы), примыкающими к Лаврской улице. Бывшее здание ипподрома, построенное в 1915–1916 годах, находится по адресу Суворова, 9.

Ипподром работал по 1950-е. Ныне под главным полем ипподрома – хранилище пресной воды Киевводоканала. На ипподроме, судя по всему, было давнее кладбище – сначала языческое, потом «великокняжеских времен». Тут найдены и каменные бабы, и могилы, как полагают, 11-13 веков.

Возможно, вымываемые в 1960х годах кости на склоне в тылу училища связи над Старонаводницкой улицей – свидетельство, что  кладбище продолжалось по это место. Или же отдельное кладбище.\\

\medskip

\textbf{Печерский мост} – находится над бульваром Дружбы Народов, с одной стороны сверху к нему подходят улицы Бастионная и Киквидзе, с другой – бульвар Леси Украинки.

В обозримом прошлом (более подробно читайте раздел про Автостраду) оврага под мостом не было, моста не было, был такой перевал холма, причем не в самой высокой его точке. По перевалу шла дорога, а по сороковые годы 20 века, время от времени даже травмайная линия. На одной стороне сего перевала в 19 веке лежало большое кладбище.

В начале 1940х годов от Демиевского путепровода стали прокладывать Автостраду, нынешний бульвар Дружбы, и пробурили перевал, сделав в нем овраг. Местность вообще сейчас резко отличается от былой, например около улицы Кургановской в самом деле был курган, какая-то горка, с вершиной на уровне высоты Печерского моста – сейчас, как мы знаем, улица Кургановская лежит ниже, а от былого там кургана остались странные бугры и валы. 

В 1942 году немцы и их союзные войска вроде венгров стали мастерить над прорытым оврагом путепровод и восстанавливать трамвайную колею. Его сделали вровень с берегами оврага, по сохранившимся снимкам можно сделать вывод, что эту конструкцию соорудили именно для трамвая, на одну колею, а пешеходы должны были идти по низу.

Когда на месте этого путепровода построили известный нам Печерский мост, я выяснить не мог. Старожили говорили, что его тоже строили немцы. По крайней мере в 1959 году Печерский мост уже был. В послевоенные годы возле него (или еще путепровода?) жил людоед, который убивал людей и делал из них колбасу.

После того, как построили музей ВОВ с Родиной Матерью и на праздники оттуда запускали салют, местные ходили к мосту смотреть этот салют, с него было удобнее всего глядеть.

Больше об окрестностях читайте в моей книге «Ересь о Киеве».\\

\textbf{Переварки} – одно из названий Приорки в 19 веке.\\

\medskip

\textbf{Пересечение} – перекрёсток проспекта Победы и Дегтярёвской.\\

\medskip

%\textbf{Печерский мост} – 

%\textbf{Печерский базар} – 

\textbf{Пещерное} – 19 век, урочище с садом на Зверинце, 12.5 десятин, принадлежало Лавре, давало доход 750 рублей. Находилось, насколько можно понять (см. выше), около Суворовского училища, хотя в Путеводителе Киева 1884 года про «пещерный сад» сказано, что «за еврейским кладбищем и лабораториею» – имеется в виду кладбище где сейчас в ботсаду участок вьющихся растений.\\

\medskip

\textbf{Пiд явором} – известная в 1980-х наливайка на Лукьяше, стояла там где сейчас Сильпо.\\


\newpage

\textbf{Пирово} – одно из старых названий Пирогово.\\

\medskip

\textbf{Пирогово} – село, бывшее владение Выдубицкого монастыря.\\


\medskip

\textbf{Плац парад} – на середину 19 века, площадь около Мариинского дворца, в парке Ватутина (Марьинском), той его части, где сейчас памятник Ватутину.\\


\medskip

\textbf{Плоское} – в середине 18 века слобода Кирилловского монастыря, позже городское селение. Улица Кирилловская, соединяющая Подол с Куренёвкой, именовалась прежде Плоской.\\


\medskip

\textbf{Победа} – основанный в конце 1950-х район (частый сектор, но есть и двухэтажки да несколько строений большей этажности), зажатый между хутором Базой, улицами Генерала Пухова и Генерала Авдеенко, а также промзоной с полиграфическим комбинатом «Пресса Украина».\\ 


\medskip

\textbf{Поганые лозы} – урочище на плане 1752 года, чуть выше Богословского монастыря (что непосредственно выше Иорданского, примыкая к нему), и примерно на уровне истока Глубочицы (урочище Хвощеватая долина по карте). Это приблизительно стык улиц Белорусской и Якира, где-то где исток Скомороха.
  
   Однако на той же карте еще выше видно «урочище Поганые», из чего можно предположить, что Погаными лозами слыла вся местность местность от парка Котляревского до Бабьего яра.

    Еще выше «урочище Поганые лозы» – урочище «Данилов крест» на «дороге Белогородской». 

А вдоль Поганых лоз, слева от них, видно «дорогу по светошислому бору». Сия дорога переходит, около истоков Сырца, в «дорогу берковецкую» (сопоставим ее с улицей Стеценко?)

\medskip

\textbf{Пожарище} – бытовавшее в послевоенное время название местности, где ныне стоит школа номер 88. Там была еще яма, как говорили, от снаряда, с озером, в ней купались. Возможно однако, что яма образовалась во время взрыва пороховых складов Зверинецкого форта в 1918.\\

\medskip

\textbf{Покал, Покол} 

50°22'55.3"N 30°33'24.4"E

Заросшее деревьями урочище напротив Лысой горы на Зверинце, к востоку от Столичного шоссе, между ним и Промышленной улицей. В его окрестностях, в лужах, я видел тритонов.

По сути, там расположен и теплый канал\footnote{Начало: 50°23'21.9"N 30°33'30.5"E} от Пятой ТЭЦ. В нем очень быстрое течение и каменистое дно. В юности, пацанами мы ходили с Бастионной улицы на него купаться – плавать там трудно, хотя и мелко. Вода была очень теплая, и в качестве забавы можно было держаться за тросы в начале канала. В конце канал имеет гидротехническое сооружение, разделяясь на пруды по разведению рыбы и на водопад в отводом уже в затоки Днепра.

На 2023 год северная часть Покала стремительно застраивается ЖК.\\

\medskip

\textbf{Полицейский садик, Полицейский сквер} – сквер на углу Федорова и Горького (Антоновича). По рассказам, место встреч киевских геев в 1970-х.

Название от Полицейской улицы (ныне Федорова). Прежде, в конце 19 века, на месте сквера была Конная площадь.\\

\medskip


\textbf{Польская слобода} – по южную часть Байковой улицы лежит старая часть кладбища, в частности там есть католические, польские участки. К ним с юго-запада примыкает частный сектор Забайковья, и некоторая его часть, например улица Волжская, слывет Польской слободой, потому что там жило и живет много поляков.\\

\medskip

\textbf{Полупьянова} – улица Полупанова на Приорке, ныне улица Приорская. Прямая, застроена была в пятидесятые трехэтажными домиками, которые по 2021 год сохранились на четной стороне. В девяностые тут была нездоровая обстановка.

В 1960-х, в конце улицы, были две военные части – понтонный полк и в/ч 63271, а от них до Днепра – луговина.\\

\medskip

\textbf{Пост-Волынский}

Окрестности станции Киев-Волынский, прежде именуемой Пост-Волынский. Промзона и небольшой жилой сектор по улице Новополевой, от двух этажей и выше – есть старенький домики, чем-то напоминающие застройку исконной Дарницы, есть советские  панельки. Эта жилая часть района очень зеленая.\\


\medskip


\textbf{Преворок} – одно из названий Приорки, по крайней мере во второй половине 19 века.\\

\medskip


\textbf{Преображение, хутор} 

50°22'09.8"N 30°31'24.7"E

Старое народное название лаврской Спасо-Преображенской пустыни, на горе в Мышеловке, между улицами Ягодной и Кащенко. 

   В бытность хутора Преображение, тут была церковь Преображения, построенная в 1873 году и разрушенная в 1938-9. Остатки пустыни, а не при ней кладбище\footnote{Рядом впрочем, через дорогу – улицу Ягодную находится Корчеватское кладбище.} уничтожены в начале 1980-х. Пустынь восстановлена с 2007 года.

Некоторое время после революции, по 1930-е годы, в Преображении размещался дом отдыха научных работников.

Екатерина Петровна Кудрявцева в воспоминаниях о своем отце, профессоре Киевской духовной академии Петре Павловиче Кудрявцеве пишет:

\begin{quotation}
А сколько радости, физического наслаждения и эстетического удовлетворения доставляли нам месяцы, проведенные в доме отдыха научных работников, который размещался тогда в живописнейшей местности в окрестностях Киева – на хуторе Преображение. Хутор этот принадлежал когда-то Киево-Печерской лавре, и еще в 27-28 годах там была и действующая церковь во имя праздника Преображения.

Вся эта местность по правому берегу Днепра, начиная от Киево-Печер­cкой Лавры, далее через Голосеево, Китаев, Преображение, Феофанию, являлась когда-то угодьями монастыря и содержалась в образцовом порядке. 

После революции в Преображении и был устроен дом отдыха научных работников. Рельеф местности там гористый, от самого Китаева идет непрерывный ряд озер, а внизу, под горой, устроен даже артезианский колодец, откуда вода подавалась наверх в дом отдыха. Столетние липы, дубы\footnote{Один такой сохранился: 50°22'11.6"N 30°31'15.8"E} буквально «не в обхват», грецкие орехи, такое изобилие зелени, самой пышной растительности делало этот уголок действительно настоящим домом «отдыха».

Отдыхали там члены секции – преподаватели и профессора вузов и просто научные сотрудники Украинской академии наук. Но в июне, когда еще не заканчивалась работа в вузах и отделах Академии наук, этот дом отдыха предоставлялся в распоряжение жен и детей, вообще членов семей научных работников. Этим правом пользовались и мы с сестрой, будучи студентками; и я до сих пор не могу без теплого чувства вспомнить время, проведенное нами в Преображении. Папа очень любил это место тоже. 

Он даже не роптал, когда порою начинались упорные дожди – в таких случаях он выходил со стулом на крылечко монастырского корпуса (все отдыхающие располагались в этих корпусах) и часами мог сидеть с книгой в руке, прислушиваясь к шуму дождя, вдыхая ароматный воздух, свежесть молодой листвы (в июне месяце). В хорошую же погоду он был
неутомимым организатором ближних и дальних прогулок по окрестностям, поскольку – скажу прямо – он был страстным любителем природы.
\end{quotation}

\newpage


\textbf{Преображения Господня церковь} деревянная

50°28'08.7"N 30°31'15.4"E

На 2021 год там пустырь за домом на Почайнинской, 18. Сгорела во время пожара на Подоле в 1811 году.\\ 

\medskip

\textbf{Прибрежная отрада} – см. Караваевщина.\\


\medskip


\textbf{Приорка}

Местность между Куренёвкой и Западинкой. Среди местных всё, что севернее железной дороги около Птички считается Приоркой. В начале 20 века местные называли свой район Преваркой. В 19 веке граница Приорки была к северу от ручья Куриный брод, а южнее считалась Куреневка – такое представление отличается от нынешнего. 

Петр Развидовский, приор (генеральный проповедник) в киевском доминиканском конвенте вел записки с 1634 по 1664 год, в которых он сообщает о Приорке, перечисляя владения доминикан:

\begin{quotation}
Грунты конвентские по фундации привиллегий начинаются от реки Сырца за Кирилловским монастырем к Приорке, вдруг за мостом перед большим крестом; в правой стороне была некогда деревушка Яцковка и с той деревушки давали мы подымное по квитанциям 1629 года. Я не застал уже оной, но на том же месте поставил дом постоялый, и был с того доход.
\end{quotation}

И далее:

\begin{quotation}
начиная об Берковца, обширные нивы конвентские, названные Уваров великий, Уваров малый, даже до самой Приорки местечка, которое я остновал; было хат двести, одного бору на две мили даже до Ирпеня в ширину и в длину.
\end{quotation}

Название местности постепенно, от исконной Приорки, превратилось в Преварку со сходности с занятиями соседней Куреневки – тут «курили», то есть работали винокурни, и варили пиво, потому и «преварка».\\

\medskip

\textbf{Притыка} – в краеведческой среде Притыка известна как некая давняя пристань в районе Подола. На плане Дебоскета 1753 года показана «река Притыка» севернее Подола в виде реки, что впадает в Днепр, а нее в свою очередь с запада впадает еще какая-то речка, а чуть южнее этого впадения расположена «Пильня» – лесопилка, приводимая в ход мельницей. Вдоль Подола же на плане Дебоскета никакой Почайны нет, равно как и самого этого названия выше по течению.\\ 

\medskip

\textbf{Пробитый вал} – невесть каких времен вал, который тянулся вдоль БЖ, мимо нынешней Львовской площади и по крайней мере до Лукьяновки. Примерно в месте, где ныне верховье Вознесенского спуска, в валу была пробоина, и оттуда по Валовой горе спускалась дорога к Щекавице и Подолу. Вал упоминается в земельных документах например 1522 года. Спустя почти сто лет, на рисунках ван Вестерфельда, мы видим тот же вал с пробоиной, значит его никто не чинил.\\ 

\medskip

\textbf{Провалье} 

50°26'50.6"N 30°32'37.9"E

В 19 веке, местность к северо-зап\-аду от Аскольдовой могилы, склоны Днепра до Городского сада. Окрестности Зеленого театра, более на северо-запад. Между Провальем и Дворцовым парком лежала улица Козловская, которая выходила к перекрестку Московской и Никольской. Козловская так называлась по фамилии землевладельцев, хотя жили там не только Козловские.

На начало 20 века Провалье адресно совместилось с Козловкой, Козловской улицей и считалось «от Никольских ворот до Царского сада», включая в себя, кроме прочего, Мариинский парк.\\

\medskip

\textbf{Провалье Ирининское} – непонятно, то ли Провалье, что «основное», или нет. Упомянуто в земельном универсале Мазепы от 15 июня 1693 года, что одно из землевладений Братского монастыря было «за провальем Ирининским».\\

\medskip

\textbf{Прозоровская башня} – в составе Новой Печерской крепости (правое крыло Васильковского укрепления, по Щорса, 34), Башня номер 3.

В ней была церковь (1840 года), где в 1863 году в нижнем этаже поместили останки фельдмаршала князя А. Прозоровского (1732-1809). В том же году некоторые помещения отвели под военный суд и содержания арестантов.

В 1897 году башню переделали в склад. После Великой Отечественной войны по 1993 в башне располагались воинская часть.\\

%Изначально башню, по приказу Николая I, хотели строить на Зверинце, но английские и французские инженеры воспротивились из-за подземных пустот в районе оного.\\

\medskip


\textbf{Прозоровская площадь} – находилась, как понимаю, там где теперь жилой квартальчик на юго-восток от Новогоспитальной улицы.\\

\medskip


\textbf{Проневщина} – местность между Верхним каскадом Совских прудов и Краснозвездным проспектом. Короче говоря, окрестности улицы Петра Радченко и оттуда к Севастопольской площади.

Еще в 19 веке тут был лес, принадлежащий Митрополичьему дому. В лесу находились курганы. В начале 20 века роща была вырублена, курганы в итоге разорены. В 1950-х Проневщина застроилась частными усадьбами.

На стыке 19-20 веков известен был также хутор Проневщина из 1 двора и 10 жителей. Где именно стояли его постройки, я не знаю.

По соседству с Проневщиной, через пруды – Совское кладбище, а по другую сторону Краснозвездного проспекта – Александровская слободка. 

Подробнее см. мою книгу «Речка Совка и ее притоки».\\

\medskip

\textbf{Прос\'еки} – в Святошино, в 20 веке, улицы на юг от проспекта Победы. Улица Петрицкого это первая прос\'ека, Крамского – вторая просека, Кричевского – третья просека, Кольцевая дорога – четвёртая просека. Живописная – пятая просека.\\

\medskip


\textbf{Протасов яр} – яр на берегу Лыбеди, известен в 19 веке как дармовое месторождение глины. Глину тут брали даже гончары из Никольской слободки. В самом яру кирпичные заводы и их глинища находились преимущественно на северном склоне, ближе к устью.

В Протасовом яру были найдены кости мамонта и каменные орудия труда. Вероятно, много памятников древности уничтожено этими самыми кирпичными заводами. Про свои раскопки 1876 года курганов Протасового яра и близлежащей Батыевой горы, где купно было около двух сотен курганов, писал археолог Я. Волошинский в работе 1876 года «Киевские курганы».

В 19 веке в яру была деревня Протасов яр, дома стояли от нынешней улицы Огородной и вверх по яру до ложбины, где делают надгробия. Докучаевский переулок относился к деревне же.

Яр делит собой две горы – с юга это Байкова гора (Клинический городок на ее склоне), а с севера – Батыева гора. По северному склону среди перестроенного частного сектора лежит улица Огородная, показанная еще на дореволюционных картах. 

На южном по верху склона идет улица Амосова. На том же склоне к улица Протасов яр (что на дне яра) сходит лыжная горка в виде буквы Л.

На весну 2019 года застроена восточная часть яра, то бишь его низовья. Остальное еще пребывает в относительно диком виде, при этом южная часть – просто склон, а вот северная сторона интереснее.

Она разделена как бы на две части, между ними приярок с фирмой по производству надгробий. Судя по старым картам, тут до революции был кирпичный завод. Еще один завод, Батухина, располагался ближе к железной дороге.

Над удольем, на юг есть огромный, покрытый кленами горб, чей северный склон сходит в яр с болотцем, где растет хвощ, а сам склон изрыт погребами, ведь неподалеку, над яром, стоят жилые дома Соломенки. Южная сторона горба нисходит в удолье, на дне коего остатки фруктового сада, еще в середине 20 века занимавшего и низину, и гору над нею. По горбу проложены тропы, там гуляют и отдыхают люди.

На 2000 год где-то на северном склоне яра, «в подножии Батыевой горы» существовала Г-образная пещера длиной 7 метров, высотой 1,2-1,3, шириной 60 сантиметров, в конце которой находилась комнатка с нишей. На стенах были граффити – кресты и сеть. Описана в книге Бобровского «Подземные сооружения Киева». Мне не удалось самостоятельно найти эту пещеру в 2019 году, а выяснить ее положение у одного из участников раскопок 2000 года тоже не получилось, вместо этого я наткнулся на какую-то иронию и браваду.

Также, в своих заметках, куда я выписывал и заносил разные любопытные вещи, я нашел невесть откуда взятое: «около Протасов яр, 30, на вершине горы 2 пещеры». Но указанный адрес это некое автохозяйство у подножия южного склона, около лыжного спуска. Если всё же адрес верен, имеет смысл исследовать склон до этого адреса от остановки «Тубинститут», что на улице Амосова.

Часть северного склона террасирована непонятно в какие времена, возможно кирпичными заводами. В середине 20 века тут были огороды. Сейчас всё заросло деревьями, большей частью кленами.

В северный склон врезается также приярком переулок Докучаевский, застроенный старенькими частными домиками. Над ним – покрывший часть горы гаражный кооператив (длится наверх до Огородной улицы) и улица Докучаевская.

Итак, по северному склону есть где полазать, а южный где сохранился в диком виде – относительно узкий, просто склон, а где не сохранился, там застроен Клиническим городком и прочими сооружениями.

Под Протасовым яром проходит коллектор длиной 1,2 километра с ручьем, устье коего в Лыбедь расположено южнее моста через улицу Ивана Федорова, позади автомойки и кафе на Федорова 31\footnote{50°25'37.8"N 30°30'34.0"E}. Среди диггеров ручей известен как Светлый песок. Начало коллектора – наверху, близ пересечения улицы Соломенской и Протасов яр\footnote{Примерно 50°25'14.0"N 30°29'27.1"E}, там где заезд к Клиническому городку.\\

\medskip

\textbf{Протасов хутор} – в 19 веке, хутор на том склоне Протасова яра, что продолжается Байковой горой.\\

\medskip

\textbf{Протва}, хутор – лежал к западу от хутора Лыбедь, примерно окрестности впадения речки Бусловки в Лыбедь, внизу Лысой горы, в сторону Стратегического шоссе.\\ 

\medskip

\textbf{Птичка} – Куренёвский птичий рынок, совмещенный с рынком хозяйственным. На выходных тут собирается толкучка, равно как на близлежащей Петровке.\\

\medskip

\textbf{Пунище} – давнее урочище на Оболони, в низовье речки Сырец, юго-восточнее Приорки.

Соседствовало с урочищем Круговина, что было где-то на излучине Сырца. Точное местоположение сказать не могу, но явно занято нынче промзоной, и не в устье Сырца. По карте 1752 года это выглядит так – двигаясь вдоль течения Сырца – мимо села Приорки, мы попадаем на пруд с мельницей Кирилловского монастыря, потом ставок Котляра Прокопа, затем по берегам Сырца идет урочище Речище, ниже его Пунище, а потом Круговина.\\

\medskip

\textbf{Пушка} – памятник погибшим в 1917 году рабочим-арсенальцам, напротив станции метро «Арсенальная». Пушка была поставлена в 1923 году на постамент от памятника Искре и Кочубею.\\

\medskip

\textbf{Пятачок} – удолье в пересечении улиц Бастионной и Киквидзе, около большого полукруглого дома с аркой посередине (Киквидзе 1/2)\footnote{Общага с высокими потолками,  широкими коридорами и дивной лестницей.} и базарчика, который из стихийного а затем навесного стал крытым в 2017 году.

Иногда название распространяется на весь квартал между улицами Киквидзе, Бастионной, Подвысоцкого, Билокур (Бош). Другое название удолья – Яма.

Когда я снял серию «Планеты Киева» про Пятачок, мне стали писать раздраженные люди, называвшиеся старожилами. Они утверждали, что местность никогда не называлась Пятачком, однако называется Ямой. 

Следовательно, некоторые старожилы считают удолье Ямой, ну и пусть считают. Для жителей другого конца Бастионной Ямой была та Яма, куда круто нисходит склон от Бастионной к домам 12 и 14. Ямой ее же называют старожилы с улицы Мичурина, поколениями живущие там с довоенных лет.\\

\medskip

\textbf{Пятничный Клов}

50°25'48"N 30°30'29"E

Урочище, место пятничного сбора диггеров. Находится в русле бетонного коллектора Лыбеди, чуть южнее старого портала речки Кловицы, известной ныне как Клов. Стена Пятничного Клова изрисована диггерскими граффити.
