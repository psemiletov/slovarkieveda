\chapter*{И}
\addcontentsline{toc}{chapter}{И}


\textbf{Ивницкий гостинец} – давняя дорога, она же Старое путище, дорога Киевская, Смоляная, Смолянская. Лежала из Киева через Лыбедь, Борщаговку, Белгородку в Ивницу и на Волынь (цитирую Руликовского «Повет Васильковский»).\\ 

\textbf{Иорданские рогатки} – бытовавшее в 19 веке название места вокруг нынешнего перекрестка улиц Нижнеюрковской и Нижнеюрковского переулка.\\

\textbf{Исаев хутор} – на стыке 18-19 веков, хутор в удольи близ южного склона Девич-горы (Лысой).\\

\textbf{Итальянский домик} – одно из названий облезлого, розово-бежевого жилого дома работников обувной фабрики № 4. После войны его реконструировали пленные итальянцы. Находится на Приорке, по адресу Сокальская улица, 1. Построен по проекту архитектора Каракиса в 1940–1941 годах. 

После войны там жили «водники» (детский сад водников стоит рядом, занимая двухэтажное старое здание), затем дали квартиры еще и пожарникам, после чего дом стали называть Пожаркой.

Лестницы в доме расположены по бокам, от лестниц отходят внешние, открытые галереи, с которых и можно попасть в квартиры. Дом четырехэтажный, с колоннами, которые как бы защищают галереи. В доме 50 квартир, на 1 и 2 комнаты, потолки высокие.

Рядом, по той же стороне улицы, двухэтажные дома, с виду тех же годов.

Большую часть сведений о доме я получил от Даши Кононюк, пытаясь поначалу уточнить – верно ли, что местные именуют этот домик Китайкой. Оказалось, что нет.\\

\textbf{Ицун, Вицун} – в 17 веке рукав Днепра, в начале 20 уже озеро. Один из потоков, питавших верховье Почайны.
