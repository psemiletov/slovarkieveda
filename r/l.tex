\chapter*{Л}
\addcontentsline{toc}{chapter}{Л}

\textbf{Лазаревщина} – название Феофании в 17 веке.\\

\medskip


\textbf{Лесопилка} – район между улицей Святошинской и Первой просекой (Петрицкого). Там была лесопилка и посёлок ее рабочих.\\

\medskip


\textbf{Ленинского комсомола}, зеленый кинотеатр на 2700 мест. Находится в парке Нивки (угол Балаклеевской и Безручко) в закрытом состоянии. Был построен в 1958 году по проекту архитектора Чуприны.\\


\medskip


\textbf{Леси} – бульвар Леси Украинки, с обсаженной тополями парковой полосой посередке. По\-днимается на гору Печерска. В советское время на бульваре располагалось много культовых среди народа магазинов, на первом-втором этажах домов.

Пройдем снизу вверх. Внизу у перекрестка кстати, где сейчас небоскреб торчит, был пустырь, а ближе к Бассейной – трамвайная остановка. Трамвай шел по Мечникова и потом на Саксаганского, а одно время и по самой Леси почти от Печерского моста. 

Первым снизу магазином был «Дом радио» (дом №3), где помню продавались телевизоры, разные радиолампы, детали, и уцененные пятидюймовые дискеты. В следующем выше доме был «Подарочный» (№5), разместившийся на двух этажах. Там торговали сувенирами, кошельками, косметикой, игрушками, пластинками и кажется какой-то одеждой. В девяностые в отделе где пластинки или рядом с ним были игровые приставки и картриджи. От прежнего Подарочного на 2017 год осталась лишь маленькая часть, раньше он был во всю протяженность дома. 

Рядом с ним – советская 15-этажка, 1976 года постройки, с библиотекой имени Салтыкова-Щедрина. Прежде она была единственной высоткой, что возвышалась с этой стороны над зеленым оврагом Клова, нависая белыми своими панелями над темной листвой могучих кленов.

Дальше и выше, да по другой стороне, в доме номер 20, был магазин «Военная книга». А на самом верху бульвара – в доме 24 располагался гастроном (от него осталась мозаика у входа), «Юный техник» (ближе к углу), и «Каштан» (на другом торце угла) – тоже мозаика сохранилась. Напротив «Юного техника», в доме 17/19 был магазин техники «Орбита», с телевизорами и магнитофонами. В ЮТ продавались разные куски фанеры, проволоки, авиационная резина – словом, весь тот хлам производства, который юные техники могли обращать в какие-нибудь планеры.\\

\medskip

\textbf{Лески, слободка}

50°25'59.6"N 30°31'28.6"E

На 1830-е – склон Черепановой горы, точно между стадионом Олимпийским и Косым Капониром. В то время это был склон горы с приярками, потом его съели и сравняли.\\


\medskip


\textbf{Ливерпуль} – стиляжное название диетического гастронома на Крещатике.\\


\medskip


\textbf{Липинка}

50°29'45"N 30°26'8"E

Гора на Приорке, между улицами Данченко и Замковецкой\footnote{На дореволюционных картах – Белическая улица. Современная же Белицкая улица это дореволюционная Подгорная.}. На 2017 год началась застройка высотками, а раньше тут были сады. Востоком граничит с урочищем Замковище, а югом с частным сектором Беличьего поля. Севером примыкает к Сукачеву яру, занятому гаражными кооперативом «Барвинок» – пойме одного из истоков речки Курячий Брод.\\

\medskip


\textbf{Липлиновка} – жилой район на Куреневке, у низовий Подольского спуска, называется от психиатрической клиники Константина Михайловича Леплинского, имевшей адрес Кирилловская, 99. Сам проживал, с 1906, по адресу Маловладимирская 31.

На карте 1914 года «частная клиника Леплинского» обозначена где-то под горой\footnote{50.48099398597106, 30.476537726811788}, около тогдашней Кирилловской площади, где ныне Кирилловская 99 и около.

Леплинский (1.61857-16.02.1919) работал с 1893 в Кирилловской больнице, будущей Павловке, а с 1895 принимал в Александровской.
 
Сын православного священника, закончил 4 класса Могилевской Духовной Семинарии, а 8.03.1883 – Императорский университет Св.Владимира в Киеве. На 1905 – приват-доцент Университета Св.Владимира, коллежский советник. 

В 1910-х – председатель Киевского общества правильной охоты и Клуба охотников. Стало быть, никакого у меня к нему уважения, одно презрение.

В 1914 год был назначен старостой церкви с. Перегоновка Васильковского уезда, в 1915 стал почетным мировым судьей, губернским гласным от Васильковского уезда.

В справочнике «Весь Киев» за 1915 год есть «Лечебница для нервно и душевнобольных К. М. Леплинскаго», Кирилловская 99, телефон 21-68\footnote{А относительно Сикорского за то же время: «Санатория для нервно и душевнобольных врачей Н.Б. Горбунова и А.Г. Сикорскаго, Лукьяновка, Овручская 25, тел 25 86».}.\\

\medskip

\textbf{Липки} – ныне под Липками смутно подразумевает правительственный квартал и окрестности. Изначально же Липками называли возвышенность над Кловским оврагом, подле Кловского дворца. В 1744 году Лавра завершила его постройку, а с северной стороны дворца высадили липовую рощу.

Закревский пишет, что рощу безжалостно истребили в 1833 году, когда рощу пустили под застройку деревянными домиками, и «столетние липы, украшавшие Киев, повалились под ударами топора».\\

\medskip


\textbf{Липовой куст} – так в середине 18 века именовали некоторую местность к западу от Верхнего каскада Совских прудов.\\

\medskip

\textbf{Лос Соломас} – первоначально, в конце нулевых, надпись оставляли на стенах граффитчики Соломенки. Теперь иногда так называют этот район.\\

\medskip

\textbf{Лыбедская площадь} – в 1880-х годах так называли нынешнюю площадь у Республиканского стадиона (НСК Олимпийский). Позже – Троицкая площадь.\\

\medskip


\textbf{Лыбедь} – наряду с Сырцом и Борщовкой, одна из крупнейших киевских малых рек, неоднократно упоминается в летописи.

В 968 году:

\begin{quotation}
\noindent И тако отступиша Печенезе за Лыбедь но не бяше лзе напоити коня на Лыбеди пред Печенеги.
\end{quotation}

980:

\begin{quotation}
\noindent Бе же Володимир побежен похотью женьскою, быша ему водимыя: Рогнедь, юже посади на Лыбеди, идеже есть ныне селце Передславино
\end{quotation}

1136:

\begin{quotation}
\noindent Паки же Олговичи с Половци переидоша Днепр, декабря в 29 день, и почаша воевати от Трьполя, около Красна и Васильева и до Белогорода, оли же до Киева и по Желань и до Вышгорода и до Дерев и чрес Лыбедь стреляхуся.
\end{quotation}

1146:

\begin{quotation}
\noindent Стоящим же еще полком межи собою, и видив Игорь вси его вои, оже Кияне пославшеся и пояша у Изяслава тысячкого и с стягом и приведоша и к собе, и потом переехавше Берендичи черес Лыбедь и взяша Игореви товары перед Золотыми вороты [...]

И приехав Улеб в свой полок, такоже Иван, и поверга стягы и поскочи к Жидовським воротом. Видив же то Игорь и Святослав и сыновец его Всеволодович, не сметошася, но поидоша противу Изяслава\footnote{А тот был вот где, ранее: «И прииде Изяслав ко валови, идеже есть Надово озеро у Шелвова борку, и ту ста полкы».}, и нелзе бы им доехати Надовым озером, и поидоша на верх озера, и ту быша им пророви от озера, а другии ис Сухой Лыбеди, и ту ся стесни полци.
\end{quotation} 

1151:

\begin{quotation}
\noindent Утрии же день исполцився приде Дюргий\footnote{Гюрги, Юрий Долгорукий.} к Кыеву, и ту сташа полкы по оной стороне Лыбеди, и начаша битися о Лыбедь. Андрей же Гюргевич и Володимир Андреевич с Половци налегоша силою, и тако переехаша Лыбедь; они переехаша, кге же есть Сухая Лыбедь. [...]
И възвратися Андрей опять неврежен [...] И стреляющимся им до вечера о Лыбедь, а инии переехаша и на болоньи\footnote{Прибрежная луговина.} бьяхуся против Вячеславлю полку и Изяславлю, а друзии противу Лядских воротом, на песцех, бьяхутся. Изяслав же, то видев, по всей своей братье и повеле нарядити дружину ис полков [...] и потекоша нань вси, и Чернии Клобуци, отвсюда, и тако вбодоша е в Лыбедь весде, инии же и брода грешиша, и тако избиша е, а другыи изломаша е [...] оттоле же ни один человек не перееха боле того на сю сторону.
\end{quotation}

1177:

\begin{quotation}
\noindent и побеже Святослав через Днепр, устья Лыбеди, и потопоша людье мнози.
\end{quotation}

Со времен летописных прошло много времени, и сейчас трудно сказать, что же именно подразумевали в те годы стародавние под Лыбедью. 

Поясню сию туманную мысль. По 20 век включительно, жители окрестностей улицы Володарского (Златоустовской), вдоль которой протекал ручей, известный на картах 19 века как Скоморох (считается притоком Лыбеди), полагали, что это сама Лыбедь. Если такое представление было справедливо в летописные времена, то «Лыбедь» летописная и относящиеся к ней события можно толковать, учитывая и Скоморох.

Попробую обрисовать «традиционную» Лыбедь в текущем состоянии, с отсылками в прошлое, ибо, например, устье Лыбеди в 19 веке было не там, где теперь, равно как постройка железной дороги внесла сумятицу в течение реки, особенно ее истоков.

У Лыбеди несколько истоков.

Исток А. Начинается на Отрадном, в районе парка «Орленок» (бывший фруктовый сад Грушки, в 2009 году яблони выкорчевали), куда со стороны школы номер 46 (Василенко 10) заходит коллектор, проходит под фонтаном\footnote{50°26'36"N 30°25'23"E} у ДК  «Меридиан» и следует на северо-восток к довольно большому пруду\footnote{50°26'41"N 30°25'37"E}. Этот пруд существует с середины 19 века.  

Затем коллектор покидает территорию парка, и от детского сада №225 (Гарматная 30А) идет строго прямо на юго-восток до точки

50°26'24.4"N 30°26'19.2"E

%около юго-западного угла школьного стадиона, между ним и небольшой военной частью, известной 1980-е как Военнаходка, ибо люди лазали туда, на склады, через забор за амуницией.

Там железобетонный коллектор высотой полтора метра, шириной 3, соединяется с коллектором другого истока, истока Б. 

Итак, до этого места, коллектор истока А от парка Орленок проходит через Поликлинику № 1 Соломенского района (Гарматная, 36), пересекает Гарматную, дальнейший ход – жилой дом Гарматная 39Б, детсад на Гарматной 41, дом на Западной 14, пересекает Западную, ныряет под дом с почтой на Борщаговской 210, пересекает Борщаговскую, уходит под дом Борщаговская 189, детскую площадку за ним, под большой дом на Леваневского (Татьянинская) 8, высотку на Леваневского 9 и выруливает к углу школьного стадиона. 

Исток Б прослеживается тоже на Отрадном, но южнее, от парка Отрадного. Там есть длинное озеро Отрадное, оно же Интеграл

50°26'1"N 30°25'32"E

из-за своей формы – по виду русло реки, около него стоит памятный современный камень, что здесь находится исток Лыбеди. Из северной части этого озера или пруда, вода коллектором отводится в лежащее к северу, в «Мамаевой слободе», озеро Красавицу. От его северного берега водный поток коллектором уходит под гаражный кооператив «Корт», где к нему присоединяется

50.4372412, 30.4289639

с запада, еще один коллектор, что начинается на перекрестке улиц Светлогорской и Радищева

50°26'47"N 30°24'37"E 

затем следует по Радищева на юг до пересечения с бульваром Лепсе и течет под/вдоль этого бульвара дугой к перекрестку с проспектом космонавта Комарова

50.4360992, 30.4113042

...по нему на восток примерно до 28 дома, но на другой стороне, и оттуда наискосок сворачивает к гаражному кооперативу Корт, а здесь вы уже знаете, сливается с водотоком от Отрадного озера.

Далее общий уже коллектор под корпусами НАУ сворачивает на северо-восток. Пересекает Гарматную возле корпусов НАУ 4 и 5, проходит мимо общаги НАУ (Гарматная 53), затем под детсадом на Нежинской 26, еще одной общагой на Нежинской 14, пересекает Нежинскую и мимо юга дома Нежинская 7, между ним и школьным стадионом, и доходит наконец до слияния с коллектором истока А, к югу от юго-восточного угла школьной футбольной площадки

50.4401633N 30.4385528E

Примерно там же, но чуть далее, ближе к перекрестка Яблонской с Лебедева-Кумача (Голего) к двум истокам подходил еще один овраг с ручьем, начинавшийся на нынешней Чоколовке в ее частном секторе, эдак в окрестностях перекрестка улиц Волынской и Донецкой.

Этот овраг купно с водотоком, назовем его Чоколовский исток, был перебит пополам железной дорогой с ее высокой насыпью. По другую сторону жд этот овраг четко продолжается улицей Татьяны Яблонской. А впадал в Лыбедь (ныне сводный поток истоков АБ) примерно тут:

50.44045051763379, 30.44301236032757

Итак, и до устроения коллекторов здесь в окрестностях было слияние самых «высших» истоков Лыбеди, но сейчас южный Чоколовский приток идет другим путем, уже не по оврагу улицы Яблонской.

%Сюда стекал по открытому руслу и исток-приток от нынешнего парка «Орленок». 

От сего места слияния трех истоков (А, Б, Чоколовский), Лыбедь протекала б\'ольшим руслом по нынешней улице Нижнеключевой.

Но когда возникла железная дорога, отрезок оврага притока Чоколовского по улице Яблонской отрезали.

Итак, Чоколовский исток или приток в первичном виде не существует, но вода его должна была куда-то деваться, распределяться? По всем соображениям этот исток теперь течет по обе стороны железной дороги, в качестве ручья Вершинки и его притока.

Название Вершинка вроде бы пошло от подписи на плане Ушакова, однако на имеющемся у меня частично смазанном скане я этого разобрать не могу. Кроме того, план Ушакова накладывается на местность лишь относительно, это словно карта, которой сопровождают книги в жанре фэнтэзи. Из плана Ушакова можно заключить что такая-то церковь стоит рядом с такой-то, ворота сякие-то через несколько домов от церкви, но точные расстояния между ними не соблюдены. Кроме того, слово «вершинка» не название, оно означает, буквально, «один из истоков» либо «приток». Но раз уж называют сей исток Вершинкой, не буду противоречить устоявшемуся.

Коллектор Вершинки начинается около пересечения улицы Мишина и Волынской:

50.42651882128974, 30.448328137916903

Идет под Новогород-Северской, под школьным стадионом и выходит на поверхность, в бетонном русле, где течет по зеленой зоне за домами начиная от Ушинского 26 в направлении Радиорынка, подле коего снова уходит под землю.

Параллельно Вершинке с другой стороны железной дороги течет, сначала на поверхности, другой ручей. На OSM он обозначен как Волочаевский ручей, но тот совсем другой водоток. Коллектор безымянного ручья начинает прослеживается за гаражами в конце улицы Суздальской, тут: 

50.42095928289456, 30.43738236756805

И течет между гаражами и железной дорогой.

По идее он должен был стекать к Лыбеди в овраге Чоколовского истока, но ему не дают, овраг перерублен и устроена насыпь, и ручей 
пускают дальше вдоль железной дороги. Он снова будет то прятаться под землю, то выходить на поверхность, уже за Караваевским путепроводом, вдоль сначала Полевой улицы, а потом Железнодорожной.

Но вернемся к соединенным истокам АБ.

После смычки они текут в коллекторе (а еще в первой половине 20 века на поверхности) далее на восток, за домами по четной стороне улицы Лебедева-Кумача (Голего), под спортивными площадками. Раньше тут всюду был частный сектор и сады. По пути коллектор пересекает улицу Генерала Тупикова, а ближе к перекрестку Лебедева-Кумача с Индустриальной подтягивается к трассе первого, и за перекрестком следует на восток вдоль четной стороны улицы Нижнеключевой, затем вдоль Полевой и пересекает частный сектор улицы Железнодорожной подле россыпи дробных домов под общим номером 16. 

Около строения на Полевой, 27А к истоку АБ с северо-запада присоединяется еще один коллектор, Волочаевский

50.44172115373779, 30.449964975858588  

Почему именно он именуется Волочаевским? Потому что на определенном, довольно небольшом отрезке он протекает вдоль бывшей улицы Волочаевской. Улица Волочаевская лежала на отрезке между 

50.447063309091604, 30.43293243270232 

50.44578376974889, 30.442196219820232

Сейчас на этом отрезке, но чуть севернее, лежит часть Машиностроительной. Западный конец улицы Волочаевской можно считать около дворца культуры  «Точэлектроприбор» на Гарматной 26/2, ныне это ДК «Росток». Череда именований улицы Волочаевской такова – 6-я Дачная, Андреевская (по 1940-е включительно), Волочаевская, затем квартал в 1970-80х застраивается высотными жилыми домами и исчезает как дорога.

Я не знаю, кто дал протекающему там поныне в подземном коллекторе ручью название Волочаевского, но это явно произошло, допустим, с 1950-х годов по время исчезновения улицы. 

%Это давний ручей, протекавший сюда в овраге, начинавшемся в промзоне окрестностей стыка переулка Радищева и улицы Радищева. 

Это давний ручей, протекавший сюда в овраге, начинавшемся в промзоне окрестностей стыка переулка Василенко и улицы Машиностроительной:

50.4502495, 30.4184389

Оттуда приток идет между Выборгской и Машиностроительной (эти улицы параллельны), потом у перекрестка Машиностроительной и Деснянской пересекает перекресток и движется мимо южной стороны небольшого дома на Индустриальной 26А, затем пересекает Индустриальную, ТЦ Аркадию, улицу Борщаговскую напротив (севернее) дома номер 143, под домом 143А, мимо дубравы, пересекает Дашавскую, проходит под южным углом сурового одноэтажного строения рядом (к востоку) с домом номер Дашавская, 25, и далее на юго-восток возле институтского спортивного здания на Полевой 24, пересекает Верхнеключевую\footnote{Возникла вроде бы в начале 20 века как Богатырская, затем распалась на две улицы – Ключевую или Новоключевую, и Ладожскую. В 1958 их объединили в Верхнеключевую. В 1980-м частный сектор на ней снесли.} и соединяется с Истоком АБ.

В районе стыка углов частного сектора и делового парка «Ладога», на восток от Полевой 14, за обочиной улицы Железнодорожной, коллектор АБ прямоугольным порталом выходит на поверхность:

50°26'36.6"N 30°27'16.5"E

...течет в бетонном коридоре и очень скоро сходится:

50°26'36.7"N 30°27'17.3"E

...с еще двумя другими истоками Лыбеди, Вершинкой и параллельным ей ручьем, образуя Т-образное соединение. Эти два истока, как сказано выше, части былого единого истока, который я именую Чоколовским. Он сущестовал цельно до устройства железной дороги.

Напротив этого слияния, к югу, лежат железнодорожные пути оборотного парка, гаражный кооператив «Первоймайский» и Кадетская роща.

Приключения Вершинки и того ручья на пути сюда \textbf{от моста путепровода Кардач} таковы.

Ручей выходит из коллектора на задворках пустыря на повороте Железнодорожной улицы, южнее дома 48А. Там пустырь и народный проход к железной дороге, тропа. По северную сторону тропы из бетонной трубы вытекает слабым ручьем и течет в земляной канаве, по овражку между гаражами (вдоль Железнодорожной) и насыпью железной дороги. На каком-то этапе он начинает течь уже у обочины улицы, между нею и гаражами. Вдоль улицы растут, кстати, несколько огромных высоченных дубов. Затем ручей странным образом, двумя потоками истекает из бетонноплитной стены упомянутого Т-образного соединения, у самого его перекрестка.

А Вершинка?

Южная или восточная часть обрубленного Чоколовского истока. Течет параллельно описанному выше ручью, но по другую сторону от железной дороги.. %Вырулив под землей к почти железной дороге, В-2 (Вершинка) следут вдоль жд на северо-восток, в зеленой зоне параллельно улице Ушинского, затем на задворках радиорынка Кардачи (между ним и железной дорогой). 

С заднего двора Радиорынка коллектор Вершинки проходит к улице Индустриальной между жд станцией Кардачи и АТБ (39 дом), пересекает Индустриальную и выходит на поверхность у подножия Кадетской рощи, держась между нею и железной дорогой.

Огибая гору с рощей, Вершинка протекает в прямоугольного сечения бетонном желобе на северо-восток, затем сворачивает на север, ныряет в коллектор под железную дорогу, и выходит из его арки близ упомянутой выше точки схождения всех трех истоков. От арки до пересечения в Т-образном соединении исток течет по бетонному желобу, по берегам коего растет хмель и прочая зелень.

%Что это за русла? Это и есть некогда искусственно разделенный железной дорогой исток В, Чоколовский. Был один, стало два – Вершинка да Волочаевский. Я уже писал, как он проходил раньше, теперь расскажу о том, как его пустили разделенным, по рукотворным руслам-каналам. 

%Сначала про часть истока В, что течет по западную, а затем северную сторону железной дороги. Далее буду именовать его В-1. На картах его именуют Волочаевским. Русло В-1 идет между железной дорогой и гаражами по Новополевой, затем Полевой. 

%Целое озеро возникло некогда при строительстве ЖК «Family and Friends» южнее пересечения улицы Волошина и Полевой.

%Где именно северо-западная часть русла В-1 уходит под землю в приближении к Индустриальной улице и станции Кардачи, я не проследил, но на поверхность она выходит уже на задворках пустыря на повороте Железнодорожной улицы, южнее дома 48А. Там пустырь и народный проход к железной дороге, тропа. По северную сторону тропы из бетонной трубы вытекает слабый ручей и течет в земляной канаве, по овражку между гаражами (вдоль Железнодорожной) и насыпью железной дороги. На каком-то этапе этот ручей начинает течь уже у обочины улицы, между нею и гаражами. Вдоль улицы растут, кстати, несколько огромных высоченных дубов.

%Затем этот Волочаевский ручей странным образом, двумя потоками истекает из бетонноплитной стены упомянутого Т-образного соединения, у самого его перекрестка.

%Южная или восточная часть истока В, В-2, он же Вершинка. Взятое в бетон, узкое русло с ручейком по его дну лежит южнее, вдоль холма Кадетской рощи. Диггеры считают его притоком Лыбеди и отдельной речкой, Вершинкой (а ее часть под железнодорожными путями называют Белой), хотя сопоставление карт времен до железной дороги показывает, что никакая речка вдоль будущего хода железной дороги от, скажем так, Пост-Волынского и до станции Кардач не протекала.

%Но стала протекать после устроения железной дороги. На оном русле, в 19 веке, при кадетском корпусе, был большой Кадетский пруд – там\footnote{50°26'46.1"N 30°28'24.6"E} сейчас железнодорожные пути оборотного парка, примерно на юг от домов 10 и 8 по Борщаговской. К стыку 19-20 веков немного выше по течению был устроен еще один пруд, общества Рыбоводства, при лежащих там огородах. Но всё это в пределах подножия Кадетской рощи. 

%Коллектор истока В-2, в отличие от исторического истока В, кажется начинается напротив Авиаремотного завода, за забором промзоны на Воздухофлотском, 92Б, где есть обросший ивами пожарный пруд\footnote{50°24'46.3"N 30°26'02.7"E}. Это около Пост-Волынского.

%Вырулив под землей к почти железной дороге, В-2 (Вершинка) следут вдоль жд на северо-восток, в зеленой зоне параллельно улице Ушинского, затем на задворках радиорынка Кардачи (между ним и железной дорогой). 

%С заднего двора Радиорынка коллектор Вершинки проходит к улице Индустриальной между жд станцией Кардачи и АТБ (39 дом), пересекает Индустриальную и выходит на поверхность у подножия Кадетской рощи. Тут он, огибая гору с рощей, скудным ручьем протекает в прямоугольного сечения бетонном желобе на северо-восток, затем сворачивает на север, ныряет в коллектор под железную дорогу, и выходит из его арки близ упомянутой выше точки схождения всех трех истоков. От арки до пересечения в Т-образном соединении исток течет по бетонному желобу, по берегам коего растет хмель и прочая зелень.

50.44357432892149, 30.454810385271458

Вообще Т-образное соединение, если бы не частный сектор, промзона и железная дорога, очень зелено, есть высокие ивы, да и вся округа в ивах, особенно окрестности улицы Полевой. Проще всего к соединению добраться от Борщаговской по улице Мамина-Сибиряка.

Итак, все истоки сходятся там, между восточным углом частного сектора Кардач и западной стороной лежащего в явном овраге бывшего русла Лыбеди, оборотного парка железной дороги. Отсюда Лыбедь в глубоком бетонном коридоре, изрисованном диковинными граффити, течет вдоль северной стороны насыпи  железной дороги на восток.

На многих картах стыка 19-20 веков у места этого нынешнего Т-образного соединения видно следующее. Соединения нет. Вместо этого, до сего места идет русло Лыбеди по левую (северную) сторону от железной дороги, параллельно нынешней Нижнеключевой улице – значит, речь идет о соединенных истоках от Отрадного, АБ и Чоколовского.

В месте же Т-образного соединения русло это раздваивалось и одна его часть продолжала течь по северную сторону, по руслу современного коллектора, а другая часть пересекала железную дорогу и, выйдя с южной стороны, протекала уже там между рельсами и холмом Кадетской рощи. На этой-то южной ветке, там где сейчас оборотный парк железной дороги, лежали два пруда, с огородами у первого, рыболовного. И немного не доходя до Кадетского шоссе, а ныне Воздухофлотского проспекта, южная ветка снова ныряла под железную дорогу присоединяясь\footnote{50°26'47.1"N 30°28'49.4"E} к северной около керосиновых складов (в 1930-х годах – нефтяных), в современных ориентирах на юго-запад от перекрестка Борщаговской и Воздухофлотского, там где гаражный кооператив «Лыбедь». Такое положение, хотя уже без прудов, видно еще на карте РККА 1930-х.

А вот до постройки железной дороги, Лыбедь, судя по картам, на этом отрезке текла именно к югу от будущих рельсов, образовывала пруд

50.44601324281866, 30.474938293269346

и потом речка вытекала из этого пруда и бежала дальше, южнее и почти параллельно сначала Борщаговской, а потом Жилянской улице.

Вернемся в настоящее время.

Через 350 метров от Т-образного соединения к Лыбеди с севера присоединяется еще один приток, Шулявица (Источная) О ней читайте отдельную статью.

Лыбедь и далее сохраняет прежнее направление. Вскоре к ней с юга, миновав железнодорожные пути, присоединяется

50°26'43.5"N 30°27'50.4"E

коллектор с Первомайского парка\footnote{Ограничен Уманской, Курской, Ереванской, Козицкого.} (некоторые краеведы считают его «Спутником», хотя последний лежит севернее), туда проще добраться, двигаясь к железной дороге на север от перекрестка Уманской и Козицкого. Это коллектор ручья, который диггеры именуют Кадетской Рощей или Железнодорожным (потому что над нижней его частью шуруют поезда). Он начинается\footnote{50°25'35.8"N 30°27'34.6"E} далеко отсюда, у Севастопольской площади, между домами Чоколовский бульвар 6 и Воздухофлотский проспект 46А/2. Оттуда коллектор идет на северо-запад, практически по руслу заметного в прошлые века длинного оврага, за домами четной стороны Чоколовского бульвара, под Площадью Космонавтов сворачивает на северо-восток и идет по оврагу вдоль четной стороны уже улицы Ереванской, переходя на нечетную после перекрестка оной улицы с Питерской и Искровской. 

Затем коллектор проходит под углом Ереванской, 15 и начинает идти на восток между домами 17, 21 и 19, 23. После, проходит под домами Ереванская 25, 27, 28, 29, пересекает улицу Козицкого, заруливает в Первомайский парк и поворачивает там на север, проходит под парком к железной дороге, под путями и выбирается к Лыбеди.

Оттуда Лыбедь продолжает в открытом кол\-лекторе-желобе движение на восток, между улицей Борщаговской и вокзальными путями.

На долготе дома по Борщаговской, 8, с севера к коллектору присоединяется коллектор ручья Песчаного, это левый приток Лыбеди:

50.4473613, 30.4742128

Лыбедь следует дальше вдоль переплетений рельсов, затем на время поворачивает к северо-востоку и близ развязки Воздухофлотского проспекта с Борщаговской уходит под землю через портал\footnote{50°26'48"N 30°28'50"E}.

Миновав проспект, Лыбедь снова отклоняется южнее, к вокзалу, и, пройдя под улицей Жилянской, следует под гаражным кооперативом и СТО, после чего появляется на поверхности – на этом отрезке над речкой есть железный мост\footnote{50°26'43.7"N 30°29'05.0"E}, примерно на долготе проспект Победы 7Б.

Затем Лыбедь в открытом коллекторе следует на юго-восток, между железнодорожными путями и промзоной. Около госпредприятия «Укррезерв», по сути на задворках северо-западнее разворотного кольца трамвая, остановки «Старовокзальная», Лыбедь принимает

50°26′40″N 30°29′9″E

в себя еще один левый приток, Скоморох.

Лыбедь течет далее на юго-восток, с юга поджимаемая вокзальными путями и сооружениями. 

У разворотного кольца трамвая на Старовокзальной улице, а точнее у сооружения перехода на вокзал, речка временно прячется под землю. У нечетной стороны улицы, перед наземным переходом на вокзал, со стороны упомянутой улицы к Лыбеди присоединяется

50.4432893, 30.4885599

очередной упрятанный под землю приток, около завода «Ленинская Кузня». Это ручей (диггеры называют его Афанасьевским), который течет сюда из Святославова яра.

За зданием наземного вокзального перехода Лыбедь снова выходит на поверхность и течет на юго-восток, с левого берега ее завод Ленинская кузня, с правого пустырь и недострой, за которым северные платформы пригородных поездов и станция метро «Вокзальная». По ходу над речкой расположены три моста.

Лыбедь отклоняется от вокзала на восток, к улице Коминтерна, и проходит под построенным в 1940 году мостом. Река, как и прежде, закована тут в бетонный желоб, сам лежащий в большем желобе из квадратных бетонных же плит. Во время паводка вода поднимается во «внешний» желоб. В нем между плитами растет трава и кустики.

За большим мостом – сразу маленький, расположенный ниже. Он на улице Вокзальной.

Лыбедь протекает далее между промзоной по левому берегу (Укрэнерго и  станция №1 теплопоставки) и рынком Привокзальным по правому. Этот рынок я помню по девяностым, мы туда ездили с мамой пару раз, там стояли коммерческие киоски, и цены были дешевле чем в городе. В киосках кроме прочего продавались кассеты, чипсы, жвачки, шоколадки. 

Лыбедь приближается к улице Льва Толстого и здесь, по левому берегу Лыбеди, к ней юго-восточнее станции теплоснабжения, около главного корпуса – гудящей котельной КиевГРЭС, построенной в 1927-1929 годах – а еще вернее чуть выше по течению у моста через Лыбедь, присоединяется приток

50.4394723, 30.4952091

текущий сюда под землей из ботсада имени Фомина вдоль улицы Льва Толстого. Диггеры называют его Паньковский по местности.

Вскоре речка ныряет под мост улицы Льва Толстого и течет на задворках улицы Семьи Праховых, между нею и улицей Лыбедской, что лежит параллельно железной дороге. Местность выглядит как унылая промзона, «освеженная» какими-то офисами и автомойками. Над коллектором кое-где нависают ивы.

На юг от большого здания по Льва Толстого 57 (если стоять к нему лицом, то надо обойти его справа) будет мостик\footnote{50°26'19"N   30°29'46"E} через Лыбедь.

На восток от Лыбеди, на протяженности отсюда и до улицы Ивана Федорова, лежат высоты Батыевой горы. На запад же находится местность, в 19 веке официально именованная как Новое строение.

За зданием Лыбедская, 1В можно видеть\footnote{50°26'16.7"N 30°29'44.9"E} открытое русло окончания сводного коллектора двух левобережных притоков Лыбеди, Богданова ручья и речки Мокрой – через него перекинут мостик. Этот коллектор впадает в Лыбедь за домом Семьи Праховых, 6.

Лыбедская улица поворачивает, на ней есть мостик через Лыбедь\footnote{50°26'12.4"N 30°29'54.7"E} – с него можно спуститься к речке. Другой подход чуть юго-восто\-чнее, у поворота\footnote{50°26'02.6"N 30°30'05.4"E} улицы Эренбурга на Набе\-режно-Жилянскую (которая идет вдоль коллектора Лыбеди).

Продолжая движение на юго-восток, вскоре, у восточной оконечности улицы Физкультуры, Лыбедь вбирает в себя крупнейший приток Клов

50°25'52"N 30°30'25"E

Отсюда на юго-восток до моста по улице Ивана Федорова Лыбедь течет по желобу в эдакой бетонной широкой улице, по обе стороны которой – глухие стены зданий окружающей промзоны. На этом отрезке находится Пятничный Клов (см. статью) и ниже его по течению, слева впадает небольшой приток из круглой, пролазной на четвереньках трубы. 

Почти сразу южнее моста на Федорова, с правой стороны, в речку впадает

50°25'37.9"N 30°30'34.3"E

ручей из Протасова яра.

Начиная отсюда и почти до истока речки Совки, по восточную сторону Лыбеди начинается теперь Байкова гора со старым, 19 века кладбищем.

Следующий после Протасова ручья весомый приток – справа, речка Ямка, впадает

50°25'16.2"N 30°30'47.3"E 

напротив перекрестка улиц Владимиро-Лыб\-едской и Ямской.

Вскоре после этого – мост над Лыбедью, по нему дорога ведет к улице Байковой, что карабкается на холм между двумя частями Байкова кладбища.

В первой половине 19 века напротив этой улицы Лыбедь разветвлялась, образуя по ходу течения вдоль горы эдакий остров – оба рукава смыкались недалеко от южной оконечности Байковой горы. Сейчас этого нет и речка шурует в прямом бетонном канале. С востока бывшая промзона, с запада Байкова гора. В 21 веке участок железной дороги тут одним из первым в городе железнодорожники закрыли оградой, и подбираться к реке стало трудно.

В том же месте на стыке 19-20 веков, а может и раньше, было большое болото, поэтому нынешняя улица Ковпака, которая туда спускается с востока, носила название Болотной. В 1901 году ее переименовали в Митрофановскую от одноименного алтаря расположенной поблизости Владимирской церкви, и название продержалось до 1968-го.

Следующий по счету левобережный приток впадает около моста близ пересечения улицы Байковой и Ямской, он течет под задворками четной стороны улицы Тверской\footnote{Бышая Зверинецкая, ибо вела от Нового Строения в сторону Зверинца.} от ее пересечения с Большой Васильковской, а возможно начинается еще выше. Открытое русло сего притока видно на карте 1837 года.

После окончания Байковой горы, с востока, в Лыбедь впадает\footnote{50°24'49.5"N 30°31'08.9"E} левобережный ее приток, непосредственно южнее дома на ул. Казимира Малевича (Бульонская, Заводская, Боженко), 86П. Этот приток прибывает сюда вдоль улицы Загородной, а между домом бульвар Дружбы Народов 19А и футбольным полем:

50.4174056, 30.5404472

Его можно проследить почти от одного из истоков Бусловки, а именно на юго-запад от Дома Мебели, между футбольным полем и домом на бульваре Дружбы Народов 19А. Коллектор проходит мимо 17А, между Чешской 1 и 3, пересекает Чешскую, затем между Чешской 4 и 6, вдоль шестого номера, и от южного его угла сворачивает на запад, вдоль улицы Щорса, сначала мимо дома 20А, затем 20, 18, напротив 14 переходя под нечетную сторону улицы, и до номера 5, где сворачивает под ЖК «Французский квартал 2» к зданию на Большой Васильковской 137, а уже от него вдоль улицы Загородной строго по прямой к Лыбеди.

У перекрестка улицы Гринченко и Краcноармейского переулка, не доходя до Демиевского путепровода, в Лыбедь впадает

50°24'35.9"N 30°31'20.0"E

речка Совка, о которой читайте отдельную мою книгу «Речка Совка и ее притоки». Там же был раньше металлический мостик через Лыбедь и ходили люди, сокращая путь от Лыбедской площади, но потом железнодорожники решили, что ходить нельзя, и оградили всё оградой, а мостик спилили.

Относительно скоро, при пересечении улиц Гринченко и Саперно-Сло\-бодской, за маргариновым заводом, в Лыбедь впадает

50°24'19"N 30°31'44"E

еще один крупный правобережный приток – Ореховатка, которой в словаре посвящена отдельная статья.

По северным склонам долины Лыбеди начинает тянуться Черная гора, а по южным – более пологие, но не менее высокие холмы Сапёрной слободки, которые вскоре перейдут в Лысую гору с уже весьма крутыми склонами. Сапёрная слободка, еще недавно бывшая частным сектором, сейчас тяжело застроена высотками.

Приняв в себя Ореховатку, Лыбедь немного пробегает еще по поверхности и уходит в подземный коллектор, где течет вдоль  железнодорожной станции Киев-Демиевский, ранее Киев-Московский.

Неподалеку, 

50°24'12.5"N 30°32'16.2"E

от одноименной остановки транспорта, по нечетной стороне Саперно-Сло\-бодской, справа от приземистого здания, где то прокат чего-либо, то автомойка, в Лыбедь впадает левобережный приток, известный среди некоторых знатоков города как Живец. Этот приток виден неподалеку отсюда на поверхности\footnote{50°24'16"N 30°32'17"E}, примыкая к восточной стороне трехэтажного поста электрической централизации – короче говоря, если от остановки дойти до здания вокзала, то этот пост будет правее вокзала. Приток приходит сюда от улицы Товарной, где протекает в метровой трубе, а к Товарной прибывает от лежащего восточнее, в овраге гаражного кооператива «Днепр». Овраг, ныне основательно засыпанный, начинался некогда чуть южнее перекрестка улицы Чешской и Ивана Кудри.

Сопоставляя с притоком название Живец, знатоки города сильно ошибаются. В земельных документах 16 века, относящихся ко владениям Михайловского монастыря, а именно к Девич горе, ныне известной как Лысая, упоминаются описания границ этих владений, в частности\footnote{Из судебного письма киевского наместника Василя Рая про розмежевание спорных Орининских грунтов и Девич горы, между Свято-Михайловским
Золотоверхим монастирем и земянином Максимом Панковичем, от 8 декабря 1572}:

\begin{quotation}
а иж се игумен и чернцы водле права ку доводу на тую землю слушне, а пристойне домовляли, сказали есмо им первей, абы они границы тое земли обрубное свое значне показали, которые стороны своее границы явные нам показуючи, повели, взявши от ставу стго Михайла Золотоверхого от млинка своего на реци Лыбеди, ричкою Моричанкою уверх до старое гребелки, от гребелки и к колодезищем студенцом тым потоком, долиною до Вычовки\footnote{Ошибка: Сычовка, ныне Совка.}, где сходитсе чотырох земль границы,
печерская граница Троецкая, Багриновская стго Михайла Выдубицкого манастыра, а по другой сторони земля Орининская Девич гора стго Михайла Золотоверхое церкви от тых границ долиною уверх \textbf{живца крыницы Лукарца}, Лукарцем уверх озера Лукарецкого от озера в перевал, перевалом у Днепр просто, уверх Днепром до Лыбеди уверх, Лыбедю до млынка Михайловского.

Мы, видечи показане слушных, а явных границ земли Орининское и Девич горы стго Михайла Золотоверхого, сказали есмо игуменови и чернцом на держанье тое земли водле права осмънадцот светков ставити людей добрих, виры годних, сусидов околичних.\end{quotation}
 
Как видим, речь идет о \textbf{живце} – ручье колодца Лукарца. Живец это не название водотока, это слово обозначает ручей. Также очевидно, что описанное выше не относится к местности протекания притока, хотя и приток, и Лукарец граничат с Девич-горой, однако с разных сторон Лыбеди.

После вливания в Лыбедь сего притока, Лыбедь под Саперно-Слободской улицей переходит на ее подгорную сторону, ближе к Лыске. Вот мы миновали вокзал, и по южную сторону начнется, если еще не началась, Лысая гора. К северу от Саперно-Слободкой, между шоссе и железной дорогой, лежит металлобаза и переулок Саперно-Слободской, подобно ужу вьющийся между старыми малоэтажными и частными домиками, среди которых есть даже деревянный.

А напротив этого райончика и металлобазы, только южнее, через Саперно-Слободскую улицу, начинает возвышаться зеленый склон Лыски, и в удолье между виднеется казенного вида бетонный забор с воротами. Это водонасосная станция «Сапёрно-Слободская», и стоит она на ручье или речушке, что начинается\footnote{Где-то тут: 50°23'46.3"N 30°32'25.6"E} южнее, на задворках Лыски – ручей там имеет открытое русло. По ходу его есть даже пруд. Этот ручей проходит вдоль автобазы и в коллекторе уже впадает

50°24'08.0"N 30°32'37.5"E

в Лыбедь немного западнее надземного пешеходного перехода. В устье этого поточка в 19 веке (в первой половине) лежал большой пруд, принадлежавший Лавре.

На Викимапии сей ручей именован «Лукарец», но я не рискую делать такой вывод.

Далее справа (по ходу Лыбеди) начинает тянуться уже крутой, суглинный склон Лысой горы, хотя там и не такой высокий, как ближе к Днепру. Лыбедь как и прежде течет под землей. Напротив начала эстакады с улицей Киквидзе, по правой же стороне есть загаженный круговой заезд под землю\footnote{50°24'4"N 30°32'50"E}. Через него вы попадаете в подземный коллектор, по коему движется Лыбедь. Света снаружи достаточно, чтобы обозреть ближайшие пару десятков метров – бетонные колонны, тоннель, вода, шприцы, мусор.

На противоположном берегу, если пройти немного дальше по течению, но не доходя до выхода из тоннеля, будет подземное устье притока – Бусловки, про которую читайте отдельную статью. Она добралась сюда от улицы Киквидзе под огромной эстакадой и железнодорожными путями. Раньше по путям можно было перебраться от низовий Киквидзе к Лысой горе, но в 2019 году рельсы отгородили забором, якобы с видеонаблюдением, и пройти нельзя иначе как по верху, по эстакаде, что очень долго и неудобно, ибо строители кажется не предусмотрели перемещение по ней пешеходов.

Вдоль Лысой горы там же тянется сухое, глубокое русло, заросшее бурьянами. Сама река однако течет то ли рядом (скорее всего), то ли под этим руслом, но под землей в коллекторе, из коего выходит только возле\footnote{50°23'59.4"N 30°33'12.5"E} гаражного кооператива «Лыбедь», на месте коего Лыбедь раньше раздваивалась и текла двумя рукавами, образуя эдакий остров.

Начинается отрезок русла, где Лыбедь протекает в более-менее природных условиях, не в бетоне, но по песчаному дну, вдоль зеленых берегов. Берег левый – рекультивированная свалка с лэпами, летом тут буйствует зелень. Свалка граничит с гаражами. Берег правый – крутой склон Лыски. У выхода из коллектора валяются обкатанные с виду камни, но кажется изделия из глины вроде обломков кирпичей. Все ветки над руслом увешаны обрывками мусора, кульков – они поднимается до уровня веток в половодье.

Так Лыбедь на свободе течет около 400 метров до мостика\footnote{50°23'51.0"N 30°33'28.4"E} с железной дорогой и затем почти сразу до Столичного шоссе. Перед мостиком в бурьянах есть подступы к широкой трубе\footnote{50°23'51.7"N 30°33'26.9"E} над речкой, по трубе при определенной смелости переходят с одной стороны на другую разные люди, которым неймется. Если обувь скользкая, можно загреметь вниз с высоты в несколько метров. Лыбедь тут широкая, около дюжины метров, а глубина летом чуть выше колена. 

Местность тут кстати называется Теличкой.

За мостиком через 15 метров, у перекрестка с улицей Промышленной, Лыбедь уходит под Столичное шоссе и выходит из-под него через 70 метров, где продолжает течь уже в бетонном желобе, вдоль улицы Промышленной, то правую ее сторону (если идти по течению).

Примечательно, что на противоположной стороне улицы, около ее стыка с Камышинской\footnote{50°23'52.4"N 30°33'34.2"E} есть участок прежнего русла устья Лыбеди, сухой и замусоренный. Левый берег этого оврага составляет железнодорожная насыпь, по ней проложены пути подвоза сырья на ТЭЦ-5. Но далее от указанной точки овраг идет вдоль насыпи рукотворный.

А в 19 веке Днепр примыкал к холмам – Зверинецкому и Лысой горе, вдоль которой протекал рукав Днепра. И прямо между горами, при выходе из долины, и было устье Лыбеди – это позже оно стало выдвигаться по новой суше на запад (до постройки ТЭЦ-5) и юго-запад (как сейчас).

...Вдоль Промышленной улицы, среди промзоны, течет Лыбедь к Днепру в искусственном, раньше его тут не было, бетонированном канале с постепенным перепадом высот. Ширина русла такая, что с разбегу можно попробовать перепрыгнуть, хотя трудно. Я не пытался. Глубина – в среднем по колено. А вот скорость течения на протяженности разная, вероятно в зависимости от глубины.

Вдоль русла на ветках висят всё те же кульки. Канал постепенно понижается, если идешь к Днепру, становится очевидным, как берега вокруг канала растут. Но это понижается дно канала, он сам. Над ним кое-где перекинуты мостики, в том числе железнодорожный, для обслуживающих заводы товарняков. Кроме обычных асфальтированных мостов, есть и металлические платформы поверх труб весьма широкого диаметра.

За последними местность, и без того какая-то похоронно-волшебная, становится по ощущениям подобна пронизанной ручьями весенней роще, в любое время года, хотя ничем свежим там не пахнет, а слева за забором прессуют металлолом, правый же берег делят гаражи, канализационная станция\footnote{Ведь вдоль Лыбеди, параллельно ей, существует Лыбедский канализационный коллектор.} да хибары бедняков.

Вдоль внутреннего бетонного желоба, по бетонным плитам, возвышаются, образуя как бы валы, грязевые наносы, кое-где поросшие жалким осокором.

На север оттуда, на территории Асфальтобетонного завода, в декабре 2019 я застал начало некоего строительства.

И вот открывается вид на устье Лыбеди – бетонное гидротехническое сооружение, исписанное граффити и напоминающее палубу авианосца, только с каналом посередине. Оттуда вода Лыбеди с шумом и бурлением попадает в глубокий с виду днепровский залив.

Напротив этого места, но на левом берегу, в дачах Осокорков, лежит улица Масловсвкая, которая по прямой линии продолжается улицей Коллекторной. Под ней проходит тот самый «лыбедский» коллектор – он добирается сюда под днепровском дном идет эдак от правобережного устья Лыбеди!\\

\medskip

\textbf{Лукарец} – Сементовский в 1852 году писал:

\begin{quotation}
У подошвы Девич-горы протекает чистый кристальный источник Лукарец; в иные годы весной он превращается от изобилия воды в небольшое озеро; красота этого источника привлекательна, самая вода в источнике имеет свойство железных вод\footnote{То бишь должна быть рыжего цвета от железистых частиц.}, и поэтому весьма полезна в некоторых болезнях. В минувшие годы соседние обыватели знали это свойство Лукарца и пользовались его врачебною водою. Недалеко от этого источника протекал у подошвы Лысой горы другой источник Живец [...]
\end{quotation}

Похилевич в книге 1865 года издания «Монастыри и церкви Киева» цитирует документ 1701 года, где описывается граница землевладения:

\begin{quotation}
взявши от гребли и от млына Михайловского на речце Лыбеди, речкою Маричанкою вверх до старой гребельки; от гребельки до студенцов потоком и колодежищем; долиною до Вычовки\footnote{Сычовки, то бишь Совки.}, где сходятся границы четырех земель\footnote{Землевладений монастырей.}: Печерской, Троицкой, Багриновской и Выдубецкой, и по другой стороне земля Ориновская Девич-гора. 

От тех границ долиною вверх живця к кренице Лукарца; Лукарцем вверх озера Лукарецкаго.

От озера на перевал, в долину, перевалом в Днепр просто; Днепром вверх до Лыбеди; Лыбедью до млына Михайловского и до ставу.
\end{quotation}

По многим соображениям, Лукарец существует поныне при Лысогорском спуске, по координатам 50°23'12.7"N 30°33'02.9"E\\

\newpage


\textbf{Лукьяновка, Лукьяша}

Нагорный район, центром коего служит Лукьяновская площадь. Прежде Лукьяновкой считалась местность большая, нежели сейчас – например, раньше Лукьяновкой считались и Татарка, и Репяхов яр. 

Название местности некоторые краеведы связывают со старостой подольского цеха шевцов Лукьяна Александровича, который после 1845 года поселился в этих краях. Иная версия – что от некоего владельца хутора, по фамилии Лукьянов. Но еще в 1820 году письменно проходит топоним «Лукояновка», а в 1824 году – предместье Лукьяновка, где был хутор киевского золотаря Самсона Стрельбицкого, от коего произошла улица Самсоновская (ныне Глебова), именуемая так с 1838 по 1952 год.

Мне кажется, что название Лукьяновка связано с корнем «кий», «киян», и означает нечто вроде искаженного предместья Киева.

На Лукьяновке начинаются относительно значительные киевские реки, текущие в разные стороны – Скоморох и Глубочица.

Северной границей нынешней Лукьяновки можно считать Кмитов яр, вернее, он лежит на стыке Лукьяши и Татарки. Татарка еще и то, что лежит по четной стороне Овручской улицы эдак от пересечения с Багговутовской и далее к Подольскому спуску.\\

\medskip

%Лукьяновское трамвайное депо РАСШИРИТЬ
%ДОПИСАТЬ\\


\textbf{Любка} – речка, протекает поныне мимо хутора Любки и впадает в Ирпень севернее Романовки. На Любке есть пруд «озеро Коцюбинское». Речка огибает южную границу Коцюбинского поселка.

Любка обозначена на карте 1746 года, где река показана полной мере – увы, я не могу сопоставить по этой карте исток. Современное русло я прослеживаю от улицы Малинской\footnote{50°28′55.5″N 30°22′03.5″E}, но очевидно, что начинается оно еще восточнее. 

Из жалованной грамоты Петра I Лавре, 1720 года:

\begin{quotation}
«на Беркове при речке Любце два человека» – сказано в Ведомостях из Москвы присланных в старых Разрядных Делах найденных, явились списки с привилев Великий князей Российских
\end{quotation} 

\medskip

\textbf{Любка} – другая речка, про которую Семеновский в «Киев, его достопамятности» 1852 года писал:

\begin{quotation}
Недалеко от этого источника (Лукарца) протекал у подошвы Девич горы другой источник Живец, а несколько далее впадает и ныне в Днепр не\-большая речка Любка, над которою издревле находилось небольшое поселение Багриново.
\end{quotation}

Девич гора ныне известна как Лысая, что близ Зверинца. Багринова гора лежит южнее Лысой. У южной части Багриновой горы протекает, прудами, ручей от Голосеево.\\

\medskip

\textbf{Лысая гора}

50°30'31.2"N 30°26'36.3"E

Еще одна Лысая гора, выше горы Западинки, условно говоря там здание Западинская, 10 – школа  №156.

Прежде, северная часть улицы Западинской называлась Песчаной, а та, что подходит к школе с юга – Лысогорской улицей.\\

\medskip

%Лысая гора (Девич) РАСШИРИТЬ

\textbf{Лыски}, слобода – известна по крайней мере в середине 19 века, проще говоря это был частный сектор по Черепановой горе. Севером Лыски граничили с Бульонной слободой.\\
%\textbf{Лукрец, Лукарц, Лукарец} – 


%\textbf{Лыбедь}, хутор (19 век) – хутор между Байковой горой (концом Кировоградской) и Московской площадью.
