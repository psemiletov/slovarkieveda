\chapter*{Ш}
\addcontentsline{toc}{chapter}{Ш}

\textbf{Шайба} – он же Барабан, двухэтажное круглое здание. Построено в 1983 году рядом с Печерским базаром как «Универсам», на 2019 год в нем тоже супермаркет. На первом этаже «Универсама» был отдел продуктов с самообслуживанием, и у входа еще небольшой отдел, где стоял автомат по жарке пончиков. На втором этаже (он представлял собой внутреннее кольцо над первым) помню только игрушечный отдел.\\

\textbf{Шалаш} - так до Великой Отечественной войны называли танцплощадку в тогдашнем парке Ежова, ныне Березовая роща, что на Приорке. Во время войны сгорела, равно как и располагавшиеся там же зеленый кинотеатр, летняя эстрада и ресторан.\\

\textbf{Шалена баба} – в 19 веке, заросший травой ров в Беличах.\\

\textbf{Шато де флер} – Chateau des Fleurs, дореволюционный парк, находился по месту нынешнего стадиона Динамо. Эту часть Царского сада арендовал артист В. Н. Дагмаров, устроив там несколько летних летних театров, пивные павильоны, тир, ресторан (цены устанавливались городской управой) и прочее. Дагмаров платил городу в год 17 000 рублей арендной платы, за вход брал в будни 30 копеек, в субботу 20, по праздникам 40, однако до 7 часов вечера вход был бесплатным. В одном из театров шла русская оперетка, в другой помеременно комедия, фарс и дивертисменты.

По сути, Шато де Флер конкурировал с Садом купеческого собрания (что был около филармонии) с его концертной эстрадой и рестораном. Днем тот был открыт бесплатно, а вечером, на концерты симфонических оркестров, стоил 40 копеек.\\ 

\textbf{Шархавщина} – название Феофании в 17 веке. Тогда ходило два названия – Лазаревщина да Шархавщина. Сильвестр Косов подарил селение домовому чиновнику своему Ширховскому (Шарховский), потому и название. После Ширховского владение перешло к Софийскому монастырю.\\

\textbf{Шахравщина} – в 19 веке часть села Крюковщины. Названа так, по местной молве, от землевладельца, князя Шахрая. Скорее всего речь идет о том же Шарховском, что владел Феофанией.\\

\textbf{Шелковичный сад}

Дал название улице Шелковичной. Примыкал, на стыке 18-19 веков, к Кловскому дворцу с юга, находился на условно говоря северном берегу речки Кловицы, на крутом склоне.\\

\textbf{Ширма} – см. мою книгу «Речка Совка и ее окрестности». Холмистая местность между Демиевкой и Совками, занята преимущественно лабиринтом частного сектора и небольшим количеством советского времени более высотных зданий. Границей между Демиевкой и Ширмой можно считать Казацкую улицу.\\

\textbf{Шияновские улицы} - они же Шияновские дома, или, по выражению Дмитрия Бибикова, Шияновские нужники.

В первой половине 19 века и до 1940-х таковыми слыли две улицы, Большая и Малая Шияновские, около Печерского базара. Подробно о них рассказывает Николай Лесков в своих воспоминаниях "Печерские антики". Деревянные дома сии, разнокалиберные и бедные, принадлежали Шияновым. Там сдавалось жилье разным людям, как обычной бедноте, так и тем, кто не хотел привлекать внимания властей - например староверам, или людям с фальшивыми паспортами (в Царской России это был способ жить где хочется, а не где положено).

На Шияновских домах, как и вообще в городе, при Бибикове висели доски с обозначением годов, когда какой дом предписано снести. Также, ветхие дома воспрещалось чинить, но местное население латало дома при помощи досок, внешне состаренных, чтобы не были заметны следы ремонта.

Шияновские улицы лежали от Печерского рынка до Засарайной улицы, названной так потому, что находилась она за сараями Военного ведомства. Ныне Засарайная улица стала частью бульвара Леси Украинки. Некоторое время Большая Шияновская носила одновременно название Воспитательного переулка, ибо на ней в середине 19 века стоял приют для бездомных детей.

Большая Шияновская именуется теперь улицей Николая Лескова, а Малая Шияновская - Немировича-Данченко, и конечно, они не сохранили колорита, описанного Лесковым, хотя на ней осталось несколько довольно старых домов, однако не бибиковской эпохи.\\

\textbf{Шоколадный домик} – построенный в 1899 году особняк по проекту архитектора Владимира Николаева, возведен для купца первой гильдии Семена Семеновича Могилевского, по слухам для его тайных встреч с замужней любовницей. После революции там обитал глава Одесской ЧК Раковский, потом Каганович, затем размещался НКВД, и с 1960-го – ЗАГС. Сейчас там музей. Адрес: Шелковичная, 17/2.\\

\textbf{Шполянка} – местность, гора между низовьем Бабьего яра и Сырцом. Бывшая дореволюционная дача землевладельца, купца Василия Ипполитовича Шполянского (?-29.01.1919), у него стоял там большой дом. В урочище Копылово, в 1894 году он арендовал, за 7 рублей 25 копеек в год, с торгов, большой участок земли - который и прослыл потом Шполянкой.

Урочище лежит вдоль улицы Копыловской и на запад от нее, там где улица Шполянская, Фруктовая, Петропавловская и окрестности. По 1940-е годы тут были частные огороды и дачи жителей Куренёвки. Ныне – частный сектор.

Исконные улицы района – Шполянская, Артезианский переулок, Фруктовая, Тагильская, Верболозная.\\

\textbf{Шулявица} – левый приток Лыбеди, название краеведческо-диггерское. Среди Диггеров известен также как Источная. Сама имеет нес\-колько притоков.

Течет в коллекторе откуда-то со стороны промзоны между проспектом Победы и Гарматной, эдак из-за Киевприбора. Пересекает Гарматную у севера здания номер 7, затем течет на юго-запад к торговому комплексу по адресу Индустриальная, 6 (бывший корпус Большевика), проходит сначала под ним, потом между ним и Т-образным (сверху) зданием Выборгская 42А, на его широте пересекает Индус\-триальную, наискосок идет через квартал Ин\-дустриальная – Выборгская – Металлистов – Кузнечный переулок, пересекает последний у Выборгская 18/20, следует за домами (на север от них) 16/15 и 12, пересекает Выборгскую около дома 10 (общага КПИ), проходит под перекрестком Борщаговской, Выборгской, Полевой, Янгеля, проходит между Борщаговской (шоссе) и домом 115 (Институт энергосбережения), и сохраняя юго-восточное направление идет к железной дороге, пересекая улицу Мамина-Сибиряка, затем между домами 99 и 97АК2 по Борщаговской, пересекает Дашавскую, идет под большим автохозяйством и за ним соединяется с открытым коллектором Лыбеди в точке примерно 50°26'40.5"N 30°27'32.9"E, то есть ближе к стыку автохозяйства с гаражами.\\

%\textbf{Шулявщина} – деревня Шулявщина, бывшая по месту нынешней Шулявки. Жителей этой деревни выселили в Белгородку в 20-30 годах 19 века по указу, наделяющему Киев выгонной землей. За отъятие земель жителям полагалось вознаграждение, а поскольку обитатели Демиевки и Зверинца таковой свободной для выгона земли не имели и занимались городскими промыслами, им разрешили записаться в мещане.

\textbf{Шулявщина}, деревня – казенная деревня, в 19 веке занимала место КПИ (до его постройки) и окрестностей оного.\\

\textbf{Шулявское старое кладбище} - находилось около КПИ, там где сейчас здание по адресу Борщаговская 16А. Небольшой погост был и при близлежащей церкви Марии Магдалины - на ее месте сейчас наземная часть станции "метро КПИ". Церковь разобрали на кирпич весной и летом 1935, а рядом из того светло-желтого кирпича возводили школу. Попутно уничтожили погост, где были похоронены основатели КПИ - прах профессоров\footnote{В.П. Ермаков, М.И. Коновалов, П.Ф. Ерченко, А.В. Кобелев, Е.П. Вотчал, М.К. Пимоненко, А.Н. Дынник, С.М. Сольский.} впрочем перенесли на Лукьяновское кладбище.

%\textbf{Шулявское кладбище}

%50°26'40"N 30°26'5"E
