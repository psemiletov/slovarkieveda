\chapter*{З}
\addcontentsline{toc}{chapter}{З}

\textbf{Забора, Забара} – урочище, известное в 19 века, перекресток нынешних Автозаводской и Луговой, еще тогда это был перекресток дорог. Через него, чуть южнее, с Приорки протекал ручей Западинка.

От названия местности происходили улицы Старозабарская (часть ее стала Автозаводской, часть исчезла в 1980-х вместе со сносом старой застройки) и параллельная ей Новозабарская.\\

\medskip

\textbf{Забара} – на плане окрестностей Киева 1850 года так обозначено некое урочище южнее Братской Борщаговки, между нею и северо-западной окраиной Жулян.\\

\medskip

\textbf{Загоровщина, Загоровка, Загара}

50°28'26.3"N 30°29'07.7"E

Местность на Татарке, заросшие деревьями и кустами склоны Кирилловских высот условно напротив Института автоматики на улице Нагорной. На начало 21 века у местных жителей в ходу название Загоровка, туда на террасу-площадку за зданиями ходят устраивать пикники. Площадка слывет как Загара или Нижняя стрелка, Стрельбище либо Кафе Прощай Молодость, из-за встреч там одноклассников.

Стрельбищем площадка именовалась от того, что там было стрельбище вышележащего по склону гостинично-спортивного комплекса «Авангард» (его построили к Олимпиаде 1980 года). Поныне земля усеяна там, около забора, разбитыми тарелочками вроде виниловых, используемых в качестве бросаемых в воздух мишеней.

Загоровщина известна по делу Бейлиса как место нахождения в пещере трупа мальчика Андрея Ющинского. Урочище простирается по Смородинский спуск включительно.

Вообще говоря, пещер на Загоровщине было много, как и на близлежащем Смородинском спуске. Но сейчас на Загоровщине пещеры, судя по всему, все засыпаны.

В начале лета 1831 года мещанин Василий Ювженко обнаружил где-то на Загоровщине, под липой, пещеру глубиной более 2.8 метра, которую позже засыпали. 

Я не знаю, почему на современных картах урочище Загоровщина обозначена в районе улиц Пугачева и Герцена, то бишь верховья Репяхова яра. Вероятно именно с карт оно стало проникать и дальше. Предположу, что корень сего мнения о расположении Загоровщины лежит в топонимическом словаре Резника и Пономаренко.

\newpage

\textbf{Замковище} – урочище южнее Мостицкого массива и горы Липинки, в районе улицы Замковецкой и Замковецкого переулка. На начало 21 века там частный сектор и застраиваемая местность. Название дает основание полагать, что здесь были остатки какого-то замка. Соседствует с урочищем Беличье поле.

На плане 1914 года «урочище Замковище» соседствует с Сукачевым яром (он лежит севернее) и подписано чуть севернее современной Замковецкой улицы, там где улица Кавалеридзе, ограниченное с запада Межедой. Словом, где-то тут:

50.494090664752164, 30.426991875178718

Замковищем издавна называли остатки земляных укреплений, валов, следы строений.\\ 

\medskip

\textbf{Замок Ричарда Львиное Сердце, Дом Ричарда} – дом на Андреевском спуске, 15, назван так по внешнему виду, напоминающему замок (раньше там была еще винтовая лестница) и жильцу по имени Ричард Матвеевич Юревич, который обитал там по крайней мере по 1970-е. Юревич некогда был уланом польской армии и воевал в Первой мировой. Знал семью Булгаковых, живших чуть ниже в 13-м доме. На первом этаже Замка Ричарда раньше была, кроме прочего, конюшня.

Построенное в начале 20 века на средства купца Орлова, здание было доходным домом, после революции в нем разместились коммунальные квартиры. Предание гласит, что вдова Орлова расплатилась со строителями несправедливо, и те «поселили» в дом привидений, насыпав в дымоход яичную скорлупу. Её нашел там профессор Киевской Духовной академии Степан Тимофеевич Голубев и объяснил, что воздух, проходя через мелкие дырочки в скорлупе, вызывал те странные
звуки, что пугали обитателей дома и создавали оному дурную славу.

Дом западом выходит к небольшой горе Уздыхальнице с плоской, срезанной верхушкой – с верхушки есть один из входов в Дом Ричарда. И у северного подножия этой горы – «дом Булгаковых».\\ 

\medskip

\textbf{Западинка, Западинский}, ручей – протекал большей частью в длинном овраге, где сейчас проспект Правды. Овраг начинался у перекрестка проспекта Правды с Межевой.
 
Название ручья происходит от местности Западинка (Западинцы). 

От прежнего частного сектора осталось две улицы – начало Западинской и Галицкая (бывшая Песчаная или Песочная). Песочная лежала выше, севернее ручья. На карте Кульженко 1894 года, по северной стороне улицы Песочной, вдоль нее и до перекрестка с Вышгородской показан короткий ручей, невесть куда впадающий.

Сам же Западинка протекала так – по ходу нынешнего проспекта Правды, по нынешней улице Ивашкевича до перехода ее в улицу Луговую, где сворачивала более на юг, следуя на запад от улицы Автозаводской (прежней Старозабарской) до соединения с Курячим Бродом, и далее общий поток шел в речку Сырец.

Некоторые диггеры имеют иное мнение про ручей Западинку, называя так ручей, что течет под Геофизприбором и улицей Западинской к Вышгородской, и затем прослеживают ход сего ручья под улицей Полупанова (ныне Приорской) и к Бережанскому рынку. Очевидно, что речь идет о безымянном ручье, что протекал вдоль Песочной улицы. Однако, возле перекрестка с Вышгородской к коллектору этого ручья присоединяется коллектор, проложенный под проспектом Правды – истинное русло Западинки.\\

\medskip
%чинается под гаражами на улице Новомостицой, затем идет южнее Брестской (параллельно ей) через местность Замковище, всё более застраиваемую высотками вместо частного сектора. В Замковище еще в 19 веке были остатки какого-то замка.\\

\textbf{Западинцы, Западинка}

50°30'20.0"N 30°25'57.0"E

Гора, урочище – холм на север от проспекта Правды, бывшего оврага, где протекала речка Западинка. На карте середины 19 века гора Западинка лежит между проспектом Правды, проспектом Свободы и Межевой улицей

%По урочищу лежала улица Лысогорская, эдак от Западинской к нынешнему проспекту Победы. На Западнике была своя Лысая гора.

По ряду старинных карт, урочище Западинцы занимает оба берега оврага проспекта Победы, и северный и южный.

Севернее располагается местная Лысая гора, никем уже не угадываемая.\\

\medskip

\textbf{Запечерка, Запещерка} – местность за Лаврой, там где сейчас музей Великой Отечественной Войны. Улица Запещерная, прежде проходящая через частный сектор, существует и сейчас в виде безымянной дороги к музею, отходящей на гору от улицы Лаврской. Там же были улицы Запещерная-Лабораторная, Ново-Петровская, а участок Лаврской снизу доверху назывался прежде Ново-Наводницкой.

В справочнике Пономаренко и Резника упоминается «народное название» этой местности – Запретка, которое лично я не слышал, и оно напоминает мне искаженную Запещерку. Справочник же производит «Запретку» от неких бывших тут языческих капищ, дескать, место было запретным для нечистой силы. Хотелось бы знать источники, откуда сие почерпнули сочинители справочника.\\

\medskip

\textbf{Запольская дорога} – известная по крайней мере в 18 веке дорога, она же Великая, шла через Вету, как и Белоцерковская, а от Веты отделялась через Юровку к городам Запольским, Заполью в Невеселовском поле над Унавой, Ирпенем и Стугной.\\

\medskip

\textbf{Зверинец Неводничий} – отдельное урочище на Зверинце, принадлежал Выдубицкому монастырю, упоминается с начала 17 века.\\

\medskip

\textbf{Звенигород} – согласно Нестору, «иже есть город мал у Киева, яко десяти веръсты въдале». По ряду списков, однако, не Звенигород, а Белгород.\\

\newpage

\textbf{Зелёнка}

Сам театр: 50°26'49"N 30°32'47"E

Верх. опорн. стена: 50°26'49"N 30°32'46"E

Нижн. опорн. стена: 50°26'52"N 30°32'49"E

В середине 19 века, в связи с сооружением Цепного моста и прикрывавшей его башни – Подольского набережного верка – на склоне над Днепром соорудили две полукруглые подпорные стены, под которыми проложили тоннели трубопровода водокачки Подольского верка. Водокачка находилась в верке, и по чугунным трубам вода из Днепра поступала в мастерские Арсенала, соединенные с верком подземным ходом. 

Подземные коридоры остались поныне. 

На нижней стене тренируются альпинисты и скалолазы. Если подняться по земляной горе к верху опорной стены, слева будет вход в стену – темный кирпичный провал ведет в коридор, где разбитая, опасная лестница сходит вниз. По правую руку – бойницы, сужающиеся наружу. Я не спустился в самый низ из-за того, что фонарик разрядился, а света фонарика в смартфоне было мало. Пол внизу усыпан битым кирпичом, и если ход есть, то между полом и потолком расстояние теперь совсем невелико и с лестницы его нельзя определить. Всё очень загажено. 

Ежели идти к нижней опорной стене со стороны Днепра, шоссе, то не доходя до нее, слева\footnote{50°26'52"N 30°32'52"E}, будет круглый вход в дренажку ЗТСМД («Зелёный театр – станция метро Днепр»), куда можно войти чуть пригнувшись. На лето 2021 вход открыт. По середине бетонной трубы течет водный поток, идти первые несколько метров можно еще по краю, а потом нужны резиновые сапоги или бахилы.

Неподалеку, но выше, над нижней стеной, есть также популярная среди диггеров ДШС №12 (50°26'52"N 30°32'48"E).

В 1949 году по проекту архитекторов А. Власова и А. Заварова и инженера Н. Пестрякова между подпорными стенами был построен Зеленый театр на 4000 мест, использовавшийся в том числе как кинотеатр под открытым небом.

Его-то развалины и были в 1990-е облюбованы неформалами и среди нефоров Зелёнкой слыли именно остатки зеленого кинотеатра. Там зарождалось диггерское движение, а про окрестности ходили слухи и страшилки, что это аномальная зона с подземными порталами куда-то. И дескать, по Зелёнке бродит потустороннее существо – Хозяин.\\

%РАСШИРИТЬ!!!!

\medskip

\textbf{Зелёнка}

50°27'38"N 30°24'29"E

Зеленый театр в парке «Нивки». Построен в 1958 году как летний, на 2700 мест, кинотеатр «Ленинского комсомола» (одноименный с этой частью парка), по проекту архитектора Чуприны.

На 2021 год закрыт и пребывает в плачевном состоянии.\\

\medskip

\textbf{Зорька} – советский кафе-бар «Зоряне», близ кинотеатра «Украина» на улице Карла Маркса.\\