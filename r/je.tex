\chapter*{Ж}
\addcontentsline{toc}{chapter}{Ж}

\textbf{Жаба}

50°27'19.2"N 30°31'45.4"E

Знаменитая в советское время танцплощадка, что была на склоне между лестницей, ведущей к памятнику Магдебургскому праву, и Аркой в честь воссоединения Украины с Россией. Просуществовала в относительно целом состоянии  примерно по 2012 год. Название возникло от скульптуры зеленой жабы (метр на полтора), стоящей неподалеку.

Над танцплощадкой Жабой раньше был ресторан Слоник, названный так от фонтана со скульптурой слона.\\

\medskip

\textbf{Жандармский сад} 

В. Л. Беренштам в «Т. Г. Шевченко и простолюдины, его знакомцы» писал в 1900 году:

\begin{quotation}
В течение года я еще раза два встречал Чапыгу: мои знакомые видели его не раз на народных гуляниях, а также на вечерах в Жандармском саду (сад находился там, где теперь Николаевская, Ольгинская и Меринговская улицы).
\end{quotation}

\medskip

\textbf{Жандармские огороды} – см. б\'ольшую статью про Арестантские огороды.\\

\medskip


\textbf{Желань} – летописное урочище, возможно нынешние Жуляны, вопреки расположению улицы Жилянской. В 1161 году Торки, преследуя Изяслава от Белгорода (современная Белогородка), настигла обоз Изяслава у Желани, а его полки – около Буличей (Беличей). Вероятно с Желанью словесным корнем связана и Шулявка (и летописный Шелвов борок). Расстояние между Жулянами и Шулявкой всего 5 километров.\\ 

\medskip

\textbf{Жеребьевщина}

50°30'03.2"N 30°25'57.3"E

На середину 19 века урочище представляло собой прямую улицу, вдоль которой стояли домики с большими садами. Ныне – многоэтажная застройка, дома 6 и 12 в Апрельском переулке.

В начале 20 века урочище с юга огибала Жеребьевская улица, ныне Новомостицкая.\\

\medskip

\textbf{Жилгородок} – район на Бусовой горе, между низовьями улицы Зверинецкой (от дома 61) и параллельном ей там же отрезке Бусловской, по улицу Соловцова. Естественная застройка советского времени – хрущовки в 3-5 этажей, и пара девятиэтажек. В Жилгородке обитали рабочие ДОКа, знавшие друг друга в лицо, а окружающая местность представляла собой частный сектор. Ныне частный сектор преобразился в терема, а частично застроен небоскребами ЖК Триумф и Бусов Хилл. Исчезла также двухэтажная общага серии 207-7 на Бусловской, 15. За нее, справа, спускались быстрым путем на Тимирязевскую (мимо дома 42 на оной).

В советское время в Жилгородок ходило ма\-ршрутное такси – небольшой автобусик – от разворота за Печерским мостом (ближе к Суворовскому училищу), и в Жилгородке около трехэтажки по адресу Зверинецкая, 32 (с продуктовым магазином на первом этаже) была другая его конечная. Название Жилгородок – чисто местное – я, живя на близлежащей Бастионной, его не слышал (а узнал о нем только в 2017 году от Entertaining stuff with eng subs), хотя в нашем доме на Бастионной 11-А жили в том числе и доковцы.\\

\medskip

\textbf{Жуковка}, лаврский хутор 19 века – от него название Жукова острова. При хуторе было урочище Чернечье, оно же Галерное.\\
