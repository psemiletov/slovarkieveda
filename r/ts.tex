\chapter*{Ц}
\addcontentsline{toc}{chapter}{Ц}

\textbf{Царский сад} – нынешний «Крещатый парк» и Городской сад. В 19, начале 20 веков – парк Царский сад. В названии том сохранялось прошлое, ведь в середине 18 века здесь, на высоченных днепровских кручах, был виноградный государев сад.\\

\medskip

\textbf{Царское село, Междужопье} – жилой район в верхней части Наводницкого оврага. Делится пополам улицей Старонаводницкой. Восточная часть ЦС, частный сектор – между Старонаводницкой и Лаврой. Западная часть (бышее урочище Монах) – между Старонаводницкой и бульваром Леси Украинки – элитные новостройки, начавшие появляться еще в начале девяностых или конце восьмидесятых взамен тоже частного сектора, раскинувшегося по уступам склона (см. про урочище Монах).

Восточная часть раскинулась на горе вдоль улицы Панфиловцев и примыкающей к ней причудливой улицей Редутной и одноименным переулком. В советское время это был частный сектор, где жили офицеры. Сейчас от былой постройки почти ничего не сохранилось и вместо небольших домиков выросли хоромы. На местности сохранились земляные редуты Печерской крепости.\\

\medskip

\textbf{Цыганская гора} – про нее мне рассказал в письме Andre Todosi, и хотя я порой бывал в том месте, однако не знал, что у него есть отдельное название. Можно сказать, что это урочище, часть восточного склона Бусовой горы.

\begin{quotation}
Цыганкой местные называли спуск по улице Николая Соловцова, от пересечения её с улицей Бусловской, до поворота в конце спуска\footnote{Спуск по проулочку между частных домов соединяется с Тимирязевской в точке 50°24'21.4"N 30°33'34.6"E}. [...] Это было одно их двух основный мест поблизости, где местные жители катались на санках (вторым
местом был спуск по Тимирязевскому переулку от пересечения с улицей Бусловской). По словам мамы, Цыганкой эту горку называли из-за того, что на этой улице в одном из домом (в конце спуска, где-то на повороте) жили цыгане.
\end{quotation}

Ромы жили также в одном, в советские времена, из домов чуть южнее пятой школы, поэтому вероятно можно говорить о своеобразной цыганской слободке. По рассказам уже моей мамы, в дом около школы люди ходили к цыганам чтобы те погадали.\\

\medskip

\textbf{Цымбалов яр} – яр и одноименная улица, а также переулок, в частном секторе урочища Добрый путь, близ Демиевки. Улица разделена на две части, каждая из которых лежит в отдельном, параллельном другому овраге. Урочище граничит востоком с Саперной слободкой, а западом – с лесом в Голосеевском парке, где границей коему служит улица Максима Рыльского с домом-музеем на ней.

Название яра возможно происходит от фамилии Цыбалов, представители которой жили на Демиевке. По другой версии – от домовладельца  Цимбаленко.

В 21 веке частный сектор яра активно застраивается ЖК.

%\textbf{Турец} – «смуговина Турец» у Иорданского озера, из актов 17-18 веков. Пролив, вытекавший из Иорданского озера по Оболони, к старинному валу, что длился от Юркового ставка.