\chapter*{М}
\addcontentsline{toc}{chapter}{М}

\textbf{Марганинка} – маргариновая фабрика, рядом с кондитерской (бывшей Карла Маркса).\\

\medskip

\textbf{Мариченка, Морычанка} – речка, приток Лыбеди. Устье Мариченки было около пруда и млына (водяной мельницы) на Лыбеди, принадлежащего Свято-Михайловскому Златоверхому монастырю. Речка упоминается в земельных документах 17 века. Всё это в окрестностях Демиевки и Лысой горы. С каким известным нам притоком Лыбеди ее сопоставить?

Из жалованной грамоты Петра I, за 1700 год, в подтверждение прав Михайловского Златоверхого монастыря на владения имуществами, размежевание земель 1576 года описано подобным образом:

\begin{quotation}
начав от ставу и от млина михайловского на реки Либеди, речкою Мариченкою в верх к Старой гребелки, к Студенцом, тем потоком и к колодязищем, долиною до Вычевки, где сходятца четирех земел границы: Печерская, Троецкая, Багриновская, стаго Михаила Выдубецкаго мнстря;
\end{quotation}

Млин Михайловский по карте де Боскета 1753 года находился на пруду чуть ниже впадения в Лыбедь речки Ореховатки (Ореховатицы), бегущей через Голосеевские пруды по парку имени Рыльского.

Это дает основание сопоставить Мариченку с Ореховаткой.

Еще более четко можно отождествить по иску от 1563 года кременецкого старосты Николая Збаразкого и киевского наместника 
Иосипа Немирича земянину Максиму Панковичу про возвращение Орининских грунтов Свято-Михайловскому Златоверхому монастырю. Орининская земля – условно говоря то, что сейчас слывет правобережной Лысой горой. И вот как описываются ее границы:

\begin{quotation}
землица звечная того манастыря свтго Михайла Золотоверхого, лежачая кгрунтом своим за Либедю прожив монастыра Печерского граню своею межи землею Печерскою и выдубицкою по чом от ставу свтго Михайла Золотоверхаго речкою Маричонкою уверх по дорогу Гостинец, што от млына Чихачовского идет мимо ниву Куриловскую дорогою около нивы Чопахи, а около Чопахи в долину тою долиною вливо на
колодезики студенци, тым потоком в Лукарец, от Лукарца в перевал, перевалом у Днепр просто, уверх Днепром до Лыбеди, уверх Лыбедю, яко ж, дей, на он час все ся покажет достаточне, на которой же земли греб[л]я и млын свтго Михайла Золотоверхого стоит церковних
\end{quotation}

Единственное, что сбивает с толку, так это упоминание горы Маричинки в рукописной книге француза де ла Флиза «Медикотопографическое описание государственных имуществ Киевского округа» 1854 года, где есть такое описание деревни Совки:

\begin{quotation}
Место положение. Деревня лежит в 5 верстах от г. Киева на юг, расположена частью в самом овраге, а частью на взгорью, окруженная горами и кустарниками. Значительных гор четыре: название коих, 1-а Маричинка, 2-ая Курган, 3-ая Вотхый Лес, 4-ая Лишкова. Все вышиною от 5 до 6 сажений\footnote{Сажень – около двух метров.}.
\end{quotation}

Гора Маричинка, и одноименная река. Возле деревни Совки, как известно, протекала речка Совка, тоже приток Лыбеди, но в земельных документах встречаются одновременно и Совка (как Сычовка) и Маричинка, стало быть это не один и тот же водоем. Хотя расстояние между деревней Совки и Голосеево, где течет Ореховатка, по прямой не так уж далеко, но между ними водораздел.\\ 

\medskip

\textbf{Маслай}, город – между Борками и Лютежем, разрушенный ордой город, где была, по словам местных, записанным Лаврентием Похилевичем, «епископия и 60 церквей и монастырей». Похилевич видел рвы, валы, кирпич и щебень остатков каких-то зданий.

Борки теперь затоплены водами Киевского водохранилища, а ведь когда-то севернее них проходило устье речки Ирпень. Сохранилась ли местность города Маслая, неизвестно.\\

\medskip

\textbf{Междужопье} – см. Царское село. Название же Междужопье возникло потому, что местность эта лежит между тылами двух статуй – Родины-матери и Леси Украинки.\\

\medskip

\textbf{Межигорская улица} – не та, что современная, а была такая на Зверинце, в первой половине 20 века, соединяла улицу Лейтенантскую (командарма Каменева) и Автостраду (бульвар Дружбы Народов), в нынешних пределах как бы спускаясь вдоль западного крыла больницы Четвертого управления. Там ведь был, около кладбища, частный сектор, его снесли примерно в начале восьмидесятых.\\

\medskip

\textbf{Метростроя поселок} – жилой район с интересной застройкой около школы 161 по улице Академика Каблукова, окрестности ее и улицы Метростроевской. Возник в 1950-х. До сих пор тут находятся некоторые учреждения от Метростроя.\\

\medskip

\textbf{Мичиган} – так стиляги, они же шузня (от «shoes»), именовали открытое кафе на открытом воздухе «Грот», что было на Крещатике.\\

\medskip

\textbf{Мировичей}, усадьба

50°27'14"N 30°22'36"E

Львовская, 15. Участок между улицами Петрицкого, Верховинной, Львовской. На 2016 год – заросшие бурьяном развалины военного госпиталя (по 2004-й), часть домов, в том числе с охранным статусом, разрушены. Были вырублены вековые дубы и сосны, земля отдана под застройку. В 2015 году там собрались делать парк.

До революции дачной усадьбой по №23 на Пушкинской, 252 владела О. Васильева, затем В. Мирович. В 1918 году дом был конфискован, усадьба объединена с соседними, и устроено детское оздоровительное учреждение. После Великой Отечественной войны – медико-санитарная часть, затем госпиталь военного городка номер №195, позже получившего уклон в лечение туберкулеза.

Неподалеку на восток, за усадьбой детского сада, находится Центральный Городской противотуберкулезный диспансер №2 (Львовская, 3). Среди сосен – желтые, вероятно сталинских времен одно-двухэтажные здания, раскрашенная скульптурная группа Красной шапочки с грустным Волком, на кирпичном постаменте, кованые ворота из рядом железных копий.\\

\medskip

%Усадьбой владел купец, почетный гражданин города Захарий Павлович Мирович, в конце 19 века ему же принадлежал и «дом Булгакова» на Андреевском спуске, перекупленный потом семьей архитектора Василия Павловича Листовничего («Василисы» из «Белой гвардии»), который вероятно погиб в 1919 году при побеге с парохода, везшего заключенных  по Припяти\footnote{Архитектор был арестован ЧК в ночь с 6 на 7 июня. Из справки 06.07.1990 года УКГБ УССР: «Сообщаем, что Листовничий Василий Павлович до ареста работал заведующим строительно-монтажной секции хозяйственного подотдела, одновременно являлся служащим губернского отдела народного образования. Арестован 7 июня 1919 года. 14 августа 1919 года Чрезвычайная Комиссия отправила Листовничего В.П. в лагерь на весь период гражданской войны».}.

\textbf{Мокрый яр} – я встретил это название в книге Головченко и Мусиенко «Золотые ворота» (о подпольщиках времен Великой Отечественной войны), как часть Соломенки. По логике это Кучмин яр, где протекает речка Мокрая по одноименной улице, и если название не выдумано писателями, то это и есть местное, на 1940-е годы, название если не всего Кучмина яра, то его части.\\

\medskip

\textbf{Мокрая, Мужичек} – правый приток Лыбеди, протекает в Кучмином яру, вдоль западного склона Батыевой горы. Название речки происходит от одноименной улицы Мокрой, что существовала уже в 19 веке, а может улица названа по речке.

В краеведение введено название «река Мужичок», которое Вортман прочитал на карте «Места вокруг Киева» 1753 года. На скане этой карты я не могу разобрать именно такую надпись. Однако Мужичек упоминается в земельных документах, относительно того же ручья или речки, значит прочтение верно.

Там же на карте, по правому берегу реки, показано село Паньковщина, а по левому, близ впадения в Лыбедь, изображено некое строение и подпись «Вавилон».

Исток в верхнем роднике\footnote{50°25'27"N   30°29'10"E} в Соломенском парке. Протекает через парк вниз к улице Кудряшова, под ней, под железной дорогой и впадает в Лыбедь за домами между Льва Толстого, Гайдара и первыми номерами Лыбедской\footnote{Примерно 50°26'16.9"N 30°29'47.8"E}. Подробнее читайте в заметке про Кучмин яр.

Около пересечения улиц Урицкого, Толстого и железной дороги к коллектору Мокрой присоединяется подземный же коллектор Богданова ручья, и общий коллектор идет к Лыбеди параллельно улице Толстого и впадает в Лыбедь за домом Семьи Праховых, 6.

%Второй исток Мокрой, взятый в метровые трубы, находится у Соломенской площади.


% Соседствовал с левобережной слободкой Паньковщиной.

Примерно в окрестностях был, на 19 век, и большой пруд, относящийся к сенокосной даче Софийского монастыря.\\

\medskip

%\textbf{Мостицкое кладбище} -

%\textbf{Мужичок} – брод на Лыбеди близ впадения Мокрой.\\


\textbf{Монах, Монашеский сад, Монастырский сад} - в середине 20 века так называли местность нынешнего "высотного" Царского села, между бульваром Леси Украинки, Суворовским училищем и Старонаводницкой. 

Я лично помню это место в прежнем виде смутно, на стыке 1970-80 мы ходили смотреть оттуда салют, а перед нами лежали голые террасы, уже подготовленные к застройке, и отделенные от нас чуть ли не забором с колючей проволокой.

А прежде там были заросли и пустыри, яры и горки, там катались на санках и лыжах, гуляли.

Старожил Николай Черныш пишет в "Уголке моего сердца", что "на западе крутой обрыв урочища начинеался наделко от того места, где сейчас стоит памятник великой поэтессе. Из-под этого обрыва вытекал ручей. Он растекался по низине, образуя болотца, густо заросшие тросником". Это, по сути, описание одного из истоков Наводницкого ручья.

Черныш далее сообщает, что в Монахе были овраги, поляны, часть его ближе к Суворовскому училищу состояла из сада, большей частью яблоневого, а ближе к Старонаводницой располагалось подсобное хозяйство училища. В 1950-х работникам училища выделяли в Монахе землю под грядки, затем решили переместить эти огороды ближе к Зверинецкому кладбищу, и где-то неподалеку оттуда находился пруд-копанка, позже обратившаяся в болотце. 

Судя по плану 1812 года и топокарте 1923-30, эта "копанка" лежала по ходу того истока Наводницкого ручья, что начинался "у Леси", и сейчас на ее месте стоит высотка по адресу Старонаводницкая 8-Б. 

А чуть севернее был другой исток, но см. статью про Наводничи.

В этой нижней части Монаха, вдоль Станонаводницкой улицы, но не на соседнем холме, в 1970-е годы стояли частные дома. 

Уничтожение Монаха производилась в несколько этапов. Сначала после ВОВ, с подачи училища, несколько лет засыпали самосвалами часть низины и устроили на получившемся плато стадион. Затем верховье с другой стороны было откушено строительством здания нынешнего ЦИКа. Также, остаток Монаха, надо сказать всё же основную его часть, собирались превратить в ландшафтный парк - тогда-то всё вырубили, домики снесли, ну а к девяностым началось почти всю местность застроили высотками, как тогда считалось, для депутатов и афганцев.

\medskip


\textbf{Музейный городок} – в 1930-х так называли территорию Лавры.\\

\medskip

\textbf{Мусмановский лес} – на середину 19 века, лес по правому берегу одного из истоков ручья Куриный Брод, примерно между улицами Брестской и Замковецкой. Сопоставим с восточной частью горы Липинки.\\


\medskip

\textbf{Мушанка, Мушенка, Мужинка} – речка, протекает в сторону Днепра со стороны Мостища, на широте между Горенкой и Котыркой, впадала где-то на Оболони, на обозримых мною картах исчезает в дореволюционных полях орошения. На землях бывшего хутора Мушинка, позже Бернера, речка существует поныне в виде цепи прудов среди соснового леса на территории госпиталя ГУ МВД вдоль Лесозащитного переулка.
