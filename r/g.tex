\chapter*{Г}
\addcontentsline{toc}{chapter}{Г}

\textbf{Галаганы, Станкозавод} – вплоть по середину 20 века, хутор примыкавший к юго-восточ\-ным околицам нынешней станции метро «Нивки». Теперь это местность окрестностей улицы Галагановской и от нее на восток и юго восток к Дружковской, к железной дороге.

Хутор в 19 веке основали братья Иван, Мусий и Егор Галаганы, что были лесниками.

В Галаганах было свое кладбище, сейчас там к северу от дома, что по адресу Галагановская, 6 – гаражный кооператив\footnote{50°27'24.1"N 30°24'15.2"E}. А на месте самого дома рос хуторской сад. Оттуда на восток протекал ручей, приток Рубежовского ручья, что и сейчас бежит течет вдоль железной дороги, дабы влиться в Сырец. Либо этот ручей можно считать одним из истоков Рубежовского. Другой исток находится у западного конца улицы Дружковской, там тоже гаражные кооперативы. Рубежовский впадает в Сырец около Рубежовской станции, там чуть ли не водопад, во всяком случае перепад высот внушительный.

Однако от северной части Галаганов, пересекая проспект Победы, у Сырца есть еще один приток – короткий ручей, что с пригорка впадает в Сырец к востоку\footnote{50°27'35.9"N 30°24'35.2"E} метрах в пятидесяти от первой гати через пруды в парке Нивки.

На хуторе Галаганы в 1893 году построили  церковь Святого Александра Невского (по проекту Владимира Николаевича Николаева). В росписи участвовали В. Васнецов, П. Сведомский, В. Котарбинский. Около церкви было кладбище. Храм закрыли в 1934, и уничтожили кладбище в 500 могил рядом с церковью. Ее помещение стало домом культуры Киевского завода станков автоматов. Вероятно, позже на его месте возвели новый ДК, имени Горького, а может это то же здание, переоборудованное. На 2021 год ДК Горького снесен, на его месте небоскреб.

От Станкозавода весь район Галаганы местные жители иногда так и называют – Станкозавод.\\

\medskip

\textbf{Гаращенко хутор} – см. Караваевщина.\\

\medskip

%Галерный остров (Водников)

\textbf{Генеральская дача} 

50°26'43"N 30°22'27"E

Дом отдыха высших военных чинов при воинской части А-3482, в северной части Казенного леса, на прудах одного из притоков речки Борщовки.

Южнее находится парк «урочище Совки», с селением Совки ничего общего не имеющий. Местных жителей на пруды не пускают. Только для людей в погонах!\\

\medskip

\textbf{Глинка}, озеро

50.409991399119036, 30.528935106396542

Водоем у западного подножия Черной горы, в ложбине, рядом с бульваром Дружбы народов. Это заполненный водой карьер кирпичного завода, частично совпадая со старым глинищем завода Эйсмана, позже Субботиной (дочь Эйсмана), еще позже Журавлева. В раскопах карьеров в окрестностях Глинки в 19 веке был найден янтарь, зубы акулы, древние следы растений – пальмы, секвойи, а также относящиеся к еще более давним временам морские растения. 

Глинка питается от родников, стекающих по склонам. На плане 19 века видно, что с северо-востока сюда стекал некий ручей.

Ложбина озера находилась прежде несколько западнее. Так на 1918 год, озеро размером с современное граничит строго с современным, но западнее, и еще западнее было два меньших озерца (а на север от них еще и озеро Бернер). Тогда же, на месте современного русла был склон горы. Подобное положение прослеживается по 1923 год, после чего, к 1930-му, два мелких озерца соединяются, а основной карьер начинает ползти на восток, поедая склон, но еще не заполняется водой.

На 1941 тогдашнее озеро Глинка уже наполовину совпадает с современным. По рассказам старожилов, туда во время войны со склона в воду съехал танк.

В начале нулевых мы со Светой Семёновой лазали по берегам озера в ходе съемок краеведческого фильма «Киевская сюита». На плоту по озеру плавали два пацана, мы стали расспрашивать их про озеро. Они поведали, что вроде бы на дне лежат машины, а также тут находили скелеты младенцев, потому что на другом берегу был роддом.

Роддома не было, но какие-то древние скелеты вполне могло вымыть водой из склонов, нарушенных добычей глины.

На склоне со стороны бульвара Дружбы Народов виднелся вход в дренажку. Берега были замусорены, со стороны переулка Филатова над озером нависала обрывом круча.

В конце 2016 года было затеяно укрепление берегов озера, сюда подвели какие-то трубы, сложившийся за десятилетия рельеф изменился бесповоротно. Еще весной 2016 года я успел вдоволь полазать по оползающим, частично превращенным в свалку, крутым и местами водоточивым и с рыжими железистыми частицами, берегам озера. Сверху же отлично обозревались окрестности в сторону Голосеево и Демиевки – высота огромная и всё видно как на ладони.

В 2019 году на берегах озера развернулись строительные работы по возведению ЖК.\\

\newpage

\textbf{Глубокий яр} – овраг на Черной горе, между улицами Тихая и Мендлеева. Занят гаражным кооперативом. В начале 20 века там был, вероятно, кирпичный завод Журавлева.\\

\medskip

\textbf{Глубокое, долина} – в 16 веке долина, описанная в документах как находящаяся между Кудрявцем и Лыбедью.\\

\medskip

\textbf{Глубочица, Глыбичица}, урочище

50°27'44.8"N 30°29'28.5"E

Окрестности низовий улицы Соляной и Пимоненко, да Кудрявского спуска = собственно тамошний овражище между горой Кудрявца и Щековицей.\\

\medskip


\textbf{Глубочица, Глыбочица} – взятая в коллектор речка, прежнее ее название – Кудрявец. Именно это название проходит в давних земельных документах и на картах. Например, на плане 1787 года русло Кудрявца в пределах Подола обозначено так: «канал чрез предместие из ручая Кудрявца». На карте же 1800 года название «Глыбiчица» приурочено к яру между холмом Кудрявца и Щекавицей, к средней его части (см. урочище Глубочица).

У речки два истока. 

Один начинается около Цветущего (Квитучего) переулка и стекает оттуда вдоль улицы Кмитов яр, где соединяется с истоком из другого отрога этого яра, что на территории Дачи Хрущова близ Института Педиатрии. О том, другом истоке читайте статьи Болото и Кмитов яр.

Миновав завод Артёма (в продолжении Кмитова яра на его нынешней территории был длинный пруд) Глубочица в подземном коллекторе пересекает улицу Татарскую и проходит в удолье меж двумя холмами до улицы Глубочицкой, через ЖК «Львовский квартал». До него там дичал пустырь, появившийся на месте разрушенного дрожжевого завода, что существовал кажется по крайней мере до конца 20 века. Завод сей возник на месте дореволюционного еще, водочного завода купчихи Чоколовой, а в удольи был большой пруд.

Другой пруд находился, по крайней мере в середине 19 века, в месте где сейчас улица Соляная соединяется с Глубочицкой. Ибо по оврагу Соляной тоже протекал крупный ручей, такой большой, что его можно считать третьим истоком Глубочицы.

Сейчас Глубочица принимает в себя несколько подземных ныне притоков, один из которых диггеры называют Кудрявцем, хотя без оснований, ибо Кудрявец это старое название самой Глубочицы. Этот приток, взятый в коллектор, начинается у сквера в самом верховье Кудрявского спуска, а затем идет под спуском до Глубочицкой улицы.

Другой приток, Киянка, выходит из урочища Кожемяки. Подробно я описываю всё это в «Ереси о Киеве» – что и где, и как Глубочица протекала в пределах Подола раньше. Здесь же добавлю, что Глубочица впадает в Киевскую гавань около Рыбальского моста.\\

\medskip

\textbf{Гнилая}, улица – улица старого, допожарного Подола\footnote{9 июля 1811 года сгорел почти весь Подол.}, шла примерно под нынешней улицей Боричев Ток. Послепожарная Гнилая – нынешняя Покровская.\\

\medskip

\textbf{Голгофа} – панорама, аттракцион, находился на Владимирской горке выше нынешнего Украинского дома, на уровне костела. Внутри павильона устанавливалась круговая картина и некоторые предметы, всё это составляло панораму. Первой темой ее были последние дни Христа, потом показывали «Вифлеем», «Поражение наполеона», «Распятие Христа».

Павильон открылся в 1902 году и был оснащен по последнего слову техники – для установки полотна картины применялась особая тележка на рельсах, темное время суток светили газовые горелки, в качестве материала для ступеней использовали модный тогда бетон. Голгофа обошлась предпринимателям-владельцам в 18 тысяч рублей, однако за два месяца они почти отбили эту сумму за счет посетителей.

Голгофа пережила революцию, множество смен власти и простояла на горке по 1934 год, продолжая приносить доход. За снос Голгофы ратовал режиссер Александр Довженко, коего избрали членом президиума Комитета содействия созданию парка и заместителем председателя проектного бюро. В 1934 году павильон разобрали, а полотно свернули в рулон и перевезли в Лавру, в успенский собор, где картина и сгинула вместе с собором в 1941 году.\\

\medskip

\textbf{Голосиевка}, совхоз – до пятидесятых годов 20 века лежал непосредственно к северу от большого пруда по нынешней улице Генерала Родимцева. Граничил с Покровской Голосеевской пустынью.\\

\medskip

\textbf{Гончары} – урочище между горами Детинкой и Старокиевской, проще говоря овраг с улицей Гончарной. Некогда тут жили гончары. В советское время на Гончарах стояли небольшие, о двух этажах кирпичные домики с садами. Тогда же местность слыла Гончаркой. Ныне урочище считают частью Воздвиженки.

Вообще прежде вся нынешняя Воздвиженка именовалась Гончары и Кожемяки, от смежных тут урочищ.\\

\medskip

\textbf{Горка} – короткий грунтовой путь по склону горы, на верхней террасе которой стоит дом по адресу Бастионная 11-А, а внизу – дом по Бастионной 13, он же Арсенальский. Если подниматься, то слева за гаражами и грушей-дичкой будет остаток частного сектора, длящийся до продолжения верхней террасы (уже под домом на Бастионном переулке, 11). Горкой пользуются в основном жители домов Бастионная 11-А и Бастионный переулок, 11. Первый из них тоже был ведомственный, построенный работниками (для них же) Полиграфкниги, ДОКа, и швейной фабрики.\\

\medskip

%Гостинец

\textbf{Город} – так жители Бастионной и окрестностей называли центр Киева вообще, а свой район не Зверинцем, а Печерском. Мы говорили: «Поехать в город», «Выберемся в город».\\

\medskip

\textbf{Госпитальное кладбище}, оно же военное. Дореволюционное. Остатки его находятся на южном краю Бусовой горы, в конце Зверинецкого переулка (бывшего Военно-Кладбищенско\-го). Лежало на месте склона Бусовой горы, изрядно съеденного добычей глины для кирпичных заводов, но вероятно зародилось не как военное, а как обыкновенное, еще в начале 19 века, поскольку на одной из могил сохранился год – 1812 и надпись, почему-то «МАНЯ».

В феврале 1878 года Киевское городское управление по просьбе Инженерного управления выделило тут землю, 4 десятины 2000 квадратных саженей под кладбище для военных, умиравших в госпитале. 

Еще в середине 20 века занимало всё низовье горы между железной дорогой, Зверинецким переулком, Соловцова, также в окрестностях улицы Сорочинской как к железной дороге, так и до Зверинецкой. Западным обрывом оно выходило к низовьям речки Бусловки, или, если угодно, к низовью улицы Киквидзе, там сейчас построили ЖК «Панорама». Теперь на месте большей части кладбище – частый сектор.

Смутно помню, что в 1990-х я, помимо креста с «Маней», видел там в зарослях конопли еще и могилы второй половины 20 века, вероятно местных жителей.

На кладбище установлен современный памятный крест в память воинов, погибших в ходе освободительной войны в Болгарии 1877-1878 годов.\\

\medskip

\textbf{Госпитальный спуск} – можно сказать, что бульвар Леси Украинки это сильно выпрямленный Госпитальный спуск. Тот шел примерно там же, но изгибаясь влево-вправо.\\

\medskip

\textbf{Государев сад}, позже Городской сад – заложен во время Петра I, занимал место, условно говоря, от Марьинского дворца по филармонию включительно, позже от него отчекрыжили, в северо-западной части, сад Купеческого собрания и Шато де Флер.

\medskip

\textbf{Граевщина} 

50°29'16.2"N 30°26'57.4"E

Так в 19 веке слыло большое, заросшее деревьями урочище на правом берегу речки Сырец, сейчас застроено разными ангарами. Напротив него на Сырце была плотина с городской мельницей, а выше её пруда, выше по течению – кожевенный завод купца Серебреникова, и выше его еще одна мельница на другом пруде. Но это отельный разговор, тогда вообще весь бедный трудяга Сырец был цепью прудов.\\

\medskip

\textbf{Графа Игнатьева} – урочище, слывущее на стыке 19-20 веков, оно же улица Большая на Верхней Соломенке. На Графе стояло тогда более 200 домов.\\

\medskip

\textbf{Графская гора} 

Условно говоря, склон над метро Крещатик, а шире – между улицами Городецкого и Круглоуниверситетской. Середина горы – улица Лютеранская, прежде Графская. На горе была Немецкая слобода.\\

\medskip

\textbf{Груша Володимирова} – урочище, в 16 веке граничащее с Тесными улицами и Старой дорогой.\\

\medskip

\textbf{Грушки}, хутор – по середину 20 века, хутор на одну улицу, между железной дорогой и началом (по перекресток с Выборгской) современной улицы Николая Василенко (и на восток от него). На восточных окраинах хутора было два жалких озера. Название хутора – от его прежних владельцев, дворянина Константина Грушко и его супруги Устины, владевших имением в 1871-1902 годах, после того как Киевская палата казенных имений выделила тут 14 земельных участков под аренду для киевлян, и Грушко взяли в свои руки большую часть. 

С 1898 по 1902 год земля (76 десятин) хутора отдана городом военно-инженерному ведомству взамен «военных» 38 десятин, отведенных под Политех.

По 1960-е в Грушках существовал частный сектор, застроенный позже промзоной.\\

\medskip

\textbf{Грушки} – район, промзона на месте бывшего хутора, ограниченная проспектом Победы, Гарматной, Машиностроительной, Василенко. Включает в себя жилой квартальчик Чугуевку\footnote{50°27'9"N 30°25'32"E}, получившую название от Чугуевского переулка. Там стоят хрущовки и двухэтажные домики. Зеленое тихое место.
