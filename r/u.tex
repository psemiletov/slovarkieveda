\chapter*{У}
\addcontentsline{toc}{chapter}{У}

\textbf{Уваров великий, Уваров малый} – в 17 веке, нивы где-то между деревней Берковец и местечком Приоркой. Принадлежали доминиканскому монастырю.\\

\textbf{Уздыхальница, Здыхальница} - 

РАСШИРИТЬ

сказать про дом Булгакова

Прежде гора была несколько выше, ее уменьшали несколько раз - например, в 19 веке при генерал-губернаторе Бибикове, но и ранее, в веке 17. В неизданной киевской летописи Ильи Кощаковского сказано: 

\begin{quotation}
Року 1616 при воеводе киевском Станиславе Жолковском, старанем и коштом панов мещан киевских: пана войта на тот час Федора Ходыки и бурмистра Матвея Мачохи и всего поспольства места Киева, копали гору Уздыхальницу, которая стоит пред замков киевским; а выкопали полшеста сажня у звыш, а нашли там печерку трох саженей у гору, и вширь сажень; там нашли горщик порожний и написано на стене имя: Павел; знать, же то колись был пустельник.\end{quotation}



\textbf{Унизовая долина} – по крайней мере в середине 18 века так называли часть оврага, где улица Нижнеюрковская проходит мимо котельной Лукьяновской.\\

\textbf{Усадьба Меринга} – известная своим прудом дореволюционная усадьба профессора медицины Киевского университета Фридриха Фридриховича Меринга, ныне окрестности театра Франко.

В начале 19 века усадьба принадлежала графу  Безбородько, затем ею владел жандармский корпус и усадьбу называли «Жандармский сад». Уже тогда, в дикой местности на картах виден пруд или озеро. В 1862 году Александр II пожаловал усадьбу (10,5 гектара) гене\-рал-майору Ф. Ф. Трепову, затем ее купил Меринг и отвел большую часть под сад для горожан (рядом был также частный парк «Тивали» с альпийской горкой). 

Зимой на пруду заливали каток. После смерти Меринга в 1887 году его наследники решили продать землю под застройку, что и осуществили в конце 19 века. Тогда на месте пруда разбили Николаевскую площадь, а по остальной усадьбе положили улицы для застройки – Николаевскую, Меринговскую, Ольгинскую, Новую. В ходе великих трудов местность обезобразили – было срыто 175 000 кубометров грунта, которым потом засыпали Афанасьевский яр и хватило еще на досыпание земли на Владимирской горке. Усадьба ушла за цену 1,8 миллиона рублей.

На месте пруда в итоге возник театр Соловцова, ныне Франко.\\

\textbf{Урод} - фонтан, бассейн на Думской площади. Существовал, кажется, еще в 1870-х.\\
