\chapter*{С}
\addcontentsline{toc}{chapter}{С}

\textbf{Сад Кащенко}

50°28'13.4"N 30°28'02.5"E

Акклиматизационный сад растений располагался на территории нынешнего Института Журналиситки. На время кончины имел площадь 5 гектаров. Уничтожен в 1975 году для постройки тут Высшей партийной школы при ЦК КПУ. Сад был заложен в конце 1920-х академиком Николаем Феофановичем Кащенко и тогда был размером около 12 гектаров.\\

\textbf{Сад Райгородок} - см. Райгородок.\\

\textbf{Садмарское}, село Троицкого Больницкого Монастыря, – на земле Садмарского «селища» потом возникла Слободка Мышеловка.\\

\textbf{Самбург}, хутор – находился к востоку от Красного трактира и юго-востоку от Голосеевской пустыни. В советское время – совхоз Самбулки, что лежал к югу от совхоза Голосиевка.\\

\textbf{Самбург}, речка – около Голосеевской пустыни.\\

\textbf{Сапёрная слободка} – причудливо застроенный зеленый район на холмах между Демиевкой и Лысой горой, вдоль железной дороги. Или, если угодно – между улицей Сапёрно-Слободской и проспектом Науки (захватывая дома по оному). Сохранилось много частного сектора и советской застройки.\\

\textbf{Сапёрное поле} – район на месте бывшего на стыке 19-20 веков саперного полигона. В 20 веке – частный сектор, хрущовки и прочие здания, несколько заводов – хлебкомбинат, винный завод. В 21 веке местность была застроена небоскребами, заводы сгинули, частный сектор на 2021 год держится в окрестностях улицы Саперное поле и низовий улицы Чигорина, да и частный сектор этот уже новый, с теремами.\\

\textbf{Сарай} – народное название клуба «Современник» в парке им. Фрунзе. Ныне в помещении – ФРEEDOM.\\

\textbf{Сахалин} – микрорайон на Новобеличах, в частности так слыли домики от 2 до 5 этажей на месте нынешнего ЖК Грюнвальд, по Клавдиевской улице. На 2021 год 3-5 этажки сохранились между углом Корсунской и Клавдиевской.

Здания были от оборонного ведомства и от беличанской исправительной колонии, потому и возникло название, а также от расположенных близ этой окраины песчаных полей. В дома переселились сотрудники Минобороны, прежде обитавшие в барачном поселке Шанхай, находившемся в лесничестве. 

В советское вреся, в околицах Сахалина были пионерлагерь "Молодая гвардия", барахолка и свалка авиазавода.

Со временем, название Сахалин расползлось, пожалуй, на всю северо-западную часть Новобеличей, а то и на весь район, как на удаленный, на отшибе.\\

\textbf{Свитанок} – он же Курятник. Летний кинотеатр на Татарке, напротив Красного дома. Существовал по начало восьмидесятых.\\

\textbf{Сверщовское место} - в конце 16 века, участок на Подоле, был продан Братскому монастырю земянами, супругами Андреем и Натальей Обуховыми.\\

\textbf{Святое озеро}

50°25'21.5"N 30°33'16.2"E

Находилось у нынешнего  перекрестка улицы Струтинского (Болсуновской) с бульваром Дружбы народов. Питалось от двух ручьев, каждый из которых стекал по упомянутым улицам. Из озера вытекал ручей к большому озеру в Наводницкой балке, у перекрестка бульвара с улицей Лаврской, под горой с Родиной-Матерью, куда по оврагу нынешней улицы Старонаводницкой, по оврагу Наводницкому, подруливал Наводницкий ручей, который тоже имел два истока, тоже в месте соединения образующих озеро. 

Из озера на площади героев ВОВ, образованного из Наводницкого ручья и ручья от Святого озера, общий поток шел к Днепру. Вся эта водная система в полной мере показана на плане «Киев на печерах» невесть какого года, относящемся к Киевской крепости.\\

\textbf{Святоозерская слободка} - поселение на холме Святым озером на месте нынешней больницы 4-го управления, рядом со Зверинецким кладбищем. При слободке была Иоанно-Предтечинская церковь.\\


\textbf{Святославский яр}

Улица Святославская в советское время именовалась Чапаева, теперь это Липинского. Перпендикулярная ей нынешняя Франко – это дореволюционная Афанасьевская. Их пересечение лежит в глубоком, относительно вышележащей улицы Ярославов вал, овраге, застроенном большей частью старыми доходными домами.

Константин Паустовский в «Книге о жизни» писал именно об этом перекрестке, что при нем упирался в пустырь:

\begin{quotation}
Святославская улица, застроенная скучными доходными домами из желтого киевского кирпича, с такими же кирпичными тротуарами, упиралась в огромный пустырь, изрезанный оврагами. Таких пустырей среди города было несколько. Назывались они «ярами».

Весь день мимо нашего дома тянулись к Святославскому яру обозы «каламашек» с глиной. Каламашками в Киеве назывались тележки для перевозки земли. Каламашники засыпали овраги в яру и ровняли его для постройки новых домов. Земля высыпалась из каламашек, на мостовой всегда было грязно, и потому я не любил Святославскую улицу.

В яр нам строго запретили ходить. Это было страшное место, приют воров и нищих. Но все же мы, мальчишки, собирались иногда отрядами и шли в яр. Мы брали с собой на всякий случай полицейский свисток. Он казался нам таким же верным оружием, как револьвер.

Сначала мы с опаской смотрели сверху в овраги. Там блестело битое стекло, валялись ржавые тазы и рылись в мусоре собаки. Они не обращали на нас внимания. Потом мы настолько осмелели, что начали спускаться в овраги, откуда тянуло дрянным желтым дымком. Дымок этот шел от землянок и лачуг. Лачуги были слеплены из чего попало – ломаной фанеры, старой жести, разбитых ящиков, сидений от венских стульев, матрацев, из которых торчали пружины. Вместо дверей висели грязные мешки. У очагов сидели простоволосые женщины в отрепьях. Они обзывали нас «барчуками» или просили «на монопольку». Только одна из них – седая косматая старуха с львиным лицом – улыбалась нам единственным зубом. Это была известная в Киеве нищенка-итальянка.
\end{quotation}

Собственно яр представлял собой систему приярков между улицами Чапаева и Чкалова, а восточная часть была и есть на задворках улицы Лысенко, близ Золотых ворот. Один отрог яра лежал по нынешней улице Чапаева-Липинского, и имел с обоих берегов множество приярков. 

Со стороны улицы Чкалова к нему присоединялся, примерно там где велотрек, большой приярок. 

Другой громадный приярок лежал в квартале между Львовской площадью, улицей Бульв\-арно-Кудрявской, Чкалова, Ярославовым валом и Чеховским переулком. То бишь где был Сенной рынок. Этот большой приярок вливался в общий яр у сквера Чкалова, около перекрестка улиц Чкалова и Коцюбинского. Оттуда общий яр, продолжая принимать меньшие приярки, шел дальше к площади Победы примерно вдоль улицы Чкалова.

Части яра в разное время засыпались для постройки домов, для засыпки использовалась, в частности, земля из усадьбы Меринга. На плане Ушакова 17 века по месту яра, вот ближе к Золотым воротам, обозначен «городок земляной», то бишь землянки.  

Название Святославов яр я встречал в старой литературе, у того же Паустовского. Афанасьевский яр нынче в ходу у краеведов и обозначено на электронных картах, в литературе я его не встречал. Насколько я понимаю, у Паустовского Святославовым яром слыли верховья той геологической системы, которую я описал выше. Насколько это название распространялось ниже и на приярки, я не знаю.\\

%по его засыпанной части проходит улица Франко, бывшая Афанасьевская. Некоторые считают, что Афанасьевский яр охватывал всю местность включая улицу Чкалова (Маловладимирскую, Столыпинскую, Гончара). У яра было несколько отрогов. Овраг сей начал застраиваться в конце 19 века, прежде сюда свозили снег и мусор

\textbf{Сенная площадь} – в 1884 году площадь около нынешнего дворца «Украина». А в начале 20 века – площадь уже неподалеку от Лукьяновской площади, там где теперь госпиталь МВД. От этой Сенной брала начало улица Бердичевская. Сенной слыла и Львовская площадь, от нее же там назывался Сенной базар, позже перемещенный в отдельное здание на Воровского (Бульварно-Кудрявской). Еще одна Сенная площадь на стыке 19-20 веков была около Башни №4, на север от нее.\\ 

\textbf{Сент-Мари} – на начало 20 века, хутор или дача, примыкавшая к северу хутора Волейки. Сент-Мари лежал вдоль реки Сырца. Можно примерно сопоставить с улицей Магистральной и большим ГСК по другую от нее сторону Сырца, т.е. ГСК по улице Рижской. Там где на Магистральной, на левом берегу Сырца стоят частные домики с номерами 37-50\footnote{50°28'19.7"N 30°25'47.9"E} на речке был пруд причудивой формы, напоминашей перевернутую букву Y.\\

\textbf{Сетомль}, Сетомля, Ситомль. Водоем, который обычно «помещают» на Оболони, ибо сказано в Ипатьевской летописи за 1150 год – «у Сетомля, на болоньи» в соседстве с Олговой могилой. 

За год 1034, в рассказе про то, как Ярослав сражается с Печенегами в поле, где потом построили Софию, написано, что когда под вечер Ярослав победил, Печенеги дали дёру, кто куда, и некоторые утонули в «Ситомли». Значит, при побеге от нынешней Софиевской площади, можно было добраться до Ситомли, и она была достаточно глубока, чтобы в ней тонули. Кроме того, за 1065 год есть известие, что из Сетомли рыбаки вытащили сетью некое загадочное «детище», стало быть в Ситомле ловили рыбу.

\begin{quotation}
В лето 6572 (1063)

В си же времена бысть детищь ввержен в Сетомль, сего же детища выволокоша в неводе, егоже позоровахом до вечера, и пакы ввергоша й в воду; бяшеть бо сиць: на лици ему срамний удове, иного нелзе казати срама ради. Пред сим же временем и солнце переменися, и не бысть светло, но акы месяць бысть, егоже невегласи глаголють снедаему сущю.
\end{quotation}


\textbf{Синее озеро} – озеро на Виноградаре между проспектом Правды и улицей Газопроводной). Северная часть водоема была выкопана во второй половине 20 века. Местность Синего озера именовалась Выгодой, между озером и до Кинь-Грусти на восток тянулись сыпучие пески. Еще в середине 19 века озеро было одним из трех кажется прудов в цепи.\\

\textbf{Склеп Качковских}

50.477376, 30.462149

Единственный сохранившийся в Кирилловской роще склеп (помимо еще с десятка могильных плит), наследние бывшего тут кладбища... 

Склеп выпотрошен, имеет подземный этаж, куда в полу вандалами пробиты дыры. Нижний этаж забит мусором и битыми бутылками, погребений насколько я знаю нет. Под потолком склепа - закопченый образ Иисуса.

Склеп был сооружен вы 1912 году, и туда перенесли, со здешнего кладбища, прах 1863, по другим данным 1865—1909

Петр Эразмович Качковский, видный киевский хирург, был приват-доцентом университета Св.Владимира. Именно ему принадлежал участок № 33 на Мало-Владимирской улице (теперь носящей имя Олеся Гончара). Здесь в 1908 году открыла двери новая хирургическая лечебница доктора Качковского

Три года спустя на могиле Петра Качковского устроили склеп, в котором был погребен также его брат – рано умерший студент Антон Качковский. 
----------




\textbf{Скольники} – они же «школьники», то бишь бурсаки. Давнее название района Подола от Могилянки до Почтовой площади.\\

\textbf{Скоморох, Скоморовка, Скомороха} – ручей, левый приток Лыбеди. Исток возле улицы Якира (Древлянская, раньше это была часть Овручской), Киевского мотоциклетного завода, на месте которого по начало 20 века были дачи да огороды, и сеть улиц вроде Дачной, Загородней. И вот исток был у перекрестка Дачной и Овручской (Овручской в ее тогдашних пределах), а точнее в месте 

50°27'57.2"N 30°27'50.4"E

Это в 24 метрах к северо-западу от дома на Семьи Хохловых, 8.

Отмечу, что на карте 1847 года огромное «Вновьотведенное кладбище» обозначено примерно между 

50°28'03.4"N 30°28'18.2"E\\
50°28'14.2"N 30°27'54.4"E\\
50°28'03.4"N 30°27'40.9"E\\
50°27'53.6"N 30°28'03.6"E\\

То есть квартал «мотозавода» и примыкающие к нему земли. Там начинался яр, по которому протекал Скоморох, и кладбище на карте показано по берегам этого яра. Однако нынешнее Лукьяновское кладбище находится западнее обозначенной местности, примыкая впрочем к ней. Официально под Лукьяновское кладбище Дума отвела землю в 1878 году, три десятины, хотя вроде бы на его месте уже хоронили с 1871. Я пока не знаю, как это осмыслить.

...От Семьи Хохловых, 8, далее ручей в овраге пересекал нынешнюю Якира-Древлянскую у дома 14/21, и между Деревлянской 17А и Белорусской 17, а затем под Белорусской 17-А устремлялся к Белорусской, которая представляет собой свод нескольких оврагов.

Скоморох протекал по оврагу Белорусской улицы, которая еще в середине 20 века была как бы глубже теперешней. Вдоль ее нечетной стороны было дополнительное углубление, как желоб, в котором по проложенным рельсам ходили трамваи. Маршрут существовал по крайней мере по начало 1960-х, а Скоморох уже тогда спрятался там в коллекторе. Потом рельсы сняли, желоб засыпали.

От Лукьяновского трамвайного депо Скоморох тёк вдоль Бердычевской (а сейчас – под территорией упраздненного депо), потом параллельно улице Володарского (Златоустовской), к западу от нее. 

Чуть западнее площади Победы сворачивал к Лыбеди. У Скомороха есть приток из Обсерваторного яра. Этот приток – конечно подземный – протекает под Павловским сквером, пересекает Гоголевскую, Тургеневскую (там есть смотровой колодец возле дома номер 32), Дмитриевскую и на уровне 20 дома на Златоустовской присоединяется к коллектору Скомороха.

Любопытно, что протекавший в открытом русле еще в середине 20 века участок Скомороха около улиц Володарского (Златоустовской) и Речной местные жители называли Лыбедью. Берега там были укреплены досками, был деревянный мост с деревянными же перилами. По словам старожилов, вода замерзала только по краям ручья, воды было немного, хотя в ливень она поднималась почти до уровня моста. Старожилы считали, что этот ручей – речка Лыбедь. 

И если такое представление считалось справедливым и в «великокняжеские» времена, то летописные описания событий, связанных с Лыбедью, могут получить непривычное толкование.

Диггерские названия коллектора Скомороха – Скоморох и Фестивальный. Длина коллектора около 3 километров.

Он проходит под улицей Белорусской, затем от первого ее номера сворачивает на юго-восток через Лукьяновскую площадь и мимо конторского дома Лукьяновского депо (здание 7А) уходит под бывшую территорию депо, ныне ТРЦ. Пройдя также под гаражным кооперативом «Коперник» коллектор пересекает улицу Коперника и под западным углом дома 15/13 следует под детский сад (Котарбинского, 20), мимо дома 22 по ней же, пересекает улицу Котарбинского около дома 21, затем под спортивной площадкой между Глебова 16 и 14. Пересекает Глебова, идет под Глебова 7, Черновола 14, 16, 20, далее мимо перекрестка Маршала Рыбалко и Черновола, оттуда на юго-восток параллельно Златоустовской, и пересекает проспект Победы напротив западного угла «Энергопроекта» на проспекте Победы, 4. На другой стороне проспекта ныряет под дом номер 3, находящийся позади него рынок Сириус, пересекает Жилянскую улицу напротив Галицкой синагоги (Жилянская 97), и под территорией завода "Транссигнал" проходит к Лыбеди, где, в относительной близости к разворотному кольцу трамвая у остановки Старовокзальная, соединяется с Лыбедью четырехугольным в сечении порталом\footnote{50°26'40"N 30°29'9"E}, загаженным черным илом.\\

\textbf{Слоник} – народное название советского летнего кафе около Арки в честь воссоединения Украины и Россией, у Филармонии. Назывался так от фонтана в виде слона. На склоне под Слоником была танцплощадка Жаба.\\

\textbf{Смородка} – Смородинский спуск.\\

\textbf{Собачья тропа} – грунтовка вдоль речки Кловицы в глубоком овраге Клова, по нынешней улице Мечникова. Существовала примерно по середину 20 века. Позже Собачкой называли часть Кловского оврага между улицей Мечникова и бульваром Леси Украинки. С последнего в овраг сходила по усаженному большими кленами склону лестница.\\

\textbf{Собачка} 

50°25'07.3"N 30°33'22.8"E

Местность на Зверинце, на задворках Бастионного переулка, примыкающая к забору ботсада. Названа так потому, что местные жители испокон веков выгуливают там собак. Представляет собой крутой склон с двумя террасами, нижняя из которых обрывается бетонной подпорной стеной. На Собачке растут вишни, грецкий орех, а обратившийся в заросший пустырь стадион был обсажен по одной стороне сиренью.

В яру, отделяющему Собачку от частного сектора по улице Мичурина, раньше было нечто вроде одной из дорог от Зверинецкого форта, а теперь в берегах яра доживают свой век погреба местных жителей. В начале нулевых в той части погребов, что была заброшена, обитали беспризорники.

По 21 век и его первое десятилетие через Собачку шло две основные тропы, нижняя (вдоль кромки подпорной стены) и верхняя (вдоль забора ботсада). На 2020 год нижняя тропа совершенно завалена буреломом, исчезла большая площадка посередине тропы, лестничка наверх к бывшей спортивной площадке. Сетчатые секции металлической ограды последней разворованы.\\

\textbf{Совка, речка} - см. мою книгу "Речка Совка и ее притоки".\\

\textbf{Совки, район} - бывшее село и совхоз между Ширмой и аэропортом Жуляны. Совхоз имел земли и в другом месте киева, см. далее.\\

\textbf{Совки, парк}

50°26'28.8"N 30°22'22.4"E

Урочище и одноименный парк (сосновый лес) на Борщаговке. Парк и окрестности это бывшие земли упомянутого выше совхоза Совки.



%\textbf{Совка}, речка – берет начало на улице Кайсарова (бывшей Совской балке), составляясь из многочисленных студеных родников, текущих из склонов. Впадает в Нижний каскад Совских прудов (у перекрестка Краснозведного проспекта, Кировоградской и Кайсарова), а после Нижнего каскада – в Лыбедь (50°24′35.57″N (50.409881), 30°31′20.5″E (30.52236)). Верхний каскад Совских прудов запитан не Совкой, но безымянной речкой, что протекала по местности Кухместровищине, беря начало у Зеленогорской улицы.

%\textbf{Совские пруды} – сложная водная система, состоящая из многих частей. Протекает по местностям, в старину принадлежавших Михайловскому монастырю, Софие, Лавре. На ней стояло множество мельниц. Составляется из ручья Проневщины (устье Зеленогорской улицы), Бычовки, Совки, ручья с улицы Луганской, ручья с Кировоградской, родника с Краснозвездного, и вонючки оттуда же. 

%\textbf{Совская пещера} – пещера в лёссе, на западном склоне Совской балке (улица Кайсарова), примерно на 50°24′15.24″N (50.404234), 30°29′26.77″E (30.490769).

\textbf{Соколий рог} – по плану де Боскета 1753 года, кажется, овраг с Кловским ручьем. А по грамоте 1694 года Братскому монастырю определение такое: «который Рог Соколей имеет себе рубеж: начать от кладовища дорогою до броду на Рутку; назад от того кладовища долиною в ручей верхняго Клова, да ручаем, в Глубокую долину, лесом на гору к великой меже, великою межею, к стародавнему путищу, в рощу лясковую, путищем в Сторожевскую дорогу, по Секирки в Лыбедь». Согласно универсалу Мазепы 1693 года, Соколий рог находился у Крещатой долины и принадлежал Братскому монастырю.\\

\textbf{Солдатская слобода} – местность где нынешний Бастионный переулок, улица Струтинского от поворота на Мичурина до Бастионной, и часть Бастионной улицы (там где ПТУ). В 19 веке тут жили солдаты из Зверинецкого форта, что находился на территории ботсада. После взрыва пороховых складов в Зверинецком форте в 1918 году была практически разрушена, и частый сектор возродился в некоторых пределах в верховье Струтинского позже.\\

\textbf{Солдатская слобода} – еще одна таковая, находилась между ручьем Скоморох, Лукьяновкой и улицей Артёма. В слободку входят, например, окрестности Астрономической обсерватории между улицами Бульварно-Кудрявской и Гоголевской.

На юго-восток от обсерватории, по адресу Гоголевская 28 есть за забором небольшие старинные домики в как бы парке. Это бывшая усадьба художника Владимира Донатовича Орловского, построена она по проекту архитектора Николаева. Участок земли городская Дума предоставила художнику в 1889 году, а 1892-му Орловский уже поселился в построенном доме, где и прожил до конца жизни (умер в 1914).

А до того, еще по середину 19 века, тут находилось кладбище. В советское время в усадьбе поместился детский сад, а ныне, официально, детский санаторий «Салют» при КГГА.\\

\textbf{Соломенка, Соломка, Солома} – возникла во второй половине 19 века на правом берегу Лыбеди как хутор Соломенки у низовий Богданова яра, эдак где сейчас улицы Брюллова и Кирпы. Позже название распространилось наверх к Тесным улицам, «поглотив» их. 

Я прослеживаю «хутор Соломенки» с карты по крайней мере 1865 года. А уже на плане 1879 года «предместье Верхняя Соломенка» занимает Кучмин Яр и правобережную Паньковщину, а «Нижняя Соломенка» показана между Лыбедью и железной дорогой, параллельно Жилянской улице в пределах улиц Паньковской и Безаковской (Коминтерна, ведет к Вокзалу). Короче говоря Нижняя Соломенка того времени это окрестности нынешних улиц Семьи Праховых и Лыбедской.

На плане 1880 года обе Соломенки отмечены как «деревня Соломенка», хотя там была церковь, а поселение с церковью должно считаться не деревней, но селом.\\

\textbf{Соломенское кладбище старое}

50°26'15.6"N 30°29'20.4"E

Занимало в 19 веке место возле нынешней почты по адресу Петрозаводская, 2. Асфальт и здания.

Существовало с 1832 года и было упразднено из-за строительства железной дороги.
\\

\textbf{Соломенское кладбище}

50°25'43.9"N 30°28'16.0"E

Расположено в самом сердце района, на запад от верховий Кучмина яра. Возникло в 1880-х взамен старого Соломенского кладбища, закрыто в 1959. В 1980х три участка кладбища отвели под застройку.

В середине 19 века на месте кладбища был огород кантонистов.

На Соломенской кладбище хоронили местных жителей, преподавателей и учеников Кадетского корпуса, затем умерших раненых Первой мировой. На кладбище погребены летчики, погибкие в авиакатастрофе на стыке Демиевки и Ширмы в 1957 году (подробнее см. мою книгу "Речка Совка"). Также здесь похоронены профессора Духовной академии Петр Кудрявцев и Василий Экземплярский, и на "кадетском" участке - архитектор Василий Осьмак\footnote{По его проектам построены: стадион "Динамо", Троицкий народный театр (театр оперетты), здания в Клиническом городке, библиотека Университета (гуманитарный корпус), доходный дом на Чкалова, 44, Здание Высших женских курсов и Технического общества (Чкалова, 55 и 55Б), центральная научная библиотека, Главпочтамт (старый, взорван в 1941-м).}\\


\textbf{Солоная горка} 

50°26'08.9"N 30°33'45.6"E

Слобода под Лаврой, на склонах Днепра, показанное на плане Ушакова 1695 года. Примерные координаты вычислены мною по указанию на карте – 660 сажень от Печерского перевоза (в устье Наводничей) до Солоной горке, и по параллельности с лаврской трапезной. По Солоной горке, судя по карте, стояли жилые домики в садах. Сейчас внизу урочища – довольно плоское место, поворот с Днепровского спуска на эстакаду к станции метро Днепр. Слобода упомянута и в генеральном описании Левобережной Украины 1765-69 годов.\\

\textbf{Спасский овраг, взвоз} - см. Красница, Николаевский спуск, Панкратьевский ручей.\\

\textbf{Сталинка} – название Демиевки и Ширмы с 1930-х годов, в ходу у старожилов и поныне.\\

%\textbf{Старая полянка}

\textbf{Старая слобода} - на стыке 18-19 веков, лежала над Аскольдовой могилой, отделяясь от Новой слободы Никольской улице. Можно сказать, что Старая слобода - окрестности станции метро Арсенальная, а южнее их была Новая слобода.\\

\textbf{Старая Антифеевская полянка, Антифеевка} – окрестности улицы Старая Поляна (прежняя Антифеевская) на Татарке, попросту говоря, а названа она по урочищу. Была еще Новая полянка. И там и там селились самосёлы. 

В начале 20 века ходило другое название Старой поляны – Антифеевка, по фамилии околоточного надзирателя Антифеева, которому самоселы давали на лапу, чтобы он их не трогал. 

В обиходе Старая поляна – часто просто Полянка. Южная часть улицы исчезла, занята ныне коленом Лукьяновской, поэтому нумерация домов на Старой Поляне начинается, на 2020 год, 38-м номером. Под застройку и перепланировку в 1980-х попал и засаженный каштанами и липами сквер\footnote{50°27'56.2"N 30°29'33.0"E}, бывший около двадцатых номеров – сейчас это Лукьяновская, 1 – и до девятиэтажки Лукьяновская, 7-А.\\

\textbf{Стародум} – «крепостца», выстроенная  над Крещатицким источником в 1713-15 годах.\\

\textbf{Стекляшка} – павильон из стали и стекла на углу около «Темпа», со стороны где улица Киквидзе подходит к Печерскому мосту. В этом магазин продавались разные замороженные продукты.\\

\textbf{Стефанеч} – монастырь где-то на Клове, назван так от епископа Стефана, бывшего прежде игуменом Печерского монастыря. В 1108 году по летописи, в монастыре был завершен купол церкви святые Богородицы Влахерны на Клове, заложенная Стефаном. На то время игумном монастыря был, вероятно, уже Петр, указанный игумном за 1115 году.\\

\textbf{Сторожовская дорога} – старинная дорога над Лыбедью, через Вету в Дикие поля.\\

\textbf{Стрекаловка} – местность на Соломенке в окрестностях Стадионной улицы (между Островского, Стадионной и Богдановской). Названа по усадьбе дворянина М. П. Стрекалова.\\

\textbf{Стрельбище}

50°28'26.6"N 30°29'06.9"E

Плоский уступ, заросший травой пустырь в местности Загоровке, похожий на заросшую спортлощадку. Окрестности усеяны осколками черных тарелочек, по которым стреляли. Стрельбище тут было при лежащем выше  гости\-нично-спортивном комплексе «Авангард», построенном к Олимпиаде 1980 года. На Стрельбище можно пройти свернув в переулок на другой стороне от Нагорной 20, а затем, не доходя до автостоянки, направо в дебри.\\

\textbf{Стрелники} – поле возле Преорки. Указано в земельных грамотах начала 18 века.\\

%1923.png, требует уточнения
\textbf{Сукачев овраг или Сухачев яр} – на старых картах так обозначен здоровенный яр на юго-запад от нынешних Ветряных гор, отделяющий оные от урочища Выгоды (в определенных пределах это нынешний Виноградарь), которое показано было на запад и на юго-запад по отношению к Ветряным горам. По карте 1902 года показан как лежащий между улицами Жеребеевской и Белицкой.

По сопоставлению с картой середины 19 века, начинался близ улицы генерала Грекова, на ее пересечении с улицей семьей Кристеров, у пруда в урочище Илейнови (?) Долинки.

Затем вдоль генерала Грекова, Сукачев яр следует на восток, между Данченко и Новомостицкой (там где ГСК), и затем по ходу нынешней Брестской улицы.

По яру протекает ручей Куриный Брод.

Некоторые краеведы, не разобрав подпись на картах, именуют яр "Суклеев яр", в то время как яр то ли Сухачев, то ли Сукачев.\\

\textbf{Сулимовка, Сулимы} - приют для бедных, со школой и столовой, основанный Киевским благотворительным обществом, находился на Университетском спуске (ныне ул. Круглоуниверситетская), можно сказать, что на Крестах.\\

\textbf{Супер-8} – парк аттракционов, построенный в 1970-е, на склонах Днепра, выше Зеленого Театра. Там были невысокие американские горы, качели-лодочки, комната смеха, небольшое колесо обозрения.\\

\textbf{Сырец} – деревня в середине 18 века, названа по речке. Граничила с Преваркой и Куреневщиной.\\

\textbf{Сырец} – одна из крупнейших малых рек Киева, подробнее см. «Ересь о Киеве».