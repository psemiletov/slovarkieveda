\chapter*{Е}
\addcontentsline{toc}{chapter}{Е}

\textbf{Евбаз} – Еврейский, или Галицкий базар, что был на площади Победы, там где цирк. На Евбазе стояла настоящая Железная церковь. В конце 1940-х общенародную барахолку ликвидировали, снеся сараи складов и ларьки. В 1952 году освободившуюся Галицкую площадь переименовали в площадь Победы.

В восьмидесятые-девяностые Евбазом называли уже небольшую толкучку около трамвайной остановки рядом с универмагом Украина.\\

\textbf{Еврейское кладбище} – кроме того, что с 1798 года было на Зверинце\footnote{Т.н. Старое еврейское кладбище, на его месте ныне участок вьющихся растений в ботсаду. В 1895 году объявлено закрытым.}, существовало еще одно иудейское кладбище, и находилось оно не по месту нынешней телевышки (там было Братское), а почти рядом, по другую сторону улицы Мельникова, правее спорткомплекса «Авангард» (построен на месте мусульманского и караимского кладбищ). От еврейского кладбища осталась его бывшая контора, двухэтажный дом по адресу Мельникова, 44. Здание телецентра рядом – Карандаш – стоит на месте еврейского же кладбища.\\

\textbf{Евсейкова долина} – урочище, упоминаемое в земельных документах 16-18 веков. В грамоте Жикгимонта I, разрешающей восстановить Михайловский Златоверхий монастырь, 15 марта 1523 года, описываются границы принадлежащей монастырю земли в Киеве:

\begin{quotation}
и земли к тому манастырю мает держати по давному, как перед тым бывало, по самый вал, и по Лядскии ворота, и по Евсийкову долину, по старую дорогу, по Михайловский ввоз.
\end{quotation}

В высочайшем Е. И. В. указе из прав. сената от 13 декабря 745 г. за № 9490, написано: 

\begin{quotation}
в прав. сенате Киево-золотоверхо-михайловского мн-ря архимандрит Си\-львестр Думницкий с братиею бил челом, объявляя, что на данную от доброхотнодателей, лежачую за старокиевскою крепостию, в даче по урочищам от долины Евсейковой, землю со всем, которая сошлась того места к берегу реки Почайны, крещацким взвозом в 1700 году блаженныя и вечнодостойныя памяти государь император Петр Великий, жалованною грамотою, по королевским привилегиям, в вечное и ненарушимое владение мн-рю Михайловскому всемилостивейше ствердил и укрепил. 

и по оной грамоте для варения в мн-рь меду, пив и прочаго, провар в 1745 году на оной крещатицкой земле начато строить, ибо в мн-ре провапа, воскобойни и винокурни за немалым в том мн-ре утеснением и за хоромным деревянными многим строением, а наипаче за великим от огня страхом содержать опасно [...]
\end{quotation}

Вероятно, Евсейкова долина это давнее название удолья, где лежит улица Крещатик.\\
