\chapter*{Е}
\addcontentsline{toc}{chapter}{Е}

\textbf{Евбаз} – Еврейский, или Галицкий базар, что был на площади Победы, там где цирк. На Евбазе стояла настоящая Железная церковь. В конце 1940-х общенародную барахолку ликвидировали, снеся сараи складов и ларьки. В 1952 году освободившуюся Галицкую площадь переименовали в площадь Победы.

В восьмидесятые-девяностые Евбазом называли уже небольшую толкучку вдоль трамвайной остановки рядом с универмагом Украина.\\

\medskip

\textbf{Еврейское кладбище} – кроме того, что с 1798 года было на Зверинце\footnote{Т.н. Старое еврейское кладбище, на его месте ныне участок вьющихся растений в ботсаду. В 1895 году объявлено закрытым.}, существовало еще одно иудейское кладбище, и находилось оно не по месту нынешней телевышки (там было Братское), а почти рядом, по другую сторону улицы Мельникова, правее спорткомплекса «Авангард» (построен на месте мусульманского и караимского кладбищ). От еврейского кладбища осталась его бывшая контора, двухэтажный дом по адресу Мельникова, 44. Здание телецентра рядом – стоит на месте еврейского же кладбища.\\

\medskip


\textbf{Евсейкова долина} – урочище, упоминаемое в земельных документах 16-18 веков. В грамоте Жикгимонта I, разрешающей восстановить Михайловский Златоверхий монастырь, 15 марта 1523 года, описываются границы принадлежащей монастырю земли в Киеве:

\begin{quotation}
и земли к тому манастырю мает держати по давному, как перед тым бывало, по самый вал, и по Лядскии ворота, и по Евсийкову долину, по старую дорогу, по Михайловский ввоз.
\end{quotation}

В высочайшем Е. И. В. указе из прав. сената от 13 декабря 745 г. за № 9490, написано: 

\begin{quotation}
в прав. сенате Киево-золотоверхо-михайловского мн-ря архимандрит Си\-львестр Думницкий с братиею бил челом, объявляя, что на данную от доброхотнодателей, лежачую за старокиевскою крепостию, в даче по урочищам от долины Евсейковой, землю со всем, которая сошлась того места к берегу реки Почайны, крещацким взвозом в 1700 году блаженныя и вечнодостойныя памяти государь император Петр Великий, жалованною грамотою, по королевским привилегиям, в вечное и ненарушимое владение мн-рю Михайловскому всемилостивейше ствердил и укрепил. 

и по оной грамоте для варения в мн-рь меду, пив и прочаго, провар в 1745 году на оной крещатицкой земле начато строить, ибо в мн-ре провара, воскобойни и винокурни за немалым в том мн-ре утеснением и за хоромным деревянными многим строением, а наипаче за великим от огня страхом содержать опасно [...]
\end{quotation}

Вероятно, Евсейкова долина это давнее название удолья, где лежит улица Крещатик.\\

\medskip


\textbf{Ермаков Евгений Федорович} 

Известный киевский епархиальный архитектор, младший брат Василия Ермакова – Михаила, митрополита Киевского, Галицкого и всея Украины.

Родился  21 января 1868 года в семье подпоручика 132-го пехотного Бендерского полка Федора Андреевича Ермакова в Могилеве-Подольском. Окончил Киевское реальное училище, в 1887-1892 учился в Санкт-Петербурге в Институте гражданских инженеров, получил звание по I разряду.

Затем служил в Киеве инженером при Городской управе, с 1898 года – архитектором Киевской епархии, с 1899 еще и архитектором Киево-Печерской лавры.

Во время Первой мировой был мобилизован в при военном ведомстве в Золотоноше и Прилуках по его проектам построили госпиталя.

Вернулся в Киев. При гетмане Скоропадском жил по адресу Владимирская, 20. Эмигрировал, жил в Югославии, умер в 1946 году в городе Нови-Сад.

Проекты Ермакова в Киеве, в хронологическом порядке:

Кельи (корпус № 2) Флоровского монастыря (1895).

Беседка на террасе Андреевского спуска (1897-1898).

Кокоревская беседка на Владимирской горке (1898, план и надзор).

Корпус настоятеля Выдубицкого монастыря (1898, перестройка).

Духовная семинария на Вознесенском спуске (1899-1901, с использованием проекта синодального архитектора Е. Морозова).

Многоэтажный дом на Софийской площади по ул. Владимирской, 20 (1898-1899, надстроен в 1904-1905).

Второе женское училище духовного ведомства на ул. Десятинной № 4-6 (1900, надстройка 4-го этажа).

Покровская церковь по ул. Мостицкой № 18 (1900-1906, заканчивал строительство Н. Казанский, росписи И. Ижакевича).

Особняк М. Бродской на углу ул. Институтской, 26 и Виноградного переулка (1900-1912).

Кельи Свято-Покровской пустыни в Голосеево (конец 19-го века).

Православное религиозно-просветительное общество на ул. Большой Житомирской № 9 (1902-1903).

Пристройка ризницы и библиотеки к колокольне Выдубицкого монастыря (1902).

Жилой дом в Рыльском переулке, 3 (1903).

Доходный дом на Софийской площади / ул. Владимирской, 22 (1903-1904, с участием А. Вербицкого).

Церковно-учительская школа / семинария на углу улиц Пугачева, 12/2 и Багговутовской (1903).

Расширение помещений бани Братского Богоявленского монастыря (1904).

Шатровая часовня с усыпальницей генерала Ф. Пышенкова (1905) на углу улиц Овручской и Багговутовской, не сохранилась.

Жилой дом на ул. Трехсвятительской, 4 (1906).

Церковь Святого Князя Владимира на Новостроенской площади (район дворца Украина, не сохранилась) (1906).

Ильинская церковь на углу ул. Безаковской, 25 и Жилянской (1908-1914), не сохранилась.

Церковь святого Симеона Столпника, в Петропавловской Борщаговке (1908-1909).

Больница и часовня (не сохранились) Киевского благотворительного общества на ул. 
Большой Васильковской, 104 (1909).

Киево-Лыбедская Троицкая церковь (начало 19-го века), не была достроена и не сохранилась.

Хирургический корпус Покровского монастыря (1910-1911).

Церковь святой Елизаветы на Трухановом 
острове (1910-1911), не сохранилась.

Церковь Живоносного Источника Пресвятой Богородицы Свято-Покровской пустыни в Голосеево (1910-1912).

Собор святого Пантелеймона в Феофании (1912).

По проектам Евгения Ермакова в Киево-Печерской лавре были построены:

Аптека (корпус № 24) (1902-1903).

Магазин икон у южных ворот (корпус № 35) (1902-1903).

Корпус певчих митрополичьего хора (корпус № 6) (1902-1904).

Переплетная мастерская (корпус № 15) (1903-1904).

Благовещенская митрополичья церковь (корпус № 86) (1905)

Ограда водосвятной часовни - кивория (корпус № 95) (1906).

Отель (корпус № 19) (1906-1908).

Библиотека Флавиана (корпус № 5) (1908-1909).

Проскурня (просвирня) (корпус № 11) (1913).

Насосная станция (корпус № 47) (1913).

Лаврская больница с церковью (корпус № 111), 1911-1914.
