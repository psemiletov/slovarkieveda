\chapter*{Ч}
\addcontentsline{toc}{chapter}{Ч}

\textbf{Чайка, мыс}

50°25'16.6"N 30°33'59.2"E

Северо-восточный отрог холма Зверинца, там где в ботсаду воссоздан Всеволодов «Красный двор». Место «двора» предполагаемое, однако возможное. 

В ходе раскопок, самые ранние находки (осколки керамики, кремневые орудия труда, глиняные грузила) здесь  археологи датируют концом 3-го тысячелетия до нашей эры, позднетрипольской культурой.

Несколько столетий до нашей эры тут было языческое кладбище, в 13-14 веках – обжитая местность, в 16-17 веках – гончарная слобода, почему-то заброшенная вместе с печами и готовой продукцией.

Название происходит, по одной версии, от дачи профессора-уролога Андроника Архиповича Чайки, состоящей из 12 комнат. При усадьбе был сад более гектара. Во время фашисткой оккупации немцы хотели отдать этот дом под жилье одного из приближенных фюрера, но потом дачу разобрали то ли немцы, то ли уже наши. По другому мнению, название возникло от фамилии капитана красноармейской батареи, стоявшей тут во время Великой Отечественной войны.

На картах того времени по месту «красного двора» видны постройки дачи\footnote{50°25'17.5"N 30°33'59.3"E} (там где сейчас дикий спуск по террасам) и неких возможно сараев.

Прямо под кромкой обрыва есть пещерка, размером такая, что поместится сидя-лежа один человек. Забраться туда без веревки можно только снизу, обходным путем.

Внизу под обрывом можно найти обломки плинфы – кирпича великокняжеских времен, где глина смешана с камешками.

Очертания оконечности мыса имеют строгие, рукотворные очертания, однако таковые учитывая вхождение мыса в состав Зверинецкого форта, трудно сказать, когда возникли эти очертания.\\ 

\medskip

\textbf{Чапаевский}, кинотеатр – в 1960-е так называли кинотеатр им. Чапаева, на Львовской площади (Большая Житомирская, 40). Построен в 1913 под названием «Лира», в 1937 переименован в им. Чапаева, ныне снова «Лира».\\

\medskip

\textbf{Чачин, Чичин} остров – находился в районе Оболони. Разросшийся ныне залив Собачье горло раньше был узким проливом на этом острове. Чачин начинался на уровне устья Десны и завершался около северной части теперешнего залива Оболонь. Улица Приречная и идущий параллельно ей отрезок Героев Сталинграда – это был Чачин остров. Озеро, которое сейчас обозначено на картах как Белое – не исконное бело, а часть пролива, отделявшего остров от материка. Белое же озеро историческое было севернее, там где сейчас залив Верблюд.\\

\medskip

\textbf{Черепанова гора}

Она же Даронова гора. Гора на восток, юго-восток от стадиона Олимпийский. Названа так в 19 веке по усадьбе губернатора Павла Сидоровича Черепанова. См. Даронова гора. Добраться туда можно по Госпитальному переулку.

Гора была существенно обрезана при устроении Республиканского стадиона с южной и восточной сторон, а наверху претерпела изменения из-за Госпитального укрепления Киевской крепости. Склон горы был там, где сейчас «зона гостеприимства», примыкающая к стадиону с востока. Крутые склоны доходили и до современного здания по адресу улица Физкультуры, 1.

В середине 19 века на Черепановой горе была слобода Лыски.\\

\medskip

\textbf{Черкеня} – гора, названия которой я почти не встречал. По времени появления в известных мне источниках, сначала она упоминается в книге Брайчевского «Когда и как возник Киев»:

\begin{quotation}
В районе Сырца также имелись поселения зарубинецкой культуры, одно из них было расположено на горе Черкеня, возвышающейся над долиной р. Сырец. В 1957 году здесь была найдена типичная керамика зарубинецкого типа.
\end{quotation}

И дана ссылка на «сообщение В.Д. Дяденко», коему в сносках ранее Брайчевский приносил искреннюю благодарность.

Про Черкеню в сети нашел сведения, что это в районе Шполянской улицы (см. Шполянка). Пачкова в сборнике «Стародавний Киев» 1975 года упоминает Черкеню и поясняет «отрог правого берега р. Сырец».

Когда возникло название Черкеня и кто его употреблял, выяснить я не смог. Если таки сопоставлять ее с местом улицы Шполянская, на начало 20 века там была улочка или дорога, от переулка Казарменного, выходящая к склону над Сырцом, где находился хутор Козинского и его же кирпичный завод, а прежде был кирпичный завод Булышкина. Разумеется, при добыче глины там образовался рай для археологических раскопок. Современная улица Шполянская, как и окрестности, застроена частным сектором, остался лишь заросший березами да прочими деревьями склон от задворков улицы Фруктовой и вниз до карьерного озера на речке Сырце. Озеро и окружающая территория считаются ландшафтным заказником местного значения «Зеленое озеро».

К слову, сама улица Шполянская как дорога к некому домику показана в совершенно современных пределах на карте середины 19 века, там же западнее подписано «урочище Копылово», а еще западнее «урочище Граевщина».\\

\medskip

\textbf{Черная гора} – холм между бульваром Дружбы Народов, Железнодорожным шоссе и улицей Киквидзе (где далее начинается Бусова гора). 

В первой половине 19 веке на берегах двух яров стояли бараки обычных солдат и саперов. По ярам текли ручьи, впадающие в Лыбедь. Под конец 19 века местность пустеет, остаются и множатся лишь саперные бараки в восточной части горы, примыкая к правому берегу речки Бусловки. Об их следе до сих пор свидетельствует Военный переулок, а севернее его в советское время была военная часть, ныне застроенная ЖК. Военная часть возникла на месте Саперных дач – садов за огородов. В то же время по другую сторону Лыбеди, напротив Саперных бараков, была Саперная Слободка, чье название сохраняется за районом до сих пор.

На начало 20 века Саперные бараки на Черной горе стали уже Саперным лагерем, но по виду ничуть не изменились – десятки бараков, выстроенных рядами. От Боенской улицы туда, через дикую местность, шла грунтовка. 

Потом бац – к 1930 году на карте появляется «поселок Черный яр» в районе нынешних улиц Черногорской, Тихой, Верхнегорской, параллельного им участка Железнодорожного шоссе, и низа улицы Товарной. Это частные домики среди садов. Современная Черногорская, вертикальная ее часть, вобрала в себя прежнюю Нежинскую улицу, а западная часть Черногорской была улицей Средней. На 1935 год сетка улиц вот этого низовья горы частично совпадает с современной. Товарная начиналась тут же, она росла снизу вверх. 

Постепенно частный сектор распространился на всю Черную гору, кроме военной части, ну и гаражных кооперативов. Сейчас восточная горы половина, где и была в/ч, покрыта высотными ЖК.

У северо-западного края горы находится карьерное озеро Глинка.

%Согласно карте 1845 года, на запад от наибольшего пересекающего гору яра (ныне занят гаражами), по склону стояли бараки 2 и 3 полка, а на восток – бараки сапёрные. По яру стекал ручей, приток Лыбеди. В низовье яра через ручей был мостик. 

Гору пересекает несколько яров, про них читайте в статье о Черном яре.\\

\medskip

\textbf{Черная грязь} – урочище, прежде даже одноименная улица у подножия Замковой горы. Лежит между горой, Андреевским спуском, Боричевым током и улицей Флоровской. Но всё же ближе к горе. На 2017 год там затеяно-остановлено строительство, там же – бывший корпус швейной фабрики «Юность». В урочище очень высок уровень грунтовых вод. В незапамятные времена тут был колодец Кошинка.\\

\medskip

\textbf{Чернечье} озеро – оно же прежнее Иорданское (не путать с современным) или Черное, питалось от по крайней мере Иорданского ручья. В 20 веке это озеро вытеснила железнодорожная станция Киев-Петровка с одной стороны, а с другой увеличиваемая на запад Гавань. К 1940-м от озера остались рожки да ножки – несколько обмельчавших водоемов в окрестностях того выхода из станции метро Петровка, что ближе к книжному рынку. Прежде озеро впадало в южную часть Кирилловского озера.\\ 

\medskip

\textbf{Черный яр} 

Некоторые его точки:

50.4119765142537, 30.541325291312766

50.41001304305175, 30.542078289311913

Местность на Черной горе. Конечно же, там поселился гаражный кооператив. Не путать с гаражами «Глубокий яр», те западнее. Черный яр начинался около улицы Чешской. По нему протекал ручей, приток Лыбеди. Он и сейчас бежит в подземном коллекторе и выходит к железнодорожному шоссе примерно тут:

50.40606980073067, 30.538773434021046

в переулке параллельном улице Товарной.\\

\medskip

\textbf{Черный яр} – урочище и одноименная улица на Щекавице, примыкала к Житнеторжской площади примерно там, где теперь от Нижнего Вала отделяется улица Олеговская. 

Улица Черный Яр шла параллельно Глубочицкой, между нею и Олеговской, и соединялась с Лукьяновской. Самое низовье современной Олеговской было улицей Черный яр. В некоторое время, до революции, из-за мрачного названия жители попросили переименовать улицу, и ей дали имя Мирной, присоединив к ней короткую, уже существовавшую улицу Мирную, что лежала несколько восточнее. Так оба отрезка, разделенные некоторым пространством, именовались Мирной улицей. На ней стояли кирпичные дома в несколько этажей.

На 2016 год старые дома на Мирной снесены, виднеются их остовы среди пустырей. До революции Черный яр был злачным местом с постоянными убийствами, грабежами и тому подобным.\\

\medskip

\textbf{Чертова долина} – на плане 1752 года так обозначен буерак вероятно по месту нынешней улицы Соляной.\\

\medskip

\textbf{Чикирдин ввоз} – подъем с Подола на Щекавицу, с нее далее по Лысой (ныне слывущей как Юрковица) к Белгородке.\\

\medskip

%\textbf{Чмелёв яр} – окрестности нынешней улицы Старая поляна, часть этой улицы раньше называлась улица Чмелёв яр. Последняя затрагивала также и урочище Новую поляну. Попробую дополнить в следующей заметке.\\

%\medskip

\textbf{Чмелёв яр} – яр и одноименная старая улица в окрестностях Старой и Новой полян. Крутая улица большей частью не существует, она исчезла при перепланировании улицы Лукьяновской, впрочем остался отрезок как часть улицы Старая поляна и как часть Лукьяновской (там где она идет вдоль домов 7А, 9, 11.

Окрестности Чмелёва яра назывались в народе Чмелёвкой.

Улица Чмелёв яр начиналась, на начало 20 века, у перекрестка Большой Юрковской и Саксонским яром (за сквером на Старой поляне, сквер более не существует), шла вниз к Лукьяновской, пересекала ее, и спускалась далее до Глубочицкой.\\

%Улица Чмелёв яр начиналась за сквером на Старой поляне (не существует) и спускалась к Глубочицкой.\\

\newpage

\textbf{Чоколовка} – название происходит от купеческого рода Чоколовых. У Чоколовых был также спиртовой завод на Глубочице, с прудом, где позже возник дрожжевой завод, а в 2016 году эта котловина между Глубочицкой улицей и Татарской застроена жилым комплексом.

Именно по предложению думе 1893 года члена городской управы М. И. Чоколова, в 1899 году был создан дачный поселок в Пуще-Водицы – местность в 220 десятин между речками Котуркой и Горенкой разделили на 600 участков, между которыми проложили 7 улиц, пересекаемых 16 линиями. Участки сдали в аренду.\\

\medskip


\textbf{Чугуевка} – жилой райончик из двух-пяти этажных домов, внутри промзоны Грушек. Название от переулка Чугуевского, в северной половине коего и находится Чугуевка.