\chapter*{Ф}
\addcontentsline{toc}{chapter}{Ф}

\textbf{Фрометовка} – окрестности улицы Фрометовской и Фрометовского спуска.

Занимает склон горы над нижним прудом нижнего каскада Совских прудов. И улица, и спуск были заняты частным сектором, сейчас от былой застройки именно улицы остались рожки да ножки. Внизу спуска некогда существовал мост через речку Совку, теперь он тоже есть, но в виде переезда над трубами, то есть не столь явный. Название от фамилии Фроммет. Подробнее написано в моей книге «Речка Совка».\\

\medskip

\textbf{Фузиковщина} – хутор на Нивках, изначально был в районе улиц Гончарова, Невская и Александровская. На начало 21 века это частный сектор, где кое-где еще можно видеть хаты из дранки.

Невская улица потому, что вела к церкви Александра Невского на соседнем к юго-востоку «нивском» хуторе – Галаганах, что лежит по другую сторону проспекта Победы. С Фузиковщины и вырос современный район Нивки.

Хутор в середине 19 века принадлежал братьям Виктору и Семену Фузикам, жителям села Беличи, владельцам близлежащего кирпичного завода. В пределах земель хутора возникла дача «Нивка» и ферма «Нивка», позже – но еще до революции – было построено трамвайное депо, питомник сыновей Вильгельма Кристера, электроподстанция и парк «Нивки», где был ресторан «Эльдорадо» – нынешний парк «Нивки» и есть продолжатель парка старинного.

%м карту 1886