\chapter*{Ю}
\addcontentsline{toc}{chapter}{Ю}

%\textbf{Ю зе фа} – выражение для счета перед играми вроде «камень, ножницы, бумага». То есть, дети считают «ю, зе, фа» и потом выкидывают свои фигуры на пальцах. Я слышал также, на Лесном, «жу зе фа». В Челябинске для этого же считают «цу е фа». В Киеве на Бастионной и вообще возможно на всем правом берегу вместо счета говорили «камень, ножницы, бумага» и после показывали руки.\\

\textbf{Юный техник}, магазин на первом этаже в доме по бульвару Леси Украинки, 24, на стороне самого бульвара. С другой стороны угла дома, на улице Щорса, был валютный магазин Каштан, а дальше по бульвару Леси, за Юным техником – продуктовый, с классной мозаикой у входа. В Юном же технике продавался всякий хлам для мастеровитых людей – предполагалось, что это некие юные техники, но в магазине постоянно отоваривались разные дядьки. Товар был такой – фанерные доски, рейки, обрезки железных листов, словом отходы промышленного производства, а также пачки авиационной резины и прочая мелочь.

Примечательно, что в совсем другом месте, около восточной стороны Арсенала, напротив дома Кловский спуск 8, есть остановка «Юный техник» – возможно, там тоже был одноименный магазин.\\

\medskip

\textbf{Юрковица}, ручей – подробнее о нем читайте в «Ереси о Киеве». Основной исток был в пруду, что располагался еще в 19 веке в овраге на задворках нынешней военной части по адресу Нижнеюрковская, 8-А (на ее месте был склон горы, позже срытый кирпичным заводом). Местность около этого истока называется Романовщина, от владельца кирпичного завода Романовского (см. мою книгу «Киевский кирпич»). Приняв еще родники, Юрковица протекает под Нижнеюрковской улицей.
