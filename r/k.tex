\chapter*{К}
\addcontentsline{toc}{chapter}{К}

\textbf{Кадетская роща} – не путать с улицей Кадетский гай в Турецком городке. Непонятно, почему так назвали улицу, лежащую много в стороне от Кадетской рощи.

Кадетская роща, вернее её остатки, лежат на склоне горы вдоль Лыбеди, между Чоколовским бульваром, Уманской улицей и улицей Железнодорожной (а затем рельсами Оборотного парка Киева-пассажирского). На карте Шуберта роща обозначена как Кадетский бор.

Официально ее именуют сейчас «парк Спутник».

Сейчас размер Кадетской рощи составляет, на глазок, 15 процентов от ее площади по карте 1842 года, когда роща покрывала нынешний Первомайский массив и часть Кардач. Но и современные остатки рощи сопоставимы с ботсадом имени Фомина, что в центре. И эти остатки теснятся гаражными кооперативами, культовыми сооружениями и АЗС.

Название свое роща получила от находящегося рядом Кадетского корпуса (сейчас это Воздухофлотский проспект, 6), от коего роща ныне лежит в полукилометре. В отличие от бытующего мнения, кто воинское учреждение завели там с 1847 года, его прекрасно видно например на плане Киева 1842 года, на краю рощи, у берега Лыбеди.

Однако история рощи древнее. Закревский в 19 веке писал:

\begin{quotation}
...на берегу топкой и болотистой Лыбеди, возле пруда, Киевский архиепископ Варлаам Ванатович (1722-1730) выстроил летний дом и вокруг него насадил березовую рощу. Усадьбу эту назвали \textit{Шулявщиною}. Произошло ли это название со времени основания упоминаемого архиерейского загородного дома с рощей, или оно существовало и прежде – этого мы не знаем. Около 1763 г. митрополит Арсений Могилянский распространил эту прекрасную рощу и развел регулярный сад, а при доме устроил церковь во имя св. Бориса и Глеба. Тут же было поселено несколько семейств служителей, приписных к Софийскому тогда монастырю. Неизвестно, с которого именно года, но давно уже Киевляне имеют обыкновение в этой роще собираться первого мая на гуляние.

В начале двадцатых годов текущего столетия в Шулявщине архиерейский дом с пристройками приходил в ветхость, а в 1847 г. место это было уступлено казне\footnote{За 30 000 рублей серебром.} [...] Между тем в Шулявщине возведено огромное и величественное здание для Кадетского корпуса или Военного училища в четыре этажа.
\end{quotation}

%Отмечу, что на середину 19 века, Шулявщиной именовалась деревня неподалеку от нынешней Кадетской рощи. Местность той деревни сейчас считается Шулявкой, а ее историческая область это скорее окрестности Шулявского кладбища\footnote{50°26'40"N 30°26'5"E}.

В 19 веке, первомайские гуляния в Кадетской роще включали в себя разные базары, театры, выставки, танцы. Приходили и съезжались люди всех сословий, устраивались благотворительные аукционы, где например за стакан лимонада платили сто рублей.

В Кадетской роще есть бетонная ротонда в конце Ленинской Аллеи, заложенной в 1969 году, а также здание летнего кинотеатра «Березка».\\

\textbf{Казачка} – окрестности улицы Казачьей.\\

\textbf{Казенные дачи} – с 1860-х так именовалась дачная местность между улицами Гарматной, Индустриальной и проспектом Победы. Там была известная дача Сан-Суси, принадлежащая А. Шедель. Улицами служили дачные «линии». В 1880-х, после строительства рядом завода Гретеля и Криванека (позже «Большевик») Казенные дачи стали частью рабочей окраины, жилым районом. Нынешняя улица Смоленская – бывшая 2-я Дачная. Там, на территории современного нам завода порционных автоматов, была первая в Киеве фабрика граммофонных пластинок Г. И. Индржишека.\\

\textbf{Каменка}

Исток (открытого русла): 

50°28'32.5"N 30°24'57.4"E

Левый приток Сырца, исток примерно возле улицы Саратовской, затем уход в коллектор под Стеценко\footnote{50°28'37.3"N 30°25'11.8"E}, затем выход в пруды на стыке Стеценко и Щусева, против станции метро «Сырец». Верхняя половина ручья течет по ложбине оврага, среди тенистой рощи, в открытом, хотя бетонированном русле. По улице Стеценко Каменка принимает в себя воды многочисленных родников, сочащихся из крутых склонов. Археологи ищут там следы поселения эпохи неолита. Я искал и не нашел.

Название Каменки прослеживается мною на картах с середины 18 века по середину 19-го. В 1865 году ручей протекал мимо хутора Гурского, на картах 1886 и 1912 годов хуторов показано больше, и ручей приходится на хутор Плюмина, а выше (восточнее) отмечены хутора Гладика, Семенова, Хмуржинского, Гурского. С противоположного берега Сырца, в окрестностях устья Каменки, были армейский лагерь и хутор Петерсона с кирпичным заводом оного же Петерсона.

На карте 1746 года на Каменке показаны два пруда с мельницами – один близ устья речки в речке Сырце, другой несколько выше по течению Каменки. Также, судя по этой карте, у Каменки было два истока.

Местные жители (расположенного выше плато, окрестностей улицы Саратовской) называют Каменку Лыбедью и Вонючкой. Последнее несправедливо, ибо вода пахнет обычно.\\

\textbf{Каменный затон} – согласно Закревскому, «в южной части Печерска, близ урочища Каменного затона», был воспитательный дом – «изрядной величины деревянный, на каменном фундаменте, дом сей основан в конце прошлого столетия; теперь сломан. Первоначально он был устроен для принятия бедных сирот и незаконнорожденных детей».\\

\textbf{Караваевка} - местное для жителей улиц Тимирязевской, Бусловской, Зверинецкой горы в ботсаду, что лежит вдоль улицы Тимирязевской ниже хоздвора. От улиуцы Караваевка отделена не только ботсадовским забором, но и ручьем Омелютинкой.

См. про местность Караваевщину.\\ 

\textbf{Карандаш} – здание телецентра (не телевышка, а на другой стороне улицы) на Мельникова, похоже на карандаш. Стоит на месте Еврейского кладбища.\\

\textbf{Карпиловка} – «посада» между «Юрковицким и Иорданским потоками» (т.е. ручьями), на начало 17 века была владением Кирилловской церкви. По сходности названий и смежности мест предполагаю Карпиловку окрестностями нынешней улицы Копыловской (слывут как Антифеевка), что лежат напротив Кирилловской церкви через низовья Бабьего яра.

К началу 18 века в Карпиловке жили люди, относящиеся к Кирилловскому и Иорданскому монастырям.

Karpilowka показана на польской карте 1650 года между Подолом и Вышгородом, на берегу странной с современной точки зрения реки Репин (Ирпень), соединенной северной стороной с Днепром, а южной доходящей до селения Borsoiwka. 

На подробном плане с Приоркой и Куреневкой за 1752 год, Карпиловка не обозначена.

Более читайте про Копылоку и Копырев конец.\\


\textbf{Кафедра} – в 1970-х народное название кафе «Киевского», что было выше валютного магазина «Каштан» на углу Крещатика и бульвара Шевченко. «Киевское» примыкало к этому магазину. Был еще один «Каштан» выше, на углу Пушкинской, но тот был ювелирным.\\

\textbf{Караванский шлях} – старинная дорога из Киева в Белогородку, Новоселки, Черногородку, за Поволожчей соединялся с татарским Черным шляхом.\\

\textbf{Кардачи} – радиорынок на Караваевых дачах.\\

\textbf{Караваевы дачи} – район между Шулявкой, Чоколовкой и Соломенкой, по нему проходит долина одного из истоков Лыбеди, с остатками Кадетской рощи по восточному берегу, между улицами Уманской и Железнодорожной. На Караваевых дачах уцелела полоса частного сектора между вдоль Железнодорожной и Полевой улицы, а так – большей частью хрущовки.

Один из истоков Лыбеди протекает вдоль железной дороги, однако есть тут и другой исток, о нем свидетельствует название улицы Нижнеключевая.

Еще в первой половине 20 века от Отрадного в открытом русле протекала речка. На Нежинской улице по месту дома 14 (общага) на нем был пруд. Затем речка шла то по частным усадьбам, то пустырями на восток, параллельно улице Лебедева-Кумача (к северу от четной стороны), затем вдоль Нижнеключевой (на юг от ее нечетной стороны). 

У перекрестка Полевой и Нижнеключевой есть одичавшая местность между ними и Железнодорожной улицей. Ее теснит стройка ЖК на месте заводского корпуса. По сей местности и протекала эта часть Лыбеди, образуя болотце к востоку от перекрестка Полевой и Верхнеключевой, на месте ангара по адресу Полевая, 25. Далее речка следовала к Железнодорожной, где ныряла под мост, а спустя короткое расстояние – под железнодорожное полотно, чтобы влиться в общий коллектор Лыбеди вдоль железной дороги\footnote{Сейчас это слияние двух русел находится по координатам 50°26'36.8"N 30°27'17.1"E, малость не доходя по Железнодорожной до Мамина-Сибиряка.}. Но это уже не Караваевы дачи!

Вернемся западнее, к ним самим. В середине 19 века там был питомник деревьев – 42,63 десятины, где на продажу выращивались кусты, парковые и садовые деревья, оранжерейные растения. В 1870 году землю купил известный тогда хирург, крупный землевладелец, профессор Владимир Караваев, присовокупив к ней близлежащий «гимназический участок» и получив таким образом шмат земли в 64 десятины. 

После смерти профессора 32 десятины достались его дочери, А. В. Караваевой, которые она начала в 1903 году продавать, разделив на 238 участков. К 1908 году там вырос рабочий поселок Караваевы дачи, к 1911 года насчитывающий 323 усадьбы, 296 домов и 5000 жителей. Там же была одноименная железнодорожная станция, в перестроенном виде сохранившаяся по наши дни.\\ 

\textbf{Караваевщина}, или еще одни Караваевы дачи – располагались на Зверинце, от нынешнего участка ботсада «Степи Украины» и на юг до Верхней Телички. Там сейчас разные плодовые сады-питомники. 

По поселку проходила Печерско-Караваевс\-кая улица, что начиналась от низа дороги, что спускается от Ионинской церкви, затем поднималась наверх между участками «Степи Украины» (слева) и «Дальний восток» (справа), а затем идущая между плодовыми питомниками. Сейчас улица эта, уже без домов (хотя есть хозяйственное здание ботсада), сохранилась в прежних пределах в виде дороги или аллеи.

Название возникло оттого, что хирург Владимир Афанасьевич Караваев владел здешней землей, дачей «Прибрежная отрада». Дача была известна с 1864 года. Более двух десятин земли. Находилась близ участка кирпичного завода Приказа общественного призрения, частного владения мещан Головацких и городского выгона, бывшего в «оброчном и потомственном содержании вышеупомянутого Караваева». Прежде того, земли арендовал Гессе.
 
«Караваеву дачу» – «Прибрежную отраду» – в 1875-79 годах арендовал крестьянин Максим Горащенко (Гаращенко), и на картах появился «хутор Горащенко». В 1879 году за неуплату аренды землю забрали в государственную собственность.

Земля была распродана по участкам, и в первом десятилетии 20 века на Печерско-Караваевской улице было примерно 30 усадеб, да 10 домов стояло на Печерско-Карава\-евском переулке.

С улицей пересекалась Еврейскокладбищенская (ныне от нее есть часть дороги от зимнего сада к перекрестку Печерско-Караваевской и дороги, сходящей от Ионинской церкви) – идущая от Еврейского кладбища, занимавшего теперешний участок вьющихся растений (по сей день там сохранились каменные надгробия, надо только внимательно смотреть). Крутая дорога – Кленовая аллея – по склону между этим участком и небольшим хвойничком это было начало Еврейскокладбищенской улицы.\\

\textbf{Катериновка} – название дачного поселка на запад за Святошином. По имени Катерины Клейглейс, жены киевского генерал-губерна\-тора. Она была главой совета Киевского благотворительного общества, получившего от казны 10 десятин в 1895 году. В 1905 общество сдало часть земли под аренду для строительства дач.

Другое название местности было – Сулимовские дачи, потому, что там отдыхали летом воспитанницы Сулимовского пансиона, опекаемого обществом. По 1960-е годы на местности сохранялись названия улиц – Катериновской и Катериновских Поперечных, после их переименовали. 

Катериновка известна ныне под названием Екатериновка и лежит на стыке проспекта Победы с Брест-Литовским шоссе, на север от Петропавловской Борщаговки\footnote{50°27'07.6"N 30°19'33.3"E}. Это частный сектор посреди леса, превратившийся в коттеджный городок с теремами.\\

\textbf{Кафедра} - народное советское название кафе "Киевское", что было в начале Крещатика.\\

\textbf{Каштан} – прежде Торгсин, сеть магазинов, где за валюту продавались разные товары, иностранные и советские (дефицитные или дорогие вроде фотоаппаратов), ювелирные изделия. Один «Каштан» был в угловом доме вместе с «Юным техником» по адресу бульвар Леси Украинки, 24, на Пушкинской (более ювелирный), и напротив памятника Ленина на бульваре Шевченко, 2 в угловом доме.\\

\textbf{Каштан}, мороженое – знаменитое киевское мороженое, выпускалось на Хладокомбинате №2 (его корпуса были между Демиевским путепроводом и Океан-плазой).\\

\textbf{Квадрат} - микрорайон 9-33 в районе площади Победы.

\textbf{Кинологическая} – пещера длиной 30 метров в суглинном склоне на Вознесенском спуске, около Кинологического центра ГУ МВД Украины в г. Киеве. Название диггерское.\\

\textbf{Кинь-грусть} – удолье с тремя прудами, ныне застроено теремами и продолжает обрастать ими со всех сторон. С юга к нему примыкает поросший дубами просторный холм Кристеровской горки, с севера – Княжая гора (ее подножие огибает улица Кобзарская)\footnote{50°31'10.9"N 30°26'41.7"E}, тоже заросшая лесом, и за нею после частного сектора начинается Пуща-Водица. 

Урочище Кинь-Грусть именуется так потому, что по преданию Екатерина сказала тут взгрустнувшему Потёмкину – кинь грусть!

Пруды лежат на верховье речки Княжихи - от которой, получается, и Княжая гора. Название я нашел в статье Вортмана, в иных источниках я его не встречал. 

Самый нижний пруд называется Кулик, у него земляные, поросшие травой берега, на которых отдыхают люди и сидят рыбаки. Кулик лежит в углу между улицей Красицкого, переулком Водников и улицей Кобзарской. Два других пруда – несколько северо-восточней, имеют бетонные берега, обсижены рыбаками. Вдоль берегов растут небольшие ивы. Исток ручья, питающего пруды, где-то между улицами Красицкого и Кобзарской.

На некоторых старых картах конца 19 века показано, что ручей из Кинь-Грусти сворачивает на юг и впадает в пруды на земле Кристера, и далее продолжается уже под именем Коноплянки. Однако, между Кинь- Грустью и прудами южнее Кристеровой горки - холм, это разные водоразделы.

Ручей из Кинь-Грусти прежде протекал на восток и терялся где-то в сторону Почайны. Сейчас подземный коллектор ручья, к востоку от прудов, доходит до Вышгородской 55, где сворачивает на юго-восток, и за школой №16, на Дубровицкой улице присоединяется к коллектору Коноплянки.

Одно из современных названий Кинь-Грусти – Триозерка.\\

\textbf{Кирилловская площадь} – дореволюционная площадь в низовьях нынешнего Подольского спуска. Вокруг стояли частные домики в садах.\\

\textbf{Кирилловский ручей} – ручей, что протекает по Бабьему яру. Взят в коллектор длиной 5 километров, принимает в себя многочисленные родники Бабьего яра. В начале 20 века протекал на поверхности. Впадает в систему озер Опечень.\\
%РАСШИРИТЬ

\textbf{Кирилловское кладбище} 

50°28'43.1"N 30°27'49.3"E

Было на запад от Павловки и Кирилловской церкви, на возвышенности между Бабьим и Репяховым ярами. Сейчас от кладбища осталось с десяток большей частью разбитых каменных надгробий да обезображенный склеп братьев Качковских (один врач и имел клинику на Маловладимирской – Чкалова, Гончара – а другой студент).

Кладбище поначалу, в 19 веке, использовалось для захоронений умерших в Кирилловской богадельни и больнице. В 1929 году его закрыли, охранять перестали, и так оно пришло до теперешнего состояния. Любопытно, что несмотря на прошедшие годы, между уцелевшими редкими надгробиями еще ощущаются, среди деревьев, былые дорожки.\\

\textbf{Киянка} – ручей, приток Глубочицы (Кудрявца), берущий начало в склоне оврага урочища Дегтяры (окрестности Дегтярной улицы). Протекает в коллекторе. В 2015 году часть вод его выбилась на поверхность на стройке, и образовала даже заросшее камышом болотце.\\

\textbf{Киянка} – известный в 1980-е хозяйственный магазин в Кловском овраге, на улице Мечникова 9. Теперь там совсем другое здание. У подножия склона горы, в большом и светлом павильоне «Киянки» продавались лопаты, посуда, перочинные ножи, обои, инструменты, моющие средства и другая всячина.

Помню, ездил туда с бабушкой – мы доезжали на троллейбусе до «Подарочного» магазина, а потом шли по большой лестнице в Кловский овраг, сейчас на месте лестницы стоят огромные дома.

Эту лестницу во второй половине 20 века некоторые называли Собачкой – на ее переползло название лежащей ниже, по нынешней Мечникова, Собачьей тропы.\\

\textbf{Кияновка} - так в обиходе называют Кияновский переулок его жители.\\

\textbf{Клинец}

50°27'36.3"N 30°30'48.8"E

Гора на Андреевскому спуске, напротив Уздыхальницы, лежит к югу от Замковой и отделена от нее овражком с перешейком. Металлическая лестница с Андреевского поднимается именно на Клинец. 

Обычно даже краеведы совмещают Клинец с Замковой, хотя это и по виду, и раньше по земельным документам и названиям были отдельные горы. Подробнее об этом в моей «Ереси о Киеве».\\

\textbf{Клинический городок} – бытующее с начала 20 века название местности в окрестностях нынешней улицы Амосова, на могучем склоне в северной части Байковой горы. Именуется так из-за обилия находящихся там медицинских учреждений. Внизу его есть старый деревянный дом, где расположена аптека. Напротив него, на территории одной из фармакологических компаний, находится дореволюционное здание Института бактериологии, построенного за деньги сахарного магната Лазаря Бродского.

Улица Амосова раньше называлась Клинической, а ее низовье, идущее от упомянутого института к улице Протасов яр, именовалось улицей Дьяковской.\\

\textbf{Клов} – летописное урочище, включающее Липки и овраг где проходит улица Мечникова. 

Изначально овраг начинался близ Никольских ворот, на северо-запад от них. Ворота эти более известны как здание военной комендатуры, которое слева от станции метро Арсенальная, если стоять к ней лицом. Изначально это была «казарма на Перешейке», с двухарочным воротами. Построили ее в середине 19 века. Еще левее находилось начало оврага с перешейком между оврагом и обрывистым берегом Днепра, по этому перешейку попадали из Старого города в Печерскую крепость. Верховье оврага засыпали в 19 веке.

%А на противоположном берегу оврага находится Кловский дворец, ныне более привычно другое его название – Марьнский. В 1744 году Елизавете Первой, видите ли, негде было в Киеве остановиться, вот после ее посещения и стали возводить этот дворец, закупая кирпичи у Лавры. До того "кловский двор" – участок – принадлежал Лавре, и непосредственно в 1740-е был под управлением лаврской типографии, используясь для разведения скота и выращивания овощей. Такие обычные типографские дела...

В восьмидесятые я считал Кловом поросший огромными кленами склон по нечетной стороне бульвара Леси Украинки (от номера 9 по 7) – сейчас этот склон весь застроен (вот почему здания между 9 и 7 имеют буквенные приставки), а раньше был дик и зелен, и по нему вниз спускалась долгая, в несколько пролетов лестница, по которой мы с бабушкой обычно добирались до улицы Мечникова и магазина «Киянка» в нем в частности.

На восточном берегу оврага (где у подножия и располагалась «Киянка») находится Александровская, в советское время Октябрьская больница. Еще в начале 20 века, на восток от нее, и на юг, произрастал виноградный сад. О нем теперь напоминает только Виноградный переулок (так же, как об огромном шелковичном саду на Клове напоминает улица Шелковичная). На южном же склоне тогда был просто сад, он лежал между улицами Рыбальской и тем отрезком Кловского спуска, что идет от Московской улицы до Мечникова.

Название Клов кажется мне искаженным Куов, Киев. Я бы не указывал на это, если б, как предполагаю, Байков яр не стал Божковым яром.\\ 

\textbf{Кловица, Кловка} – историческое название речки или ручья, протекавшего в овраге Клова. Ныне этот ручей, взятый в коллектор, слывет как просто Клов, а Кловицей многие диггеры невесть почему называют другой ручей, приток что стекает по Шелковичной улице. Я же придерживаюсь исторического названия.

Кловица – левый приток Лыбеди. Еще в первой половине 20 века Кловица имела открытый участок русла с истоком\footnote{50°26'41.2"N 30°32'36.2"E} в яру, что прежде начинался слева от здания военной комендатуры на улице Январского восстания, 1 – дореволюционные Никольские ворота, от которых шла в сторону Лавры улица Никольская, названная так от Никольского же собора.

Потом этот участок яра засыпали, его сейчас пересекает улица Грушевского. Затем яр и ручей в нем продолжались вдоль или по несуществующему тогда стадиона завода «Арсенал», у западной стороны стадиона\footnote{50°26'32.3"N 30°32'22.8"E}, что выходит на Кловский спуск. После засыпки верхней части яра, речка начиналась эдак за домом по адресу Кловский спуск, 3\footnote{Вероятно, был еще один исток со стороны улицы Липской, тогда он должен был протекать по оврагу возле дома номер 6 по Виноградному переулку – овраг через улицу напротив медучреждения СБУ.}. 

Чуть южнее, в первой половине 20 века, она вбирала в себя ручей, что сочился от южной же стороны дома\footnote{50°26'28.2"N 30°32'23.4"E} за длинным зданием на Кловском спуске, 7. Оба здания это корпуса завода «Арсенал».

У перекрестка Кловского спуска с улицей Мечникова с востока присоединялся приток, вытекавший из общественного колодца в недрах частного сектора на месте нынешних корпусов завода «Арсенал». Этот ручей показан на плане 1803 года Меленского, на позднейших планах его не видно.

От перекрестка речка протекала к Бессарабке в долине урочища Клов, под склоном вдоль теперешней нечетной стороны Мечникова (стороне Октябрьской больницы), а дорога была проложена как раз по Мечникова, несколько в стороне от русла. По крайней мере во второй половине 19 века на склоне удолья по стороне больницы был виноградный сад, а на противоположной, что взбирается к бульвару Леси – сад плодовый. По дну оврага проходила грунтовая дорога, слывшая Собачьей тропой.

Следующий большой приток присоединялся на перекрестке Мечникова с Первомайского, он показан на плане 1803, однако на 1930-е его уже не видно. Приток начинался в яру, что существовал в окрестностях нынешнего дома по адресу бульвар Леси Украинки, 20/22 (там где была Военная книга). Бульвара, понятное дело, тогда не было, и вот овраг пересекал его и продолжался в теперешней Первомайского – она-то и лежит в сохранившейся части оврага, что принимал в себя приярки, а потом вливался в огромный овраг Клова.

Не доходя до территории Октябрьской больницы, Кловица сначала принимала в себя с ее стороны небольшой приток, а потом у ограды больницы, около моста, уходила в коллектор.

Далее Кловица, как и прежде, но сейчас под землей, следует до перекрестка с Госпитальной и Леси Украинки. От дома номер 10 по Леси к речке присоединялся еще один приток, некогда будучи восточной границей еще одного садового участка, простиравшегося до Бассейной.

У пересечения нескольких широких улиц – Мечникова, бульвара Леси Украинки, Госпитальной, Шелковичной – Кловица сворачивает под землей на юг, к Лыбеди, огибая с востока Дворец Спорта, а затем мимо Республиканского стадиона течет под прежней Лыбедской площадью, чуть южнее Свято-Троицкой церкви (стояла рядом с театром оперетты), от которой площадь потом переименовали в Троицкую.

Далее Кловица, в 19 веке как-то вдруг переименованная в Клов, протекала вдоль улицы Совской, нынешней Физкультуры, и за газовым заводом совсем уж приближалась к Лыбеди. На месте газового завода потом возникло трамвайное депо им. Тараса Шевченко (на углу Физкультуры и Горького), которое было снесено в 2005-2006 годах. Там собирались возводить очередной жилищный комплекс. В 2008 году котлован под него заполнился водой, и на 2020 год там здоровенное прямоугольное озеро.

Современное устье Кловицы в Лыбедь находится на задворках переулка Физкультуры, в виде бетонного портала Прозоровского коллектора дождевой канализации\footnote{50°25'52"N 30°30'25"E}. Чуть юго-восточнее вдоль русла Лыбеди находится Пятничный Клов – место сбора (по пятницам) диггеров. Между ними\footnote{50°25'49.8"N 30°30'28.7"E} прежде был отрезок старого коллектора Кловицы, его след виден до сих пор, равно как и старый портал устья. На середину 19 века тут была роща, на конец 19 века – дача Скотного двора Софийского митрополичьего двора.

В коллекторе Кловицы бывает страшное явление – коллекторная волна, когда например во время дождя коллектор быстро заполняется водой под потолок и сильнейший поток уносит с собой всё, что может унести, включая человеческие жизни. Так в 2003 году погибли диггеры Хартман и Ампер.

Описание маршрута коллектора Кловицы.

Начинается за Домом офицеров, затем на юг, под углом дома Институтская, 29/3 и вниз, на юг, вдоль нечетной стороны Кловского спуска, между шоссейной частью и домами. Затем коллектор сворачивает на юго-запад вдоль шоссейной части четной стороны улицы Мечникова.

Около перекрестка Мечникова, Шелковичной и Бассейной к коллектору сверху, с севера, присоединяется приток. Его начало прослеживается от «дома плачущей вдовы»\footnote{А может быть и дальше, от театра Франко.} на Лютеранской 23, построенном в 1907 году по заказу купца Сергей Аршевского. На фронтоне дома есть барельеф с женским лицом, во время дождя кажется, что из его глаз текут слезы. В усадьбе дома есть фонтан.
От дома 23 (он кстати граничит со зданием администрации президента) коллектор притока идет под домом 25 и 27-29, пересекает Лютеранскую и уходит под дом на Шелковичной 32, 32/34, следует вдоль него и смежного 36/7, пересекает переулок Козловского, границу сквера и проходит далее идет по западной, четной стороне Шелковичной улицы, по границе домов.

Далее общий коллектор Клова пересекает Бассейную, газон, снова Бассейную, и начинает ход на юго-запад по четной стороне Эспланадной улицы, между домами и шоссе.

От дома 8/10 (МинСоцПолитики) к коллектору присоединяется еще один приток, слывущий среди диггеров Крещатиком. Он начинается далеко, в самом верху Михайловской улицы, возле Михайловской площади. По оной улице коллектор притока идет вниз по четной стороне к Майдану. Около фонтана и стеклянного купола «Глобуса», почти напротив низа улица Костельной, в него впадает короткий приток с Софиевской улицы.

Затем коллектор этих соединенных притоков Клова выруливает к Крещатику в районе напротив угла Главпочтамта. Где-то тут к ним с северо-востока присоединяется еще один коллектор. Он начинается\footnote{50°27'04.4"N 30°31'48.7"E} за Парламентской библиотекой, на Петровской аллее, близ теннисных кортов, там где некое хозяйственное сооружение в склоне. Оттуда коллектор поворачивает к библиотеке, вдоль нее проходит под улицей Грушевского и выруливает на Европейскую площадь, огибая гостиницу «Днепр» и далее следует вдоль нечетной стороны Крещатика к Майдану. 

Так составляется ручей Крещатика\footnote{Не совсем понятно, куда делся также ручей, протекавший около Национальная художественного музея Украины, что на Грушевского 6.}. 

Коллектор Крещатика идет на юг под Крещатиком. У пересечения Крещатика с Хмельницкого, он принимает в себя коллектор с улицы Хмельницкого – диггеры называют его «ручей Луга». Далее общий коллектор следует мимо Бессарабского рынка, Арена Сити, Мандар\-ин-плаза, под перекрестком Большой Васильковской с Скоропадского, на юг под западным углом комплекса «Макулан» (Большая Васильковская ул., 9/2), затем под Рогнединской 3, наконец ул. Шота Руставели, 9А и под упомянутой уже Эспланадной 8/10.

Обогащенный притоком Крещатиком, Клов идет дальше на юго-запад, пересекая Эспланадную и следуя на юг вдоль ее восточной стороны (та, что на стороне Дворца Спорта).

От перекрестка с Саксаганского Кловица в коллекторе продолжает на юг, затем юго-запад идти между ул. Саксаганского, 1 и НСК Олимпийский, мимо последнего наискось пересекает Троицкую площадь, и выходит к перекрестку Большой Васильковской и Физкультуры.

Затем следует на запад вдоль улицы Физкультуры, а после перекрестка с Антоновича – между кинотеатром «Батерфляй» и «Мегамаркетом», и мимо южной части Владимирская 101. От перекрестка Короленковской и Физкультуры коллектор продолжает движение под последней и впадает в Лыбедь у моста, за автомойкой.\\ 

\textbf{Кловский дворец} – построен Лаврой в середине 18 века для приема важных лаврских гостей. Ныне в этом здании на Орлика (бывшая Кловская), 8, размещается Верховный суд Украины. 

Кловский дворец соперничал с возводящимся в то же время (и частично теми же людьми) Царским (Марьинским) дворцом. Разбили рядом с ним и сад с оранжереей. В саду росли лимоны, апельсины, мандарины. Заведовал им поначалу Василий Скабеев (он продолжал работать и над садом при Марьинском дворце), потом Иоганн Блех, затем Онуфрий Литонёвский, при котором сад пополнился на 600 груш и яблонь, 400 груш-венгерок, 200 черносливов, 650 вишен, 120 марелек, 300 орехов и 55 кустов винограда. Для работы в саду, с 1762 года, привлекали жителей принадлежащего Лавре села Совки.

Позже дворец забрали себе военные под госпиталь, а на стыке 18-19 веков в нем обосновались гражданские губернаторы, а затем гимназия и так далее. Здание неоднократно горело. В последние десятилетия СССР и по 2004 тут был музей Истории Киева.\\

\textbf{Кмитов яр} – между Лукьяновкой и Татаркой, состоит из нескольких отрогов. Один 
лежит напротив парка Котляревского, на другой стороне улицы Герцена. По второму отрогу проложена улица Кмитов яр. Далее сводный уже овраг продолжался там, где теперь завод Артема.

В первом отроге Кмитова яра находится бывшая усадьба «дача Хрущева», а также каскад прудов, питаемых ручьем из трубы в северной части оврага. Этот ручей – один из истоков речки Глубочицы. Откуда он течет до этой трубы, читайте в заметке «Болото». Попасть в ту часть Кмитова яра, к прудам, можно через Институт педиатрии и акушерства, на территории коего и находится северная часть яра.

Далее на юго-восток яр продолжается улицей Лермонтовской (близ которой ручей из прудов уходит в коллектор), а затем одноименной улицей Кмитов яр, выруливающей сюда с Татарки по другому отрогу яра. Затем улица эта заходит на территорию завода Артёма, а Глубочица в коллекторе течет непосредственно под заводом.\\

\textbf{Княжая гора} 

50°31'11.0"N 30°26'42.5"E

Примечательный холм, лежит в восточной части парка «Кинь-грусть», между улицами Сошенка и Кобзарской. Вообще там вся местность это холм, но с запада на холму – гора более отчетливая. Ее описывают овалами то ли окопы, то ли небольшие валы со рвами. На самой вершине – плоское место, откуда никакой вид не открывается из-за деревьев. Тут растут большие высокие дубы и липы. Склоны не обрывисты, круты, однако покаты, на них ведет несколько широких троп. 

«Круговое» строение горы прослеживается на старых картах, например на плане РККА 1937 года.

На дореволюционной карте (я не знаю, какого года) там показан «Английский сад», поэтому замеченные мною валы и окопы могут быть следами его оформления.

Сейчас рядом с горой разбит сквер.\\


\textbf{Кожемяки} – урочище смежное с Гончарами. Расположено по Кожемяцкой улице, где некогда жили кожемяки. Лежит в огромном овраге между горами – с севера это Замковая и Клинец, с юга Детинка и Валовая. Обычно урочища совмещают в единое название – Гончары и Кожемяки. Ныне они более известны как Воздвиженка, по новостроенному кварталу, а тот по названию улицы, а та по близлежащей церкви. В Кожемяках, из урочища Дегтяры (что рядом) протекает ручей Киянка, приток Глубочицы. Он взят в коллектор.\\


\textbf{Козье болото} – несмотря на расхожее мнение, что болото было на месте нынешнего Майдана, оно было не там, хотя неподалеку, на склоне горы под земляным крепостным валом. План Меленского 1803 года показывает болото в области примерно между переулком Тараса Шевченко (бывшая Козеболотовская улица), Паторжинского, Малоподвальная – внутреннее пространство между ними, а еще вернее окрестности домов по адресам Паторжинского 14, Малоподвальная 10/12, 12, 8, но также и угол квартала между Ирининской и Михайловским переулком. Сейчас в связи с чудовищной застройкой района сложно представить былое.

Музей Шевченко – дом Житницкого, где квартировал Тарас – еще тогда, в 1803, находился в местности застроенной частными домами среди садов, а до болота надо было еще немного пройти по Малоподвальной к нынешнему дому номер 8, кстати улица тогда существовала почти в современных пределах, на том отрезке. И вот во дворе за домом номер 8 было болото.

Улица же так назвала не от слова «подвал», а потом, что она «под валом». Вал, вернее часть его, проходила по нечетной стороне Малоподвальной, а по четной было болото. Этот вал, пришедший сюда от Золотых ворот, спускался к Майдану и потом шел вдоль улицы Костельной, между нею и Крещатиком. Крещатик был вне вала. На 1803 год, вместо Крещатика была дорога по дикой местности, существовала и развязка около Европейской площади, то есть по улице Грушевского была дорога, и по Владимирскому спуску. На месте Майдана, по нынешней четной части Крещатика, в валу были Печерские ворота, потому что от них из низины шла дорога наверх, к Печерску, ставшая потом Институтской улицей.

В широком смысле Козьим болотом называлась вся местность включая улицу Козоболотную. Близ нее, на нынешнем Майдане, в 19 веке собирался Крещатицкий базар.\\

%\textbf{Козовища} – урочище, скорее всего старое название Козинки.\\

\textbf{Козовища, Козовица} – урочище, владение Выдубицкого монастыря.

Из выписки из земских книг Киевского воеводства, 27 июня 1629 года:

\begin{quotation}
року тисеча шестсот двадцать осмого мсца мая двадцать девятого дня кгвалтоване на добра митрополиї Киевское монастира Видубицкого подле монастыря их Печерского лежачие, наслали слугу свого Юрка Замалинского с колкодесят иних бояр и подданих своїх монастирских, тамже тиї насланци з власного росказаня велм. вашой в тих Видубищах атамана видубицкого, которий там перевоз и пором митрополей на реці Днепру во власном кгрунте видубицком на урочищу у Козовищ Микулинца и Стефана Корца окрутне збылы, змордовали [...] 
\end{quotation}

%Из чего следует, что в урочище Козовищ были паром и перевоз, а урочище находилось возле Печерского монастыря, то есть Лавры. Известна давняя переправа через Днепр чуть севернее Наводничей, и вероятно речь идет о перевозе там же, принадлежавшем Выдубицкого монастырю.

Известен перевоз Выдубицикого монастыря около Лыбеди, в 17 веке, в местах граничащих с владениями Лавры. Там было два парома. В 1713 году между Козовищей и Лыбедью начали строить кирпичные заводы.\\


\textbf{Козловка} – от сгинувшей в 20 веке улицы Козловской, окрестности Зеленого театра. Также местность слыла как Провалье. В советское время на Козловке были два ресторана, «Кукушка» и «Курени». Местность представляет собой склоны над Днепром.

Название улица получила от того, что 1804 году там приобрел участок коллежский регистратор Козловский.\\

\textbf{Коммунар} - кинотеатр, стоял на месте нынешнего кинотеатра "Киевская Русь".\\

\textbf{Кониченка} – на карте предместий Киева 1850-60 годов так подписан ручей Коноплянка.\\

\textbf{Конная площадь} - нынешний Полицейский сквер.\\

\textbf{Колос} – около Житнего рынка, поперек Верхнего и Нижнего валов, стоял деревянный кинотеатр «Колос», чьи задворки притягивали пьяниц и блатных. Кинотеатр возник в тридцатых в помещении магазина по продаже тканей. В 1953 году он сгорел, однако был восстановлен. В 1977 в нем снова случился пожар, «Колос» сначала закрыли, а потом снесли.\\

\textbf{Коноплянка}, урочище – выгон по обеим сторонам Лыбеди, в 18-19 веках принадлежащий Лавре.\\

\textbf{Коноплянка}, хутор – небольшой хутор с деревянным домиком и хозяйственным постройками, известный в 188х, в роще около Лысогорского форта, там где сейчас радиовышка.\\  

\textbf{Коноплянка}, урочище – урочище по ходу течения ручья Коноплянки, ныне в углу между улицами Луговой и Бережанской. Имеет отношение к нижнему течению речки Коноплянки.\\

\textbf{Коноплянка}, речка

Исток: на восток от парка «Кристеровская горка», за гаражами на Краснопольской, 2, который образует несколько прудов в упомянутом парке, куда вход только для жителей окрестных элитных ЖК. Раньше эта местность относилась к садоводству Кристера (см. Кристерская горка).

Устье: 50°30'45.5"N 30°28'28.98"E

Бывший западный приток Почайны. Теперь протекает в коллекторе и впадает в озеро Луговое со стороны улицы Петра Дегтяренко. От выхода из коллектора до озера, между гаражными кооперативами, 200 метров идет открытое, шириной метра 3-4 русло.\\

\textbf{Консерва} – консерватория.\\

\textbf{Конча-заспа} – было два озера, южное Конча (Корнча) и северное Заспа. Местность сильно изменилась стараниями застройщиков, а обозначения озер перепутаны на картах. Конча, или Глушец – ближайшее к Столичному шоссе, вдоль него. Длинное (6 километров), узкое, некогда глубокое до 13 метров. А Заспа – это старуха.

Озеро Заспа известно по крайней мере с конца 16 века, как и Калной Луг.

В Кончу в прошлые века впадала речка Вита, ранее Вета.\\

\textbf{Конюшня} – видеотека 1990-х на первом этаже в доме на Мечникова, 16.\\

\textbf{Копыловка} – окрестности улицы Копыловской, прежде, а хотя бы в 17 веке, представляла собой отдельное селение Карпиловка и отображена на польских картах того времени. Местность лежит на левом берегу низовья Бабьего яра, на пологом склоне. Граничит со Шполянкой.

Но еще на середину 19 века, когда Шполянка не существовала как поселение, ее местность тоже слыла урочищем Копылово.

Первые этаже некоторых домов в Копыловке расположены ниже уровня земли, к парадным переброшены мостки. Это потому, что уровень земли поднялся из-за пульпы, хлынувшей сюда во время Куреневской трагедии.\\

\textbf{Копырев конец} – некое место перед «городом», как ехать со стороны Вышегорода. Ипатьевская летопись за 1140 год: «Поиде Всеволод Олгович из Вышегорода к Кыеву, изрядив полкы, и пришед ста у города в Копыревом конци, и начал зажигать дворы, иже суть пред городом в Копыревом конци».

И за 1121: «Заложена бысть церкви святаго Иоанна на Копыревом конце».

Современные сопоставления Копырева конца кажутся мне нелепыми, за недостатком исходных данных.

Хотя созвучие имен и указания, что Всеволод шел от Вышегорода, дают повод задуматься, а не Копыловка ли древний Копырев конец? Ведь если со стороны Вышгорода двигаться по улице Вышгородской, мы прибудем как раз к Копыловке...\\

\textbf{Кореневщина} – в конце 18 века одно из названий Куреневки.\\

%\textbf{Коровье озеро} – озеро в северной части Телички, по восточную сторону железной дороги, ныне застроено промзоной.\\

\textbf{Корича} - озеро, принадлежавшее Выдубицкому монастырю на середину 17 века, упомянуто в документе того времени вместе с другими монастырскими озерами: "Заспа, Корича, Глущец, Потяж, Плоская и Святище". Вероятно, от Корича произошло название хутора Корчеватого.\\

\textbf{Корчеватое} 

Изначально, на середину 19 века, хутор Корчеватое лежали у южного подножия Лысой горы, между нею и улицей Лысогорский спуск, примерно по координатам:

50°23'22.9"N 30°33'06.4"E

На месте Лысогорского спуска был заросший деревьями и кустами яр с парой приярков, а от него в высотам Лысой горы шли по склону поля. Строения были внизу, около нышнего 
шоссе, и потом населенный пункт стал расползаться на юг, в низовье между Днепром и склоном Багриновой горы, следующей к югу после Лысой.\\

\textbf{Корчи, Дниструиха}

50°29'0"N 30°25'54"E

Заболоченное озеро, пруд в низовье яра, на речке Рогостинке при ее приближении к тоннелю под железной дорогой. Показано еще на карте 1799 года, хотя на последующих картах оно то есть, то нет до конца 19 века. Прежде чем там появилась железная дорога, Рогостинка, после своего пруда, впадала в огромный пруд на Сырце, по другую сторону от нынешней железной дороги. До сооружения последней, между Корчами и Сырцом по насыпи шла грунтовка.

В советское время это Корчи были большим прудом с пляжем. Название Дниструиха бытовало у жителей улицы Белицкой. Некогда в Корчах купались жители всей окрестности, в том числе Сырца.

В 1960-х котлован его берегов покрывала трава, из-под которой выбивались суглинок и песок. В жару было много людей - не как на Гидропарке, а скорее как на Радунке. Южная часть пруда уже тогда начала зарастать камышом. К семидесятым годам северный, восточный и западный берега, где и располагался основной пляж стали покрываться кустами, деревья окрестным склонам заматерели и озеро уже напоминало место скорее для пикников на природе, нежели на прежний пляж.

Однако по начало восьмидесятых это было именно озеро, вернее пруд, с чистой водой, не болото.

Сейчас берега все в зарослях, воды не видно - в ней болотная трава. Судя по всему, озеро также очень обмелело - я смог дойти примерно до середины по топкой почве и каким-то бревнышкам. Раньше же местами было "с головой".\\



\textbf{Космодром} – площадь Космонавтов.\\

\textbf{Косогорка} – Косогорный переулок, местное название.\\

\textbf{Костопальня, Костопаловка} – см.  Александровская слободка.\\

\textbf{Косточка} – улица Константиновская.\\

%\textbf{Кардачи, Каравайские дачи} – местность на Индустриальной улице, с одноименным радиорынком. Прежде ею владел тот же врач Караваев. Также его именем была названа улица Караваевская, ныне Льва Толстого.\\

%\textbf{Красная горка} – 

\textbf{Котлярка, Котляра} – парк Котлеревского на Лукьяше.\\

\textbf{Котурка, Котырь, Котер} – ручей, что начинается в Берковцах и течет на север через Пущу. Прежде принадлежала Выдубицкому монастырю. Впадает в реку Ирпень.\\

\textbf{Красница, Красницы} 

50°26'29.3"N 30°33'14.9"E

Урочище, примыкающее непосредственно на юго-восток к Аскольдовой могиле. Показано на ряде карт начала 19 века. Представляет собой уступ горы. На стыке 18-20 веков тут был пруд под обрывом или некой кручей, а вообще сверху и снизу Красницы, окружая некий пустырь на уступе, лепились один к другому дворики усадеб.

Через Красницу проходил к Днепру Спасский спуск.\\

\textbf{Красный дом} – дом на Татарке, над верхом Смородинского спуска. Адрес Нагорная 8/32. Сталинка, состояла из коммуналок, построена в 1940 году для работников Кабельного завода. Изначально дом не был оштукатурен и являл миру красный кирпич, отчего и получил название. По начало 21 века на первом этаже дома был продуктовый магазин, а справа от него на углу – аптека.

На стыке веков был надстроен этаж вокруг небольшой башни со шпилем, венчавшей дом. Во дворе Красного дома некогда был фонтан.\\    

\textbf{Красный трактир} – село, существовавшее до пятидесятых годов 20 века между Теремками, Новоселицами и Феофанией. Занимало территорию нынешнего ВДНХ.\\

\textbf{Кресты} – в записках Автонома Антиповича Солтановского о 1846 году:

\begin{quotation}
По левую сторону пустыря\footnote{Перед университетом св. Владимира.} тянулся к Крещатику бульвар, засаженный молодыми чахлыми каштанами, а за ним под старокиевскими валами вытягивалась улица к Крещатику же со вновь строющимися каменными домами (Фундуклеевская, ранее Кадетская – прим. Семилетов). Прямо с бульвара можно было, поднявшись по тропинкам в гору\footnote{В сторону Печерской крепости.}, пройти к крепости через так называемые «кресты».

Кресты располагались в оврагах и состояли из развалившихся лачужек и землянок, где жил самый забубенный люд: низшаго сорта проститутки, пьяницы, воры, спившиеся отставные чиновники, строчившие ябеды, и самые бедные рабочие и ремесленники. Через «кресты» проходить днем было небезопасно, а ночью положительно невозможно.

Около крепости на горе была какая-то бедная часть города с каменной церковью св. Ильи, толкучим рынком, где ежедневно продавались жителями «крестов» и солдатами ворованные вещи, и где осенью и весной люди, без преувеличения, тонули в грязи.
\end{quotation}

\textbf{Кривой мост} - соединял гору Клинец с Замковой.\\

\textbf{Кристерова горка} 

50°30'54"N 30°26'47"E

Бывшая дача Кристера – поросший деревьями, в частности дубами, холм между улицей Осиповского, Вышгородской и Красицкого. Обустроены тенистые аллеи, на самом верху вроде бы раньше стоял семейный склеп Кристеров, но где именно – непонятно. У подножия холма на улице Осиповского, за оградой Института пищевой биотехнологии, в 2017 году видны развалины кирпично-деревянного особняка семьи Кристеров.

По другую, южную сторону улицы Осиповского – парк «Кристерова гора» с четырьмя прудами. Вход туда – только для избранных, для жителей близлежащего высотного ЖК «Парковый город», который купно с парком разместились на земле бывшего совхоза «Троянда». Там выращивались разные овощи-фрукты, стояли теплицы. До революции, эта местность тоже относилась к «садовому заведению Кристера».\\

\textbf{Крещатицкие горы} – в конце 18 века так называли склоны Днепра от источника Крещатика (ныне колонна Магдебургскому праву) вдоль Царского сада (теперь Крещатый парк) – словом, в сторону стадиона Динамо.\\

\textbf{Кругликов яр} 

50°21'57.8"N 30°31'14.0"E

Расположен в лесу между Самбурками и Мышеловкой, к югу от последней, параллелен улице Ягодной. В яру начинается ручей, питающий Китаевские пруды.\\

\textbf{Крутая гора} – урочище слыло в середине 19 века, ныне покрытая березовой рощей гора на улице Белицкой, от угла между нею и Вышгородской. У подножия горы находится музей Тарасова хата.\\

\textbf{Кудрявец} – прежнее название речки Глубочицы. См. Глубочица.\\

\textbf{Кудрявец} – урочище, по которому протекала речка Кудрявец (Глубочица), ныне заключенная в коллектор под улицей Глубочицкой. Вот южная (четная) сторона этой улицы, то бишь противоположный Щекавице склон, и есть Кудрявец.

По карте 1803 года Меленского, Кудрявец – местность, очерченная улицей Кудрявской и Несторовским переулком.\\

\textbf{Кудрявица}, гора – на картах 19 века так обозначен нынешний Кудрявец, то бишь теперешний склон четной стороны Глубочицкой улицы.\\

\textbf{Кудрявский мост}

50°27'27.9"N 30°30'18.1"E

Перегорожен, но существует. Находится у самого перекрестка Вознесенского спуска и Кудрявской улицы. Старинный мостик, построенный по проекту инженера В. Бессмертного в 1897 году над верховьем Вознесенского яра, для подъезда к спиртовому складу на Кудрявской, 14. Одно время по мосту даже ходил трамвай.

Под мостик ведет спуск на Петровкую улицу, ее заброшенную верхнюю часть, куда снизу не попасть.\\


\textbf{Кузнецы} – ими в начале 19 века слыла местность от Почтовой площади до колонны Магдебургскому праву.\\

\textbf{Кукушкина дача} – окрестности Аскольдовой могилы и в сторону Зеленого театра, четких границ не скажет никто. Название ходило по крайней мере с начала 20 века, когда место это слыло прибежищем босяков и бродяг. В 50-е годы время возле Аскольдовой могилы (чуть ниже церкви) был даже ресторан «Кукушка» с круглой эстрадой небольшой открытой танцплощадки. \\

\textbf{Кулёк} – университет культуры.\\

\textbf{Кулик} – нижний из прудов в Кинь-грусти.\\

\textbf{Кулюженко площадь} – площадь Шевченко, старое название – площадь Кульженко (от близлежащей дачи Кульженко на Кинь-Грусти), или, как говорят старожилы – Кулюженко.\\

\textbf{Куреневская площадь} – на стыке 19-20 веков, находилась в южной части нынешнего Куреневского парка.\\

\textbf{Куренёвщина} – в конце 18 века одно из названий Куренёвки.\\

\textbf{Куренёвка} – район южнее Приорки, границей сейчас считается железная дорога и улица Добрынинская. Название происходит от слова «курить» в значении «гнать», ибо тут, при речке Сырце, стояло много винокурень. В том же значении Приорку называли Преваркой, хотя изначально Приорка от приора доминиканского монастыря.\\

\textbf{Кучменный, Кучмин яр} – яр к западу от Батыевой горы, верховья его приярков начинаются от Соломенской улицы, в Соломенском ландшафтном парке.

Парк был разбит в 1986 году с изменением рельефа. На дне яра вырубили множество деревьев, склоны сделали более покатыми, устроили дорожки и большую лестницу, что спускается от улицы Волгоградской. 

Для протекавшего по яру ручья, истоков речки Мокрой, известной в прежние века как Мужичек, придумали пустить его по трубам, а из самого верхнего ручья соорудить череду бассейнов, один ниже другого, но дальше второго бассейна вода не шла.

Прежде, правая часть яра была застроена хатками.

Ниже парка, яр застроен по большому счету частным сектором. По нему под улицей Кудряшова протекает в коллекторе речка Мокрая. 

В 19 веке в Кучмином яру было небольшое кладбище, сейчас на его месте школа номер 221 по улице Кудряшова. Южнее кладбища находилась местная Лысая гора, на пересечении улицы Кудряшова и Шаповала, застроена ЖК «Поры года».\\

\textbf{Купеческий сад} – парк около Арки Дружбы Народов. Позже его назвали Пионерским парком, ныне – Крещатый парк.\\

\textbf{Куреневский гражданский аэродром} - находился около Петровки, построен в 1910 на деньги Федора Терещенко, открыт в 1911 году. Тогда же, в октябре, там приземлился дирижабль Ф. Андерса (Купеческий сад - Куреневка - Вышгород - Куреневка.\\ 

\textbf{Куренёвское кладбище} 

50°29'32.4"N 30°27'00.4"E

Расположено между улицами Белицкой, Сырецкой и Валковской, добираться туда по Сырецкой и потом свернуть мимо частного сектора Валковской. Разделено на две большие части – православную и иудейскую. 

Существует по крайней мере с 19 века. Однако параллельно с ним тогда было более старое кладбище Куренёвки и Приорки, разделенное пополам между этими предместьями. Куренёвское – севернее, Приорская половина – южнее. Оно находилось, еще в середине 19 века, на северном берегу ручья Коноплянки, примерно где сейчас обширный ГСК и часть бывшего военного городка:

50°30'47.7"N 30°27'53.8"E

Короче говоря южнее жилого дома на Петра Панча, 9.\\

\textbf{Курячий Брод}, ручей – смыкается с Западинкой и общим потоком они впадают в Сырец.

Ручей имеет два ручья-истока, которые протекают разделенные местностью Замковище. 

Один исток начинается на восток от нагромождения гаражей, что протянулись вдоль Новомостицкой улицы. Там, за Межевой улицей, начинался овраг ручья\footnote{Примерно 50°29'46.3"N 30°25'18.2"E}, Сукачев яр. Еще в нулевых тут зеленели поля, а в 2017 год по обеим сторонам Межевой затеяли строительство жилых комплексов «Кристер град» и «Варшавский» (к 2019 первый уже построили). За зеленым, острым сверху строительным забором можно было наблюдать замусоренные остатки овражьего берега.

Далее ручей течет к югу от Новомостицкой и Брестской, параллельно им, на восток, к пруду за «Фуршетом» у перекрестка Мостицкой и Вышгородской\footnote{50°29'57.5"N 30°26'57.7"E}. Кстати этот «Фуршет», о чем мне поведала Даша Кононюк – дореволюционное здание, один из корпусов завода «Цепи Галля» (сам завод, под названием уже ЗАО «Завод штампов и пресс-форм Инструментальщик»  расположен ныне рядом).

%Один исток начинается у нагромождения гаражей близ Брестской, 25, и протекает к югу от этой улицы, параллельно ей, на восток, к пруду за «Фуршетом» у перекрестка Мостицкой и Вышгородской (50°29′57.5″N 30°26′57.7″E 50.499310, 30.449354).

Другой исток, более короткий, начинается у пересечения улицы Замковецкой и переулка Замковецкого, в яру на задворках частного сектора. Вдоль улицы Замковецкой течет на восток в упомянутый пруд. 

Затем общим потоком вода идет на юго-вос\-ток. Поблизости от перекрестка на юго-восток есть улица Боровиковского, вдоль которой ручей бежит в коллекторе. Боровиковского с конца 19 века по 1952 год называлась Курячий Брод.

По старым картам, далее Курячий Брод пересекала переулок Кошичев (Кожищев) – теперь это (соединенная к востоку с бывшей Вербовой) улица Казанская. Она лежит параллельно Боровиковского на юг.

Ручей Курячий Брод иногда путают с просто Бродом – другим названием речки Рогостинки, тоже притока Сырца. 
