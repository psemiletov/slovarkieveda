\chapter*{Р}
\addcontentsline{toc}{chapter}{Р}

\textbf{Райгородок}

50°27'57.6"N 30°28'40.4"E

Усадьба, склон над Кмитовым яром, на границе Лукьяновки с Татаркой. Был на месте нынешних зданий по адресам Мельникова 12-А и юго-восточной части 18-Б. Незастроенная зеленая часть склона за этими домами тоже была частью сада. Подойти туда снизу мешает ГСК

Участок самовольно захватил коллекционер древностей Турвонт Кибальчич (1848-1913), назвал его Райгородком и на основании сего землевладения избирался потом в гласные городской Думы. В Райгородке Кибальчич находил кости "неких доисторических животных".

На картах начала 20 века обозначен как "сад Райгородок".\\


\textbf{Рваный парус} – народное название летнего кафе «Аэлита» около озера в Голосеево (близ Голосеевской площади), существовало по середину 1980-х.\\

\textbf{Редьки, хутор} - поселок на четыре улицы, на северо-запад от озера Министерки, на краю основательно в тех  местах вырубленного леса Пущи-Водицы. Частный сектор.\\

\textbf{Резники} - подольское урочище, связанное с мясниками. Известно с 17 века, Максим Берлинский сопоставляет его место с "бискупским в старину дворцом и особенного того места рынком".\\ 

\textbf{Репяхов яр} – до революции Репяховым яром считалось только низовье нынешнего Репяхова яра, а нынешний подписывался на картах Кирилловским яром. Иначе говоря, исконный Репяхов яр это вот:

50°28'52.3"N 30°28'23.3"E

Где овраг уже подошел под гору с Павловкой.

В наше время же, Репяховым яром считается весь огромный, параллельный Бабьему, яр на юг от горы с психиатрической больницей.

Состоит из двух больших отрогов, один начинается от улицы Герцена, другой от Пугачева, и конец яра вливается в низовье Подольского спуска.

На крутом левом (по ходу как идти к Павловке) берегу западного отрога яра было Кирилловское кладбище, а также, почти над обрывом, находится Дурка – занимающее огромную площадь недостроенное здание института социальной и судебной психиатрии и наркологии\footnote{50°28'42"N 30°28'1"E}.

По краю восточного отрога идет гм... улица Врубелевский спуск (считается, что Врубель добирался до Кирилловской церкви этой дорогой) – сначала мимо гаражей, потом по дикому, замусоренному склону. До революции там была трамвайная линия. Краеведы любят называть это место Киевской Швейцарией, хотя оно совсем не похоже на Швейцарию.\\

\textbf{Риф} – некогда часть днепровского, западного берега острова Муромец. По сопоставлению с картой 1932 года, урочище находилось напротив правобережного лесопильного завода, непосредственно над смычкой Чертороя с Днепром, в районе нынешнего пляжа Черторой, к северу у Московского моста. На плане же 1902 года Риф показан напротив современного залива Верблюд.\\

\textbf{Рогатки} – в 19 веке так называли трехрогую развилку на современной Нижнеюрковской, где от улицы отделяется одноименный переулок.\\

\textbf{Рогатинская земля} – то же, что Рогостинская.\\

\textbf{Рогнединская гора} – в середине 19 века так именовали гору возле Троицкого базара (ны\-не у Троицкой площади, Республиканского стадиона). И неясно, что назвали так раньше – улицу Рогнединскую от горы, или гору от улицы. Рогнедой же звали одну из жен Владимира Красно Солнышко.\\

\textbf{Рогостинка, Рогостена}, ручей – левый приток Сырца. Петр Развидовский в своих записках середины 16 вероятно называет этот же ручей Рокошанкой.

Со второй половины 20 века ручей стали именовать Бродом (не Курячим Бродом, а просто Бродом, Курячий то другой). Впадает в Сырец чуть южнее\footnote{50°29'05.0"N 30°26'13.5"E} стыка улиц Тираспольской и Сырецкой, пройдя под железнодорожным полотном в яйцевидном тоннеле\footnote{50°29'01.0"N 30°25'58.0"E}.

Около восточной стороны тоннеля раньше была станция Сырец. По западную – заболоченное озеро Корчи (см. отдельную заметку). Крутой склон над ним испорчен велосипедистами, которые перекопали его трамплинами и трассами.

По Рогозову яру с крутым и высоченным северным склоном (южный более покатый, хотя тоже высокий), ручей протекает в русле, которое то петляет, как бы прорезанное по дну яра, то растекается по нему болотом.

К Рогозову яру, на горе по правому берегу речки, примыкает северная часть Сырецкого дендропарка, в основу коего легло цветочное хозяйство, основанное в 1875 году банкиром, немцем Карлом Мейером.

Из документа, дела «по рапортам киевских земского исправника и управы благочиния о появившемся в уезде киевском и городе Киеве скотском падеже» 1790 года узнаем сведения, где какие паслись стада:

\begin{quotation}
стада куреневские, преорские и сырецкие с подольскими на Оболоне по выше Подола и нигде между собою не сближаются, ходят бо куреневские по выше Сырца к святошинскому бору, сырецкие около речки Рогостянки по тамошним полям и чигирам, а преорские в бор к вершине речки Котурки, на Оболонь же никогда до того времени не были гоняемы.
\end{quotation}

На плане 1746 года, близ впадения Рогостены в Сырец, показаны пруд с мельницей.

На 2019 исток\footnote{50°29'00.1"N 30°24'52.0"E} Рогостинки находится в частном секторе, отгороженном от любопытных сетчатым забором. Там этот исток зажат между огородами с севера, с запада гаражным кооперативом «Тюльпан» (около улицы Маршала Гречко), а с юга высотками ЖК «Город цветов», возникшим на месте питомника растений. Дно в приближении к уходу в частные владения усеяно битыми горшками, дореволюционными кирпичами и камнями.\\ 

\textbf{Родники}

50°24'10"N 30°29'23"E

Так жители Ширмы и примыкающей к ней части Краснозвездного проспекта называют студеные родники в склоне горы на улице Кайсарова. На 2015 год там было два обустроенных родника, в 2017 остался один, но еще один появился несколько севернее, прямо из травы (его немножко облагородили).\\

\textbf{Рожница} 

50°25'25.7"N 30°33'46.4"E

В давние времена, священное для язычников урочище в Наводничах. Представляет собой ложбину в склоне холма, с родником, знаменитым на всю округу. Ныне занято АЗС. Подробности в «Ереси о Киеве».\\

\textbf{Романовка} 

50°28'07.2"N 30°29'40.9"E

Террасированные склоны глиняного карьера над кирпичным заводом, на север от пересечения Нижнеюрковской и Отто Шмидта. Заросли деревьями, в том числе хвойными. В окрестностях можно найти старинные кирпичи.

Название происходит от землевладельца, "киевского гражданина" Романовского, который с 1816 года купил в этой местности кирпичный завод, и владел им по крайней мере по половину 19 века. Позже на его месте известен кирпичный завод баронессы Фиркс - усадьбу и завод она купила у наследников Романовского (на 1882 год адрес Нижняя Юрковица, 4).

За помощь в создании на Юрковице Макарьевской церкви баронесса была почетным членом Свято-Макарьевского братства, куда также входили житель Юрковской улицы художник Николай Пимоненко, инженер кирпичного завода Якубенко А. Ф. (непонятно, завода Фиркс или Рихерта), подольский священник Едлинский и многие другие.

Усадьба Романовского включала в себя урочище Юрковицу, где и добывались глина и песок, необходимые для производства кафеля и кирпича.

Урочище Романовка известна также истоком ручья Юрковицы (взят в коллектор, начинается на задворках Нижнеюрковской, 8, где прежде был овраг между оставшимся поныне склоном и горы и соседним, восточным, который был срыт).

На Романовке в верхней ее части, в 1899 году группа археолога Николая Беляшевского   проводила раскопки Кургана-Могикана, с подачи и при участии Антонины Скрыленко (жила на Верхо-Юрковской, 8 в частном доме). Скрыленко была археологом и сотрудничала с Хвойкой, оставшись при этом в тени, хотя по некоторым соображениям роль ее была не меньшей.\\
  
\textbf{Рубежевка} 

Окрестности улиц Стрыйской, Чистяковской, Кулибина. Часть урочище - бывший завод Красный экскаватор.

В конце 19 века тут была детская колония, позже переместившаяся восточнее, где сейчас железнодорожная станция Рубежевский и одноименный ручей.\\

\textbf{Рубежевский}, ручей – приток Сырца. Назван так от земель одноименной колонии для малолетних, переведенной сюда в 1884 году из деревни Михайловская Рубежовка, что была в 35 километрах от Киева. Впадает в Сырец в районе парка Нивки около железнодорожной платформы «Рубежовский».\\

%К колонии также относилась земля, где ныне проходит улица Рубежовская в Новобеличах (это 4 километра от железнодорожной платформы на запад).

\textbf{Рудица} – летописное урочище. В 1097 году: «И прииде Василько в 4 ноября и перевезеся на Выдобич, иде поклонится к святому Михаилу в монастырь, и ужина ту, а товары своя постави на Рудици; вечеру же бывшю, прииде в товар свой».\\

\textbf{Рулетка} – несуществующий ныне, советский фонтан «Дружбы народов» в дальней от Крещатика части Майдана, прозванный Рулеткой из-за внешнего сходства – к фонтану спускались ступени, образуя круг, а вода била тоже из круга, откуда во все стороны глядели эдакие прямоугольные лепестки. Рулетка служила ориентиром для встреч. Примерно там сейчас купол, что за аркой с архангелом.
