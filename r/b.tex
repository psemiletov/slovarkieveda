\chapter*{Б}
\addcontentsline{toc}{chapter}{Б}

%\textbf{Бабин торжок} – летописное урочище, упоминается в Повести временных лет один раз, за 6654 (1146) год, где кроме прочего рассказывается, как избитого, но живого князя Игоря Олговича тащили с Мьстиславля двора "чересъ Бабинъ торжекъ на кн҃жь дворъ и тоу прикончаша и", из чего следует – Бабин торжок находился где-то между двором Мьстислава и княжим двором.

%Далее, вероятно с княжего двора, "ѿтоуда възложиша и на кола и везоша и на
%Подолье на торговище и повергоша пороуганью" – но тождественно ли подольское торговище Бабиному торжку, сказать трудно. После тело уже покойного Игоря перевезли в Федоров монастырь, откуда его перевезли в Семенов (на Копыревом конце).

%Где же был Игорь до Мьстьславлял двора? В той же церкви святого Федора, постриженный в монаха. По Ипатьевскому списку, каменная церковь св. Федора была заложена Мьстиславом в 6637 году, где его потом и похоронили. Церковь была одновременно монастырем.



\textbf{Багриново} – гора на юг от Лысой, между нею и Мышеловкой. Принадлежала прежде, по крайней мере с 1541 года, Выдубицкому монастырю. Западной частью примыкает к Цымбалову валу.

Малоэтажная застройка сохранилась на улицах Ракетной и Панорамной в образе поселка имени Хрущева, или же поселка Октябрьское, построенного однако до Хрущова, в 1948-53 годах, когда были воздвигнуты двухэтажные домики, школа и дом культуры. В поселке есть улица Жигулёвская аж на 4 дома. Западной частью поселок граничит с зеленой территорией Института ядерных исследований.

До постройки посёлка Октябрьский там были земли совхоза фабрики Карла Маркса.

Восточная часть горы, её обращенный к Днепру склон, в 21 веке застраивается ЖК по той же Ракетной и Панорамной, а ближе к Столичному шоссе занят старой промзоной Корчеватое, сожравшей часть склонов своими кирпичными заводами.

На середину 19 века, когда заводы пребывали еще в зачаточном состоянии, на юго-восточ\-ной части горы были арестантские казармы\footnote{Примерно тут: 50°22'55.3"N 30°32'47.5"E}.

Между Корчеватым и Багриново найдены Корчеватский могильник (древнее кладбище площадью 7000 квадратных метров), кости мамонта, а также памятники славянской трипольской культуры.

Из универсала гетмана Ивана Скоропадского 1712 года на владения Выдубицкого монастыря, границы Багриново:

\begin{quotation}
На селище Багрінов над иншие кріпости первейшим короля полского Жигимонта привилеом монастиреві Видубицкому в вічность еще року 1541 потверженное [...] именно взявши од реци Дніпра за Либедю низом Днепра Жуковку, Жуковкою в Попово озеро, тим озером в ричку Лювну, Лювною до яру глубокого, которим яром гаразд попоехавши, мимо Голосіев удатися на гору на дуби аж до дороги з Голосіева идучи, тоею ж дорогою на Дубровку до Вичовки\footnote{См. Бычовка.}. З Дубровки через дорогу в долину Глубокую к Днепру на криницу, з якои жерелце Лукарец идет, Лукарцем вверх озера Лукаріцкого. От верха озера в перевал дніпрвский и перевалом знову в Дніпр.\end{quotation}

\medskip

\textbf{Базарная улица} – в первой половине 20 века, одна из основных улиц на Демиевке, сейчас на ее месте библиотека Вернадского.\\

\medskip


\textbf{Байкова роща} – по начало 20 века, покрывала склон холма между Байковым кладбищем и Протасовым яром. Ныне это окрестности улицы Амосова. См. Клинический городок.\\

\textbf{Байковщина, Байков}, хутор – принадлежал генерал-майору Сергею Васильевичу Байкову (1772-1848), а не генералу Петр Ивановичу Байкову, как некоторые пишут.

Хутор лежал у подножия холма, точно напротив Лабораторной улицы, только через Лыбедь. Окрестности нынешнего адреса Гринченко, 5.

Выйдя в отставку, Байков поселился в Киеве, где некогда лечился в госпитале. Байков жил в собственном доме на Печерске, а при нынешней Байковой горы у него было землевладение. Умер и похоронен Байков в Питере. Николай I передал земли покойного Военно-инженерному ведомству.

Землю же, прослывшую Байковой горой, Байков приобрел следующим образом. У титулярного советника Деминского с 1825 года над Лыбедью был хутор в 64 десятины, с виноградником, фруктовым и шелковичным садами. В апреле 1831 года Байков одолжил титулярному советнику Деминскому деньги, а затем выкупил у него хутор.

По имени Байкова стали называться и роща на горе, и кладбище.

А учитывая близость оных к Демиевке, предположу что от Деминского и Демиевка...\\


\medskip

\textbf{Байково кладбище, Новостроенское кладбище}

Образовано в 1837 году на шести десятинах хутора генерала Байкова, как три новых кладбища – православное, греко-католическое и лютеранское. Мысль устроить его здесь подал в 1833 году митрополит Киевский и Галицкий Евгений Болховитинов, по причине упраздняемых в связи со строительством Новой Печерской крепости кладбищ:

\begin{quotation}
Представляется выгоднейшей возвышенность за Лыбедью, близ хутора генерала Байкова, влево от мостика и тропинки, идущей в гору. Место это, будучи вне города, не может быть впоследствии занято строениями и, состоя почти в средоточии между крепостью и поселениями, по берегу Лыбеди раскинутыми, представляется менее затруднительным при сопровождении мертвых.
\end{quotation}


\medskip


\textbf{База}, хутор

50°26'23"N 30°23'51"E

Клочок частного сектора, прячется за березняком, расположен между Победой и железнодорожной станцией Южная Борщаговка (напротив завода Киевтрактор деталь), от которой на хуторе стоит двухэтажное здание станции. База состоит из менее чем десяти усадеб, лежащих вдоль переулка Жмеринского и железной дороги. Газа и канализации нет. Есть около сторожки «колодец машинистов» – колодец, близ которого останавливаются электрички и машинисты набирают себе воду.\\

\medskip


\textbf{Базарная площадь} – около Черепановой горы, название на середину 19 века. Позже – Троицкая площадь, сейчас тоже. Возле Олимпийского стадиона.\\

\medskip

\textbf{Балашова овраг} – яр, в котором ныне проходит Смородинский спуск. Относился к северной части Загоровщины. Название овраг Балашова проходит на планах 19 века, Балашов или Балаш был майором, позже земля перешла к его дочери. Верховье оврага исчезло при прокладывании Подольского спуска. 

Подробнее читайте в «Ереси о Киеве» в разделе про Логово Змиево.\\

\medskip


\textbf{Балка} – в 1970-е так называлось место, где собирались фарцовщики виниловыми пластинками и меломаны-покупатели. 

По начало 70-х была «пластиночная» толкучка на Беличах, около железной дороги – туда из Киева ездили на электричке или маршруткой. Когда ее закрыли, нелегальная продажа пластинок переместилась в ботсад на бульваре Шевченко (по вечерам в воскресенье?). Эту виниловую барахолку назвали Балкой, потому что с аллеи, которая у всех на виду, фарцовщики спрятались в овраг, в балку.

В середине 70-х фарцовщиков оттуда изгнали, и Балка разделилась на две, Дневную и Вечернюю.

На Дневную Балку сходились на аллее под тополями на бульваре Дружбы Народов, около дома номер 14, где на первом этаже располагался магазин грампластинок «Мелодия».

Вечерняя Балка собиралась по вечерам у магазина «Грампластинки» напротив кинотеатра Киев.

В начале восьмидесятых Балка снова переместилась в ботсад. 

В девяностых – к железной дороге у Кардач, ближе к улице Полевой.\\

%Батыевы ворота
%см закревский том 1 стр 201

\medskip

\textbf{Барабан} – народное название Печерского универсама.\\

\medskip

\textbf{Бассейный овраг} – на 1870-е года, овраг от Бессарабского рынка до русла речки Клов, пролегал точно по нынешней улице Бассейной и соединялся с Кловом примерно около Дворца Спорта. По оврагу протекал ручей, что начинался около Майдана.\\

\medskip

\textbf{Бассейн Урод} – располагался, по крайне мере в 1871 году, на нынешнем Майдане, на площади, по четной стороне Крещатика, на уровне фасадов зданий, ограничивающих площадь по бокам. Название обозначено даже на картах.\\


\medskip


\textbf{Батыева гора, Батыева могила} –  гора на север от Протасова яра, в сторону Кучмина яра и вокзала.

На части Батыевой горы раньше было кладбище – на карте 1847 года оно обозначено у перекрестка нынешних улиц Проводницкой и Общественной\footnote{50°25'55.9"N 30°29'49.7"E}, и оттуда по холму на северо-восток до самой железной дороги. По 2016 год на склоне между Проводницкой до ЖД было несколько замусоренных террас с деревьями и травяной поляной, с 2016 внизу стало расти здание новостроя. Можно сказать, что место кладбища по 2016 год оставалось незанятым, ибо даже в частных усадьбах, где расположена было южная часть кладбища, дома стояли по обочине места кладбища, а не по самим могилам.\\

\medskip


\textbf{Батыевы могилы}

Судя по всему, это урочище следует отличать от Батыевой горы-могилы – находится оно по другую сторону Лыбеди, примерно в пяти километрах на юго-восток.

Закревский в своем «Описании Киева» 1868 года издания пишет:

\begin{quotation}
\noindent Куча курганов за Лыбедью, близь Васильковской дороги, носит название Батыевых.
\end{quotation}

За 1884 год в «Путеводителе» Тарановского сохранилось краткое описание тогдашней Демиевки:

\begin{quotation}
Чрез предместье Демиевку идет большая почтовая дорога, называемая Васильковский тракт, ведущая в город по плотине, через историческую реку Лыбедь. Плотина пересекается железной дорогою и около пересечения находится деревянный мост; от этого моста до Печерской лавры считается: по кратчайшей, изрытой глубокими рытвинами, дороге\footnote{Дорога не сохранилась и лежала как бы наискосок относительно бульвара Дружбы, начинаясь вдоль ОкеанПлазы, от современного места впадения Совки в Лыбедь.}, выходящей на Печерск через гору, мимо Васильковского укрепления и башни 4 версты 280 сажней. По ней проходят богомольцы, идущие прямо в Лавру. [...]

При въезде в город, как оканчивается плотина, направо находятся кирпичные заводы Субботина и Шатовой\footnote{Нынешнее озеро Глинка частично занимает место бывшего глиняного карьера завода Субботиной.}, возле которых, по берегу ручья Лыбедь, идет шоссированная дорога к товарной станции Киев II. В усадьбах этих заводов, а также в усадьбе Чернышова, что при въезде влево\footnote{Окрестности станции метро «Лыбедская».}, находятся довольно высокие, покрытые садами и лесами, курганы, называемые Батыевыми могилами. Из этих кирпичных заводов более известный первый, называвшийся Эйсмановским\footnote{Эмилия Субботина – в девичестве Эйсман, внучка основателя завода, аптекаря Иоханна (Ивана) Федоровича Эйсмана и дочь городского головы Густава Ивановича Эйсмана, вышедшая замуж за Виктора Андреевича Субботина, профессора Университета святого Владимира.}. 
\end{quotation}

Таким образом у озера Глинки и вообще окрестностей Лыбедской площади и следует понимать урочище Батыевы могилы, сами которые вероятно давно срыты и неразличимы в застройке.\\


\medskip


\textbf{Барселона}, 1960-е – жаргонное название шашлычной возле Республиканского, тогда Центрального стадиона (а ныне Олимпийского). Находилась на Красноармейской улице, потом там построили дом с магазином строительной книги на первом этаже.\\


\medskip


\textbf{Бегичевская гора} – название было в ходу в первой половине 19 века. Холм где стоит Институт благородный девиц – Октябрьский дворец, между Институтской, Крещатиком и Грушевского. Улица Институтская называлась прежде Бегичевской, и по месту адреса Институтская 1 была усадьба семьи Бегичевых, которые жили тут с 1782 года. Генерал-поручик Матвей Семенович Бегичев (1724-1791) был военным инженером, руководил арсеналом и заведовал реконструкцией Печерской крепости. Переводил с немецкого военные и философские труды. Похоронен был в Успенском соборе Лавры.

Его сын, генерал-майор Дмитрий Бегичев (1768-1836), в 1834-е передал усадьбу для нужд Киевского университета, а киевский военный губернатор Левашов предложил использовать ее для устройства там женского учебного заведения. Спустя 4 года началось обучение девушек, сначала в одноэтажном здании на углу Липской и Институтской, а в 1839 в усадьбе Бегичевых заложили здание Института благородных девиц, по проекту  В. И. Беретти. От института улицу стали называть Институтской, а иногда Девичьей (это название ходило в середине 19 века).

До того усадьба представляла собой гору с парком и трехэтажным особняком, где собирался «философический кружок». Бегичев-младший был масон, магнетизер и мистик, и гости ему под стать. По некоторым сведениям, собрания проходили и/или в другом тогдашнем имении Бегичевых, Кинь-Грусти. Собирались также в доме другого члена Киевского тайного общества, генерала Николая Николаевича Раевского, который жил в Киеве в начале 19 века. Раевский – друг Пушкина, брат Давыдовых.

Прежнее название Бегичевской горы – Иванова гора, от Ивановской дороги, ставшей улицей Бегичевской...

Н. Тумасов в статье 1885 года «История Киевской второй гимназии» писал о здешних местах – а напомню, мы сейчас рассматриваем изображение 1850 года:

\begin{quotation}
50 лет назад Киев имел иную физиономию, не ту, что теперь. Крещатик и Лютеранская гора состояли из трех-четырех каменных домов, между которыми там и сям виднелись незначительные домики, все, разумеется, одноэтажные. Александровского спуска, Царско-садской и Институтской улиц не существовало.

Где теперь женский институт, там 50 лет назад были дебри, среди коих поднимался деревянный, неоконченною постройкою дом генерала Бегичева, пристанище бродяг, наводившее трепет на запоздалого путника. Гора Михайловского монастыря соединялась с Царским садом, что препятствовало стоку крещатицкой воды в Днепр.

Один старик недавно рассказывал нам, что помнит то время, когда на месте нынешней думы находилось озеро, а на нем мельница. После проливного дождя киевляне катались по Крещатику в лодках – в 1845 году целых две недели, пока не прорыли нынешний Александровский спуск.
\end{quotation}

Необходимо сделать несколько пояснений. Александровский спуск – нынешний Владимирский спуск. В статье сказано, что Владимирская горка была соединена с противоположной, там где филармония у подножия. Но это не соответствует планам Киева 18 века, где показана какая-никакая, но дорога по месту Владимирского спуска.

Что до озера с мельницей посреди будущего Майдана – охотно верю, диггеры хорошо изучили ручей, что под Крещатиком следует к другой заточенной под землю речке – Кловице, более известной как Клов.\\

\medskip


\textbf{Беличье поле} – местность к югу от Замковища, примыкающая к улице Замковицкой. На 2017 год застроена частным сектором. Название, вероятно, от слова «белица» – так называли жительниц монастыря, не принявших однако монашеский постриг. Черницы – принимали, белицы – просто жили при монастыре.\\


\medskip


%УТОЧНИТЬ!!!!!
%\textbf{Белое озеро}  

%50°27′34″N 30°20′25″E 

%Заболоченный пруд на остатках Святошинского ручья, через улицу от Прилужной, 14-А.\\


\textbf{Белое озеро} – от него остался кусочек, примыкает к заливу Верблюд с севера. На картах, на Оболони показано другой Белое, по месту бывшего пролива Ицун, омывавшего Чачин остров с запада.\\

\medskip


\textbf{Белоцерковская дорога} – шла вместе с Запольской дорогой к Вете, через Невеселовское поле, потом на Васильков и Белую церковь.\\

\medskip

\textbf{Бережанский рынок}

50°30'56"N 30°27'25"E

Некогда толкучка на улице Бережанской (по адресу 15), на 2020 пустующая территория, огражденная мафами.\\

\medskip


\textbf{Берестове, Берестовѣ, Берестовоє} – урочище, примыкающее к северу Лавры. Именно тут, вопреки мнению науки, располагался княжий двор (по крайней мере по Ярослава Мудрого). Помимо двора с собственно престолом, на Берестове было одноименное сельцо, где жило 200 наложниц Владимира. Нестор пишет:

\begin{quotation}
\noindent на Берестовем в сельци еже зовут и ныне Берестовое
\end{quotation}

Положение летописного Берестова вычисляется, кроме прочего, по положению церкви Спаса на Берестове, построенной Владимиром Красно Солнышко, и восстановленной при Петре Могиле в 1640 году. В книге 1638 года издания монах Афанасий Кальнифойский писал в легенде к карте, приложенной им к труду «Тератургима»:

\begin{quotation}
Между западом и севером дорога идет через кат. (катедру?) Спасcкую, то есть мимо церкви Преображения Господня, которую Владимир построил. Но её стены едва стоят сейчас, руины покрыты землёй, и по положению к той же Святыне Господней относятся.
\end{quotation}

Тем же путем Кальнифойский выводит читателя, далее, к монастырю Николая Пустынного, то бишь ранее речь в самом деле идет про разрушенную церковь, известную Кальнифойсому как Спаса на Берестове.

Всегда закрадывается мысль, что если нет непрерывного существования объекта или пользования им, если он не на устах и не письменно непрерывно на протяжении веков, то – верно ли при Петре Могиле трактовали развалины некой церкви?

Как бы ни было, урочище находилось где-то там, и точно там, если развалины относились именно к церкви Спаса.

В Повести временных лет за 6559 (1051) год, в Ипатьевском списке сказано:

\begin{quotation}
боголюбивому князю Ярославу . любяще Берестовоє . и црьквь сущоую святыхъ апостолъ . и попы многы набдящю . и в них же бѣ прозвутерь именемь . Ларионъ . мужь благъ и книженъ . и постникъ . и хожаше с Берестового . на Дьнепръ . на холмъ . кде ныне ветхыи манастырь Печерьскыи . и ту молитвы творяше . бѣ бо лѣсъ ту великъ . иськопа ту печеръку малу . 2 . саженю . и приходя с Берестового .
\end{quotation}

Ларион ходил с Берестового «на Днепр», на холм, где при летописце стоял ветхий (старый) монастырь, молился там, и летописец поясняет, что там был большой лес, бѣ бо лѣсъ ту великъ.

Значит, при Несторе леса, по крайней мере большого, уже не было, а Берестове соседствовало с холмом, «где ветхий монастырь», который ныне известен как Дальние пещеры.

Какой холм примыкает к холму с Дальними пещерами? Соседний, там где основная Лавра, с Ближними пещерами, и где вне Лавры церковь Спаса, а еще севернее Аскольдова могила и был Николаевский монастырь.

Ларион ходил с холма с Берестово, где при хождении Лариона еще не было Ближних пещер и Лавры.

В широком смысле, урочище Берестове можно считать верхом горы, известной как урочище Угорьское. Подробнее читайте в «Ереси о Киеве». 

Основная часть Лавры заняла часть южную часть Берестово, как бы откусила от урочища кусочек. Кусок. Кусище.\\

\medskip


\textbf{Беретти Александр Викеньевич}

Родился 5 октября 1817, умер 6 июня 1895. Сын архитектора Викентия Беретти.

В 1827 году поступил в Питере в Академию Художеств, в 1839 году стал «назначенным», в 1840 получил звание академика за «проект кадетского корпуса на 100 человек».

В Киеве преподавал архитектуру в Университете святого Владимира. При генерал-губер\-наторе Бибикове спроектировал:

\noindent • Собственный двухэтажный особняк на Владимирской улице, 35. 1848 год, существует поныне. Слева от дома был сад с беседкой, а во дворе колодец. В 1858 году Беретти продал дом с усадьбой чиновнику Протоповову.

\noindent • Первую киевскую гимназию (1850-53), сейчас  это гуманитарный корпус Университета Шевченко (бывший св. Владимира).

\noindent • Анатомический театр Киевского университета Св. Владимира (1851-1853), ул. Богдана Хмельницкого, 37.

\noindent • Здание Дворянского собрания (1851 год, снесено в 1976), известное как «дом Понятовского».

\noindent • Пансион графини Левашовой (1850-е), ныне здание Президиума НАН Украины, Владимирская 54.

\noindent • Реальное училище (1850-е, ныне Дипломатическая академия Большая Житомирская 2.

\noindent • Владимирский собор – строительство в 1862 году началось по проекту Штрома и Спарро, под местным руководством Беретти, который внес в проект изменения, приведшие в 1866-м к тому, что выстроенный уже до куполов собор дал трещины, и стало ясно, что водружение куполов совсем разрушит его. Из Питера вызвали Штрома, он показал ошибки во внесенных изменениях, и Беретти отстранили от проекта. А собор достраивали аж в 1870-е, ибо Штром на ошибки указал, а поправить дело не брались, покуда Александр II не прогневался, увидев заброшенную стройку. Из Питера вызвали Рудольфа Бернгарда, который провел нужные расчеты и дал выкладки, как всё починить и достроить. 

После неудачи с собором Беретти практически не работал по специальности. Еще ранее, в 1860-м, Беретти руководил начальной реставрацией фресок Кирилловской церкви, да так плохо руководил, что часть фресок загубили, после чего-то дело и поручили Адриану Прахову.

Беретти умер в 1895 и похоронен на Байковом.\\

\medskip

\textbf{Беретти Викентий}

Викентий (Винченцо) Иванович Беретти (14 июня 1781, Дамазо-де-Урбе, Рим – 18 августа 1842, Киев) – архитектор, академик архитектуры и профессор Императорской Академии художеств. Сын уроженца Италии, бриллиантовых дел мастера Джованни Беретти, переехавшего в Россию в 1780-е годы.

Киевские работы:

\noindent • Астрономическая обсерватория Киевского университета (1841-45), Обсерваторная 3-В. 

\noindent • Институт Благородных девиц  (1843).

\noindent • Первая мужская гимназия (с 1959 года Желтый, гуманитарный корпус Университета Шевченка) (1847, гимназия въехала туда в 1857, а до того помещался Владимирский кадетский корпус и была казённая квартира попечителя Киевского учебного округа). Гимназия, позже названная Александровской, служит одним из мест действий «Белой гвардии» Булгакова.

\noindent • Главный корпус Университета Святого Владимира в Киеве (проект 1835 года).

\noindent • Планировка Владимирской улицы. 

\noindent • Планировка Бибиковского бульвара.

\noindent • Планировка ботсада при Университете.

%Завершение строительства костёла св. Александра (освящение в 1842), Костельная 17.

\noindent • Укрепление Золотых ворот.\\

\medskip



\textbf{Берёзка} 

50°26'26"N 30°27'30"E

Летний кинотеатр в Кадетской роще, на 2017 год представляет собой покрытое трещинами кирпичное здание, используемое в хозяйстве шиносервиса и автостоянки.\\

\medskip


\textbf{Берковцы, Берковец, Бирковец} – большой район на северо-западной окраине Киева, состоит из частного сектора, дач и кладбища. Примыкает к лесу Пущи-Водицы. Название Берковцов происходит, вероятно, от еврея-арендатора Берка.

Берковцы сейчас занимают значительное пространство, а на первую половину 20 века сосредотачивались у пересечения нынешних улиц Стеценко и Городской, то бишь где АШАН, ЛавинаМолл, и дачный кооператив 40-летия Октября. 

Оттуда же к западу, среди дач между Стеценко, Синеозерной и Газопроводной, прямо по дачным участкам лежит спокойный, около метра шириной, мутный исток речки Котурки, что течет потом через лес на север к Пуще.

Добираются на Берковцы, в зависимости от их части, по Большой Окружной маршрутками от Академки (до Ашана), либо по Стеценко от Нивок и Сырца.

Лес около Берковцов, не знаю с какой стороны, в 19 веке назывался Братским. Тогда же, не знаю, существует ли он сейчас, был в самом Берковце некий длинный и глубокий ров, слывший как урочище Шалена Баба.\\

\medskip


\textbf{Берлизовы огороды} – местность Изюмского рынка, близ Демиевской площади. Названы так от бывшего землевладельца, дворянина Гаврилы Берлизова и его сына. Раньше, в 1950-х, сюда ходил трамвай №10 и была одноименная остановка. Берлизовым принадлежала также мельница, сохранившаяся на время выхода этой книги, слева от ДК Батюка, по координатам:

50°24'28.7"N 30°30'28.1"E\\

\medskip

\textbf{Бермудский треугольник} – здание дворца бракосочетаний на проспекте Победы, 11. Построено в 1982 году на месте пустыря, странным образом почти совпадающего с новым зданием своей треугольной формой.

Между пустырем и проспектом было несколько жилых домов, по наше время не сохранились. Еще раньше, на стыке 19-20 веков, в той околице находились вкопанные в землю круглые резервуары – склады керосина, они примыкали к железной дороге, от них дорога вела на север к Брест-Литовскому шоссе (проспекту Победы, часть шоссе была переименована в проспект). Вокруг стояли сараи, потом их снесли и к 1960-м там-то и был упомянутый пустырь.\\

\medskip

\textbf{БЖ} – Большая Житомирская, но под БЖ обычно подразумевают аллею между самой улицей и склоном горы, до лестницы в Десятинном переулке. БЖ и деревянная лестница лестница некоторое время служили местом собрания нефоров. В 2018 году вместо деревянной лестницы соорудили другую, очень дорогую, о которой и писать не хочется. И не буду.\\

\medskip


\textbf{Бернер} – так жители Демиевки называли, вплоть по послевоенное время, озеро в карьере кирпичного завода Бернера. Озеро находилось на современной Лыбедской площади, занимая кусок станции метро Лыбедская и участка с «Океан Плаза».\\ 

\medskip

\textbf{Блощаговка} – одно из местных названий Борщаговки.\\

\medskip

\textbf{Божков яр} 

50°24'55"N 30°29'43"E

Здоровенный яр на север от Монтажника, и на юг от Байкова кладбища. Подпирается чуть ли не со всех сторон гаражными кооперативами. Постепенно засыпается землей и мусором. На дне яра существует довольно мощный ручей, приток Совки. К яру подходят улицы Гаевая и Божков яр. 

По дну яра протекает ручей (приток речки Совки), образуя кое-где значительные по площади заводи. В пойме буйная и нетронутая растительность. На северном, наиболее крутом и высоком берегу стоят деревья, вообще под Байковым кладбищем находится серединная, наиболее дикая часть яра.

Подробно о Божковом яру читайте в моей книге «Речка Совка и ее притоки». Впервые встречаю название Божков яр на карте середины 19 века. Возможно, название Божков яр это искаженное Байков яр, ибо он служит границей Байковой горы, отделяя ее от Забайковья.\\

\medskip

\textbf{Богданов яр} – относится к Соломенке или граничит с нею, как угодно. В нем пролегает улица Богдановская (с 1928 года, прежде Лермонтовская), Калининградская, Стадионная, Шовкуненко. Яр нисходит, условно говоря, от высот Соломенки, Соломенской площади в сторону вокзала. В 20-е годы 20 века яр застроили частным сектором, который снесли в 1970-80 и построили многоэтажки.

По дну яра протекал, и ныне течет в коллекторе, ручей, приток Лыбеди.\\ 

На середину 19 века в низовьях яра, примерно тут – 50.43765964737409, 30.489083528062327 – находилось небольшое кладбище.
\medskip

\textbf{Богданов ручей} – протекает под землей в коллекторе по одноименному яру, название ручья дано мною, поскольку уж никто его толком не именует. Диггеры считают его «веткой» реки Мокрой. В 19 веке эти водотоки соединялись где-то в месте нынешнего пересечения Урицкого, Толстого и железной дороги, там был мост, и потом общее русло уходило к Лыбеди на юго-восток примерно к тому месту, куда теперь доходит южный конец улицы Эренбурга.

Сейчас коллектор спрямлен параллельно Толстого и впадает в Лыбедь за домом Семьи Праховых, 6, то есть присоединяется к Лыбеди выше по течению, чем было давнее устье двух притоков.\\

\medskip

\textbf{Болото} – по 1970-е, урочище на Лукьяновке, верховье Глубочицы. На Ютубе, BigFire505 в 2018 году сообщил мне:

\begin{quotation}
Переулочек возле дома 28 и 28-Б по Овручской тоже не совсем обычный. Если посмотреть на одну из старых карт, где р. Глыбочица течет еще по поверхности, можно увидеть, что ее исток находится возле этого переулка. А точнее под самой спортплощадкой СШ1. 

В моем детстве там был виден еще коллектор, в котором журчала вода. Между школой и домами по улице Овручской и Тропинина шел глубокий яр, заросший коноплей, а по краям его стояли огромные вековые дубы. 

Чуть дальше по яру ручей выходил на поверхность и тек примерно до конца Делегатского переулка. Яр в этом месте расширялся и посреди его было небольшое озеро, а простонародье – Болото. 

У нас это было как название географического пункта, потому что оч\-ень круто было идти в школу с ул. Январской (потом Буденного, потом снова Баггоутовской) не через Делегатский, а болотом:))

На этом болоте мы устраивали морские бои на плотах из деревянных кругов от огромных катушек Кабельного завода. Катушки эти тогда валялись везде. Они легко разбирались и все детали от них шли в дело. Даже металлические шайбы от болтов служили нам битами в игре в «бляхи». Бляхи — это пробки от бутылок. Каждая имела свою цену: Берзовская, к примеру шла 10 к 1, а редкая тогда Пепси стоила 300 очков:))) На месте этого болота теперь автопарк коммунальщиков. Из болота ручей уходил под землю и вытекал снова в Кмитовом Яру. На этом месте сейчас котельная с огромной трубой. [...]

Кстати, на месте яра возле Овручской стоят гаражные кооперативы. Под них тогда и срубили дубы. Так вот как раз над тем местом, где проходит ручей, гаражи наклонились друг к другу по всему проходу.
\end{quotation} 

На Болото можно было добраться по Делегатскому переулку. Старожилы рассказывали, что некогда там было не болото, но озеро, питавшееся от ключей, а по воде плавали даже лебеди.

Замечу однако, что до устроения Подольского спуска, а точнее на 1861 год, исток сего ручья (назовем его Верхний Исток), питающего Болото, оказался бы сейчас к западу от Подольского спуска, а именно около одного из сооружений водопроводной станции \footnote{50.47477771982014, 30.474820341989776}, оттуда протекал ручей к современному перекрестку Овручской и спуска, там на перекрестке в него вливался еще ручеек, и сводный поток уходил овражком в самом деле где школьная спортплощадка. Затем в районе гаражного кооператива\footnote{50.47268155841699, 30.479932449062556} к нему присоединялся ручеек с севера от адреса Тропинина, 5 (тогда и улицы такой не было).

Далее ручей следовал до задворок Цветущего переулка 12, где в него с северо-востока струился ручей примерно от северного крыла Института автоматики.

От задворков ручей тёк через Цветущий переулок, вдоль (с востока) его отрезка что идет к Отто Шмидта, и потом почти строго на юг параллельно улице Кмитов яр, начиная петлять уже согласно изгибам яра, а потом поворачивал там где расположен завод Артема.

Примечательно, что на карте 1861 года никакого ручья и прудов от нынешней Дачи Хрущова нет, хотя показаны мельчайшие ручейки, описанные мною выше. 

А ручей в месте Дачи Хрущова есть на карте 1874 года, равно как и ручей Верхний Исток, выходит это два разных ручья, а сходились они у крайней северной точки завода Артема, 50.46475366350877, 30.481532948581798 – то есть где улица Кмитов яр поворачивает и овраг принимает в себя корпуса завода.\\

\medskip

\textbf{Большой Николай} – на месте советского Дворца Пионеров был Никольский военный собор, или Большой Николай, основанный Мазепой в конце 17 века. Военным собором он стал в 1831 году. Пожалуй, наиболее заметным строением при нем была трехъярусная колокольня, что находилась примерно где сейчас гостиница «Салют». Колокольня построена в 1750 году средствами епископа Смоленского Гедеона. Снесена вместе с собором в 1930-е. Впрочем, колокольня сильно пострадала прежде, во время бомбардировки в 1918 году – тогда снесло купол и верхнюю часть, и ее принялись восстанавливать, но был ли завершен ремонт – не знаю.\\

\medskip

\textbf{Босяцкий} – продуктовый магазин на месте дома по Дмитриевской, 1. Там раньше, по 1980-е стоял двухэтажный дом, с этим гастрономом на первом этаже. Чуть дальше от входа, по улице Менжинского, была зарядка сифонов.\\

\medskip

\textbf{Борщовка, она же Нивка} – река, протекающая вдоль Борщаговок и Беличей. Длина около 20 километров. Начинается к западу от ипподрома, на окраине ВДНХ, южнее павильона 24\footnote{\textasciitilde{} 50°22'11"N 30°28'27"E}. Впадает в Ирпень. На пути к нему образует множество больших прудов. Заслуживает более обширной статьи.\\

\medskip

\textbf{Ботаника, Боташа, Ботсад} – ботанический сад на Зверинце, название местное для Бастионной и окрестностей.\\

\medskip


\textbf{Братское кладбище} – находилось на территории нынешнего Института проблем прочности, Тимирязевская 2. 

На нем с 1916 года хоронили погибших в Первой мировой. Разграблено еще до строительства института, который уничтожил кладбище окончательно, хотя от него осталась недостроенная Братская церковь – храм Николая Чудотворца. Приспособлена, без купола, под нужны института.

Подробнее читайте в моей книге «Ересь о Киеве».\\

\medskip

\textbf{Братское кладбище} – ныне существующее, находится в Братской Борщаговке (улица Григо\-ровича-Барского), название обоих происходит от Братского монастыря, владевшего этой землей и селом Братская Борщаговка на ней. Ныне считается Южной Борщаговкой. Кладбище изначально было сельским.\\

\medskip

\textbf{Братское кладбище} – было на месте нынешней телебашни. Еврейское находилось в стороне.\\

\medskip

\textbf{Брест-Литовский} – дореволюционное, да и местное кое у кого поныне, название проспекта Победы.\\

\medskip

\textbf{Брёвнышки}

50°25'15.4"N 30°33'04.6"E

Покатый склон над местом, где улица Болсуновская (Струтинского) вливается в бульвар Дружбы Народов, несколько выше, ближе к улице Кургановской. Там долго, по нулевые, сохранялся осколок частного сектора. На 2021 год частный дом на Кургановской 7 снесен, и между Струтинского, Кургановским переулком и выросшей на месте снесенного домика высоткой остался заросший деревьями пригорок, южной частью представляющий собой ничейный, с деревьями пустырь со следами построек и некоего вала, а северная часть – некая частная усадьба с небольшим домиком. Вот там-то и находились Бревнышки, куда люди приходили посидеть-отдохнуть.

На месте трех советских панелек по адресу Кургановская, 3, примыкающих к пустырю, раньше была горка выше пустыря, может и впрямь курган или какое-то городище.

Еще севернее, в низинке в устье улицы Струтинского, там где черная высотка ПечерСкай, стоял частный дом, где жила бабка, у нее было стадо коз. Козы паслись на лужайке вдоль трассы как раз под горбом Бревнышек.

Севернее упомянутого перекрестка лежало Святое озеро, о нем читайте в отдельной заметке.\\

\medskip

\textbf{Бульонка, Бульонная слобода} – так в середине 19 века называлась местность между Кловским ручьем и Госпитальным укреплением. Домики, частные усадьбы, что лежали вокруг современных низовий бульвара Леси Украинки и ее стыка с улицей Мечникова.\\

\medskip

\textbf{Бурса} 

50°28'03.5"N 30°31'26.3"E

Общежитие для бурсаков, учащихся Могилянской академии. Каменное задние, ныне один из корпусов Могилянки, имеет адрес Набережно-Крещатицкая, 9. 

Первоначальное здание Бурсы было деревянным, оно сгорела 1 августа 1775 года. Деревянная бурса располагалась, по словам Закревского, вот где – «на сем месте ныне в каменном здании помещаются духовныя приходские и уездные Подольские училища». 

Закревский же пишет, что каменную бурсу построили на том же месте после пожара, сначала один этаж, а в 1816 второй.

Однако Аскоченский в «Киев с его академией» сообщает:

\begin{quotation}
Лука Белоусович подарил академии свой двор, находившийся на самом берегу Днепра, близ церкви Николая Набережного; академия, прикупив к тому еще три двора у полковника Иоакима Кононовича, заложила в 1760 г. каменную Бурсу, оконченную и освященную в 1765 г. Первоначальная Бурса была близ Братской Богоявленной церкви, с южной стороны, между улицами, идущею мимо монастыря и Духовскою, которая шла от Свято-Духовской церкви, стоявшей на самом берегу Днепра, там, где недавно был дом дворянина Холодовского.
\end{quotation}

\medskip

\textbf{Бусловка} – речка, левый приток Лыбеди, ныне спрятана в коллектор. Имеет два истока, западный и восточный.

Западный протекал по оврагу, который примерно от Дома мебели на бульваре Дружбы Народов выруливал к оврагу улицы Киквидзе и вливался в него, а там присоединяется еще восточный исток с горы Бусовой.

Коллектор западного истока начинается за седьмым домом на бульваре Приймаченко, между ним и стоящей на пригорке 84-й школой. Около домов 5 и 6 коллектор пересекает бульвар и вдоль улицы Неманской обходит Дом Мебели с северо-востока.

Если встать у Дома мебели лицом к бульвару Дружбы народов, то станет очевидным, что мы находимся в ложе оврага. Раньше тут сходилось несколько приярков, теперь они сильно сглажены. Там же находится станция метро «Дружбы Народов». 

Это верховье оврага западного истока Бусловки в 19 веке отделял Православное кладбище от Немецкого и Католического, о чем вы подробно читали в разделе про Автостраду.

...Пройдя под бульваром Дружбы народов, у его перекрестка с Неманской, коллектор Бусловки пересекает эту улицу между домами по бульвару Дружбы народов 30/1 и 28 (за углом НИИ), откуда сворачивает на юго-запад за НИИ, во дворы райончика хрущовок, где поворачивает на восток около дома номер 3 (стоит в ощутимом котловане) и вдоль улицы Подвысоцкого, а напротив 3А сворачивает на юг и следует мимо домов Драгомирова, 8 (затем пересекает Драгомирова) и Киквидзе 10А.

За 10А в направлении 18Б – глубокий (хотя несопоставимый конечно с Репяховым яром например, а так, «яма») тенистый овраг, в коем стоит дом 18Б. Между ним и крутым склоном, где прорыты погреба, вдоль западного берега (т.е. напротив дома) идет широкий, безводный желоб из бетонных плит, такой же желоб спускается в него по горке от улицы Драгомирова.  

Люки вдоль желоба в овраге выдают и подземную часть коллектора. Желоб проложен дальше оврагом в направлении юго-востока, находясь к западу за домом Драгомирова 18А, а потом следует уже только под землей к западу же от домов 20, 22, 26 по Киквидзе, проходит между 26 и 28, затем под каштанами между шоссейной частью Киквидзе и домом 28А, пока не подходит к шоссейной части возле дома №30, около остановки «Институт».

Затем коллектор выруливает к железной дороге и пересекает Железнодорожное шоссе, сохранился его короткий, осушенный наземный участок\footnote{50°24'11.5"N 30°33'11.7"E}, а затем впадает под землей в Лыбедский коллектор южнее эстакады низовья улицы Киквидзе.

На уровне низовий улицы Киквидзе, эдак дома 34, а может быть и раньше, выше, восточнее, с этим руслом Бусловки что «от бульвара Дружбы Народов» должны были соединяться Бусловские ключи – ручьи водосбора с лежащей восточнее Бусовой горы, среди них и знаменитый источник Бусловка, да собственно и сама гора дала название ручью, во всяком случае ручьем или речкой Бусловкой считался уже тот водоток, что шел от низовий Киквидзе к Лыбеди.

В первой половине 19 века, за домом номер 34, на Бусловке был пруд.

Пришло время поговорить о восточном истоке Бусловки.

Ныне коллектор восточного истока начинается около Печерского моста, напротив западного угла дома номер 1 по Бастионной, и потом идет всё время вниз под улицей Киквидзе. 

В прошлом, этот исток начинал прослеживаться, по картам, много ниже. Обо всем по порядку.

Источник или родник Бусловка, славился вкусной водой, откуда водовозы ее доставляли на продажу по всему Киеву. Эту воду городской голова Эйсман предпочитал пить даже во время эпидемий холеры. К слову, между хутором Эйсмана в пределах нынешнего озера Глинка, и источником Бусловки лежала Черная гора.

На месте современного Универа транспорта (Киквидзе 40), там где здание напротив бювета и остановки «Источник», а также на месте общаги по Киквидзе 36,  на ручье из родника был пруд, еще в 20 веке.

Между домами 35 и 37 по улице Киквидзе участок склона Бусовой горы, на 2019 год, занят истинными трущобами, иного слова не подберешь. По крайней мере по весну 2017-го там был идущий наверх переулочек, а внизу стоял, с конца нулевых, двухэтажный дом с мансардой, гаражом, и на первом этаже располагался магазин садовой техники и электроинструмента. Почему это всё снесли и разрушили, я не знаю, но к 2019 году был обнажен фундамент сего дома, и рядом с остатками внутреннего гаража, из трубы течет сильный родник\footnote{Примерно 50°24'28.0"N 30°33'09.9"E} и уходит в коллектор. Вода в этом роднике очень вкусная (я пробовал) и славна этим среди обитателей трущоб. Мне сказали, что она вымывает из склона серебро, это очищенная серебром вода.

В десятке-полтора метров от этого родника на север, за общагой по адресу Киквидзе 35, между домом и подпорной стеной, откуда сочится вода, есть массивная металлическая фиговина, накрытая доской. Если отодвинуть ее, обнаружится спуск в широкую трубу, с металлическими ступенями. Труба затоплена, в нее течет мощный поток – другой родник.

Сей родник или первый, или оба, и есть тот источник, в котором брали воду водовозы. При наложении точной военной карты 19 века фон Руге на спутник, «источник Бусловка» и оба родника совпадают. Также видно, что пруд образуется на дальнейшем русле от этих родников, а после пруда идет короткое русло и соединяется с руслом западного истока, однако со смещением относительно современного коллектора. Если не ошибаюсь, современный коллектор как бы подтащил то, западное «дружбонародовское» русло ближе к шоссейной части Киквидзе.

Однако чуть далее по улице, за домом номер 41, есть бювет, и рядом остановка 38 троллейбуса под названием «Источник» (вероятно соответствует советской остановке «Общежитие» 15-го троллейбуса) – так что возможно, знаменитый бусловский источник был именно на месте этого бювета.

Речка Бусловка названа так по горе Бусловке или Бусовой, или гора по речке, неясно. Возможно, именуется от слова «бусел», означающем птицу аиста.

На некоторых картах, например 1865 года Киева и окрестностей, Бусловкой подписано озеро Святое на Зверинце, у перекрестка улицы Струтинского и бульвара Дружбы Народов.

В некоторые годы 19 века, по крайней мере в его середине, Бусловка не впадала в Лыбедь, а была пущена прямым каналом, параллельно Лыбеди, вдоль Бусовой горы к Днепру, вероятно для нужд лаврских кирпичных заводов по течению этого канала. Рукотворное устье в таком случае находилось в месте нынешней автомобильной развязки около станции метро Выдубичи, примерно тут:\\ 50.40084041264537, 30.561896916294366\\

\medskip

\textbf{Бусловка, Бусова, Бусовица} – гора на Зверинце, в окрестностях речки Бусловки, лежит по левому берегу Лыбеди, напротив зверинецкой Лысой (Девичь) горы. Примыкает ко главному холму Зверинца, отделяясь от него низовьем улицы Тимирязевской. С другой стороны границей горы служит улица Киквидзе. 

Впервые упоминание об этой горе я встретил в не\-опубликованном (мне известно, подчеркиваю, упоминание) письме 21 марта 1615 года игумена Выдубицкого Антония Грековича «с позволением орать поле на гору Бусовицi» Гришке Охременковому.

Бусова гора по 21 век была застроена частным сектором, сейчас же там сплошь терема, посольства да высотки. По главной улице горы, Бусловской, на 2017 год осталась впрочем некая деревянная хата, да несколько частных домов не шибко роскошных. А так – всюду охранники выглядывают из будок.\\ 

\medskip

\textbf{Бусловка}, хутор – из справочника 1884 года: «находится в овраге около Саперного лагеря. Здесь на ручье, текущем от знаменитого Бусловского колодца, устроен пруд и около него огород с домиком, а повыше, при дороге, старые деревянный дом, прежде бывший постоялым двором».\\
\medskip


\textbf{Бурбулаевка} – народное название Борщаговки в 20-21 веках.\\

\medskip


\textbf{Бухара} – рынок рыбацких принадлежностей около станции метро Днепр.\\

\medskip

\textbf{Быковщина} – на стыке 18, 19 веков – урочище у подножия горы Щекавицы, околицы здания на Глубочицкой, 53. Отрезок же нынешней Глубочицкой назывался Быковщинской улицей, и начиналась она, на начало 19 века, от перекрестка современных Глубочицкой и Студентской, и считалась по перекресток Глубочицкой и Верхнего и Нижнего Валов.\\

\medskip

\textbf{Быковщина} – некий ручей, связанный с Глубочицей. Иначе сказать трудно. Вообще название урочища Быковщины и одноименного ручья я встречал только на плане Меленского 1803 года, да еще в закромах Вернадки есть «План канала из ручья Кудрявца и Быковщины в реку Днепр» невесть какого года, я видел часть этого плана лишь в статье Елены Попельницкой «Ремонт Глибочицького каналу в 1782–1785 роках: деякі аспекти історичної топографії київського Подолу ХVIII століття», и часть весьма совпадает с местом слияния, по плану 1803 года, «традиционных» Глубочицы и Киянки, близ Житнего рынка.

Про урочище и улицу Быковщинскую я писал выше. Что до ручья, то в легенде к карте 1803 года сказано, относительно Канавы (канал, по которому по Подолу текла речка, что мы ныне называем Глубочицей):

\begin{quotation}
Канал проведенный чрез Подол из ручьев Кудрявца и Быковщины и при оном плотина об мучною мельницею находящеюся в ведомстве магистрата.
\end{quotation}

При сопоставлении карт видно, что пруд с плотиной\footnote{На картине Сажина с видом с Щекавицы мы видим внизу пруд, обсаженный деревьями.} был между горами Замковой и Щекавицей напротив того места, где с последней сейчас нисходит улица Олеговская. Там-то, чуть юго-западнее, соединялись два ручья, речки – одна от современной улицы Глубочицкой, другая из урочища Гончары и Кожемяки. Первую мы отождествляем обычно с речкой Глубочицей, вторую с Киянкой.

По логике плана Меленского, раз Быковщанская улица это нынешняя Глубочицкая, то ручьем Быковщиной следует считать именно текущий по Глубочицкой ручей. Однако по более давним источникам этот ручей именуется Кудрявцем!

Но на плане Меленского ручьем Кудрявец, вероятно, надо считать Киянку из Гончаров и Кожемяк, раз уж название «Быковщины» отождествлено там с другим ручьем. Либо я не так понимаю план Меленского и он не отождествляет Киянку ни с чем, оставляя безымянной, а Быковщиной и Кудрявцем считает некий свод ручьев по течению улицей Глубочицкой?

На карте 1787 года есть «канал чрез предместье из ручья Кудрявца».

В первой четверти 18 века на «речке Кудрявке» мельницами владел Федор Быкович\footnote{Супругой его была Анна Степановна Забела, дочь Нежинского полковника.} – очевидно, название Быковщина от этого Быковича.

На упомянутом, невесть какого года и авторства, «Плане канала из ручья Кудрявца и Быковщины в реку Днепр», есть подписи к потокам – Кудрявец, если проследить выше по течению, выходит из Гончаров-Кожемяк, Быковщина с улицы Глубочицкой. Это не добавляет ясности к вопросу, поэтому перейдем к следующей статье.\\

\medskip

\textbf{Бычовка} – несуществующий в природе приток речки Совки, введенный мною в краеведение Киева по невежеству. На деле это описка в давних земельных документах, которую следует читать как Сычовку – другое название Совки.
