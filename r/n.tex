\chapter*{Н}
\addcontentsline{toc}{chapter}{Н}

\textbf{Наводничи}, Неводничи, Неводичи,  известны также как Наводницкая балка – овраг, выходящий от Царского села, по Старонаводницкой улице вдоль подножия холма со Зверинецким кладбищем, а затем по отрезку бульвара Дружбы Народов к мосту Патона. 

Овраг служит северной границей Зверинца. Под землей, от перекрестка Старонаводницкой и Панфиловцев, вдоль Старонаводницкой, а затем бульвара Дружбы Народов и Надднепрянским шоссе, в коллекторе протекает Наводницкий ручей, на котором раньше было Проклятое озеро, возможно около нынешней площади Героев Великой Отечественной войны, там где большой перекресток и к оврагу присоединяется бульвар Дружбы Народов. Впадает в Днепр в бухточке южнее моста Патона\footnote{50°25'21"N 30°34'16"E}.

У ручья два истока, спущенные теперь под землю. Один начинался на задворках ЦИК и протекал к Старонаводницкой, 6-Б-В, а затем к остановке «Водоканал» на Старонаводницкой, там к нему присоединялся другой исток, что вытекал из окрестностей гостиницы «Черное море» на Старонаводницкой 11 (на стыке 18-19 веков там был колодец среди частного сектора). На карте 1812 года он показан несколько ниже чем я сказал ранее, между шоссе Старонаводницкой и высоткой по адресу Старонаводницкая 6-А, если я верно сопоставил карты, словом тут:

50.429285497199515, 30.547117039556774

Наводницкий ручей по 2013 год в какой-то мере сохранялся на поверхности, вернее часть его русла на протяженности от поворота Старонаводницкой\footnote{50°25'33.61"N, 30°32'58.42"E} по бывшее кольцо десятого троллейбуса\footnote{ 50°25'28.34"N, 30°33'29.02"E}, пока около последнего не затеяли строительство, подрезав холм со Зверинецким кладбищем и засыпав овраг русла, сам по себе лежащий в б\'ольшем овраге. 

А так был овраг русла, прятался за какими-то автобазами, между Старонаводницкой и кладбищенским холмом. Из подножия холма туда сочились родники, а основной поток уже тогда протекал под землей в коллекторе. Всё это можно посмотреть в фильме «Киевская сюита», снятом мною с друзьями – Колей Арестовым и Светой Семеновой.

Можно было пробраться туда по превращенному в свалку пустырю и потом следовать в овраге до самого троллейбусного кольца. Внизу росли огромные деревья и проходил бетонный желоб для стоков. Я помню, как в конце девяностых весь этот нижний овраг заполнялся у дна талой водой, она замерзала, потом снова таяла.

Северный склон большого оврага Наводничей занимает Царское село – бывший частный сектор, ныне застроенный коттеджами и теремами. Раньше там жили военные, а до Великой Отечественной вообще была дикая местность, взбирающаяся к террасированному многоугольнику Печерской крепости близ Лавры. А вот по южной стороне улицы, около ручья, стояли домики.

Частный сектор, заросли и пустыри по конец восьмидесятых был и в районе нынешнего «высотного» Царского села, того что поднимается к бульвару Леси Украинки. Подробнее читайте статью про урочище Монах. Несколько позже эта местность прослыла как Междужопье, ибо находится между тылов двух памятников – упомянутой поэтессе и Родине-Матери.

В документах Наводничи мне встречаются, по крайней мере, с начала 17 века, в бумагах касательно владений Выдубицкого монастыря. 

За 1612 год упомянут:

\begin{quotation}
\noindent кгрунт монастира Видубицкого Зверинец, Неводичи, там же овоцов садовиi, груши, дерег и иний овоч [...]  
\end{quotation}

И в другом документе за тот же год иначе именовано:

\begin{quotation}
\noindent был ести на кгрунти монастиря Видубицкого [...], на Зверинцу, Наводичий названом
\end{quotation}

В одной из групп на Facebook Павел Пауль написал мне, что \textit{«с тыльной части училища связи в сторону Старонаводницкой из склона вымывало постоянно кости. Видимо тоже когда-то было кладбище»}. Предположу, что это мог быть погост ближайшей к тому месту Пятницкой церкви, церкви Параскевы-Пятницы. На старых картах на месте Училища связи (не путать с Суворовским) – Юнкерское училище, еще раньше, на середину 19 века – казармы военных кантонистов, там такой пригорок одного из берегов Наводничи, но сейчас укреплен подпорной стеной, даже несколькими, ибо на одну из террас склона есть заедз.

А напротив через нынешнюю Старонаводницкую улицу еще c начала 19 веке было одно из сосредоточий домиков с двориками. Стали бы оттуда хоронить в склоне в десятке метров от жилья? На карте 1803 года там вообще всё по вышележащую крепостную стену занято частными усадьбами, большей частью и склон тоже, хотя по склону лежат и сады-огороды. Итак, если там было кладбище, то какое-то древнее. Или не совсем кладбище, а нечто связанное с названием боярака Наводничей, известном в 17 веке – Душегубица.

Дополнительно о Наводничах читайте в моей «Ереси о Киеве».\\

\medskip

\textbf{Наводницкая пристань} – было два паромных перевоза, в устье Наводничей и в устье Лыбеди. Последний перевоз издавна принадлежал Выдубицкому монастырю, в Наводничах кажется тоже, но с 1742 года по прошению законников Лавры перевозы стали пополам с Лаврой.\\

\medskip

\textbf{Наводницкий мост} 

Сведения о нем отрывочны. Попытаюсь свести воедино.

По крайней мере в 18 веке Наводницкая пристань была в устье Наводницкого ручья, грубо говорят, там где сейчас мост Патона присоединяется к правому берегу. От пристани ходил паром на левый берег.

С 1702, 6 или 13 года, от Наводницкой пристани летом стали наводить казенный наплавной мост, разбираемый на время ледохода. 

Затем, в 1744 Лавра построила там деревянный, со свитыми из лозы канатами, мост длиной в 450 сажней. Подробностей о нем я не знаю. Об этом или другом мосте в том же месте, Закревский пишет, что его разбирали с установлением льда, и устанавливали когда спадало половодье, не раньше июня.

В начале 19 века на картах Киева в устье Наводницкого ручья появляется большой намыв, на картине 1820 года с Наводницким ручьем показанный как заливной луг. Наводницкий ручей шел по нему дальше прямо, а вот дорога, Петербуржское шоссе, отклонялось наискось, вверх по течению Днепра, и примерно на середине пути от устья Наводничей до Лавры дорога эта продолжается по Днепру то ли мостом, то ли паромной переправой, то ли обеими сразу.

На редких живописных полотнах и рисунках 19 века, где изображена переправа при Наводничах, обычно показана паромная переправа, хотя на одной картине есть именно мост. Предположу, что низкой посадки, лежащий в воде мост должен был мешать судоходству, и мост наводили не всегда. Или в определенное время его сегмент отводили в сторону. Я не знаю, как удавалось подружить наплавной мост и судоходство.

В 1915 году там построили мост на сваях, который летом 1920 сожгли отступающие польскими войска по приказу генерала Э. Ридз-Смиглы. Остались обгоревшие до воды сваи и ледорезы. Железные пролеты упали в воду. Мост однако восстановили к 20 марта 1921 года. Трудилось около 3000 рабочих. 
Наводницкий стратегический мост просуществовал до середины 1930-х годов. В 1935-м вместо него ввели в действие новый деревянный мост. Уничтожен в сентябре 1941.\\

\medskip


% –  –  –  – -
% Был длиной 834,5 саж., из которых 50 саж. были покрыты двумя железными пролетными строениями на 25 саж. отверстием каждый, 656 саж. ригельно-подкосными фермами 8-ми саж. пролетов и другие 128 саж., расположены на острове между Русановкой и Днепром, 2,5 саж. балочными пролетами. Летом 1920 года мост был сожжен отступающими польскими войсками по приказу генерала Э. Ридз-Смиглы. От моста остались только обгоревшие до уровня воды сваи и авансовые ледорезы, оба железные пролетные строения упали в воду.

%Сразу были начаты работы по его восстановлению, которые закончились 20 марта 1921 года. В работах было задействовано 3000 рабочих. Мост восстановлен примерно в предыдущем виде.



%https://kyivpastfuture.com.ua/ru/navodnytskyj-most/

% середине 1930-х годов был разработан проект, а в 1939 году начато строительство постоянного металлического моста с опорами на кессонной основе сквозной балочной системы с небольшими прогонами, позволяющими производить быструю замену поврежденных элементов (авторы: инженер В. М. Вахуркин, архитектор К. М. Яковлев).

%К началу войны строительство не было окончено. При немецких властях на его опорах был построен временный мост фон Райхенау (нем. Brücke von Reichenau), обозначенный на карте 1943 года, названный в честь немецкого военачальника генерал-фельдмаршала Вальтера фон Райхенау. Мост был взорван во время боев за Киев осенью 1943 года.


\textbf{Надово озеро} – летописное урочище.\\

\medskip


\textbf{Нархоз} – Национальный экономический университет, бывший Народного хозяйства.\\


\medskip


\textbf{Наталка} – урочище, между нынешним озером Вербным и улицей Александра Архипенко, застроено жилыми домами, западом выходит к парку Наталка и Днепру.\\
\medskip

\textbf{Немецкая слобода} – см. Графская гора.\\


\medskip


\textbf{Нежинская слобода} – сейчас там Резиницкая (до 1830х – Нежинская) и Рыбальская улицы. На Рыбальской улице была усадьба потомка нежинских купцов Г. И. Рыбальского. Слобода снесена во время постройки Печерской крепости в 1830-40 годы.\\ 

\textbf{Немецкое кладбище} 

50.441865583770955, 30.52048388067659

Одно из таковых, лютеранское кладбище, находилось где сейчас квартал на углу Большой Васильковской и Бессарабской площадью. 
Рассматривая план Киева, составленный архитектором Андреем Меленским в 1803 году,  я неожиданно наткнулся на неизвестное мне ранее кладбище в районе Крещатика, около Бессарабки. Тогда, согласно плану, оно находилось при Малой Васильковской дороге, еще никакой застройки – и Крещатика не было, шла себе по пустырям дорога.

Немного восточнее, с другой стороны нынешнего квартала где было кладбище, по улице Бассейной, с Крещатика на юг протекал ручей, диггеры называют его Крещатиком, который начинается в самом верху Михайловской улицы, затем протекал вдоль Крещатика и по оврагу современной Бассейной. Ручей и ныне течет в коллекторе, несколько иначе, но как и в прошлом впадает в речку Кловицу, около Дворца Спорта.
  
Не ошибусь, если именно к этому кладбищу с карты 1803 года относятся следующие сведения Закревского, где он пишет однако про местность Паньковщину:

\begin{quotation}
Здесь, как было упомянуто, впереди университетского здания, в том месте, где кончается Крещатицкая улица и начинается обсаженное тополями университетское шоссе, а по замечанию Киевлян, именно на месте дома и усадьбы портного Кальбе, в половине прошедшего столетия городом было отведено отдельное кладбище для Лютеран; которое однакож в 1812 г. упразднено, потому что Крещатцкая местность стала по немногу заселяться и застраиваться. 

Впрочем долгое время оставленное это Немецкое кладбище представляло весьма печальный вид разрушения, оно находилось в песчаной котловине, дождевая и снеговая вода, стремясь с возвышенных мест, уносила с собой песок, подмывала кресты, надгробные плиты, кой-какие памятники и обнажала гробы, а нередко и самое скелеты.

В 1825 г. было уже весьма малое число гробов; таким образом кладбище само собою уничтожалось. Мы только помним это место, но едва ли там остались какие либо покойники. С 1834 года эта вся эта местность стала быстро застраиваться. В замен этого кладбище было отведено другое, на юге Печерска, близ Зверинца\footnote{Начиная от Печерского моста и примерно под Подвысоцкого было православное кладбище, а оттуда до Чешской – Немецкое, по четной стороне бульвара Дружбы Народов.}.
\end{quotation}

\medskip

\textbf{Нивка}, река – см. Борщовка.\\


\medskip

\textbf{Нивки} – район, прежде хутор, принадлежавший братьям, крестьянам Фузикам из Беличей. 

На 2017 год – между проспектом Победы, улицей академика Туполева, Эстонской и Уссурийской. В первой половине 20 века – между Гончарова и Невской (искаженное «Нивская»). Эти улицы да Александровская – старые, с тех еще времен. Район застроен преимущественно частным сектором и хрущовками.\\

\medskip


\textbf{Нижняя Теличка} – местность к югу и востоку от Верхней Телички, то бишь промзона под низовьями ботсада. В начале 20 века представляла собой полуостров, отделенный от материка длинным озером Старик, доходящим до уровня нынешнего Днепровского залива, или улицы Стройиндустрии. 

Теперь от этого озера сохранилась лишь средняя часть, в виде Выдубицкого озера (кстати поглотившего прибрежный остров). Через Старик было два моста – железнодорожный в районе современного Дарницкого, и по дамбе около развязки у моста Патона. Озеро было большим заливом Днепра, старицей – рукавом русла, что зашло в тупик.\\


\medskip

\textbf{Нижняя Соломенка} – окрестности центра\-льного вокзала по правому берегу Лыбеди, от железной дороги до самой Лыбеди, параллельно улице Жилянской.\\


\medskip

\textbf{Николаевский взвоз}

На плане Ушакова 1695 года, «Николаевский взвоз» показан строго вниз к Днепру вдоль усадьбы Никольского монастыря, начиная от «Столба Николаевского». От устья же Наводницкого ручья, по сведениям из того же плана, до этого ввоза, в переводе на метры – 1817 метров, что примерно соотносится с местом будущего Спасского спуска. 

Однако Спасский спуск, более известный по картам 18 века, находился к югу от построенного Мазепой Николаевского, будущего Военного собора. А на плане Ушакова Николаевский ввоз показан севернее. Собор же стоял между нынешними бывшим Дворцом Пионеров и гостиницей Салют. За сим толкование Николаевского ввоза, изображенного на плане Ушакова, я оставляю. \\


\medskip

% – это известный поныне путь, которым можно спуститься от задворков станции метро Арсенальная и далее буквально по Зеленому театру к Днепру.

%Спуск этот хорошо показан и на планах 19 века, в частности 1846 года, и вполне прослеживается на местности даже в наши дни. 

%На начало 20 века на картах, чуть северо-западнее былого старого Никольского спуска, показывался другой заметный объект – водокачка, от лежащих внизу Подольских Никольсикх ворот (Подольский набережный верк) до Печерских Никольских ворот (около ст. м. Арсенальная) наверху. Проще говоря, старая водокачка проходит под Зеленым театром.


\textbf{Николаевский спуск} – ныне Днепровский спуск. Назывался так от Никольского монастыря, что на конец 19 века занимал территорию начиная с Аскольдовой могилы (кладбищенская Никольская церковь на ней относилась к монастырю же) и наверх, до бывшего Дворца Пионеров и гостиницы «Салют» – на месте последней стояла огромная колокольня монастыря, величественностью своей соревнующаяся с большой Лаврской колокольней.

Из путаного плана Ушакова (см. предыдущий раздел о Николаевском ввозе) можно предположить, что в этих краях уже был давний, неудобный и очень крутой спуск, возможно совпадающий с тем, что на середину 18 века известен под именем Спасского, от церкви Спаса на Берестове. Начинаясь неподалеку к северо-востоку от нее, он сходил по горе почти к мосту метро, чуть-чуть южнее оного. 

Спасский спуск называется также Панкратьевским. Сходя Спасским взвозом, вы проходили урочище Красницу, где на террасах склона стояли хатки в садах. Собственно, Спасский спуск уже в 19 веке был улицей в Краснице, что переходила в яр, по коему бежал ручей. Такой же яр, даже больше, и тоже с ручьем, был южнее.

С 1848 года вместо Спасского спуска начали, посредством удлинения дороги и засыпания яров, прокладывать новый Никольский спуск, при этом в 1853 году нашли пещеру, предположительно 12 века, о которой пишет Закревский в разделе «Иванова пещера» своего двухтомника про Киев.\\


\medskip


\textbf{Никольский Старый ввоз} 

50°28'19.8"N 30°29'29.3"E

Подъем от Плоского к Лукьяновке. Туда можно попасть с задворков Мыльного переулка, поднявшись и шагая наверх оврагом между Иорданским кладбищем и дачным кооперативом «Кожевник». Дорога и овраг имеют меньший угол наклона, нежели окружающие склоны, поэтому ввоз как бы вписывается в холм, врезается, и позволяет плавно подниматься на него, а не карабкаться по горе.

Дорога там грунтовая, довольно широкая для проезда воза, усеяна обломками старинных кирпичей.

После года эдак 2015, точно не помню, около того, ввоз перестал быть сквозным, наверху ход перегородила частная усадьба. Дорога там упиралась в невысокий, но крутой склон – судя по всему, засыпанную вверху часть оврага, который и дальше продолжался плавно. Около склона валялись сломанные деревья, но была тропка, позволявшая выбраться к частному сектору и через него на улицу Отто Шмидта (названа так потому, что он там жил).\\

\medskip

\textbf{Никольская Борщаговка} – окрестности пересечения Отрадного проспекта, проспекта Академика Королева, и улицы Симиренко.\\

\medskip

\textbf{Новая гребля} – бытовавшее по крайней мере на стыке 19-20 веков название речки Борщовки на отрезке от нынешнего проспекта Победы и до впадения в Ирпень.\\

\medskip

\textbf{Новая поляна} – урочище на Щекавице, до революции считалось как местность между улицами Саксонский яр (ныне Соляная), Чмелёв Яр и Глубочицкая. В современных ориентирах это примерно место между такой петлей, которую делает улица Лукьяновская вокруг домов 13, 11, 11А, 7, 7А, ну и вниз до Глубочицкой. Урочище названо в противовес близлежащей Старой Антифеевской Поляне, от которой осталось название улицы.\\

%\textbf{Новое строение} – 

