\chapter*{Я}
\addcontentsline{toc}{chapter}{Я}

\textbf{Яма} – см. «Пятачок».\\

\medskip

\textbf{Яма} – название низины рядом с домами по адресам Бастионная 12, 14. В Яме в советское время в доме 12 на первом этаже были, справа налево, магазины: «Культтовары» (он же Игрушечный), хлебный, молочный. В соседнем же 14-м, на углу ближе к детскому саду работала детская кухня, где младенцам бесплатно выдавали молоко в бутылочках, какие-то баночки с пюре и тому подобное.\\

\medskip

\textbf{Ямка} – река, левый приток Лыбеди. Взята в коллектор, диггерское название коего – Enemy Cave. Имела два истока, оба в районе отрогов V-образного Новопечерского переулка, под горой. Сходясь, общим руслом речка шла бы по улице Немецкой до пересечения с Предславинской, и далее мимо Дворца Украина идет вдоль Владимиро-Лыбедской по улицу Ямскую, что следует параллельно Лыбеди. Устье близ соединения улицы Владимиро-Лыбедской с Ямской, непосредственно к юго-западу от здания 58-60 на Ямской (одно время называлась Батыевой).

Современный коллектор немного отличается ходом от природного русла. Вот основные составляющие коллектора. Начало – за Домом проектов (Леси Украинки, 26), что фасадом выходит на площадь Леси Украинки. Собственно, через площадь от здания ЦИК, за которым рождается один из истоков Наводницкого ручья.

Далее Ямка идет на северо-запад под углом Института проблем международной безопасности (Новогоспитальная 1/21), пересекает улицу Щорса и обходит с севера Прозоровскую башню (Башня №3), проходя между нею и южным углом Щорса 34А. 

Следуя далее на юг, коллектор скрывается под территорией завода «Радар» и покидает ее на улице Предславинской, у корпуса 2 по адресу Предславинская 35. Далее он идет на запад под северной частью Дворца Украина, пересекает Большую Васильковскую, проходит под северным углом дома 114 на ней же, затем, огибая с севера Владимирский рынок, идет под южной стороной улицы Владимиро-Лыбедской до перекрестка с улицей Антоновича, и по Владимиро-Лыбедской выходит на финишную прямую к Лыбеди, причем около перекрестка с Ямской течет точно под зеленым бульварчиком между обеими шоссейными полотнами улицы.

Ямская улица до революции была «известна» находившимися там, кроме обычных усадеб, публичными домами, равно как и кстати Андреевский спуск. В 1905 году домовладельцы с Ямской добились от властей переноса борделей с Ямское в другое место, ибо квартиры домовладельцев обесценивались, никто не хотел снимать там жилье. А еще в 80-х годах 19 века тамошние же домовладельцы сами предложили устраивать бордели на Ямской, надеясь что такое соседство будет денежно выгодным.

Киевский роман Куприна «Яма» не зря носит такое название.\\

\medskip

\textbf{Ямки} – поселок в районе нынешней Рыбальской улицы. Писатель Николай Лесков, ребенком в 1849 году переехавший в Киев, писал о Ямках, сетуя на снос зданий при Бибикове:

\begin{quotation}
Мне жаль, например, лишенного жизни Печерска и облегавших его урочищ, которые были застроены  как попало, но очень живописно. Из них некоторые  имели также замечательно своеобразное и характерное  население, жившее неодобрительною  и даже буйною жизнью в стародавнем  запорожском  духе. Таковы были, например, удалые Кресты и Ямки, где «мешкали бессоромние дiвчата», составлявшие любопытное соединение городской, культурной проституции с казаческим простоплетством и хлебосольством. К этим дамам,  носившим не европейские, а национальные малороссийские уборы, или так называемое  «простое  платье», добрые люди хаживали в гости с  своею «горiлкою, с ковбасами, с салом и рыбицею», и «крестовские дiвчатки» из всей этой приносной провизии искусно готовили смачные снеди и  проводили с своими посетителями часы удовольствия «по-фамильному». 

Были из них даже по-своему благочестивые: эти открывали свои радушные хаты для пиров только до «благодатной», то есть до второго утреннего звона в лавре. А как только раздавался этот звон, казачка  крестилась, громко произносила: «радуйся, благодатная, господь с тобою» и сейчас же всех гостей выгоняла, а огни гасила. 

Это называлося «досидеть до благодатной».

И гости – трезвые и пьяные – этому подчинялися. 

Теперь этого оригинального типа непосредственной старожилой киевской культуры с запорожской  заправкой уже нет и следа. Он исчез, как в Париже исчез тип мюзаровской гризеты, с которою у киевских «крестовых дiвчат» было нечто сходственное в их простосердечии. 

Жаль мне тоже живописных надбережных хаток, которые лепились по обрывам над днепровской кручей: они придавали прекрасному киевскому пейзажу особенный  теплый характер и служили жилищем для большого числа бедняков, которые  хотя  и получили какое-то вознаграждение за свои «поламанные дома», но не могли за эти деньги построить себе новых домов в городе и слепили себе гнёзда  над  кручею. 

А между тем эти живописные хаточки никому и ничему не мешали\footnote{С этими хатками боролись еще с тридцатых годов 19 века, при губернаторе Левашове, усмотрев в них причину разрушения склонов, хотя оные склоны были сильно подточены глинищами казенных кирпичных заводов, работавших на постройку Киевской крепости}.
\end{quotation}

В связи с постройкой Новопечерской крепости бывшие тут кузни и каретные мастерские перенесли на Ямскую в Новом строении.\\

\medskip
%Совсем другую – мрачную, хотя и позднейшую Яму описал Куприн в повести «Яма».\\

\textbf{Яровцы} 

50°20'59.5"N 30°31'38.9"E

Исчезнувший хутор к северо-востоку от пироговского кладбища. Название от близлежащего, еще севернее, яра. Раньше за яром была гора, сейчас на месте горы – огромная котловина глиняного карьера, со свалкой на северном склоне.\\

\medskip

\textbf{Яцковка, Иакиновка} – исчезнувшая в 17 веке деревушка на Сырце, где-то около моста и большого креста. Позже доминикане поставили на месте Яцковки шинок.