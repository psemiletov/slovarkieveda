\chapter*{Предисловие}
\markboth{\MakeUppercase{Предисловие}}{}
\addcontentsline{toc}{chapter}{Предисловие}

Эта книга родилась из приложения к моей «Ереси о Киеве» – некоторые читатели помнят там эдакий словарик в конце, исчезнувший в одной из редакций книги. На деле он не исчез, а был отведен в сторонку и стал дополняться, превращаясь в самостоятельный труд.

Я не очень люблю справочники, они разрушают пространственное представление о местности, ибо сведения в них упорядочены по алфавиту, а не, допустим, с юга на север или с запада на восток. Однако упорядочение по алфавиту позволяет быстро найти нужную статью.

«Словарь киеведа» – книга неправильная, неправильная по названию и по содержимому. Название засело у меня в голове и никак не менялось, поэтому я решил оставить его как есть. 

Что до изложенных в словаре сведений, то это самый неполный и неряшливый справочник из всех возможных, частью похожий на заметки на полях, сделанные для себя самого. Я никоим образом не ставил целью охватить все урочища Киева, все местные названия, все исчезнувшие или существующие поныне здания. Более того, некоторые вещи я нарочно не занес в словарь, поскольку более подробно описал их в других краеведческих книгах моего пера. Это служит постоянной причиной моей неудовлетворености словарем. Итак, если нечто большое и важное упомянуто в словаре кратко, вроде Демиевки, скорее всего это означает, что сему отведено много места в прочих моих работах.

Благодарю всех кто снабжал меня разными местными названиями, а кроме «всех» – поименно родителей, Настю, Колю Арестова и его маму, а также Дашу Кононюк. Много полезного сообщили зрители «Киевской амплитуды» и «Планеты Киева» на Ютубе – спасибо вам!

«Словарь киеведа» набран в разрабатываемом мною текстовом редакторе TEA, свёрстан в системе вёрстки LaTeX (с движком lualatex), и всё это разумеется в операционной системе Linux.

Новые редакции книги, по мере их выхода, будут выкладываться на моем сайте:\\ 
\href{http://semiletov.org}{http://semiletov.org}\\
Ну или просто где-нибудь в сети, например на телеграм-канале Киевоград - \href{t.me/kievograd}{t.me/kievograd}\\

Буду рад отзывам и дополнениям по электронной почте: \href{mailto:peter.semiletov@gmail.com}{peter.semiletov@gmail.com}\\
в Телегу: @petersemiletov\\
или ФБ \href{https://www.facebook.com/peter.semiletov}{www.facebook.com/peter.semiletov}
