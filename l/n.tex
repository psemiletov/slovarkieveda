\chapter*{Н}
\addcontentsline{toc}{chapter}{Н}

\textbf{Научпоп} – Киевская (ордена Трудового красного знамени) киностудия научно-популярных фильмов, Киевнаучфильм, близ леса, на углу улиц Киото и Мурманской. Тут снимались замечательные документальные фильмы, а также мультики Дахно про козаков и Черкасского – «Приключения капитана Врунгеля» и другие. На 2020 год, съемочные павильоны сдаются в аренду другим студиям.\\

\medskip

\textbf{Нижнее Выгуровское озеро} – условно говоря, между улицами Электротехнической и Крайней. Искусственный водоем. До постройки жилмассива, вместо верхней части озера, в окрестностях, была ферма, а в нижней (там где Киевводоканал) – частный сектор села Выгуровщина. Луговина была также на месте и Среднего и Верхнего Выгуровских озер.\\

\medskip

\textbf{Никольский лес} – лес на Бориспольском шоссе, в окрестностях частного сектора Красного хутора. Лес был включен в состав Дарницкого района в 1965 году, в части леса устроен парк Партизанской славы, примыкающий к Красному хутору с северо-востока.\\

\medskip

\textbf{Никольская слободка}

Название происходит от первого владельца – Пустынно-Никольского монастыря\footnote{Окрестности Аскольдовой могилы.}. В 1506 году монастырь купил один участок земли у мещанина Митька Григоровича, а в 1508 другой участок, «землю Полукнязевскую», у толмача Солтана Албеева, «жены его Куньки и сыновей Ивашка и Васька». Там купно и возникла слободка.

В 19 и начале 20 веков Слободка служила местом жительства крестьян, а также рабочих с правого берега, в частности «Арсенала». Тогда же имела три урочища – улица Дачная, «Пожарище» и «За кладбищем». Пожарище вероятно относится к пожару, случившемуся 6 августа 1907 года, когда сгорело 80 зданий и погибло много домашних животных. Из Киева тушить приехала только одна пожарная команда.

Сейчас Никольская Слободка – район станции метро Левобережная и оттуда к Русановским садам на север, а к Русановскому каналу на юг.

Львиная доля частного сектор слободки была застроена еще в советское время Левобережным массивом, сейчас продолжается застройка ЖК, однако северная часть слободки в виде частного сектора сохраняется, на 2021, на границе с Русановскими садами, по улицам Сагайдака и Каховской. Напоминанием служат и Слободской переулок и Никольско-слободская улица.

Осколок частного сектора сохранялся по нулевые к северо-западу от станции метро «Левобережная», там где сейчас «Новус» на Броварском проспекте 17. В 2006 году там была уже огороженная строительным зеленым дощатым забором усадьба, за которой приютился полуразрушенный домик и остатки сада.

Рядом – недостроенное высотное здание, которое начало возводиться еще в 1986 году. Оно строилось для Госснаба УССР, перестало строиться с распадом СССР в 1991, потом принадлежало Нафтогазу, потом кажется полиции.

На месте ЖК «Садовый» и «Русановская гавань» располагался кирпичный завод, к нему вела ветка железной дороги, на моей памяти, в 90-е годы уже заброшенная. От завода и название улицы Комбинатной. Помню, с заводом соседствовала военная часть, а на берегу Русановского пролива были горы песка.

Там где сейчас сквер\footnote{50°27'29"N 30°35'23"E} перед поликлиникой, прежде постройки жилмассива, было слободское кладбище.

Южная граница Никольской слободки – улица Ованеса Туманяна. Еще в 1930-х на Туманяна 2 и 4, там где школа номер 208, находилось иудейское кладбище, существовавшее по крайней мере с 19 века.

Часть Русановского канала, к которой примыкает улица Туманяна, в прошлых веках, по крайней мере с 18, была озером Светищевым или Святищем, а возможно параллельно называлось «озеро Русаново». Подробнее об озере читайте в заметке про Святище.

В доме на Туманяна, 8 жил актер и режиссер Леонид Быков – ему дали 4-х комнатную квартиру на 10 этаже. Дом этот именовался в народе «совминовским» и там селили большей частью партноменклатуру, к которой Быков однако не принадлежал, и даже не был членом партии. Дом был построен не в 1975, как иногда считают, но по крайней мере в 1972.

Православная Свято-Николаевская церковь слободки стояла по координатам

50°26'56.3"N 30°35'18.9"E

Сейчас это середина длинного дома по адресу Раисы Окипной, 8В. В церкви венчались в 1910 году Николай Гумилев и Анна Ахматова. Деревянная однокупольная (однако со вторым куполом колокольни в том же здании), крытая железом церковь была построена в 1880 году. В длину церковь имела 35 аршинов, в ширину 33, колокольня же имела 70 квадратных аршин. При церкви было кладбище – погост, с кирпичной оградой и железными решетками. 

Церковь снесли в 1969 году при строительстве жилмассива.\\ 

\medskip


\textbf{Новая Дарница} – район между Сентябрьской, Тростянецкой, Ялтинской, Симферопольской, Привокзальной и Бориспольской. 

Возникла как дачный поселок в 1896 году, рядом с разъездом №12 Московско-Киево-Воронежской железной дороги. 43 человека арендовали выделенные 80 участков земли по 800 квадратных сажней каждый. Между ними были проложены прямые улице. К 1897 году застроили 40 участков, через год уже 60, и отвели под аренду еще 85. Стоимость аренды за год лежала в пределах от 16.35 до 75.45 рублей. Вскоре в поселке завели аптеку, почту, на станции поставили телефон, появился железнодорожный театр, обустроили парк. На 1909 год в поселке было 13 улиц и 134 участка.

Прежде, поселок Новая Дарница примыкала югом к хутору Шевченко. 

На 2021 год – много застроена старыми двухэтажными домами, которых начали строить кажется еще в 1930-е. Чем дальше от Харьковского шоссе на северо-запад, тем новее дома. На Новодарницкой улице трехэтажки соседствуют уже с четырехтажными. На Заслонова – хрущовки. На Волгодонской – высотки советского времени, по 9 и 16 этажей. К самому Харьковскому шоссе примыкают желтые дома в 2 этажа, как в Аварийном поселке (часто его называют «Соцгород», хотя тот чуть дальше) около станции метро «Черниговская».
