\chapter*{О}
\addcontentsline{toc}{chapter}{О}

\textbf{Ольхово} – озеро на Позняках, на запад от озера Прорвы, в окрестностях улицы Ивана Кочерги. Непонятно как ставить ударение в Ольхово. Вероятно его часть это теперь озеро Утиное напротив, у перекрестка Григоренко и Здолбуновской.\\

\medskip

%\textbf{Осещина} – \\

\textbf{Осокорки} – остров, владение Выдубицкого монастыря в начале 17 века.\\

\medskip


\textbf{Осокорки}, хутор – ходившее по 1940-е название Красного хутора, одновременно с ним. Подробнее см. заметку о Красном хуторе. Не путать с привычными Осокорками.\\

\medskip


\textbf{Осокорки} – бывшее село, ныне дачный поселок и сильно застроенный ЖК район.

Из универсала гетмана Ивана Скоропадского 1712 года на владения Выдубицкого монастыря:

\begin{quotation}
Напрод на селище Осокорки по той стороні Дніпра за Киевом будучое з грунтами сими именно: берег Дніпра против самой церкви Вилубицкой и у придушной обрітаючийся, Телячое, Грузная, Плоское, Подбурное, Синятин, озерем Гачища, именуемие Клешня, другая Подбурная, речище Осокорское об самого устя од верха Дніпра обадва береги, а один берег от дутни до вербника.

Остров Осокорский против Либеди, нарічіем Миколаев, криниці Вирища, Чеповаха, Татарка Великая и Малая порубіже Поповое, Калита, Кривое, Затисок, против Жуковка, Ставное, сіножати протчие там монастиреві Видубыцкому прислужаючие угодия, якие кгрунта ведлуг тих же за лядской держави – полским, а инние от протчиих многих людей прежде помянутие видубыцкие добра в аренді міючих руским характером справленних писем: которые тут были презентовани.

Таковим преділом граничатся, почавши Дніпром второговое Днеприще. Днеприщем на Липки, од Липок в Быстрий перевал\footnote{Граничил с озером Подборным.}, одтель в Вубчое, а далей в другое Вубчое. З сего на Коропов\footnote{Озеро.} в Чорние Лозы\footnote{Граничили с дорогой Бортницкой и Убщовым озером.}. З Чорных Лоз на Кривую Сосну, потом на два Осокорки великие. Од Осокорков врачища и в річку Позняковскую. Річкою в Тельбин, Тельбином в Коростишов, Коростишовом в Глубокий Колтов, з Колтова в Доманское жерело, жерелом в Дніпр и знову Дніпром в Неводницкую просто.
\end{quotation}

\medskip


\textbf{Остров страданий} – народное название Русановки в первые годы после постройки жилмассива, из-за его плохого транспортного сообщения с городом.

%\textbf{Осокорки}, село