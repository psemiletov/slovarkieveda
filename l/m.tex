\chapter*{М}
\addcontentsline{toc}{chapter}{М}

\textbf{Малиновка}, озеро 

50°29'09.9"N 30°34'17.1"E

Озеро на запад от озера Радунки, в 17 веке  Меленовка, «речка Меленовка», вероятно прежний, заозёрившийся рукав Радунки. Окружено дачами, имеет вид вытянутой с севера на юг старицы. С незапамятных времен соединялось с Радункой каналом, частично сохранившимся вдоль улицы Пилот.

В земельных документах 17 века упоминался также остров Меленовский Ленковщина (от землевладельцев Ленковичей), «за Днепром и Чарторыею лежачий».\\

\textbf{Милославское} – летописное село, было на месте села Выгуровщина. См. «Ересь о Киеве».\\

\textbf{Михайловский луг} 

50°29'39.3"N 30°34'21.5"E

Известное по начало 20 века урочище непосредственно на север от урочища Боярского. Окрестности перекрестка проспекта Ватутина и улицы Бальзака. Частично съедено заливом Десенка.\\

\textbf{Млынное}, озеро

50°20'46.8"N 30°40'12.3"E

Озеро на юго-запад от Бортничей, ныне известно как Островки. Название Млынное произошло, судя по всему, от водяной мельницы, млына.\\

\textbf{Моложи} 

50°30'50"N 30°34'6"E

Урочище к западу от села Троещны, между Ситняками, Городищем и Десенкой. Большое, окруженное рощами поле.\\

\textbf{Мостище}

50°27'35"N 30°43'18"E

Урочище между Княжичами и хутором Рыбный, лежит в пойме речки Дарницы, в лесу между нею и железной дорогой. При Мостище на Дарнице расположено несколько давних прудов. Название, вероятно, намекает на бывший тут мост.\\

\textbf{Муромец} – остров севернее Труханова, название от хутора Муравца, прежде тут существовавшего. Подробности в моей книге «Ересь о Киеве».
