\chapter*{Т}
\addcontentsline{toc}{chapter}{Т}


\textbf{Толока}, болото – в начале 20 века находилось около метро «Позняки» между улицами Олены Пчилки и Анны Ахматовой.\\

\textbf{Толока}

50°31'26"N 30°34'51"E

Урочище к северу от села Троещина. На 2021 год - разграниченные улицы в чистом поле, участки под застройку.\\

\textbf{Туберкулёзка}, озеро 

50°24'51"N 30°40'28"E

Наибольшее из озер в Парке Партизанской Славы, длинное, а остальные два меньше, более круглые. Туберкулёзка и смежное с ним к северу озеро – естественное, а то, что восточнее – рукотворное. В первой половине 20 века было еще одно, примерно в квартале Красного хутора между улицами Елочной и Армянской. В озерах этих водятся болотные черепахи и утки.
