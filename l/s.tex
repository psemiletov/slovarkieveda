\chapter*{C}
\addcontentsline{toc}{chapter}{C}

\textbf{Салют}, пионерлагерь

50°23'1"N 30°46'9"E

Лагерь от Киевского Радиозавода, закрыт в 2004 году. Находится на речке Демидовке, на коей стояли другие пионерлагеря (Луч, Алмазный). «Салют» лежит в запустении. Остатки зеленого театра, кирпичные корпуса в несколько этажей.\\

\medskip

%\textbf{Свидловищина} -

\textbf{Святая гора}

50°26'18"N 30°44'53"E (при сопоставлении с картой Шуберта)

Холм в Никольском лесу. Сосновый лес на горе вырублен, земля распахана под засевку нового.\\

\medskip


\textbf{Святище} – озеро 

50°21'00.6"N 30°38'17.0"E

Лежит на запад от Бортничей. Западный берег стремительно застраивается.\\

\medskip


\textbf{Святыще} – озеро, оно же Святищево, Святище, Святое. 

Существовало по самое устроение Русановского канала на его отрезке между улицами Раисы Окипной и Туманяна. По высокому берегу, где ныне улица Флоренции, стояли дома частного сектора Никольской слободки, ее северной окраины. Противоположный берег представлял собой пески.

Как отдельный водоем озеро показано на картах еще 18 века, но по виду и смежным с ним тогда и позже водоемам был частью русла, длящегося от Долбички и продолжавшегося, южнее Святыща, Тельбином (Толбином, явное сходство с Долбичкой по корню). Подробнее об этом – в «Ереси о Киеве».

На карте 1750 года, северный берег Святыща (возле улицы Флоренции) подписан Лысой горой. Представления о левобережной Лысой горы 19 века, по источникам, относятся к возвышенности от Никольской слободки более на север, нежели к южной ее околице. Однако такая подпись купно с названием Святыще наводят на мысль, что озеро имело какое-то языческое значение. Упомянутый берег, судя по археологическим раскопкам (см. «Ересь о Киеве»), был населен еще много тысячелетий назад.

Севернее Святыща, то есть ближе к линии метро, было еще Васильевское озеро, вероятно оно же Русаново.

Из низа именно Русанова озера некий ручей перетекал в северную часть озера Святища. На некоторых картах один и тот же водоем, однако, подписан то Святищем, то Русановым.

На плане Фон Руге первой половины 19 века (легенду см. в Статистическом описании Киевской губернии, изданном Фундуклеем), озером Русановым подписано озеро-старица близ Никольской Слободки, ныне занято Русановским проливом, от 50°27'16.5"N 30°34'54.3"E и примерно по 50°26'47.0"N 30°35'18.2"E (начало Русановского канала). 

По совокупности данных можно понять, что озеро Русанов было вышележащей частью старицы, а Святище – нижележащей, и далее она продолжалась Тельбином.

Частный сектор южной части Никольской слободки на бывшем берегу Святища сохранялся по конец 1960-х, в пределах между каналом и нынешними улицами Окипной и Туманяна, между коими была совершенно другая, старая сетка улиц и переулков.

Их названия можно восстановить по немецкой карте 1943 года. Непосредственно вдоль берега шла улица Uwjatyzka (она же Будённого по карте 1935 года), а восточнее ее – Лозовая, пересекаемая множеством переулков. Параллельно этим двум улицам, но севернее, проходила улица Шевченко.

Южная окраина слободки была застроена в первой половине 20 века. Прежде там никто не обитал (во время существования слободки), было только иудейское кладбище – между нынешними адресами Туманяна 4 и 2, то бишь между школами 208 и 125. Старый корпус 208-й школы построили пленные немцы, новый корпус – на стыке 1970-80, при денежной помощи «шефов» – завода «Дормаш». Улица Туманяна проходит точно по части кладбища. Во время бытности слободки, улицы там не было.\\

\medskip

\textbf{Серебряный кол}, озеро – продолговатый водоем между станциями метро «Осокорки» и «Позняки», в нынешних своих пределах творение искусственное, возникшее при добыче песка.

Исторический Серебряный Кол упомянут в грамоте 1694 года царей Иоанна и Петра Алексеевичей, подтверждающей Киево-Братскому монастырю права на земли и угодья, сказано кроме прочего про сеножать Подкурье, границы которой связаны с Тельбином:

\begin{quotation}
село Позняки [...]

Другая сеножать Подкурье по дальнюю гать рубежем, почав от озера Тербина жерелом Дарницею до нижней мельницы Печерской, до Воскресенщины, от мельницы убедью до Ост\-рого Рога, а от Острого Рога до Шаломин, тою убедью, от Шаломин в конец Урлева гради, от Урлева в Пенные лозы, а от Пенных лоз в Довгушку озеро, из Довгушки жерелом в Серебренный-Кол, от Сребренного Кола длиной малою, в конец Вязок, в Княжей Затон, от Княжего Затону на брод под Троецкий футор, от брода жерелом до Весняка речки, Весняком до Синятина, от Синятина до Порубежнаго, от Порубежнаго до Телячева, от Телячева, речкою Позняковкую, до моста на Тербин речке, да в речку Дарницу.
\end{quotation}

\medskip

\textbf{Ситняки} 

50°30'25"N 30°34'24"E

Поле между Гнилушей и заливом (около Десёнки), на западном берегу Гнилуши. Заметны следы городища – сглаженные валы, и очагов давней металлургии. Я нашел там сыродутный горн, крицу. Подробнее в моей «Ереси», в части про Городец.\\

\medskip

\textbf{Сказка} – условно говоря, окрестности Лесного проспекта, 23 и 25. Торговое сосредоточие, мафы, банк, почта, небольшой базарчик. Одноименное кафе.\\

\medskip

\textbf{Сладкоежка} – известное в советское время кафе около станции метро Дарница, за Детским миром, по адресу Малышко 25/1.\\

\medskip

\textbf{Солнечное} – озеро на Позняках, как отдельный водоем возникло примерно во второй половине 20 века, в заболоченной местности. К концу 1960-х это было озеро, вытянувшееся вдоль северо-восточной стороны отдельного района частного сектора Позняков – а именно райончика, словно полуостровом уходящего в болото. Он лежал примерно между нынешним Солнечным и улицей Драгоманова.

В 1970-х Солнечное называли Старой Прирвой, позже – Зем, от «Земснаряда», что был в озере, вымывая оттуда песок для намыва окрестностей.

Исходное озеро занимало отрезок нынешнего, примерно от широты дома на Драгоманова, 1-Г и по северный конец озера. И было вполовину у\'же.\\

\medskip


%\textbf{Соцгород} – 


\textbf{Сорокасемиквартирный} – дом в Соцгороде по адресу Бульвар Верховного совета, 13. Раньше тут жили партийные деятели. Четырехэтажный, с зеленым двором, примыкающим к усадьбе музыкальной школы.\\

\medskip


\textbf{София}, болото – было в Гидропарке на юго-восток от озера Берёзки\footnote{Координаты озера: 50°26'14.0"N 30°34'41.5"E}. Как топоним существовало еще в первой половине 20 века.\\

\medskip

 
\textbf{Сторожевые горы} – длинный холм между Алмазным озером (бывшее болото Куричево) и осушенным болотом Ковпытом. Горы поросли сосновым лесом, в северной их части расположена ТЭЦ-6, а южнее ее, в лесу – кладбище домашних животных, частью стихийное, частью недостроенное официальное. У подножия горы, между ним и дорогой к улице Пуховской, проходит заболоченный канал. По другую сторону от этой дороги лежат заболоченные остатки верховий болота Куричева. На самой дороге нередко проводятся гонки.\\

\medskip

\textbf{Строителей поселок} – небольшой частный сектор в конце улицы Строителей, между нею, Попудренко, Красноткацкой и БВС. Граничит с метрополитеновским Электродепо ТЧ-1 «Дарница» с запада и с Соцгородом на востоке. Простых старых домов осталось мало, большей частью возникли элитные особняки. Основные внутренние улицы поселка – Дубового и переулок Строителей, по которому сохранилось несколько двухэтажных старых многоквартирных домов. Частный сектор в начале Дубового (ближе к БВС) снесли в 1980-90, и построили там высотки.

Улица Строителей, что начинается около Ленинградки и давшая название поселку, именуется так потому, что поселили там строителей Дарницкой ТЭЦ, хотя часть строителей жила также в бараках возле самой ТЭЦ. Поселок же возник с 40-х, активно застраивался в 50-е.

Еще раньше, на стыке 19-20 веков, там, в конце улицы Строителей, на карте видны казармы лагеря юнкерского училища.\\

\medskip

\textbf{Стыгле, Стигло, Стиглая}

50°29'17"N 30°41'12"E

Урочище между Быковней и Броварами, в нем находится озеро Помильное. Слово «стиглий» в переводе с украинского означает «спелый». Урочище заросло соснами, березами и акацией, есть травяные лужайки и песочная полянка около озера.

В Стиглом по 1960-е стоял древний курган (7 метров высотой, 40 окружность), его разрыли бульдозером, некие местные братья, искавшие там клад. В кургане оказался горшок с пеплом и костями.

Существует байка, что Помильное названо так потому, будто в нем мыли карету Екатерины во время ея поездки в Киев.

Однако вероятно название восходит к юридическому термину «помильное» времен Великого княжества Литовского – так называли оплату властям за каждую милю при выезде децкого (судебный исполнитель наместника, воеводы или князя) на место происшествия. Например, в 1529 году помильное составляло по грошу за милю. «Помильным» также называлась плата извозчикам, помильное уплачивалось и мещанам либо волощанам, которые временно предоставляли властям лошадей для перевозки чего или кого-либо. Такое помильное исчислялось меньшей денежной суммой.

Озеро имеет крутые и относительно высокие, метров шесть, берега, то есть озеро лежит как бы в углублении относительно местности, впрочем спустившись к воде мы видим берег пологий.
 
По непроверенным мною данным из озера брали песок, что разумеется могло повлиять на рельеф. Однако на той же глубине, где лежит водная поверхность, залегает и некое русло, присоединяющееся к озеру приблизительно с юго-востока. Ширина русла эдак как Лыбедь в районе Бусловки, высота примерно западного берега метров пять-шесть, противоположного чуть меньше. Русло заполнено водой то шириной что можно перепрыгнуть, то на всю ширину русла. Течение я не замечал. За низким берегом русла, и за восточным берегом озера, отделяясь небольшой полоской суши, лежит березовое болото.

По карте РККА 1930-х годов вместо озера – причудливое болото до самого Броварского шоссе, а вот «озеро» показано еще северо-вос\-точнее, однако на трехверстовках Шуберта 19 века озеро Помильное показано в текущих пределах. Следовательно, оно могло заболотиться, а после быть расчищено и углублено при добыче песка.\\

\medskip

% однако скорее всего название происходит от слова «милость», то бишь озером кого-то миловали, жаловали, подарили его кому-то, присудили. «Помильное» – частый юридический термин в документах Великого княжества Литовского.

%Также слово "помильное" означало, в роськой мове княжества Литовского, плату извозчикам. 


\textbf{Сухие горы} – гряда высоких песчаных холмов, тянущихся от южной границы Лесного кладбища\footnote{\textasciitilde{} 50°29'45.5"N 30°38'15.0"E} и по сосновому лесу на юго-восток, параллельно Лесному массиву, особенно улице космонавта Волкова. Восточной стороной эти горы еще в первой половине 20 века представляли собой берег большого болота, стало быть прежнего озера. Из болота, как видно по карте РККА 1930-х, истекала вода в болото, что было на месте Алмазного озера. Полагаю, что дюны Сторожевых гор некогда были берегом водоема большого и могучего, а болото, затем озеро в прошлом являлось заводью древней, огромной Десны. В какое время? Не в отдаленной древности, поскольку холмы еще шибко не сгладились.

Между кладбищем и Лесным, севернее ГСК, есть также некое широкое русло засохшего водотока, а также сухой мелиоративный канал. 

Вдоль гор, у низовий, между ними и бывшим болотом, идет некий ров, который подходит к кладбищу. Ров отделен от болота валом. 
Подобный ров, перпендикулярный этому через горы к Лесному массиву. Очевидно что это не мелиоканал, ибо вода по нему поступать не могла из-за высоты. Его часть видна также параллельно Лесному проспекту около большого пустыря около церкви, напротив Сельпо.

\medskip
 
\textbf{Сынатян} – озеро в Осокорках. Озером владел Выдубицкий монастырь в 17 веке, при водоеме была сеножать.\\
