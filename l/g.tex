\chapter*{Г}
\addcontentsline{toc}{chapter}{Г}

\textbf{Голопузовка}

50°26'09.2"N 30°40'44.0"E

Район между ДВРЗ и Радиозаводом, вписанный между улицами Ремонтной и Оросительной, лежит на юг от железнодорожных станций Дарница-Депо и ДВРЗ. Занят, в основном, птицефабрикой и мясокомбинатом, который перестал действовать с 2016 года. В южной части Голопузовки, что ближе к Радиозаводу, есть и жилые дома.

Другой Голопузовкой вроде бы слыли окрестности Русановки и/или она сама до постройки одноименного квартала, но эти сведения требуют уточнения.\\

\medskip

\textbf{Горбачиха}

50°28'13.0"N 30°33'57.2"E

Урочище в Русановских садах. С севера ограничено озером Русановским и противопаводковой дамбой, с востока и юга – Десёнкой. Перерезано строящимся мостом.

Представляет собой плавно сходящий от дамбы к реке песчаный пустырь, поросший верболозом. На Горбачихе есть большой дикий пляж, куда ходили  купаться, кроме дачников, обитатели близлежащих баз отдыха. Раньше там существовало по крайней мере две такие, на восточном берегу Русановского озера. Одна была от Академии Наук\footnote{50°28'20.6"N 30°34'13.0"E}, а другая рядом, не помню какого ведомства.

О первой до сих пор напоминают виднеющиеся издали высокие тополя. Вместо обеих баз теперь – частные владения. База Академии Наук находилась в конце 13-й Линии и состояла из домиков, столовой, да административного корпуса на введенной в Русановское озеро баржи (ее туда затащили до устроения дамбы, перекрывшей соединение озера с Десёнкой).

Жители второй базы готовили себе еду сами на плитах с газовыми баллонами. Между второй базой и дамбой, как идти на пляж, был заросший у берега озера вербами пустырь, на коем при дороге стояли качели. Оттуда на саму дамбу дорожка была выложена бетонной плиткой. Дорожка сохранилась поныне.\\

\medskip

\textbf{Горка}

50°29'22.4"N 30°35'24.7"E

Народное название кургана на Воскресенке, во дворе дома на Перова, 25. Курган, во всяком случае возвышенность именно на его месте существовала издавна, еще до застройки жилмассива, а чуть южнее находилось кладбище Воскресенской слободки, но курган был за его пределами. Среди старожилов жилмассива ходят рассказы, что во время дождя раньше из кургана вымывались человеческие кости. Другое местное предание гласит – сюда свезли в кучу кладбищенскую землю (ведь окрестные дома в сторону улицу Серова стоят на прежнем месте кладбища).\\

\medskip

%После того как я выложил фильм "Курганы и горы Воскресенки", в комментариях к нему Сергей Балюк написал: "В начале  восьмидесятых вдоль дороги экскаватор копал. Были и кости и черепа".

\textbf{Городище}

50°30'53"N 30°34'36"E

Равнина на запад от села Троещины, в 21 веке начала застраиваться коттеджами, прежде тут были огороды. Место некоего изученного археологами городища, примыкавшего к погребальному валу на восточном берегу озера Гнилуши. Подробнее см. «Ересь о Киеве», часть про Городок. Нет, летописный Городец был не там, а около Погребов.

Урочище Городище лежит ниже возвышенности села Троещина, по краю которой идет улица Деснянская. Создается впечатление большого вала, хотя там просто приподнятая местность. Но тогда получается, что в древности Городище было дном какого-то русла.\\

\medskip

\textbf{Горыще}, ресторан – советский ресторан на улице Шолом-Алейхема, 4. Теперь там АТБ.\\

\medskip

\textbf{Горячка}

50°25'14"N 30°37'49"E

Озеро на Позняках, смежное с Прорвой, отделенной от него насыпью. Используется под золоотвалы Дарницкой ТЭЦ. Выкопано во второй половине 20 века на месте заболоченных луга и рощи. Мрачной коркой озеро стало покрываться с 2000 года, и на 2017 год от былой водной поверхности осталась лишь незначительная часть.\\

\medskip

\textbf{Градина} – бывшее урочище близ села Выгуровщина, сейчас его место занимает улица Градинская на жилмассиве Выгуровщина-Троещина, или просто Троещина. Градина была луговиной и полями, отделялась от лежащих севернее таких же лугов урочища Лисковщины речушкой.
