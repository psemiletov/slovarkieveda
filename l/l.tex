\chapter*{Л}
\addcontentsline{toc}{chapter}{Л}

\textbf{Липецкой хутор} – хутор, известный на стыке 18-19 веков, по моим соображениям где-то в районе хладкомбината №3 или западнее в Русановских садах, в районе пересечения Каховской и 1-й Садовой. При хуторе было озеро, перпендикулярное Радужному. Еще южнее находилось озеро Очереватое (от слова «очерет»).\\

\medskip

\textbf{Лиски (Лески)} – некогда это было урочище, часть заболоченной поймы речки Дарницы, к западу от нынешнего ДВРЗ, а также одноименный хутор из десятка домов на север от современной улицы Новаторов, от дома 41/3. На 1951 год хутор разросся до ста с лишком домов, а затем постепенно сросся с лежащей чуть западнее Старой Дарницей.

Теперь урочище (не хутор) занято железнодорожной станцией Киев-Лиски и железнодорожным парком Киев-Лиски – последний под землей пересекается речкой Дарницей\footnote{50°26'54"N 30°39'19"E}, которая дальше в открытом канале протекает мимо Дарницкой ТЭЦ, рядом с которой прежде было Дарницкое озеро.

Что до жилого района Лисок, прежнего хутора – он лежит между Старой Дарницей и ДВРЗ. Эдак между железной дорогой, улицей Алма-Атинской и Гродненской. Лиски это также улица Сновская, Люботинская, Калачёвская, Трактористов, Новаторов, Щепкина и прилегающие в описанных выше границах.\\

\medskip

\textbf{Лисковщина} – урочище, бывшее некогда в селе Выгуровщина, ныне занято улицей Лисковской на жилмассиве Троещине.\\

\medskip

\textbf{Ленинградка} – Ленинградская площадь.\\

\medskip

\textbf{Летняя гора} – на восток от Детского мира, вдоль Малышко и Броварского проспекта, но не доходя до Волчьей горы, было еще одно возвышение, Летняя гора.\\

\medskip

\textbf{Лопутино}

50°26'44"N 30°46'40"E

Урочище на стыке юга Княжичей и севера Никольского леса.\\

\medskip

\textbf{Литвиново}, озеро – в начале 20 века, озеро в окрестностях нынешнего перекрестка Днепровского шоссе и улицы Княжий затон.\\

\medskip

\textbf{Лысая гора} - большая возвышенность, условно говоря, вдоль бульвара Перова, с пиком около авторынка и бывшего кинотеатра «Аврора». Отмечена на картах 19 века и начала 20-го. Именно с этой Лысой горой и связаны многие предания про киевских ведьм. Подробнее о Лысых горах Киева читайте конечно же в «Ереси о Киеве».\\

\medskip

\textbf{Луч}, пионерлагерь

50°23'11"N 30°44'46"E

Пионерлагерь предприятия «Химволокно», был расположен в Никольском лесу. После там поселился христианский детский лагерь «Радуга», а на 2018 год – полигон для страйкбола. «Луч» стоит на ручье Демидовке, что впадает в озеро Вырлицу. На этом ручье, с запада на восток (против течения), были и другие пионерлагеря: Родина, Алмазный, и лечебно-диагностический центр «Шахтер».
