\chapter*{Р}
\addcontentsline{toc}{chapter}{Р}

\textbf{Радикал}, он же Тысячный – знаменитый завод оборонной промышленности. Некогда занимал огромное пространство между улицами Красноткацкой и Вискозной, примыкая к заводу Химволокно, лежа непосредственно на восток от него. Территория завода выходила также на улицы Мурманскую и Беломорскую.

Строился в 1949-52 годах среди соснового леса, изначально был секретным. Выпускал химию –  хлор, каустическую соду, серную и соляную кислоты, бертолетовую соль, ДДТ и другое. Из-за вредного производства рабочим не разрешалось трудиться на нем более 10 лет подряд, а на пенсию с завода уходили в 45.

В 1996-2000 годах обанкротился.

За годы работы на территории завода и в почве под ним накопилось много вредных элементов, включая ртуть.

На 2017 год, часть заводской земли вместе с корпусами распродана, часть сдается в аренду. Среди корпусов проложена развитая сеть железнодорожных путей, соединенных в итоге с железнодорожным парком «Лиски» станции «Дарница». В нему же идут и рельсы от Дарницкой ТЭЦ.\\

\medskip

\textbf{Радосынь} – старица и урочище в Погребах. Подробнее о Радосыни, и отличной от нее речке Радонке, а также об озере Радунке (Радужном) читайте в «Ереси о Киеве».\\

\medskip

\textbf{Рембаза} – район в Дарнице, окрестности улицы Полесской и часть примыкающей к ней Бориспольской улицы. Жилье и промзона, свой парк, а также кладбище бронетехники. Название произошло от военно-ремонтной механической базы № 7.

Жилая застройка пёстрая – хрущовки, высотки всех времен, частный сектор.

Южной частью Рембаза граничит с Парком партизанской славы, востоком выходит к Никольскому лесу и станции метро Красный хутор (станция отделена от хутора полосой леса упомянутого парка). Северо-востоком Рембаза примыкает к району Шлакоблока, который официально входит в состав Рембазы как микрорайон.\\

\medskip

\textbf{Ремзавода} поселок – райончик из одной старенькой двухэтажки и частного сектора, по улице Воскресенской с дюжинного номера по второй. Примыкает к микрорайону Воскресенский 1 с одной стороны, и ЖК «Парковые озера» с другой. На месте этого ЖК еще по 2009 год и располагались заводские корпуса Ремзавода. Последний на 2020 год именуется «Агромаш» и занимает гораздо меньшую территорию.

Напротив ЖК лежит парк Победы и цепь водоемов на нем. Эти водоемы питаются от текущего на север ручья, и ручей раньше, до создания в 2005 году прудов, просто бежал по парку через болота, обросшие березовой рощей. И он шел вдоль северо-восточной границы завода, от чего даже остался кусочек русла\footnote{50°28'04.6"N 30°36'19.6"E} близ ресторана «Подкова». Этот рукав окончательно исчез в 2008-м. 

А восточный рукав, ныне основной и единственный в том месте, тоже существовал – в виде канала с болотом по бокам. Вообще всё русло ручья существенно расширили, хотя нынешние пруды в парке Победы лишь частично покрывают былую пойму болота по берегам ручья.

Ремзавод сбрасывал в ручей отработанную воду.

Воскресенская улица существовала в нынешних пределах как дорога в пустырной местности, по крайней мере в 1930-х. В качестве улицы, кажется,  с 1960-х.\\

\medskip

\textbf{Риф} – урочище на левом берегу выше устья Десны, известное на стыке 19-20 веков.\\

\medskip

\textbf{Родина}, пионерлагерь

50°23'32"N 30°42'27"E

Заброшенный пионерлагерь у северной окраины Бортничей, в Никольском лесу. В советское время принадлежал заводу «Радикал». Между улицей Лесной и развалинами лагеря – пруд ручье, что впадает в озеро Вырлицу. На том же ручье, на восток, против течения, были расположены и другие пионерлагери и оздоровительные учреждения, например пионерлагерь «Алмазный».\\

\medskip

%\textbf{Русановка}
%трактир Резанова

\textbf{Русанов} - на первую половину 19 века, озеро-старица близ Никольской Слободки, ныне занято Русановским проливом, от 50°27'16.5"N 30°34'54.3"E и примерно по 50°26'47.0"N 30°35'18.2"E (начало Русановского канала).\\

\medskip


\textbf{Рыбное} 

50.46755080080097, 30.705770062201562

Большое озеро, ранее (конец 19 века) болото, в лесу, к западу от одноименного хутора между Броварами и Быковней. Рядом с озером заброшенная ферма, принадлежавшая хутору.
