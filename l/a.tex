\chapter*{А}
\addcontentsline{toc}{chapter}{А}

\textbf{Алексеевская слободка} –  дореволюционное официальное название слободки на Трухановом острове.\\

\medskip

\textbf{Аварийный поселок}

Известен также как Поселок аварийщиков и Немецкий квартал. Бешеными темпами исчезающий райончик желтоватых домиков в 2-3 этажа между Соцгородом и линией метро около станции Черниговская. Прежде ограничивался улицами Красногвардейской, Попудренко, Минина, Красноткацкой. 

Был застроен известными домиками в 1947-52 годах для рабочих близлежащих заводов ДШК, «Радикал» и «Химволокно», сначала пленными, а затем уже вольнонаемными немцами. Однако еще в 1935 году «хутор Аварийный» был включен в состав Дарницкого района Киева, и на карте того времени на месте нынешнего Аварийного – пустыри Дарницких лагерей, а вот вдоль восточной стороны улицы Магнитогорской стояли какие-то домики. К 1944 году на месте нынешнего Аварийного поселка, между Краковской и Попудренко, уже было десятка два жилых домиков. От Броварского шоссе они отделялись лесочком. Северо-восточной стороной тот поселок граничил с грунтовой дорогой (нынешняя Красногвардейская улица), дальше за которой шли пески с сосняком, такая же местность была и на юго-запад, где сейчас улица Пожарского. Тогда же в классическом Соцгороде, в конце современных Краковской и Попудренко уже возникли большие дома, например Краковская 5, Попудренко 20 и 18-А, Строителей 34/1.

Сносить Аварийный поселок или отдельные дома собирались еще в 1970-е, но взялись кажется с половины 2008 года. Первыми жертвами пали здания по Красноткацкой и вдоль Красногвардейской (до Краковской), где на месте снесенных домов воздвигли высотки. С 2016 новая застройка поползла по самой Краковской.\\

\medskip

\textbf{Аврора} – небольшой парк на Воскресенке. В западной его части по 2005 год стоял одноименный кинотеатр, теперь на его месте ТРЦ. Широкоэкранный кинотеатр на 800 мест был построен в 1966 году. Фасад над входом украшало мозаичное панно, изображавшее девушку с факелом в руке, на фоне крейсера Авроры, впрочем революционную девушку можно было считать и богиней Авророй, как кому угодно. Кинотеатр перестал показывать фильмы еще в начале-середине девяностых.

В парке находится разворотное кольцо трамваем 8 и 22, на Позняки и ЖБК (завод Железобетонных конструкций).\\

\medskip

\textbf{Алмазное озеро} – крупнейшее озеро в Киеве, примыкает к северной части Лесного кладбища, занимает существенно углубленное добычей сначала торфа, а потом песка русло болота Куричева, давнего рукава Десны. До углубления, в 20 веке народное название местности было Торфы. Имя свое озеро получило от завода «Алмаз» (ныне Монетный двор) на северном берегу озера.

Полумесяц Алмазного продолжается на север болотом, тоже остатком русла Десны. На северо-восточном его берегу, на поросших сосной, песчаных Сторожевых горах, стоит ТЭЦ-6, вероятно там же располагался один из летописных Городков. Сторожевые горы отделяют это болото от другого – частично осушенного, истощенного торфодобычей болота Ковпыт, что некогда соединялось каналом с Куричевым.

Северо-западный и западный берега озера – равнинны, а южный, юго-восточный и восточный представляют – гора, где растут сосны, а кое-где тополями. Гора то пологая, то местами с высокими песчаными обрывами – точно как известный «классический» современный берег Десны. 

На озере есть остров, отделенный от берега проливом.

Длина озера около 3,5 километров, ширина около 600 метров. Глубина достигает 35-40 метров, что вызвано добычей тут песка.

В озеро с северной стороны впадает ручей, текущий вдоль заболоченной части озера и дороги у подножия Сторожевых гор, со стороны ТЭЦ-6.

На южном берегу озера, в некотором отдалении от воды, находится Лесное кладбище, отгороженное от соснового леса сетчатым забором. Среди покрытых хвоей дюн за этой оградой можно найти обломки надгробных памятников.\\

\medskip

%\textbf{Алто, Олто} – старое название Борисполя.

\textbf{Алмазный}, пионерлагерь

50°23'25"N 30°43'21"E

Бывший пионерлагерь у северной окраины Бортничей, в Никольском лесу, у речки Демидовки.

В пионерлагере был свой сад. В 1970-х руководил лагерем Леонид Вацлавович Конисевич, воспитанник коммуны Макаренко. Сюда приезжали как советские школьники, так и заграничные – от шотландских профсоюзов, из Польши и других стран.

В середине 90-х в лагере жили офицеры-студенты из Института МВД (на Коллекторной улице).

На 2010-е, на территории работал вероятно филиал психбольницы имени Павлова – реабилитировали наркоманов и вроде бы пациентов с тяжелыми расстройствами психики.
