\chapter*{В}
\addcontentsline{toc}{chapter}{В}

\textbf{Васильевское озеро} – более не существует. Находилось севернее озера Святища, в Никольской слободке, как понимаю между Броварским шоссе и улицей Раисы Окипной. Вероятно, это от него осталась впадина\footnote{50°29'45.5"N 30°38'15.0"E}, занятая теннисными кортами, между домами по Раисы Окипной 7А, 5Б, 5А, 3В, 3Б, 12А.\\

\medskip

\textbf{Витовская}

50°32'6"N 30°36'55"E

Урочище. Заросшее кустами поле, с остатками стариц, на север от 13 микрорайона Выгуров\-щины-Троещины.\\

\medskip

\textbf{Водопарк} – старое название части Лесного массива. Водопарк, затем Комсомольский, затем Лесной. Начал строиться в 1965-м. В узком смысле, Водопарком именовался район возле улиц Братиславской, Шолом-Алейхема, Милютенка, Киото, Жукова. Он включает в себя насосную станцию «Северный Водопарк» и 8 артезианских скважин.

У коренных жителей улицы Братиславской долгое время сохранялся обычай, относящийся ко времени, когда улица только прокладывалась и не существовало тротуаров – рано утром люди шли не по тротуарам, а по шоссейной части.

На Лесном массиве в первой половине 20 века было озеро. Поперек его сейчас проходит улица Шолом-Алейхема. Озеро было там, где ныне стоят дома 22, 20, и между 19 и 15 (по месту АТБ и базарчика).

Несколько юго-восточнее тогда же лежало болото – по училище Физкультуры и оттуда на север. Эта последняя часть болота, по четной стороне улицы Жукова, сохранялась по зиму 2016 года, когда на болоте вырубили огромные тополя и березы, а местность стали осушать. На 2021 год это огромный пустырь с кочками и торфом, покрытый травой и мусором. А напротив, около футбольного поля на углу Жукова и Матеюка поныне стоит несколько могучих, чуть ли не столетних верб.

Озеро было в первой половине 20 века и на месте станции метро Лесная, его уже тогда перерубило пополам Броварское шоссе.\\

\medskip

\textbf{Волчья гора} 

50°27'34.2"N 30°37'18.9"E

Небольшой холм между станциями метро «Дарница» и «Черниговская», на север от линии метро. Высота его около пяти метров, порос соснами, обустроен вроде скверика – есть скамейки. На вершине установлен памятный камень с надписью: 

\begin{quotation}
\noindent Историческая местность «Волчья гора». На этом месте в 1913 году проведены первые в истории военно-техни\-ческие эксперименты по корректировке артиллерийского огня под руководством авиаторов П. Нестерова и Е. Крутня.
\end{quotation}

Осенью 2017 года Волчью гору еще более «облагородили», а на близлежащем пустыре разбили дополнительный сквер с уложенными плиткой дорожками, которые проложили и на сам холм.

Местность входила в состав артиллерийского полигона, занимавшего пространство между улицами Жукова и бульваром Перова. То есть  полигоном был занят и Водопарк, и жилмассив с улицами Малышко, Бойченко, Юности, Дарницким бульваром. А непосредственно за «Детским миром» или на его месте была артиллерийская вышка.\\

\medskip


\textbf{Воскресенская слободка} – в 16 веке, поначалу от шляхтича Евстафия Дашкевича, а затем в 1577 году от его зятя, князя Евстафия Ружинского, земля «близ Киева, за Днепром состоящая», именуемая Евстафиевской, была передана церкви Киево-подольской Воскресенской (с устроенным при ней приделом св. великомученика Евстафия). Церковь тоже построили на средства Дашкевича.

За полтораста лет владения церковью, на земле той населилась слободка и прозвалась Воскресенской. В 1645 (королем польским Владиславом 4-м), 1650 (Яном Казимиром) и 1718 (гетманом Иваном Скоропадским) годах граница оной земли в документах определялась так:

\begin{quotation}
\noindent начав от глинищ\footnote{Речь идет о Гнилуше? Где-то там была глина, поскольку я находил остатки сыродутных горнов.} по дороге Остролуцкой по бор Миленовский, а от бору долиною живцем\footnote{Ручьем.} в чрез Криничину в речку Мелиновку\footnote{См. про озеро Малиновку.}, в гору Миленовкою в речку Радунку, Радункою же в прудец в гору, прудцем до болот сухих, а чрез болота сухие и через дорогу Басканскую\footnote{Басканский шлях, на местечко Баскань Переяславского полка.} к дороге Остролуцкой, а тою дорогою опять к глинищам, откуда границ начиналась.
\end{quotation}

Подробный разбор этих именований приведен мною в книге «Ересь о Киеве».

В 1719 году киевский губернатор Д. М. Голицына забрал слободку у церкви, и с переименованием в Губернаторскую слободку перевел в ранговую усадьбу киевских губернаторов и комендантов Печерской крепости. Название впрочем не прижилось, а в 1786 году слободка стала просто казенной.

Сейчас от Воскресенской слободки остался частный сектор по улице Марка Черемшины, к западу от озера Радунки, а на восток от него в 1960-х был построен жилмассив Воскресенка. 

В его недрах однако по 2021 год, рядом с жилым зданием 1980-х на Петра Запорожца 9Б, находятся развалины частного домика (Петра Запорожца 9В)\footnote{50°28'53.1"N 30°35'57.2"E}. По словам местной жительницы, там обитал некий «заслуженный человек», отказавшийся переселяться при застройке массива. Дом этот виден на аэрофотоснимках 1960-х.\\

\medskip


\textbf{Вьетнам} – микрорайон Воскресенский-1, построенный вдоль узкой улицы Воскресенской. По устроения на территории завода Ремзавод нового ЖК «Парковые озера», Вьетнам был тихим спальным районом с застройкой из панельных хрущовок и советской девятиэтажки. Вьетнам по улице Воскресенской граничит, кроме ЖК, еще с поселком Ремзавода, а западом выходит на бульвар Перова.\\

\medskip


\textbf{Вырлица} – озеро между Позняками и Бортнической станцией аэрации. Непомерно раздулось от добычи тут песка. Б\'ольшая часть нынешней Вырлицы находится на месте бывшего тут до строительства жилых кварталов болота, а исконная, маленькая Вырлица лежала непосредственно к югу от Мусоросжигательного завода «Энергия» (Коллекторная, 46) – на ее месте сейчас болотце\footnote{50°23'19"N 30°39'44"E} и грунтовая дорога.\\

\medskip


\textbf{Вязки}

50°29'55"N 30°34'18"E

Заросшие деревьями пески и рудная земля между проспектом Ватутина и бурой речкой\footnote{ 50°30'2"N 30°34'18"E}, где обитает колония железобактерий. Добраться туда можно от развязки улицы Бальзака с Ватутина. Подробнее см. мою «Ересь о Киеве», часть про Городок.
