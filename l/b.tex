\chapter*{Б}
\addcontentsline{toc}{chapter}{Б}

\textbf{БВС} – народное название бульвара Верховного совета в Соцгороде. Когда идешь по бульвару от проспекта Гагарина, то раньше прямо по курсу было видно Лавру на правом берегу, а сейчас обзор заслоняет ЖК, выстроенный вместо молокозавода.

Как дорога, БВС существовал уже в 1930-е, будучи разделительной линией между полосой леса с юга и пустошами с севера, и относился к территории Дарницких военных лагерей (до революции – лагерь юнкерского училища). Современное название носит с 1949 года. Застроен в основном панельными хрущовками с тихими, зелеными дворами. В западном конце есть несколько домов сталинской эпохи, один такой слывет Розовым домом.

Последними номерами (восточная сторона улицы) БВС выходит к проспекту Юрия Гагарина, и там в доме 31 дробь 1 с советских времен по 2020 год сохранялся спортивный магазин «Олимпиец».

Начало бульвара упирается в железную дорогу, огороженную бетонной стеной, впрочем оборудован переезд, через него люди ходят на Левобережку.

К востоку от рельсов, вдоль улицы Марины Расковой (ведущей к Левобережке) с 2018 года земля застроена ЖК «Галактика», на месте корпусов молокозавода №2, который был запущен в 1961 году, а в 1995 превратился в «Галактон». После ряда перепродаж завод соединился с «Даноном», а в 2015 производственные мощности были перенесены в Кременчуг и Херсон.\\

\medskip

\textbf{Белый дом} – дом на улице Юности, 8/2, у перекрестка с Космической. Называется так по виду и дал народное имя маршруточной остановке (обычный городской транспорт по этой узкой улице с односторонним движением не ходит). Иногда говорят еще – Серый дом. \\

\medskip

\textbf{Березник} – хутор, давший имя району Березняков. В первой половине 20 века находился между улицами Серафимовича, Березняковской (вдоль железной дороги), Бучмы и озером Тельбин. Кроме Тельбина, на хуторе было множество других озер.

Тельбин (остатки одноименной речки и пролива) тогда начинался от нынешнего проспекта Воссоединения, и даже севернее, от дома номер 11 на улице Энтузиастов на Русановке, перерезанный насыпью дороги, где потом возник проспект. А на месте детсада номер 577 были воды Тельбина!

Южнее хутора, Тельбин перерубила железная дорога, отделив от него Нижний Тельбин, тогда имевший, как и верхний, долгое русло как у луговой речки.

Хуторское кладбище\footnote{50°26'16.1"N 30°36'42.7"E} было на возвышенном поле-пустыре около нынешнего пересечения проспекта Воссоединения и улицы Березняковской, около моста.

На немецкой карте 1943 года, в хуторе Березник обозначены следующие улицы: 

Шевченко – примерно нынешние адреса Тычины 7, 11, 26. Судя по одному письму 1945 года, в то время хутор Березняк воспринимался местными как Кухмистерская слободка, ибо они считали улицу Шевченко – что она в слободке, и на конвертах так писали.

Железнодорожная – шла по восточному берегу озера Тельбин, в современной южной его половине. Югом упиралась в железную дорогу.

Коротенко – перпендикулярно примыкала к Железнодорожной, а начиналась там, где ныне пустырь между домами 12 и 16 на Березняковской. Этой восточной частью улица упиралась в железную дорогу.

Заозерная – находилась между Шевченко и Коротенко, параллельно им. Востоком примыкала к железной дороге.

Хутор ограничивался с востока и юга железной дорогой, с запада – Тельбином, с севера – Наводницким шоссе.

Первым высотным домом, построенным уже на жилмассиве Березняки, было здание на нынешней Бучмы, 1. Новая улица тогда называлась Разливной. Тельбин верхом достигал окре\-стностей этого дома, постепенно озеро засып\'али, изгоняя к югу.\\

\medskip

\textbf{Берёзка} 

50°32'23"N 30°32'25"E

Заброшенная (на 2020) база отдыха на острове Муромец, на западном берегу длинного озера-старицы Кинище. Три ряда домиков, всего около двадцати штук. Относится к ведомству Боярского отделения связи.\\

\medskip

\textbf{Берёзка}, оно же Веселка, озеро

50°27'47"N 30°40'27"E

Большой проточный водоем в лесу между Быковней и ДВРЗ, представляет собой пруд на речке Дарнице, расчищенный в 1954 году работниками ДВРЗ. На берегу есть лодочная станция и кафе. К озеру ведут грунтовые дороги. Глубина мне неизвестна, однако в озере тонут купальщики.

Из озера вода дальше течет по двум руслам. Одно, Дарницкий мелиоканал, следует на запад, затем на юг и на запад к Днепру. Более-менее лежит в исторической пойме реки Дарницы. А другое, Северодарницкий мелиоканал – на северо-запад, через Быковню, затем на север вдоль Лесного массива, и потом на запад в Воскресенский коллектор, через который попадает в залив Десёнку (отличается от одноименного рукава) на Радужном массиве. 

В Северодарницкий канал включен и ручей, питающий пруды в парке Победы – сей ручей тоже попадает в итоге в Воскресенский коллектор, истекая из цепи прудов и каналом пройдя по лесу вдоль улицы Курнатовского от клиники на Алишера Навои, 1 до морга и роддома на Петра Запорожца. Поэтому диггерские слухи о том, что в ручей спускают некие стоки из морга лишены оснований, ибо морг находится ближе к месту, где ручей скрывается в подземный коллектор, \\

\medskip

\textbf{Берики} – народное название Березняков.\\

\medskip

\textbf{Берковищина}, залив – примыкает к пересечению Днепровской набережной и 
улицы Анны Ахматовой.\\

\medskip

\textbf{Бетонка} – несколько небольших бетонных блоков около высотки по адресу Милютенко, 17В. Служит местом, где собираются потусить школьники.\\

\medskip

\textbf{Большая поляна}

50°29'26.7"N 30°39'24.5"E

Народное название здоровенной поляны на восток от Лесного массива, южнее Алмазного озера. На картах ошибочно подписывается как урочище Куричево, хотя Куричево – это как раз Алмазное озеро и есть. Поляна есть осушенное болото, а в лиственном лесу на запад от луговины до сих пор можно найти места, где следы от шагов в черной торфянистой почве наполняются водой. С востока к поляне подступает березняк, с севера – хвойный лес. Очень зеленое место, с густым разнотравьем, местами очень высоким.

Прежнее болото было соединено каналом с истинным болотом Куричевым, превратившимся в Алмазное озеро. Канал сохранился слева от дороги, как идти от Большой поляны к озеру.

На Большой поляне собирают целебные травы, в березняке у окраины устраивают пикники, а иногда снимают кино.\\

\medskip

\textbf{Боярское}, урочище – западный берег озера Малиновки, между озером и Чертороем. Занято дачами мелкого калибра и коттеджами, которые произросли на месте оных дач.\\

\medskip

%\textbf{Бортничи}

\textbf{БСП} – больница скорой помощи на улице Братиславской. Выходить однако на остановке «Медучилище», ибо напротив больницы расположено это самое медучилище, а рядом еще и «Центр Сердца» – все эти здания съели часть близлежащего осколка соснового леса.\\

\medskip

\textbf{Будища} – урочище в селе Выгуровщине, ныне занято троещинской улицей Будищанской.\\

\medskip

\textbf{Буратино} – советский магазин культтоваров на проспекте Воссоединения, 14. Ныне не существует.\\

\medskip

\textbf{Быковня} – поселок около Киева на восток от Лесного массива, недолго идти пешком через березняк да сосняк или по Броварскому шоссе. Застроен в основном частными домами, однако есть несколько многоэтажек.

В 19 веке и начале 20-го, тогдашний хутор Быковня относился к Броварской волости Ост\-рожского уезда. На 1858 год насчитывал 3 двора. В 1897 году тут было 35 душ. В 1938 году Быковню включили в городскую черту.
 
Изначально Быковня лежала вдоль дороги, ныне Броварского проспекта или шоссе. На тридцатые годы 20 века не было улиц Путивльской, Кисловодской, Бобринецкой – дома стояли именно вдоль шоссе, а в месте присоединения к нему нынешней Кузбасской улицы находилось озеро, по обе стороны от шоссе. Сейчас там под дорогой проходит часть Северодарницкого мелиоканала от озера Березки, так что давнее озеро лежало на месте этого проложенного несколько позже канала. Подробнее о мелиоканале читайте в моей «Ереси о Киеве».
   
Вообще Быковню окружали болота – с юга Пляховские, что развились в пойме речки Дарницы, а на западе болото занимало окрестности угла между улицами Жукова и Матеюка (с озером на месте станции метро Лесная, озеро тоже было по обе стороны от шоссе). По 2000 годы сохранялось болотце на север от Быковнянского кладбища, там даже бабка пасла стадо коз, но ко второму десятилетию 21 века болотце высохло. Кладбище старое, на немного возвышенном по отношению к болоту пригорке, показано еще на картах 19 века. Отгорожено от леса бетонным забором.

Перпендикулярно к «старой» Быковне примыкает возникший в 1950-х поселок Радистов с улицами Радистов и Ушицкой, построенный для служащих радиостанции. 

%В начале Ушицкой расположена заброшенная территория передающего радиоцентра в/ч А-0799 – тут были казармы, гараж, командный пункт и бомбоубежище. На 2021 год сохранились остатки полуразобранных зданий и КПП. Году эдак в 2006 воинская часть еще стояла сохранной за сетчатым забором с колючей проволокой. Дорожки были аккуратно заметены, деревья побелены. Потом дорожки засыпала хвоя, ограда покосилась, и началось расхищение бесхозной собственности.

%В другом конце поселка Радистов на 2021 год есть закрытый, полузаброшенный радиопередающий цех №1 Киевского радиопередающего центра, то есть аппаратная часть радиостанции, с охлаждающими резервуарами для некогда стоявших тут высоких (до 110 метров) мощных антенн. Внутри и снаружи – повышенный радиационный фон.

%Центр работал с 1953 по 1998 год. К антеннам были подключены ламповые передатчики «Снег», вещавшие в КВ и СВ диапазонах. Могли держать связь, например, с атомными ледоколами. Некоторые считают, однако, что центр играл роль обычной глушилки вражеских голосов.

В нулевых на запад от поселка Радистов в лесу лежало множество полуразобранных автомобилей, как понимаю ворованных.

Из достопримечательностей Быковни отмечу церковь святого мученика Иоанна Воина, расположенную в синем железнодорожном вагоне. Я видел вагоны-гаражи на ДВРЗ, но вот церковь в вагоне прежде никогда. Не знаю, существует ли она до сих пор.

В Быковне прожил последние годы жизни актер Борислав Брондуков – из поселка родом его жена Катерина.\\
