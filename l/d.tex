\chapter*{Д}
\addcontentsline{toc}{chapter}{Д}

\textbf{Дарница}, речка – подробно о ней я рассказал в книге «Ересь о Киеве». Увы, я не проследил Дарницу полностью от двух ее устий до истока, который, по словам профессора Виктора Вишневского, лежит в Броварах и вытекает из озера под именем Оболонь в виде ручья, пересекающего улицу Киевскую около дома номер № 125.

Я не успел еще там побывать, а проследил Дарницу только от Княжичей, вернее от Ялынки, после которой речка сия протекает через цепь прудов по местности, что в 19 веке слыла болотом Пляховским с одноименным ручьем.

Что до Княжичей, в них мне известен участок Дарницы около начала улицы Киевской, где перекресток с улицей Лагуновой\footnote{Мария Лагунова, танкистка, освобождала во время Великой Отечественной войны Бровары и потом жила в них.}. Справа от остановки на Лагуновой – переезд через овражек с этой самой Дарницей. Она течет тут с запада на восток, и вот откуда течет сюда с востока, я не знаю, туда еще не ходил.

В низовьях Пляховского болота Дарница впадает, условно говоря около ДВРЗ, в озеро Березку, из которого вытекает уже двумя руслами – мелиоканалами. Обо всём читайте в «Ереси».\\

\medskip

\textbf{Дарница}, район – лежит много в стороне от одноименной станции метро. Есть Старая Дарница и Новая Дарница. Последняя – кварталы вдоль улицы Бориспольской. Исконная Старая Дарница – там где улица Азербайджанская, параллельный ей отрезок Пражской, Гродненская, Алма-Атинская (до ДВРЗ). Между Новой и Старой Дарницами расположен Дарницкий вокзал. То, что в 19 веке называлось просто Дарницей, теперь Старая Дарница.\\

\medskip

\textbf{Дарница} – обиходное название окрестностей одноименной станции метро. Якоже рекохом выше, к настоящей Дарнице отношения не имеет, но те люди, которые никогда не ездят в ту, настоящую Дарницу, под Дарницей понимают эти вот окрестности метро.\\

\medskip


\textbf{Дарницкое}, озеро – исчезнувшее огромное озеро, лежало между нынешним южным углом Дарницкой ТЭЦ и оттуда на юг почти до Пражской. Сейчас на его месте, кроме улиц, разные гаражи и автомобильные хозяйства.

Озеро питалось от речки Дарницы, пущенной туда в спрямленном русле через болото урочища Лески (к северо-востоку от озера). На юго-восточном берегу озера был частный сектор Старой Дарницы, сейчас там Гродненский переулок с высотками. Из озера, речка текла дальше лесом на юг, до железной дороги, и затем под нею, вдоль нее (с южной стороны железнодорожного полотна) до нижней половины перерезанного той же железной дорогой пополам Тельбина.\\

\medskip


\textbf{ДВРЗ} – название это относится более к поселку Дарницкого вагоно-ремонтного завода, нежели к самому предприятию, своими корпусами и сетью рельсов примыкающему к южной части поселка. В ДВРЗ можно добраться на трамвае (номера 23 и 33) или маршрутке от Ленинградской площади.

Поселок растянулся вдоль основной своей улицы – Алма-Атинской, а к ней примыкают другие улицы – Паровозная, Макаренко, Рогозовская, Изобретателей, Марганецкая, Инженера Бородина, Волховская, Семафорная, Машинистовская и прочие.

Застройка различна – есть районы двухэтажек с уютными двориками и палисадниками, есть добротные сталинские здания, а также хрущовки и высотки. Северная и восточная часть ДВРЗ, выходящая к сосновому лесу, занята частным сектором. Некоторые гаражи в ДВРЗ сделаны из грузовых вагонов.

В поселке сонно, тихо и зелено, ощущение советского времени. Есть своя водонапорная башня, стадион «Днепровец». За клубом\footnote{Архитектор С. Соколовский.} ДВРЗ, который был торжественно открыт в 1954 году еще Будённым, расположен небольшой и холмистый парк «Сосновый», который ощутимо уменьшился из-за строительства церкви рядом с клубом. В ДВРЗ работают пять детских садов, две школы, Киевская городская клиническая больница № 11.

В 1960-е, ДВРЗ мог похвастать даже собственной газетой на украинском языке – копеечным «Дарницким вагоноремотником», освещавшим жизнь поселка и завода.

На пути от Лисок к ДВРЗ в стороне от пустынной, лежащей среди промзоны улицы Довбуша, есть небольшое озеро Калдёпка, на берегах которого отдыхают жители поселка. К озеру примыкают улицы Паровозная и Марганецкая с переулками. Оно лежит в пойме речки Дарницы, и выкопано во второй половине 20 века, будучи краем болота урочища Лиски.

Другое озеро – Берёзка – большее, находится севернее в лесу, туда ведут грунтовки.

В еще в первой половине 20 века в ДВРЗ было третье озеро, теперь на его месте гаражный кооператив\footnote{50°27'06.6"N 30°41'17.9"E} к востоку от Алма-Атинской 1-А (однако не часть этого кооператива к югу).\\

\medskip


\textbf{Демидовка} – речка, который начинается в Счастливом, течет на запад через Никольский лес и впадает в озеро Вырлицу около Бортничей. На Демидовке было много пионерлагерей – Луч, Алмазный, Родина и другие.\\

\medskip


\textbf{Долбичка} – пролив Черторыи-Десёнки между Трухановым и Долобецким островами, также название пляжа нудистов. Про Долбичку в историческом ракурсе читайте в «Ереси о Киеве».\\

\medskip


\textbf{Дом на ножках} – массивный, огромный дом на Березняках, по адресу Бучмы, 13. Стоит на бетонных опорах. Построен в 1970-м.\\

\medskip


\textbf{ДК Дарницкой ТЭЦ} – двухэтажное здание на Павла Усенко, 3, давно уже не дом культуры. Открылся в 1955 году, был культурным центром района, заодно и кинотеатром (а близлежащий кинотеатр «Ленинград» открылся в 1961).\\ 

\medskip


\textbf{Дормаш} – в Новой Дарнице, окрестности завода «Ремдортехника» на Бориспольская, 7. При СССР он именовался «Дарницким заводом по ремонту дорожной техники». В народе завод и близлежащую остановка называют «Дормаш». Граничит с Киевским Радиозаводом (Бориспольская, 9).\\

\medskip


\textbf{ДНД} – двухэтажный старый дом на углу Попудренко и Красногвардейской, ближайший к метро «Черниговская». Тут в советское время размещался штаб Добровольной Народной Дружины. Рядом была доска «Окно сатиры», на которой обличались районные тунеядцы, драчуны и бюрократы. ДНД – один из уже немногих домов, оставшихся от Аварийного поселка.\\

\medskip


\textbf{Днепровский}, советский ресторан – на его месте теперь «Фуршет» по Братиславской, 14.\\

\medskip


\textbf{Диброва}

50°31'16"N 30°38'32"E 

Урочище по прямой и пустынной улице Пуховской (читай – дороге в поле), примыкает к Федорковщине на северо-восток, своим северным углом выходит к отдаленному гаражному комплексу. Поле урочища пересекается водным каналом, близ Пуховской заболоченным. Противоположная сторона местности превращена в свалку.\\

\medskip


\textbf{ДШК} – Дарницкий шелковый комбинат, ныне на месте его прежних корпусов – «Дарынок» и арт-завод «Платформа». ДШК возник в 1947 году.

А на месте большого стадиона ДШК, который находился в четырехугольнике между Красногвардейской, Красноткацкой, Магнитогорским переулком и Магнитогорской улицей, теперь стоит ТРК «Проспект» и «Ашан». Напротив от них через Красноткацкую сохранился сквер ДШК. 

Напротив от «Ашана» через Магнитогорскую лежат скверик и сталинской архитектуры, с колоннами дом культуры другого левобережного предприятия, былого гиганта – «Киевхимволокна», созданного в 1937 году, от него кстати названа и улица Вискозная. Вискозное производство завод остановил в 1996 году, с девяностых пошел спад производства и в начале нулевых была продана часть корпусов.