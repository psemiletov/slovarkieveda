\chapter*{К}
\addcontentsline{toc}{chapter}{К}

\textbf{Казённое}, болото

50°29'04.5"N 30°37'07.0"E

Местами сохранившееся болото на восток от Куликова, между улицами Братиславской и Крайней. По землям заросшего березняком и сосняком болота проложена часть Северо-Дарницкого мелиоканала. Казенное питалось от ручья с устьем около дома по адресу улица Жукова, 20, между ним и улицей. 

Часть русла сего ручья стала частью Северодарницкого мелиоканала, она выходит черным от ила и торфа руслом из коллектора за телефонной станцией на улице Милютенко, 29 и шурует дальше по лесу на запад до впадения в Воскресенский коллектор за обочиной Крайней улицы.

Считается, что по этому руслу протекает одна из ветвей речки Дарницы, однако воды Дарницы там очень мало и то не каждый год. В последнее время канал, по которому течет вода Дарницы из озера Берёзки на отрезке от Быковни до улицы Жукова совсем пересох.\\

\medskip


\textbf{Калдёпка}

50°26'47.9"N 30°39'53.7"E

Озеро на стыке Лисок и ДВРЗ. На карте 1930-х на том месте – эдакий язык болота в пойме речки Дарницы.\\

\medskip


\textbf{Келийки}

50°31'48"N 30°38'31"E

Урочище непосредственно к северу от Дибровы. Болотистые луга на запад от горы с ТЭЦ-6 и к югу от Погребов. В давнее время – часть поймы старого русла Десны.\\

\medskip


\textbf{Клунища}

50°32'3"N 30°37"36"E

Местность в поле за ЖК «Милославичи», к востоку от урочище Витовское.\\

\medskip


\textbf{Ковпыт}, болото

50°30'46"N 30°40'22"E

Болото за ТЭЦ-6, частично убитое добычей торфа. Бывшее озеро, владение Лавры. С Ковпытом связано предание, что в болоте затонула целая деревня. Своеобразный град Китеж. Некоторые осушенные части Ковпыта напоминают высохшее песчаное русло широкой реки. Подробнее о болоте читайте в «Ереси о Киеве». На уцелевшей части Ковпыта сохраняется уголок нетронутой природы, растут редкие травы, можно услышать странные звуки.\\

\medskip

\textbf{Кодачок} – см. Горбачиха.\\

\medskip

\textbf{Комсомольский массив} – советское название жилмассива вдоль улицы Малышко, от Детского мира до Братиславской. Достопримечательности – Волчья гора и дом Полукружка.\\

\medskip

\textbf{Коник} (19 век) – приток Днепра на Осокорках.\\

\medskip


\textbf{Конка} (19 век) – приток Днепра на 
Осокорках, не то же самое, что Коник.\\

\medskip

\textbf{Контейнерка}, озеро

50°26'57"N 30°39'45"E

Находится в юго-восточной части железнодорожной станции Киев-Лиски. Занимает небо\-льшую часть прежней заболоченной поймы речки Дарницы.\\

%\textbf{Королёк, озеро}
\medskip


\textbf{Корчовня}, болото – известно с 17 века по первую половину 20-го, было между улицей Кибальчича и проспектом Ватутина. Застроено советскими панельными домами примерно как на Березняках.\\

\medskip

\textbf{Коса} – см. Плавни.\\

\medskip

\textbf{КП} – народное название Ленинградской площади со времен войны по конец 1960-х, от бывшего тут, в Великую Отечественную войну, кон\-трольно-пропускного пункта.\\ 

\medskip


\textbf{Красный хутор} – он же хутор Осокорки. К городу был причислен в 1923 году. На 2021 год это частный сектор с промзоной, лежащий вдоль улицы Ташкентской, по которой до разворота проложена трамвайная линия. У разворота – станция метро «Бориспольская», а вот станция «Красный хутор» – на северо-восток оттуда и от истинного Красного хутора отделена участком Никольского леса, где разбит Парк Партизанской Славы.

На хуторе есть еще несколько улиц – это граничащие с лесом Боровая, Елочная (на ней ближе к Староболгарской находится небольшой пруд), также по юго-западной стороне хутора проходит Харьковское шоссе. В 1940-х его отрезок на этой протяженности был Русановской улицей. Ташкентская тогда же носила имя Чаркова. Переулки между ними носили имена пронумерованных «Новых» улиц, Кронштадская была Мостовой.

Половина домов – уже современные особняки, старых мало. В лесу пасутся козы, на окраинных улицах можно увидеть кур. На хуторе растет какой-то особенный сорт красно-бордовой розы.

2016 год. У обочины Ташкентской стоит пустой серый домик, без забора, без окон. Синие рамы только остались, да нехитрый узор по стенам. Синяя же деревянная веранда сбоку. Там возятся котята с кошкой. Желтые цветы по обеим сторонам дорожки, что вела от калитки. Куст знаменитых роз – глубже в садике. Когда-то за ними следил глаз, ухаживала рука.

На карте «Археологические памятники на территории Киева с древнейших времен до середины 1-го тысячелетия нашей эры» из книги «Киев: карты, иллюстрации, документы» 1982 года в лесу около хутора, в парке Партизанской славы, эдак между озерами и улицей Боровой, обозначен «могильник (195 захоронений)». Археологи раскопали их в 1950-х годах. Отнесли к «медному веку». Обряд захоронения – трупосожжение, в погребальных урнах. По найденным остаткам костей ученые решили, что среди трупов были взрослые, подростки и дети.

Другое древнее поселение на Красном хуторе найдено на юго-запад от озер, то бишь по другую сторону от хутора, за перекрестком улицы Грузинской с Харьковским шоссе. Там теперь квартал высоток. На его углу, там где другой перекресток – с улицей Архитектора Вербицкого – вековой дуб, но может быть его уже спилили.\\

\medskip


\textbf{Красный трактир} – на стыке 18-19 веков «трактир городской» примерно в окрестностях между улицами Шлихтера и Окипной, находился у перекрестка, откуда одна дорога шла южнее озера Святища и потом разветвлялась к Никольской слободке и берегу Днепра; другая шла на Москву, на восток, третья же восток, но южнее, к озеру Дарнице и шинку при нем.\\ 

\medskip


\textbf{Криничка}, озеро – еще в начале 20 века это озеро было на Позняках, по месту нынешнего перекрестка Анны Ахматовой и Григоренко.\\ 


\medskip

\textbf{Куба} – местное название окрестностей улицы Кибальчича.\\

\medskip

\textbf{Кубик Рубика} – Детский мир около ст. метро Дарница. Он же Соты, за подобное сотам декоративное оформление.\\

\medskip


\textbf{Куликово}, болото – пределы его на первую половину 20 века можно описать как – с юга от перекрестка улиц Петра Запорожского и Курнатовского, затем к перекрестку Братиславской и Крайней (у «Эпицентра»), где смыкаются две ветви болота. Одна из ветвей начиналась от дамбы около Крайней, 11, другая от Братиславской, 18Б, и между ними тоже была заболоченная местность, где среди проток болота были разбросаны усадьбы, домики хутора Куликово.

От русла болота сохранилось общее русло наискось от Братиславской до Курнатовского – большая ложбина за жилым районом, между ними и сосновым лесом. В ложбине иногда течет некая вода, видны следы хозяйственной деятельности – некие перегородки поперек русла. По примыкающему лесу параллельно бывшему руслу идет земляная дамба, от нее перпендикулярно к руслу проведены сухие уже мелиоративные каналы.

К востоку от улицы Крайней, в топком березняке, болото Куликово сохранилось по 21 век в прежних пределах, хотя и основательно подсушенное.

С юго-востока к нему примыкает болото Казенное, питаемое от ручья и подведенной сюда части Северодарницкого мелиоканала.

Болото Куликово, вероятно, занимает ветвь давнего русла Десны. Продолжение того же русла – водоем в парке Победы близ метро «Дарница», только вода в нем течет в другую сторону, условно говоря на север.

В 20 веке, в военные и послевоенные годы в районе Куликово размещались позиции дежурных зенитных батарей.\\

\medskip


\textbf{Куликово}, хутор – поселок рядом с одноименным болотом, между нынешними улицами Крайней, Братиславской и Закревского. Некогда хутор от Воскресенской слободки, затем – просто хутор Куликово. До сноса частного сектора в 1970-х, в нем были такие улицы: Беговая, Богушевская (и Богушевский переулок), Бузковая (Сиреневая), Остапа Вересая, Грузецкая, Красноярский переулок, Леониевская, Мостищенская (и одноименный переулок), Тютюновая (Табачная), Яворовский переулок, Ярошевская.

На 21 век уцелели остатки частного сектора на Вересая, Мостищенской, Беговой, Томашевской. Мостищанская улица была на месте нынешнего Троещинского рынка. Весной дворы некоторых домов затапливает талой водой.

На месте современной школы №4 стоял частный дом, на самом берегу болота (к югу от дома).\\

\medskip


\textbf{Куричево}, болото – нынешнее озеро Алмазное, а не Большая поляна южнее, обозначенная на картах как Куричево.\\

\medskip


\textbf{Кут}, остров

50°27'44.2"N 30°34'14.0"E

Остров на Десёнке, ныне – северная часть Долобецкого острова. Некогда отделялся от него полностью, пролив сохраняется частично\footnote{50°27'35.4"N 30°34'23.3"E} и сейчас.\\

\medskip


\textbf{Кухмистерская слободка} – прежде, по карте 1750 года, Печерская слободка. Однако на карте допустим 1897-1918 годов, Печерской слободкой подписан хутор Березняк. А согласно карте Шуберта, всё это вместе – Кухмистерское. Таковы карты.

Кухмистерская слободка – поселок южнее хутора Березняки, просуществовал до строительства жилмассива Березняки. Частный сектор находился между, в современных пределах, проспектом Воссоединения, Тельбином, Днепровским шоссе и железной дорогой. Домики стояли на месте Sea Breeze. После их сноса там, перед домом 1 на Тычины, долгое время, по нулевые, был здоровенный луг с грунтовкой наискосок, по которой ходили на пляж.

Хутор имел несколько улиц, причудливо изогнутых между Днепром, старицами (ныне не существуют), а со стороны Тельбина – болотистыми лугами.

Перед домом на Днепровском шоссе 5 работала лесопилка, к ней подходила ветка железной дороги, достающая до начала нынешней улицы Тычины.

С 1923 года слободку занесли в городскую черту.

27 декабря 1960 году за подписью главы исполкома Киевского городского совета Давыдова и секретаря исполкома Б. Ермоловича, исполкомом было принято решение по выделению участка в 235 гектаров, занимаемых Кухмистерской слободкой (на западному берегу Тельбина) и селом Березняки (на восточном берегу Тельбина в тогдашних его границах), под застройку 5-8 этажными домами. Так в итоге и возник микрорайон Березняки. 

Из обоих поселков отселению подлежало 2705 человек.

Названия улиц по немецкой карте 1941 года: Октябрьская (примыкала к лесопилке), Пролетарская, Комсомольская, Красно-???.

Названия улиц по немецкой же карте 1943 года: 

Трипольская – около нынешнего Silver Breeze, примерно от моста Патона до Тычины. 

Трипольский переулок – лежал в углу между Днепровской набережной и Дарницким шоссе. На 2018 год там, кроме прочего, большой заросший пустырь с лежащими параллельно шоссе железнодорожными рельсами.

Кухмистерский приплав (причал) – эдак от конца Березняковской до железной дороги и Дарницкого шоссе. 

Юрия Федковиченка – лежала в южной части слободки, наискосок пересекая нынешние улицы Березняковскую (в ее 20-30 номерах) и Шумского. В районе дома Шумского, 8-А была сельская Трехсвятительская церковь. 

Каневская – между домами Днепровское шоссе 9А и Березняковская 38. 

Базарная (перпендикулярная будущему проспекту Воссоединения) – большая улица, идущая от упомянутого проспекта, а тогда Наводницкого шоссе (эдак от остановки «Русановка») на юг к железной дороге, переваливая через нее в месте, где ныне станция «Левый берег». Улица пересекала бы современную улицу Шумского. Часть Базарной (а по документам – Базарной 2) стало частью современной улицы Серафимовича.

Проспект Воссоединения тогда представлял собой дорогу по насыпи и назывался Наводницкое шоссе, ибо вел по сути к Наводницкому мосту (а не Патона) и на той стороне продолжался Старонаводницкой улицей.

Названия улиц по советской карте 1935 года:

Червоный шлях, Пролетарская, Октябрьская, Комсомольская, Карла Маркса, Червоный приплав, Базарная – из них Комсомольская это начало «оккупационной» Базарной, а «Червоный приплав» – это «Кухмистерский приплав», а Пролетарская скорее всего Каневская.

На 1966 год, в Кухмистерскую слободку ходил автобус номер 5, от другой конечной у пересечения Бастионной и Киквидзе, откуда автобус съезжал на бульвар Дружбы Народов вдоль Кургановской улицы. Маршрут прослеживается по 69 год, на 70-й исчезает, зато на Серафимовича пускают автобусный маршрут 49.

На стыке 18-19 веков, западный берег слободы омывался не Днепром, но длинной узкой старицей или остатком какого-то иного, не днепровского, русла. К западу от оного лежал большой остров, а уже за ним был широкий Днепр. Тогда же, с севера, слобода ограничивалась заливом Ровчаком – его можно соотнести с Русановским каналов в пределах нынешних Березняков от моста Патона по мост у бульвара Давыдова. Этот Ровчак вытекал из Тельбина. 

А на восточном берегу Тельбина находились тогда «двор и пасека» – прообраз хутора Зательбин Березняк.

На 1914 год, на северо-восток от Тельбина и его восточного же берега, считалась Верхне-Кухмистерская слободка, а «обычная» – Нижней. В глубине последней, по крайней мере с 19 века слыло озеро Попово, с центром примерно где здание Березняковская 36-А, по виду старица. В его нижней части был завод по пропитке шпал, лежащий чуть севернее железной дороги, южнее которой была лесопилка.