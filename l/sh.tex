\chapter*{Ш}
\addcontentsline{toc}{chapter}{Ш}

\textbf{Шаломка} – улица Шалом-Алейхема на Лесном.\\

\textbf{Шевченко}, хутор – существовал в первой половине 20 века. Одна его часть лежала на северо-запад от улицы Тростянецкой, другая на юго-запад, где границей хутора служило болото, отделявшее хутор от Позняков. На 2021 год по месту этого бывшего болота лежат, кроме суши, озера Солнечное, Прорва и Горячка. Местность хутора застроена высотными домами первого микрорайона Новой Дарницы и седьмого микрорайона Харьковского массива.\\

\textbf{Шелковый}, пионерлагерь

50°22'57"N 30°45'40"E

Ныне – оздоровительный лагерь, работает по назначению. Еще один из череды детских лагерей, лежащих на речке, что впадает в Вырлицу. Добротные кирпичные строения.\\

\textbf{Шлакоблок} – 10-й микрорайон Новой Дарницы, прячется между лесом и промзоной. Народное название – от 135-го завода железобетонных конструкций МО СССР, выпускавшего шлакобетонные блоки. Район находится по улице Бориспольской, номера домов 27, 29, 31, 33, 35 и так далее. Двухэтажные кирпичные домики, панельные пятиэтажки – их теснят новые высотки.
