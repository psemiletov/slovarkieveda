\chapter*{Ф}
\addcontentsline{toc}{chapter}{Ф}

\textbf{Фанерный} – он же Фанерное, поселок на пологом склоне по правому берегу речки Дарницы, около нынешнего завода Фанплит, что на Фанерной улице. Завод возник в 1909 году как предприятие братьев Габель по производству фанеры. Около завода вырос поселок, к 1950-му насчитывающий шесть улиц частного сектора. Основой поселка была местность между теперешним Фанерным переулком и Каунасской улицей, да по северную сторону от примыкающей к ним Мартовской. Под конец 50-х поселок из частных домов снесли, взамен построили несколько пятиэтажек по улице Мартовской. В 2020 году на Фанерке началось строительство ЖК.

Завод и некоторые старые дома сохранились поныне. Современное народное название местности – Фанерка.\\

\medskip

\textbf{Федорковщина}

50°31'1"N 30°38'19"E 

Поле вдоль улицы Пуховской, примыкает на северо-восток к Монетному двору. Вероятно, возникло как отмель поворота давнего русла Десны (которое проходило там, где ныне озеро Алмазное). Еще в середине 20 века в Федорковщине было несколько остатков заболоченных проток. Сейчас они частью застроены гаражным кооперативом, а частью приметны на местности.
