\chapter*{П}
\addcontentsline{toc}{chapter}{П}

\textbf{Паньковщина} – о левобережной Паньковщине читайте раздел «Позняковщина».\\

\medskip

\textbf{Пискун} – цепь озер непосредственно к югу от Ерёмина через Коллекторную улицу, по сути продолжение былой речки.\\

\medskip

\textbf{Плавни, Коса}

50°26'31.3"N 30°35'20.4"E

Местность, издавна заросшая верболозом, между Русановской набережной и Русановским проливом. Жители Березняков называют ее Плавнями, Русановки – Косой.\\

\medskip

\textbf{Погребы} – бывшее село Печерского монастыря, по описи 1545 года в нем было 4 человека (точнее, плательщика дани), а дань была положена 24 гроша. Очень важная с исторической и языческой точки зрения местность. Подробнее см. «Ересь о Киеве», часть про Городок.\\

\medskip

\textbf{Позняковщина (Поздняки, Паньковщина)}, село – было отдано в 17 веке Петром Могилой Киево-Братскому монастырю. Это село дожило как Позняки до конца 20 века, располагаясь между нынешним Дарницким шоссе и улицей Анны Ахматовой между остатками речки Тельбин и болотами, в которые разливалась речка Дарница. От сих старых Позняков на 2018 год остался частный сектор по улице Любарской (возле озера Прорвы) да Ивана Кочерги.

Позняки были раскинуты до современного днепровского залива Берковщина, который еще в первой половине 20 века был ручьем с продолговатыми прудами, что ли, который отделял Позняки от лежащих южнее Осокорков. В советское время, Осокорки постепенно из села превратилось в дачный поселок. Станция метро «Осокорки» стоит на восточной окраине бывшего села. А вот станция метро «Позняки» – южнее села Позняки, на прежнем месте болотистой луговины.\\

\medskip


\textbf{Полёт} – остановка на углу дома по адресу Проспект мира, 1. Оттуда, кроме прочего, издавна ходит автобус в аэропорт «Борисполь». Официально остановка называется иначе.\\ 

\medskip


\textbf{Поле чудес}

50°26'40"N 30°40'38"E

Заводская свалка около южной части ДВРЗ, между самим заводом и Нечетным парком прибытия, бандитское место.\\

\medskip


\textbf{Полукружка} – она же Колизей, полукруглый при виде сверху дом около станции метро «Дарница», за Детским миром, адрес Андрея Малышко, 3.\\

\medskip


\textbf{Помильное озеро}

50°29'19.9"N 30°41'13.1"E

См. урочище Стыгле.\\

\medskip


\textbf{ПОХ} – собирательное сокращение от Позняки, Осокорки, Харьковский.\\

\medskip


\textbf{Предмостная слободка} – нынешний Гидропарк. На некоторых дореволюционных картах обозначена как Никольская слободка ближняя, в то время как левобережная, исконная Никольская слободка подписана как «дальняя».

Слободка, частный сектор, была сожжена немцами в 1943 году. Отличалась чудовищной плотностью и хаотичностью застройки, в отличие от более распланированной слободки Труханова острова. 

Была даже своя церковь, Иоанна Рыльского, ее место видно, когда едешь на метро с правого берега, по правую сторону, почти около моста: 50°26'37.3"N 30°34'11.4"E

Заложена 3 мая 1909 года, построена на средства  (20500 рублей) Варвары Бобриковой, вдовы из Петербурга в память об ее погибшем на войне муже Иване. Чтобы церковь не затопляло, ее поставили на 5-метровый постамент. Украшением церкви служил хрустальный иконостас. Рядом возвели деревянную колокольню. В 1935 году храм закрыт и переоборудован под общагу. Разрушен вероятно в 1943.

А на месте теннисных кортов, которые видно по той же стороне, когда едешь, было озеро.\\

\medskip

%расширить в следующей редакции

\textbf{Прорва} – озеро на Позняках около проспекта Григоренко. Естественной является только западная его часть\footnote{50°25'18.81"N 30°37'22.52"E}, восточная была выкопана во второй половине 20 века, тогда же вдоль южного берега обоих частей проложили канал\footnote{Выходит из земли в точке 50°25'10.03"N 30°37'42.18"E, уходит в коллектор тут – 50°25'16.84"N 30°37'18.12"E, чтобы присоединиться к речке Дарнице.}, где вода очень мутная. Южнее и параллельно каналу лежит частный сектор улицы Любарской. Через озеро был перекинут мост от дороги, остатки которой находятся между улицей Тепловозной и зданием номер 18 по ней же. Чуть восточнее моста был перекресток, другая дорога отходила на юг – по ее ходу ныне два смыкающихся полуострова с водотоком посередине, соединяющим обе части озера.\\

\medskip


\textbf{Прудок} – сгинувший водоем в виде полумесяца, в селе Выгуровщина по месту нынешних улиц Каштановой и Бальзака (до пересечения его с Драйзера). Подробнее читайте в «Ереси о Киеве».